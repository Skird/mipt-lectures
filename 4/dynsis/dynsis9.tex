\documentclass{article}
\usepackage[utf8x]{inputenc}
\usepackage[english,russian]{babel}
\usepackage{amsmath,amscd}
\usepackage{amsthm}
\usepackage{mathtools}
\usepackage{amsfonts}
\usepackage{amssymb}
\usepackage{cmap}
\usepackage{centernot}
\usepackage{enumitem}
\usepackage{perpage}
\usepackage{chngcntr}
%\usepackage{minted}
\usepackage[bookmarks=true,pdfborder={0 0 0 }]{hyperref}
\usepackage{indentfirst}
\hypersetup{
  colorlinks,
  citecolor=black,
  filecolor=black,
  linkcolor=black,
  urlcolor=black
}

\newtheorem*{conclusion}{Вывод}
\newtheorem{theorem}{Теорема}
\newtheorem{lemma}{Лемма}
\newtheorem*{corollary}{Следствие}

\theoremstyle{definition}
\newtheorem*{problem}{Задача}
\newtheorem{claim}{Утверждение}
\newtheorem{exercise}{Упражнение}
\newtheorem{definition}{Определение}
\newtheorem{example}{Пример}

\theoremstyle{remark}
\newtheorem*{remark}{Замечание}

\newcommand{\doublearrow}{\twoheadrightarrow}
\renewcommand{\le}{\leqslant}
\renewcommand{\ge}{\geqslant}
\newcommand{\eps}{\varepsilon}
\renewcommand{\phi}{\varphi}
\newcommand{\ndiv}{\centernot\mid}

\MakePerPage{footnote}
\renewcommand*{\thefootnote}{\fnsymbol{footnote}}

\newcommand{\resetcntrs}{\setcounter{theorem}{0}\setcounter{definition}{0}
\setcounter{claim}{0}\setcounter{exercise}{0}}

\DeclareMathOperator{\aut}{aut}
\DeclareMathOperator{\cov}{cov}
\DeclareMathOperator{\argmin}{argmin}
\DeclareMathOperator{\argmax}{argmax}
\DeclareMathOperator*\lowlim{\underline{lim}}
\DeclareMathOperator*\uplim{\overline{lim}}
\DeclareMathOperator{\re}{Re}
\DeclareMathOperator{\im}{Im}

\frenchspacing


\begin{document}

\section*{Лекция 9. Топологическая динамика}
\addcontentsline{toc}{section}{Лекция 9. Топологическая динамика}
\resetcntrs

\section{Топологические динамические системы и~их свойства}

Рассмотрим поведение точки, траектория которой задаётся дифференциальным
уравнением $\dot{x} = v(x)$. Простой пример локального исследования такой
системы из дифференциальных уравнений~---фазовые портреты дифференциальных
уравнений на плоскости. Также можно изучать более глобальные параметры,
например, <<предельные циклы>>.

Вспомним, что динамическая система в~этом случае это $(X, \tau, T^t, G)$, где
$\tau$~--- топология, а~$T^t$~--- поток непрерывных преобразований.

Общие конструкции (считаем, что время~--- $\mathbb{Z}$):
\begin{itemize}
	\item $\omega$-предельные точки $\omega_T(X) = \{y: \exists x \in X, k_j
		\rightarrow \infty: y = T^{k_j}x \}$. Обозначим также $X' = \omega_T(X)$.
	\item $\alpha$-предельные точки $\alpha_T(X) = \{y: \exists x \in X, k_j
		\rightarrow -\infty: y = T^{k_j}x \}$.
	\item $T$ называется \emph{транзитивным}, если $\exists x_0: \omega_T(\{x\}) =
		X$
	\item $T$ называется \emph{минимальным}, если $\forall x_0: \omega_T(\{x\}) =
		X$
\end{itemize}

\section{Символические системы}

\begin{definition}
	Пусть $\mathbb{A}$~--- конечный алфавит. $\Omega = \{(\ldots x_0 x_1 x_2
	\ldots) \mid x_t \in \mathbb{A}\}, S: (x_i) \mapsto (x_{i+1})$. Тогда
	$(\Omega, S)$~--- \emph{топологическая система Бернулли}.

	Топология слабая: база $W_u = \{x: u \preceq x\}$. Также можно задать метрикой
	$d(x, y) = \frac{1}{2} |x_0 - y_0| + \sum\limits_{i \ne 0} \frac{|x_i - y_i|}
	{2^{|i|} + 2}$.
\end{definition}

\begin{exercise}
	Изучить все описанные свойства для бернуллиевской системы.
\end{exercise}
\begin{exercise}$(\ast)$ Изучить все описанные свойства для динамической
	системы из лекции 7 ($1 \rightarrow 1, 0 \rightarrow 0010$).
\end{exercise}
\begin{exercise} Пусть $\overline{Y} =$ замыкание $\{S^k y: y \in Y\}$.
	Показать, что $\overline{Y}$~--- компакт и~найти $S(Y)$ для следующих $Y$:
	\begin{itemize}
		\item $Y = \{(\ldots 0101010 \ldots)\}$.
		\item $Y = \{(\ldots 0001000 \ldots)\}$.
		\item $Y = \{(0\ldots010\ldots010\ldots)\}$ (единицы на местах $i$ и~$j$).
	\end{itemize}
\end{exercise}
\begin{exercise}
	Привести пример $X \subset \Omega$, которая была бы транзитивна, но не
	минимальна.
\end{exercise}
\begin{exercise}$(\ast)$
	Привести пример системы $X \subset \Omega$, такой что $X \ne X' \ne X''$.
\end{exercise}

Рассмотрим динамическую систему <<подкова Смейла>>: $\Omega = [0; 1] \times [0;
1]$, $T(x, y) = (\frac{x}{3} + \frac{2y}{3}, 3y)$. Множество $K \times K$
получается инвариантным, и~на нём возникает интересная динамика, на самом деле
на этом подмножестве система бернуллиевская.

\begin{definition}
	\emph{Топологическая марковская цепь} это слова $\ldots x_0 x_1 x_2 \ldots$
	над $\mathbb{A}$, притом слово $\alpha \beta$ разрешено, если в~некотором
	заданном графе на всех словах $\mathbb{A}^\ast$ есть стрелка $\alpha
	\rightarrow \beta$.

	Для обощения на большую размерность стоит упомянуть, что лучше использовать не
	<<разрешённые>> переходы, а~наоборот, <<запрещённые>>.
\end{definition}

Пример двумерной марковской цепи: $\mathbb{Z}^2$ и~запрещены конфигурации, где
в~клетках $(i,j), (i+1,j), (i+1,j+1)$ сумма равна 1 по модулю 2 (алфавит
бинарный). Такая цепь, наример, является контрпримером к~проблеме Рохлина
о~кратном перемешивании, так как является двукратно перемещивающей, но не
трехкратно.

Следующий пример принадлежит Аносову: дело происходит на $\mathbb{T}^2 =
\mathbb{R}^2 / \mathbb{Z}^2$.

$$ A: \begin{bmatrix}x\\y\end{bmatrix} \mapsto \begin{bmatrix}2 & 1\\
1 & 1\end{bmatrix}\begin{bmatrix}x\\y\end{bmatrix}$$

Интересные свойства:
\begin{itemize}
	\item $A\mu = \mu$.
	\item $A$~--- автоморфимзм, диффеоморфизм.
	\item $A \cong $ марковскому процессу.
	\item $\exists A^{-1}$.
\end{itemize}

\end{document}
