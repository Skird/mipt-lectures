\documentclass{article}
\usepackage[utf8x]{inputenc}
\usepackage[english,russian]{babel}
\usepackage{amsmath,amscd}
\usepackage{amsthm}
\usepackage{amsfonts}
\usepackage{amssymb}
\usepackage{cmap}
\usepackage{centernot}
\usepackage{enumitem}
\usepackage{perpage}
\usepackage{chngcntr}
%\usepackage{minted}
\usepackage[bookmarks=true,pdfborder={0 0 0 }]{hyperref}
\usepackage{indentfirst}
\hypersetup{
  colorlinks,
  citecolor=black,
  filecolor=black,
  linkcolor=black,
  urlcolor=black
}

\newtheorem*{conclusion}{Вывод}
\newtheorem{theorem}{Теорема}
\newtheorem{lemma}{Лемма}
\newtheorem*{corollary}{Следствие}

\theoremstyle{definition}
\newtheorem*{problem}{Задача}
\newtheorem{claim}{Утверждение}
\newtheorem{exercise}{Упражнение}
\newtheorem{definition}{Определение}
\newtheorem{example}{Пример}

\theoremstyle{remark}
\newtheorem*{remark}{Замечание}

\renewcommand{\le}{\leqslant}
\renewcommand{\ge}{\geqslant}
\newcommand{\eps}{\varepsilon}
\renewcommand{\phi}{\varphi}
\newcommand{\ndiv}{\centernot\mid}

\MakePerPage{footnote}
\renewcommand*{\thefootnote}{\fnsymbol{footnote}}

\newcommand{\resetcntrs}{\setcounter{theorem}{0}\setcounter{definition}{0}
\setcounter{claim}{0}\setcounter{exercise}{0}}

\DeclareMathOperator{\aut}{aut}
\DeclareMathOperator{\cov}{cov}
\DeclareMathOperator{\chos}{ch}
\DeclareMathOperator{\argmin}{argmin}
\DeclareMathOperator{\argmax}{argmax}
\DeclareMathOperator*\lowlim{\underline{lim}}
\DeclareMathOperator*\uplim{\overline{lim}}
\DeclareMathOperator{\re}{Re}
\DeclareMathOperator{\im}{Im}

\frenchspacing


\begin{document}

\section*{Лекция 5. Спектральная теорема II}
\addcontentsline{toc}{section}{Лекция 5. Спектральная теорема II}
\resetcntrs

\section{Общая спектральная теорема}

Список праздных фактов:
\begin{itemize}
	\item $\text{supp }\sigma = \bigcup\limits_{\substack{ \sigma(X
		\setminus K) = 0,\\\text{K~--- замкнуто}}}$
	\item $\sigma_1 \ast \sigma_2$~--- распределение случайной величины~$\xi_1 +
		\xi_2, \xi_1 \bot \xi_2$.
	\item (абсолютная непрерывность мер) $\nu \ll \mu \Leftrightarrow \nu = p(z)
		\mu, p \in L^1(\mu)$.
	\item (сингулярность мер) $\nu \bot \eta \Leftrightarrow \exists $борелевское
		$F: \nu(F) = 1, \eta(\Omega \setminus F) = 1$
	\item Любые две меры $\sigma_1, \sigma_2$ можно представить как $\sigma_1 =
		\nu_1 + \omega_1, \sigma_2 = \nu_2 + \omega_2$, притом $\omega_1 \sim
		\omega_2 \bot \nu_1 \bot \nu_2$.
\end{itemize}

\begin{theorem}
	$\sigma = \sigma_d + \sigma_s + \sigma_{ac}$ (представляется
	в~виде суммы дискретной составляющей, сингулярной составляюшей и~абсолютно
	непрерывной составляющией), притом $(\sigma_d, \sigma_s) \bot \sigma_{ac}$.

	При этом $\sigma_d \sim 1_\Lambda, \Lambda < S^1$~--- дискретная.
\end{theorem}

\begin{theorem}[$\ast$]
	Если~$\hat T$~--- эргодическое в~бесконечномерном $L^2(x, \mu)$, тогда
	$Sp(\hat T) = \text{supp }\sigma = S^1$.
\end{theorem}

Если рассматривать системы с~кратностью, получается картина, которую можно
воспринимать двумя способами:
\begin{itemize}
	\item Есть меры $\sigma_1, \ldots, \sigma_{\infty}$, попарно сингулярные,
		притом у~нас есть по $n$ копий пространства $V_n$: $\hat T\mid_{V_{2,j}}
		\cong (L^2(\sigma_2), \mu_2)$.
	\item У~нас есть $\sigma_1, \ldots, \sigma_{\infty}$, притом $\sigma_{k+1} \ll
		\sigma_k, \sigma_1 = \sigma$. В~терминах предыдущего случая $\sigma =
		\frac{\sigma_1}{4} + \ldots + \frac{\sigma_k}{2^{k+1}} + \ldots +
		\frac{\sigma_\infty}{2}$.
\end{itemize}

Спектральный инвариант тогда имеет вид $(\sigma, M(z))$, где $M(z)$~---
измеримая функция кратности.

\section{Семинарская часть}

\begin{definition}
	Пусть $T: (X, \mu), S(Y, \nu)$. $\eta$ есть джойнинг $T, S$ если $\pi_x
	\eta = \mu, \pi_y \eta = \nu, (T \times S) \eta = \eta$.
\end{definition}

Диагональный автоджойнинг: $\Delta(A \times B) = \mu(A \cap B)$.

\begin{exercise}~
	\begin{itemize}
		\item $\Delta_S (A \times B) = \mu(AS \cap B)$~--- джойнинг, если $ST = TS$,
			$S$ сохраняет меру. Замечание: $\Delta_{T^k}(A \times B) = \mu(T^k A \cap
			B) \rightarrow \mu(A) \mu(B) \Leftrightarrow \Delta_{T^k}
			\overset{w}\rightarrow \mu \times \mu$~--- джойнинговое определение
			перемешивания.
		\item $T: \sigma_1, S: \sigma_2, \sigma_1 \bot \sigma_2 \Rightarrow T \bot
			S$~--- дизъюнктны, то есть единственный джойнинг~--- это $\mu \times \nu$.
	\end{itemize}
\end{exercise}

Если есть $\beta(f(x), g(y)) = \int\limits_{X \times X} f(x)g(y) d\eta$, то она
представима как $\left< J_\eta, g \right>$. Например, $J_{\mu \times \nu} =
\Theta$~--- ортопроектор на константу, $J_\Delta = Id, J_{\Delta_S} = \hat S$.

\end{document}
