\documentclass{article}
\usepackage[utf8x]{inputenc}
\usepackage[english,russian]{babel}
\usepackage{amsmath,amscd}
\usepackage{amsthm}
\usepackage{mathtools}
\usepackage{amsfonts}
\usepackage{amssymb}
\usepackage{cmap}
\usepackage{centernot}
\usepackage{enumitem}
\usepackage{perpage}
\usepackage{chngcntr}
%\usepackage{minted}
\usepackage[bookmarks=true,pdfborder={0 0 0 }]{hyperref}
\usepackage{indentfirst}
\hypersetup{
  colorlinks,
  citecolor=black,
  filecolor=black,
  linkcolor=black,
  urlcolor=black
}

\newtheorem*{conclusion}{Вывод}
\newtheorem{theorem}{Теорема}
\newtheorem{lemma}{Лемма}
\newtheorem*{corollary}{Следствие}

\theoremstyle{definition}
\newtheorem*{problem}{Задача}
\newtheorem{claim}{Утверждение}
\newtheorem{exercise}{Упражнение}
\newtheorem{definition}{Определение}
\newtheorem{example}{Пример}

\theoremstyle{remark}
\newtheorem*{remark}{Замечание}

\newcommand{\doublearrow}{\twoheadrightarrow}
\renewcommand{\le}{\leqslant}
\renewcommand{\ge}{\geqslant}
\newcommand{\eps}{\varepsilon}
\renewcommand{\phi}{\varphi}
\newcommand{\ndiv}{\centernot\mid}

\MakePerPage{footnote}
\renewcommand*{\thefootnote}{\fnsymbol{footnote}}

\newcommand{\resetcntrs}{\setcounter{theorem}{0}\setcounter{definition}{0}
\setcounter{claim}{0}\setcounter{exercise}{0}}

\DeclareMathOperator{\aut}{aut}
\DeclareMathOperator{\cov}{cov}
\DeclareMathOperator{\argmin}{argmin}
\DeclareMathOperator{\argmax}{argmax}
\DeclareMathOperator*\lowlim{\underline{lim}}
\DeclareMathOperator*\uplim{\overline{lim}}
\DeclareMathOperator{\re}{Re}
\DeclareMathOperator{\im}{Im}

\frenchspacing


\begin{document}

\section*{Лекция 3. Эргодическая теорема II}
\addcontentsline{toc}{section}{Лекция 3. Эргодическая теорема II}
\resetcntrs

\section{Общие соображения}

Имеем группу (полугруппу) $G$, которая отвечает за время. В~статистическом
случае мы делаем~$N$ наблюдений и~усредняем результаты согласно равномерному
распределению. Причем тут вообще равномерность? Это инвариантность относительно
естественного действия группы, то есть это так называемая мера Хаара:
$\lambda(\delta + A) = \lambda(A)$.

Значит в~гипотетическом общем случае наш план таков: выделить область времени
$U$ и~посчитать $\frac{1}{\lambda(U)} \int\limits_U f(T^t x) d\lambda(t)$
ожидая, что это сходится к~$E_\lambda f$ при $U \rightarrow \infty$ в~каком-то
смысле.

\begin{definition}
	Пусть есть последовательность $F_n$, тогда если $\frac{\lambda(F_n \oplus
	(\delta + F_n))}{\lambda(F_n)} \rightarrow 0$ для всех $z \in K$-компакта, то
	эти множества называются Фёльнеровскими.
\end{definition}

\begin{definition}
	Аменабельная группа~$G$~--- такая группа, в~которой есть
	<<последовательность>> Фёльнеровских множеств $F_n$.
\end{definition}

Один из главных примеров не аменабельной группы~--- это свободная группа $F_2$,
где граница шара по размеру сопоставима с~самим шаром. Оказывается, что
аменабельность замкнута относительно всех разумных теоретико-групповых операций,
а~значит те группы, которые содержат~$F_2$ не аменабельны, в~частности группа
$\text{SO}(3)$.

\begin{exercise}
	Есть ли в~группе $\left< \begin{bmatrix} 1 & 1 \\ 0 & 1\end{bmatrix},
		\begin{bmatrix} 1 & 0 \\ 1 & 1 \end{bmatrix} \right>$ соотношения?
\end{exercise}

Напоминание: унитарный оператор~--- такой, что~$A^\ast = A^{-1}$ или $\left< Af,
g \right> = \left<f, A^\ast g\right>$.

Опретор Купмана сохраняет скалярные произведение:
$\int\limits_X f(Tx) \overline{g(Tx)} d\mu(x)$ по теореме о~замене переменных
в~интеграле Лебега есть $\left<f, g\right>$, что и~нужно.

\begin{exercise}
	$T$~--- эргодично $\Leftrightarrow \exists \phi \ne const, \hat T = \phi
	\Leftrightarrow \ker(\hat T - 1) = \{const\}$.
\end{exercise}

Докажем теперь, что из эргодичности следует перемешивание. Без ограничения
общности~$E_\lambda \phi = 0$. $T \in Mix \Leftrightarrow \forall f, g
\rightarrow \left<\hat T^k f, g\right> \rightarrow E_\lambda f E_\lambda g$, то
есть $\hat T^k \underset{w}\rightarrow \Theta$ (ортопроектор на константу).

$$\left< \Theta f, g \right> = \int (E_\lambda f) 1 g d\lambda =
E_\lambda f \int g d\lambda$$.

\section{Теорема фон Неймана, лемма Рохлина-Халмоша}

\begin{theorem}[Эргодическая теорема фон Неймана]
	$\frac{1}{n} \sum\limits_{k=0}^{n-1} \hat T f \rightarrow E_\mu f$ в~$L^2(X,
	\mu)$, если $T$~--- эргодично.
\end{theorem}
\begin{proof}
	Пусть $T$ не обязательно эргодично. Пусть~$I = \{ \phi:
	\hat T \phi = \phi \}$. $L^2(\mu) = I \oplus I^\bot, \hat TI = I$.

	Если $\phi \in I$, то $\frac{1}{n} \sum \hat T \phi = \frac{1}{n} (\phi +
	\ldots \phi) = \phi \rightarrow \phi$.

	Иначе, пусть функция имеет вид $f = h - \hat T h$. Сумма примет
	вид~$\frac{1}{n}\left(h - \hat Th + \hat Th - \hat T^2h + \ldots \right) =
	\frac{1}{n}(h - \hat T^n h) \rightarrow 0$, так как $\hat T$~--- унитарный.

	Дальнейшая идея в~том, чтобы показать, что функции такого вида плотны
	в~$I^\bot$. Рассмотрим замыкание $M = \overline{\{h - \hat T h\}}$.
	Если оно не совпадает с~$I$, то все пространство разбивается на три: $L^2(\mu)
	= I \oplus M \oplus F$ для какого-то $F$. $\exists f \in F: f \bot (h - \hat T
	h)$ для любого $h$, тогда $\left<f, h - \hat Th\right> = \left<f, h\right> -
	\left<f, \hat Th\right> = \left<f, h\right> - \left<\hat T^{-1}f, g\right> =
	\left<f - \hat T^{-1}h\right>$, значит, так как $h$ любое, то $f -
	\hat T^{-1}f = 0 \Rightarrow \hat T f = f$, противоречие.
\end{proof}

\begin{lemma}[Рохлина-Халмоша]
	Пусть $T$~--- эргодическая и~заданы параметры $\eps > 0, h \in \mathbb{N}$,
	тогда существует башня высоты $h$, такая что $\mu(\bigsqcup\limits_{k=0}^{h-1}
	T^k B) > 1 - \eps$.
\end{lemma}
\begin{proof}
	Выберем произвольное множество $B$ и~построим $L_k = T L_{k-1} \setminus B,
	L_0 = B$. Тогда множество $D = \bigsqcup_{k=0}^{\infty} L_k$ инвариантно, $TD
	= D$. Отсюда $D = X$ в~силу эргодичности.

	Выберем теперь любое множество $B$, такое, что $\mu(B)(h - 1) <
	\eps$ и~рассмотрим $B, TB, \ldots$. Спроецируем каждый $h$-й уровень на $h$
	вниз и~выберем $h$ множеств, из урезанных слоев, построив таким образом новую
	башню~$C, TC, T^2C, \ldots$. Она дизъюнктна и~мера её объединения не меньше
	чем $1 - \mu(E) = 1 - \mu(B) (h - 1) \le 1 - \eps$, так как ошибочные кусочки
	накрывают~$B$ не более, чем~$h$ раз.
\end{proof}

\section{Доказательство эргодической теоремы}
%\begin{theorem}[Эргодическая теорема фон Неймана]
%  Если в~эргодической динамической системе фильтрация $F_n$~--- Фёльнеровская
%  относительно меры Хаара $\lambda$ (полу)группы~$G$, то
%  $$\frac{1}{\lambda(U_n)} \int\limits_{U_n} f(T^t x) d\lambda(t) \rightarrow
%  E_\lambda f, n \rightarrow \infty$$.
%\end{theorem}
\begin{proof}
	Рассмотрим $f = I_{X_0}$, $\mu(X_0) = 1 - \mu(X_0) = \mu(X_1) = \frac{1}{2}$.
	Вместо одной второй можно взять любую положительную константу. Такими
	индикаторами можно аппроксимировать любую функцию.

	$$\mu( x: \lim \sup\limits_{n \rightarrow \infty} \underbrace{\frac{1}{n}
	\sum\limits_{k=0}^{n-1} f(T^k x)}_{S_n} \ge \frac{1}{2} + c) > 0, c > 0$$.

	Это значит $\forall x \exists n_j \rightarrow +\infty: S_j \ge \frac{1}{2} +
	\frac{c}{2}$.

	Рассмотрим башню~$H$ меры, близкой к~1 с~достаточно большой высотой. $n_1(x)$
	измеримо, $\mu(x: n_1(x) > M) = 0, M \rightarrow \infty$. $M$ нужно выбрать
	таким, чтобы эта мера была меньше какого-то малого $\delta$. Хотим разбивать
	нашу башню на множества меры 0 по вертикали. Сверху нужно отступить $M$,
	чтобы не испортить статистику, потеряем на этом не больше $\frac{M}{H}$ меры.
	Важно аккуратно сделать это разбиение измеримым и~пропускать те моменты, где
	в~нужном столбике меньше $\frac{1}{2} + \frac{c}{2}$ единичек.
\end{proof}

\end{document}
