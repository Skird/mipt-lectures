\documentclass{article}
\usepackage[utf8x]{inputenc}
\usepackage[english,russian]{babel}
\usepackage{amsmath,amscd}
\usepackage{amsthm}
\usepackage{mathtools}
\usepackage{amsfonts}
\usepackage{amssymb}
\usepackage{cmap}
\usepackage{centernot}
\usepackage{enumitem}
\usepackage{perpage}
\usepackage{chngcntr}
%\usepackage{minted}
\usepackage[bookmarks=true,pdfborder={0 0 0 }]{hyperref}
\usepackage{indentfirst}
\hypersetup{
  colorlinks,
  citecolor=black,
  filecolor=black,
  linkcolor=black,
  urlcolor=black
}

\newtheorem*{conclusion}{Вывод}
\newtheorem{theorem}{Теорема}
\newtheorem{lemma}{Лемма}
\newtheorem*{corollary}{Следствие}

\theoremstyle{definition}
\newtheorem*{problem}{Задача}
\newtheorem{claim}{Утверждение}
\newtheorem{exercise}{Упражнение}
\newtheorem{definition}{Определение}
\newtheorem{example}{Пример}

\theoremstyle{remark}
\newtheorem*{remark}{Замечание}

\newcommand{\doublearrow}{\twoheadrightarrow}
\renewcommand{\le}{\leqslant}
\renewcommand{\ge}{\geqslant}
\newcommand{\eps}{\varepsilon}
\renewcommand{\phi}{\varphi}
\newcommand{\ndiv}{\centernot\mid}

\MakePerPage{footnote}
\renewcommand*{\thefootnote}{\fnsymbol{footnote}}

\newcommand{\resetcntrs}{\setcounter{theorem}{0}\setcounter{definition}{0}
\setcounter{claim}{0}\setcounter{exercise}{0}}

\DeclareMathOperator{\aut}{aut}
\DeclareMathOperator{\cov}{cov}
\DeclareMathOperator{\argmin}{argmin}
\DeclareMathOperator{\argmax}{argmax}
\DeclareMathOperator*\lowlim{\underline{lim}}
\DeclareMathOperator*\uplim{\overline{lim}}
\DeclareMathOperator{\re}{Re}
\DeclareMathOperator{\im}{Im}

\frenchspacing


\begin{document}

\section*{Лекция 6. Немного про джойнинги}
\addcontentsline{toc}{section}{Лекция 6. Немного про джойнинги}
\resetcntrs

\section{Структура пространства всех джойнингов}

Праздный факт:
\begin{theorem}[Рохлин]
	Любое пространство Лебега (конечной меры) без атомов (точек ненулевой меры)
	изоморфно $[0; 1]$.
\end{theorem}

Рассмотрим множество всех джойнингов $J_{T,S}$. Оно выпукло и~компактно
(множество всех мер компактно в~силу слабой компактности шара в~гильбертовом
пространстве, а~замкунтое подмножество компакта компактно). Всякий джойнинг
может быть тогда выражен как
$$ \eta = \int\limits_{\partial J_{T,S}} \alpha d\xi_{(\eta)}. $$

Утверждается, что $\partial J_{T,S}$ представляет собой все эргодические
джойнинги.

Пусть есть две эргодические системы $T$ и~$S$ с~общим дискретным спектром
$\Lambda$. Тогда $J_{T,S}$ непусто и~компактно, то есть имеет границу. Значит
существует какой-то эргодический джойнинг $\eta$.

$L^2(\mu) = Span\{ \phi_\lambda(x): \lambda \in \Lambda \}$,
$L^2(\nu) = Span\{ \psi_\lambda(x): \lambda \in \Lambda \}$.

$(\hat T \times \hat S) \phi_{\lambda_1}(x) \psi_{\lambda_2}(y) = \lambda_1
\lambda_2 \phi_{\lambda_1} \psi_{\lambda_2}$. $L^2(\eta) = \left<
\phi_{\lambda_1} \otimes \psi_{\lambda_2} \right>$.

Утверждается, что канонические вложения $\hat \pi_1: L^2(\mu) \rightarrow
L^2(\eta), \hat \pi_2: L^2(\nu) \rightarrow L^2(\eta)$ являются изоморфизмами.

\end{document}
