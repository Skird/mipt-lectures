\documentclass{article}
\usepackage[utf8x]{inputenc}
\usepackage[english,russian]{babel}
\usepackage{amsmath,amscd}
\usepackage{amsthm}
\usepackage{mathtools}
\usepackage{amsfonts}
\usepackage{amssymb}
\usepackage{cmap}
\usepackage{centernot}
\usepackage{enumitem}
\usepackage{perpage}
\usepackage{chngcntr}
%\usepackage{minted}
\usepackage[bookmarks=true,pdfborder={0 0 0 }]{hyperref}
\usepackage{indentfirst}
\hypersetup{
  colorlinks,
  citecolor=black,
  filecolor=black,
  linkcolor=black,
  urlcolor=black
}

\newtheorem*{conclusion}{Вывод}
\newtheorem{theorem}{Теорема}
\newtheorem{lemma}{Лемма}
\newtheorem*{corollary}{Следствие}

\theoremstyle{definition}
\newtheorem*{problem}{Задача}
\newtheorem{claim}{Утверждение}
\newtheorem{exercise}{Упражнение}
\newtheorem{definition}{Определение}
\newtheorem{example}{Пример}

\theoremstyle{remark}
\newtheorem*{remark}{Замечание}

\newcommand{\doublearrow}{\twoheadrightarrow}
\renewcommand{\le}{\leqslant}
\renewcommand{\ge}{\geqslant}
\newcommand{\eps}{\varepsilon}
\renewcommand{\phi}{\varphi}
\newcommand{\ndiv}{\centernot\mid}

\MakePerPage{footnote}
\renewcommand*{\thefootnote}{\fnsymbol{footnote}}

\newcommand{\resetcntrs}{\setcounter{theorem}{0}\setcounter{definition}{0}
\setcounter{claim}{0}\setcounter{exercise}{0}}

\DeclareMathOperator{\aut}{aut}
\DeclareMathOperator{\cov}{cov}
\DeclareMathOperator{\argmin}{argmin}
\DeclareMathOperator{\argmax}{argmax}
\DeclareMathOperator*\lowlim{\underline{lim}}
\DeclareMathOperator*\uplim{\overline{lim}}
\DeclareMathOperator{\re}{Re}
\DeclareMathOperator{\im}{Im}

\frenchspacing


\begin{document}

\section*{Лекция 7. Про слабоперемешивающие системы}
\addcontentsline{toc}{section}{Лекция 7. Про слабоперемешивающие системы}
\resetcntrs

\section{Построение слабоперемешивающей, но не перемешивающией системы}

Хотим построить слабо-перемешивающую, но не перемешивающую систему.

Рассмотрим следующий процесс преобразования слов и~будем его анализировать.
$w_0 = 0, w_1 = 0010, \ldots, w_{n+1} = w_n w_n 1 w_n$.

Определим эмпирическое распределение: $p(u \Vert w_\infty) =
\lim\limits_{n \rightarrow \infty} p(u \Vert w_n)$, где
$p(u \Vert v) = \frac{\#\{\text{вхождений u в~v}\}}{|v| - |u| + 1}$.

\begin{lemma}
	Меры $\mu_n(u) = p(u \Vert w_\infty), |u| = n$ согласованы, то есть если по
	большей мере рассмотреть маленькое слово, то получится то же, что по меньшей
	мере (или, что существует мера $\mu$, проекцией которой являются все данные
	меры). Таким образом имеется динамическая система на пространстве
	$(\mathbb{A}^\mathbb{Z}, \mathcal{B}, \mu)$ с~оператором Купмана $T: (x_i)
	\mapsto (x_{i+1})$.
\end{lemma}

\begin{theorem}~\\
	\begin{itemize}
		\item $T$~--- эргодическое;
		\item $\hat T^{h_n} \underset{w}\rightarrow \frac{\hat T + 1}{2}$;
		\item $T \in WMix, T \notin Mix$.
	\end{itemize}
\end{theorem}

Процесс можно представить так: имеем слово $w$, записанное в~башню снизу вверх.
За 1 шаг мы должны скопировать $w$, получив три башни рядом. Далее на среднюю
башню нужно дописать $1$ и~склеить все в~один столбик.

\begin{definition}
	$T \in Rang(1)$, если $\exists \xi_n = \{B_n, TB_n, \ldots, T^{h_n-1}B_n,
	\eps_n\} \rightarrow \eps$, то есть $\forall A \exists \xi_n$~--- измеримая на
	$A$ и~$\mu(A \Delta A_n) \rightarrow 0$.
\end{definition}

Рассмотрим $\left< \hat T^{-h_n}f, g\right> = \int\limits_X
\hat T^{h_n} f(x)g(x) d\mu = {1\over2} \left<f,g\right> +
{1\over2} \left<\hat Tf, g\right>$. То есть второй пункт доказан.

Этюд: характеризация полиномов, таких, что $\lim \hat T^{-mh_n} = P_m(\hat T)$.
Можно получить, что $P_{3m} = P_m$ и~выразить $P_{3m+1}, P_{3m+2}$ через
$P_{m}=P_{3m}$ и~$P_{m+1}=P{3m+3}$ и~изобразить их как результат случайного
блуждания на графе Шреера $BS(1,3) / \left<t\right>$, где $BS(1,3)=\left<a,t\mid
tat^{-1} = a^3\right>$ (стандартное действие $a: x \mapsto x + 1$, $t: x \mapsto
3x$).

\begin{theorem}
	Для эргодических $T$ следующие утверждения эквивалентны:
	\begin{itemize}
		\item $T \in WMix$;
		\item $\sigma_T = \sigma_s + \sigma_{ac} (\nexists \phi: \hat T\phi =
			\lambda \phi, \lambda \ne 1)$;
		\item Джойнинг $\mu \times \mu$ эргодичен относительно $T \times T$;
		\item (Н. Винер) $\forall \eps > 0 \rightarrow \mathcal F = \{t:
			|\left<\hat T^t f, g\right>| > \eps \}$ имеет нулевую плотность:
			$\frac{\mathcal{F} \cap [1, N]}{N} \rightarrow 0$.
	\end{itemize}
\end{theorem}

\end{document}
