\documentclass{article}
\usepackage[utf8x]{inputenc}
\usepackage[english,russian]{babel}
\usepackage{amsmath,amscd}
\usepackage{amsthm}
\usepackage{amsfonts}
\usepackage{amssymb}
\usepackage{cmap}
\usepackage{centernot}
\usepackage{enumitem}
\usepackage{perpage}
\usepackage{chngcntr}
%\usepackage{minted}
\usepackage[bookmarks=true,pdfborder={0 0 0 }]{hyperref}
\usepackage{indentfirst}
\hypersetup{
  colorlinks,
  citecolor=black,
  filecolor=black,
  linkcolor=black,
  urlcolor=black
}

\newtheorem*{conclusion}{Вывод}
\newtheorem{theorem}{Теорема}
\newtheorem{lemma}{Лемма}
\newtheorem*{corollary}{Следствие}

\theoremstyle{definition}
\newtheorem*{problem}{Задача}
\newtheorem{claim}{Утверждение}
\newtheorem{exercise}{Упражнение}
\newtheorem{definition}{Определение}
\newtheorem{example}{Пример}

\theoremstyle{remark}
\newtheorem*{remark}{Замечание}

\renewcommand{\le}{\leqslant}
\renewcommand{\ge}{\geqslant}
\newcommand{\eps}{\varepsilon}
\renewcommand{\phi}{\varphi}
\newcommand{\ndiv}{\centernot\mid}

\MakePerPage{footnote}
\renewcommand*{\thefootnote}{\fnsymbol{footnote}}

\newcommand{\resetcntrs}{\setcounter{theorem}{0}\setcounter{definition}{0}
\setcounter{claim}{0}\setcounter{exercise}{0}}

\DeclareMathOperator{\aut}{aut}
\DeclareMathOperator{\cov}{cov}
\DeclareMathOperator{\chos}{ch}
\DeclareMathOperator{\argmin}{argmin}
\DeclareMathOperator{\argmax}{argmax}
\DeclareMathOperator*\lowlim{\underline{lim}}
\DeclareMathOperator*\uplim{\overline{lim}}
\DeclareMathOperator{\re}{Re}
\DeclareMathOperator{\im}{Im}

\frenchspacing


\begin{document}

\section*{Лекция 8. Метрическая энтропия}
\addcontentsline{toc}{section}{Лекция 8. Метрическая энтропия}
\resetcntrs

\section{Энтропия динамической системы}

Рассмотрим фазовое пространство~$(X, \mathcal{B}, \mu)$ и~конечное разбиение
$P = \{C_1, \ldots, C_n\}$.

\begin{definition}
	\emph{Энтропия} разбиения $P$~--- это $H(P) = -\sum \mu(C_i) \log_2 \mu(C_i)$.
\end{definition}

\begin{remark}
	В~определении основание логарифма не так важно, но мы принимаем его за~2,
	чтобы <<измерять>> информацию в~привычных битах.

	Нетрудно понять, что $H(P) = E_\mu I_P(x)$, где $I(x)$~--- информационная
	функция: $I_P(x) = -\log_2 \mu(C(x))$, где $C(x): X \rightarrow P, C(x) \ni
	x$.
\end{remark}

Возьмём разбиение $P_0$ и~построим $Q_N = P_0 \lor T^{-1} P_0 \lor \ldots \lor
T^{-N + 1} P_0$, где $A \lor B$ обозначает взятие минимального разбиения,
содержащего~$A$ и~$B$.

\begin{definition}
	Пусть $(X, T)$~--- динамическая система, $P = \{C_\alpha, \alpha \in
	\mathbb{A}\}$~--- разбиение. Пусть также~$x_j \in \mathbb{A}: C_{x_j} \ni
	T^j(x)$. Тогда бесконечная последовательность $\{x_j \mid j \in G\}$
	называется \emph{кодом} точки~$x$.
\end{definition}

Для $w \in \mathbb{A}^\ast$ примем обозначение $[w] = \{x: x_0 \ldots x_{|w|-1}
= w\}$.

Можно рассматривать проецсс кодирования как случайный процесс
в~следующем смысле: пусть $\xi_0(x) = \alpha, x \in C_\alpha$~--- кодирующая
случайная величина, тогда процесс кодирования $\xi_j = \xi_0 T^i$.

\begin{definition}
	\emph{Энтропией} процесса $\{\xi_j\}$ называется $h(\xi) = h_{P_0}(T) =
	\lim\limits_{N \rightarrow \infty} \frac{1}{N} H(Q_N)$.
\end{definition}

Отметим также свойство: $H(P_0) \le \log_2 |P_0|$, достигается оценка очевидно
для равномерного разбиения.

Условную энтропию можно ввести с~помощью условное матожидание. Точно, через
матожидание, также можно ввести энтропию бесконечного разбиения, однако, она уже
не обязана быть конечной.

\begin{theorem}
	$H_P(T) = H(P \mid \bigvee\limits_{-\infty}^{-1} T^{i} P)$.
\end{theorem}

\begin{remark}
	Представить себе процесс с~пространством, неизоморфным пространству Лебега не
	очень просто, но такие есть (как пример, Винеровский процесс).
\end{remark}

\begin{definition}
	$\eta$~--- \emph{измеримое разбиение}, если $\eta = \{f^{-1}(\{y\})\}$ для
	некоторого измеримого $f: X \rightarrow Y$.
\end{definition}

\begin{center}
	\begin{tabular}{|c|c|}
		\hline
		Разбиения & $\sigma$-алгебры\\
		\hline
		\multicolumn{2}{|c|}{пространство Лебега $\cong [0; 1]$ }\\
		\hline
		$\bbnu = \{X\}$ & $\{\varnothing, X\}$\\
		\hline
		$\bbespilon = \{\{x\}\mid x \in X\}$ & $\mathcal{A}$~--- полная на $[0;1]$\\
		\hline
		$\xi \lor \eta = \{C_1 \cap C_2\}$ & $\mathcal{B}_1 \land \mathcal{B}_2$\\
		\hline
		$\xi \land \eta$ & $\mathcal{B}_1 \cap \mathcal{B}_2$\\
		\hline
	\end{tabular}
\end{center}

\begin{definition}
	$\mathcal{A}$ называется \emph{$\mu$-полной}, если $\mathcal{A} =
	[\mathcal{A}]_\mu$.
\end{definition}
\begin{definition}
	Система $(Y, \mathcal{B}, \gamma, S)$ является \emph{фактором}
	$(X, \mathcal{A}, \mu, T)$, если существует <<вложение>>:
	$\exists \phi: X \rightarrow Y, \mathcal{B} \le \mathcal{A}$.

	Или: $\mathcal{B} \le \mathcal{A}, S = T\mid_\mathcal{B}$ при условии, что
	$T\mathcal{B} = \mathcal{B}$.
\end{definition}

\begin{definition}
	\emph{Энтропия динамической системы} $h(T) = \sup\limits_P h_P(T)$, где $P$~--
	конечное разбиение.
\end{definition}

\begin{center}
	\begin{tabular}{|c|c|}
		\hline
		$h > 0$ & Спектр $(Leb, \infty)$\\
		\hline
		$h = 0$ & Самые разные спектры, разные кратности,\\
		        & примеры субэкспоненциального роста сложности\\
		\hline
	\end{tabular}
\end{center}

\begin{theorem}[Шеннон-Макмилан-Брейман]
	Энтропию процесса можно эквивалентно определить следующим образом:
	$\exists h \ge 0: \forall \eps > 0 \rightarrow $ для $\mu$-большинства блоков
	$[w]$ длины $n$ (мера всех остальных блоков стремится к~0): $2^{-n(h+\eps)} <
	\mu([w]) < 2^{-n(h-\eps)}$. Такое $h$ и~есть энтропия процесса.
\end{theorem}

\section{Семинарская часть}

Символическая сложность: $P_w(n) = \#L_n(w), L_n(w) = \{u \le w: |u| = n\}$.

\begin{exercise}~\\
	\begin{itemize}
		\item $\min p(n)$ для непериодических слов?
		\item построить экспоненциальный рост: $2^{h_n}$
		\item построить линейный рост
		\item достигается ли наименьшая скорость роста в~первом пункте
		\item добиться квадратичной скрости
		\item добиться кубической скорости
		\item добиться субэкспоненциальной скорости
		\item$(^\ast)$ какова скорость роста для слабо-перемешивающей системы из
			предыдущей лекции? Для последовательности Туве-Морса.
	\end{itemize}
\end{exercise}

\end{document}
