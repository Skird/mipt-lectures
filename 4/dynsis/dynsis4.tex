\documentclass{article}
\usepackage[utf8x]{inputenc}
\usepackage[english,russian]{babel}
\usepackage{amsmath,amscd}
\usepackage{amsthm}
\usepackage{mathtools}
\usepackage{amsfonts}
\usepackage{amssymb}
\usepackage{cmap}
\usepackage{centernot}
\usepackage{enumitem}
\usepackage{perpage}
\usepackage{chngcntr}
%\usepackage{minted}
\usepackage[bookmarks=true,pdfborder={0 0 0 }]{hyperref}
\usepackage{indentfirst}
\hypersetup{
  colorlinks,
  citecolor=black,
  filecolor=black,
  linkcolor=black,
  urlcolor=black
}

\newtheorem*{conclusion}{Вывод}
\newtheorem{theorem}{Теорема}
\newtheorem{lemma}{Лемма}
\newtheorem*{corollary}{Следствие}

\theoremstyle{definition}
\newtheorem*{problem}{Задача}
\newtheorem{claim}{Утверждение}
\newtheorem{exercise}{Упражнение}
\newtheorem{definition}{Определение}
\newtheorem{example}{Пример}

\theoremstyle{remark}
\newtheorem*{remark}{Замечание}

\newcommand{\doublearrow}{\twoheadrightarrow}
\renewcommand{\le}{\leqslant}
\renewcommand{\ge}{\geqslant}
\newcommand{\eps}{\varepsilon}
\renewcommand{\phi}{\varphi}
\newcommand{\ndiv}{\centernot\mid}

\MakePerPage{footnote}
\renewcommand*{\thefootnote}{\fnsymbol{footnote}}

\newcommand{\resetcntrs}{\setcounter{theorem}{0}\setcounter{definition}{0}
\setcounter{claim}{0}\setcounter{exercise}{0}}

\DeclareMathOperator{\aut}{aut}
\DeclareMathOperator{\cov}{cov}
\DeclareMathOperator{\argmin}{argmin}
\DeclareMathOperator{\argmax}{argmax}
\DeclareMathOperator*\lowlim{\underline{lim}}
\DeclareMathOperator*\uplim{\overline{lim}}
\DeclareMathOperator{\re}{Re}
\DeclareMathOperator{\im}{Im}

\frenchspacing


\begin{document}

\section*{Лекция 4. Спектральная теорема}
\addcontentsline{toc}{section}{Лекция 4. Спектральная теорема}
\resetcntrs

\section{Спектральная теорема}

Перед тем, как сформулировать спектальную теорему, вспомним некотрорые факты:
\begin{itemize}
	\item Оператор Купмана~$\hat T^t$ унитарен.
	\item Под гильбертовым пространством~$H$ понимаем линейное пространство
		над~$\mathbb{C}$ с~эрмитовым скалряным произведением $\left< u, v \right>$,
		полное относительно порождённой метрики.

		Мы будем интересоваться как правило пространством~$H = L^2(X, \mathcal{B},
		\mu) = \left\{f: \Vert f \Vert^2 = \int\limits_X |f|^2 d\mu < \infty
		\right\} /_\sim$.
	\item Циклическое пространство $Z(h) = \overline{Span}(\hat T^k h: k \in
		\mathbb{Z})$~--- минимальное замкнутое инвариантное подпространство.
\end{itemize}

\begin{theorem}[Спектральная теорема для систем с~простым спектром]
	Пусть есть унитарный оператор в~гильбертовом пространстве и~пусть $\exists h:
	Z(h) = H$, тогда имеет место коммутативная диаграмма:

	$$\begin{CD}
	H           @>\hat T>>  H\\
	@VV\psi V              @VV\psi V\\
	H @> M_z: \phi(\lambda) \mapsto \lambda \phi(\lambda) >> L^2(S^1,\sigma_h)
	\end{CD}$$

	Где $\sigma_h: \left<\hat T^kh, h\right> = \int\limits_{S^1} z^k d\sigma_h$.

	Притом, если есть циклический вектор~$h'$, то $\sigma_h \sim \sigma_{h'}$, где
	$\hat T \sim \hat S = \Phi^{-1} \hat T \Phi \Leftrightarrow \sigma^{(\hat
	T)}_h \sim \sigma^{(\hat S)}_h$, $\nu \sim \mu \Leftrightarrow \nu \ll \mu,
	\mu \ll \nu$; $\nu \ll \mu: \nu = p(\lambda) \mu$.
\end{theorem}

Что такое кратность? Модельный пример~--- система из двух одинаковых маятников,
независимо колеблющихся. Более формально,
$H > L_1 \oplus \ldots L_m: \hat T\mid_{H_i} \sim \hat T\mid_{H_j}$.

Общие спектральные инварианты: $([\sigma], \mathcal{M}(\lambda))$,
$\mathcal{M}$~--- функция кратности, $\mathcal{M}: S^1 \rightarrow \mathbb{N}
\sqcup \{\infty\}$.

\section{Семинарская часть}

$\begin{bmatrix}1 & 1\\ 1 & -1\end{bmatrix}$ имеет СЗ $\pm\sqrt{2}$. Спектр
	этого оператора $\sigma = \frac{\delta_{-\sqrt{2}} + \delta_{\sqrt{2}}}{2}$.

\begin{exercise}[$\ast$]
	Придумать вектор: $h$, дающий такую $\sigma$.
\end{exercise}

$\Lambda = \{\pm \sqrt{2}\}, L^2(\Lambda, \sigma) = \{\phi: \Lambda \rightarrow
\mathbb{C} \} \cong \mathbb{C}^2$. $M_\lambda(\phi) = \lambda \phi(\lambda)$.
Если функцию $\phi$ представлять столбцом значений в~$\pm\sqrt{2}$, то
$$M_\lambda(\begin{bmatrix}\phi(-\sqrt{2})\\ \phi(\sqrt{2})\end{bmatrix}) =
\begin{bmatrix} -\sqrt{2}\phi(-\sqrt{2})\\ \sqrt{2}\phi(\sqrt{2}) \end{bmatrix}
= \begin{bmatrix} -\sqrt{2} & 0\\ 0 & \sqrt{2}\end{bmatrix}
	\begin{bmatrix}\phi(-\sqrt{2})\\ \phi(\sqrt{2})\end{bmatrix}$$.

Рассмотрим теперь группу~$\mathbb{T} = \mathbb{R}^2 / \mathbb{Z}^2$.
И~матрицу~$A(x, y) = \begin{bmatrix}2&1\\1&1\end{bmatrix}
	\begin{bmatrix}x\\y\end{bmatrix}$

Характеры тора (гомоморфизмы в~комплексную единичную окружность) выглядят так:
$\gamma_{j,k}(x, y) = \exp(2\pi i(jx + ky))$. Возьмём один характер и~рассмотрим
его динамику под действием оператора Купмана.

Можно проверить, что $\gamma_{j_0,k_0}(A(x, y)) = \gamma_{A^T(j_0,k_0)}(x, y)$.
Действие корректно задано на торе, если все элементы~$A$~--- целые. Оно
обратимо, если обратная матрица тоже целочисленна, то есть $\det A = 1$.

Орбиты этого действия на~$\mathbb{Z}^2$ получились гиперболами на
соответствующих целых точках (плюс одна стационарная орбита). Спектральный тип
этой системы получился~$(Leb, \infty)$.

\begin{exercise}
	Возмем последовательность характеров на одной гиперболе: $\xi_j: \hat A\xi_i =
	\xi_{i+1}$. Посчитать $\sigma_{\xi_0}$ и~$\left<\hat A^k \xi_i, \xi_i
	\right>$.
\end{exercise}

\end{document}
