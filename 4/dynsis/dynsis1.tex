\documentclass{article}
\usepackage[utf8x]{inputenc}
\usepackage[english,russian]{babel}
\usepackage{amsmath,amscd}
\usepackage{amsthm}
\usepackage{mathtools}
\usepackage{amsfonts}
\usepackage{amssymb}
\usepackage{cmap}
\usepackage{centernot}
\usepackage{enumitem}
\usepackage{perpage}
\usepackage{chngcntr}
%\usepackage{minted}
\usepackage[bookmarks=true,pdfborder={0 0 0 }]{hyperref}
\usepackage{indentfirst}
\hypersetup{
  colorlinks,
  citecolor=black,
  filecolor=black,
  linkcolor=black,
  urlcolor=black
}

\newtheorem*{conclusion}{Вывод}
\newtheorem{theorem}{Теорема}
\newtheorem{lemma}{Лемма}
\newtheorem*{corollary}{Следствие}

\theoremstyle{definition}
\newtheorem*{problem}{Задача}
\newtheorem{claim}{Утверждение}
\newtheorem{exercise}{Упражнение}
\newtheorem{definition}{Определение}
\newtheorem{example}{Пример}

\theoremstyle{remark}
\newtheorem*{remark}{Замечание}

\newcommand{\doublearrow}{\twoheadrightarrow}
\renewcommand{\le}{\leqslant}
\renewcommand{\ge}{\geqslant}
\newcommand{\eps}{\varepsilon}
\renewcommand{\phi}{\varphi}
\newcommand{\ndiv}{\centernot\mid}

\MakePerPage{footnote}
\renewcommand*{\thefootnote}{\fnsymbol{footnote}}

\newcommand{\resetcntrs}{\setcounter{theorem}{0}\setcounter{definition}{0}
\setcounter{claim}{0}\setcounter{exercise}{0}}

\DeclareMathOperator{\aut}{aut}
\DeclareMathOperator{\cov}{cov}
\DeclareMathOperator{\argmin}{argmin}
\DeclareMathOperator{\argmax}{argmax}
\DeclareMathOperator*\lowlim{\underline{lim}}
\DeclareMathOperator*\uplim{\overline{lim}}
\DeclareMathOperator{\re}{Re}
\DeclareMathOperator{\im}{Im}

\frenchspacing


\begin{document}

\section*{Лекция 1. Определение и~примеры динамических}
\addcontentsline{toc}{section}{Лекция 1. Определение и~примеры динамических
систем}
\resetcntrs

\section{Введение}

Литература:
\begin{itemize}
	\item (!) Синай <<Эргодическая теория>>, издательство <<Фазис>> или
		<<РХД>>~--- брошюра из нескольких лекций.
	\item Каток---Хиссельблат <<Введение в~современную теорию динамических
		систем>>~--- энциклопедического плана, есть материал про топологическую
		динамику.
	\item Коррфельд---Синай---Фомин <<Эргодическая теория>>~--- учебник, однако
		старый.
	\item <<Динамические системы-2>> издательства <<ВИНИТИ>>.
	\item (!) Халмош <<Лекции по эргодической теории>>.
	\item Мартин-Итленд <<Математическая теория энтропии>>.
	\item Арнольд <<Математические методы классической механики>>.
	\item Shields <<Ergodic theory of discrete sample paths>>.
\end{itemize}

Динамические системы~--- где-то 1920е, Банах, фон Нейман и~другие, рассмотрение
различных объектов как процессов. Что характерно:
\begin{itemize}
	\item время
	\item состояние
	\item эволюция/динамика
\end{itemize}

\begin{example}
	$x'' + \omega^2 x = 0 \Rightarrow x = c_1 \cos \omega t + c_2 \sin \omega t$.
	Можно переписать как~$x' = y, y' = -\omega^2 x$, тогда динамика будет
	определяться однозначно состоянием.
\end{example}

Исторически пытались решать такие задачи грубо говоря формулой. Но даже имея
точную формулу решения, мы можем ошибаться сильно, если плохо измерены
начальные условия, если со временем теряется много информации (не говоря уже о
том, что большинство задач аналитически не решается).  Значит по возможности
рассматривать динамику как-то глобально.

\begin{definition}
	Фазовый поток $T^t: (x_0, y_0) \mapsto (x(t), y(t))$, где $(x(t), y(t))$~---
	решение с~начальными условиями $(x_0, y_0)$.
\end{definition}

\begin{itemize}
	\item $T^{t+s} = T^t \circ T^s$.
	\item $(T^t)^{-1} = T^{-t}$.
	\item $T^t \in Diff(\mathbb{R}^2)$
	\item То есть $T^t$ есть гомоморфизм. $T^t: \mathbb{R} \rightarrow
		Diff(\mathbb{R}^2)$.
\end{itemize}

Это и~есть основная модель динамической системы. Время не обязательно
непрерывно: мы можем рассматривать осциллятор только в~целые моменты времени.
То есть у~нас будет фигурировать множество поворотов окружноти на угол~$\alpha$.
Есть еще много других, естественных примеров: конечные автоматы, случайные
процессы с~дискретным временем.

Под классическим временем понимается $\mathbb{Z}$ или $\mathbb{R}$, в~общем
случае можно брать другие группы.

\begin{definition}
	Динамическая система~--- следующая четвека:
	\begin{itemize}
		\item $G$~--- группа или полугруппа времени (обычно коммутативная),
		\item $X$~--- фазововое пространство,
		\item $\mathcal{S}$~--- структура,
		\item $T^t$~--- фазовый поток.
	\end{itemize}
	Со следующими свойствами:
	\begin{itemize}
		\item $T^0 = Id$,
		\item $T^{t+s} = T^t \circ T^s$,
		\item $(T^t)^{-1} = T^{-t}$,
		\item $T^t$~--- сохраняет структуру~$\mathcal{S}$.
	\end{itemize}
\end{definition}
Примеры структур:
\begin{itemize}
	\item тривиальная структура. $T^t \in X^X$,
	\item топология, прообраз открытого открыт,
	\item вероятнсть $(X, \mathcal{B}, \mu)$,
	\item линейное пространство,
	\item группа.
\end{itemize}

\begin{definition}
	В~случае $\mathbb{Z}$ процесс ассоциирован с~некоторым измерением $f:
	X \rightarrow \mathbb{C}$, порождающим последовательность~$f_k(x) = f(T^k x)$.
\end{definition}

\section{Эргодическая теория}

Сперва из всех динамических систем предпочтем те, в~которых стуктура имеет
вероятностный характер, чтобы изучить ряд количественных методов, а~также из-за
близости к~дискретной математике.

\begin{example}[Динамические системы по случайным процессам]
Рассмотрим переход от случайного процесса к~динамической системе. Имеем
случайный процесс с~дискрентым временем: $\ldots, \xi_{-1}, \xi_0, \xi_1,
\ldots, \xi_t: \Omega \rightarrow \mathbb{A}$.

Рассмотрим фазовое пространство~$X = \mathbb{A}^\mathbb{Z} = (\ldots, x_0, x_1,
\ldots)$, на нём можно ввести цилиндрическую сигма-алгебру и~индуцировать меру.
Один шаг процесса определим как $S: (x_i) \mapsto (x_{i+1})$.
\end{example}

\begin{theorem}[Пуанкаре о~возвращении]
	Пусть $\mu(X) = 1, T: X \rightarrow X, \mu(T^{-1}A) = \mu(TA) = \mu(A)$,
	тогда для любого измеримого~$A, \mu(A) > 0$ найдётся $n > 0:
	\mu(A \cap T^n A) > 0$.
\end{theorem}
\begin{proof}
	$\mu(T^{n+k}A \cap T^nA) = \mu(TA^k)$, то есть если $\mu(TA^k) = 0$ для
	всех~$k$, то все множества $T^nA$ попарно неперескаются, что невозможно, так
	как множество $X$ конечной меры, а~множество $A$~--- положительной.
\end{proof}

Визуально, рассмотрим конструкцию, которая называется <<башня>> или
<<здание>>. Изобразим~$A$ как первый этаж $L_0$, $TA$ частично перейдет в~само
$A$, частично в~новые точки $TA \setminus A$. Далее, некотрые точки перейдут
в~новые, некоторые перейдут обратно в~$A$. Так строим этажи $L_0, L_1, \ldots$.
Некоторые свойства полученной конструкции:

\begin{itemize}
	\item $T(\bigcup L_n) = \bigcup L_n$.
	\item $r_A(x) = $ время возвращения в~$A$ корректно определено почти всюду и
		измеримо.
	\item $\mu(\bigcup L_n) = \int\limits_A r_A(x) d\mu$.
	\item $E(r_A \mid A) \le \frac{1}{\mu(A)}$.
\end{itemize}

\begin{definition}
	Башня высоты~$h$ с~основанием~$A$ есть неперескающиеся множества
	$A, \ldots, T^{h-1}A$.
\end{definition}

\begin{exercise}[Лемма Рохлина-Халмоша]
	$\forall \eps > 0, h \in \mathbb{N} \rightarrow \exists$ башня, такая что
	$\mu(\bigsqcup\limits_{k=0}^{h-1} T^kA) > 1 - \eps$ ($T$ сохраняет меру, и,
	возможно, еще что-нибудь).
\end{exercise}

\section{Семинарская часть}

\begin{definition}
	Биллиард~--- система из точки и~множества на плоскости, в~котором движется
	точка, отражаясь по стандартным законам.
\end{definition}

\begin{exercise}
	Есть биллиард в~угле величины~$\alpha < \frac{\pi}{2}$. Уйдет ли точка на
	бесконечность?
\end{exercise}
\begin{exercise}
	Есть биллиард во множестве~$X = \{(x, y) \mid x \ge 0, y \ge 0, y \le
	\frac{1}{1 + x^2}$. Уйдет ли точка, выпущенная из $(0, 0)$, на бесконечность?
\end{exercise}
\begin{exercise}
	Есть биллиард в~прямоугольнике. Есть ли замнутые траектории? Есть ли
	незамкнутые? Есть ли всюду плотные?
\end{exercise}
\begin{exercise}
	Есть биллиард в~круге. Какие бывают траектории? А~в~эллипсе?
\end{exercise}

\end{document}
