\documentclass{article}
\usepackage[utf8x]{inputenc}
\usepackage[english,russian]{babel}
\usepackage{amsmath,amscd}
\usepackage{amsthm}
\usepackage{mathtools}
\usepackage{amsfonts}
\usepackage{amssymb}
\usepackage{cmap}
\usepackage{centernot}
\usepackage{enumitem}
\usepackage{perpage}
\usepackage{chngcntr}
%\usepackage{minted}
\usepackage[bookmarks=true,pdfborder={0 0 0 }]{hyperref}
\usepackage{indentfirst}
\hypersetup{
  colorlinks,
  citecolor=black,
  filecolor=black,
  linkcolor=black,
  urlcolor=black
}

\newtheorem*{conclusion}{Вывод}
\newtheorem{theorem}{Теорема}
\newtheorem{lemma}{Лемма}
\newtheorem*{corollary}{Следствие}

\theoremstyle{definition}
\newtheorem*{problem}{Задача}
\newtheorem{claim}{Утверждение}
\newtheorem{exercise}{Упражнение}
\newtheorem{definition}{Определение}
\newtheorem{example}{Пример}

\theoremstyle{remark}
\newtheorem*{remark}{Замечание}

\newcommand{\doublearrow}{\twoheadrightarrow}
\renewcommand{\le}{\leqslant}
\renewcommand{\ge}{\geqslant}
\newcommand{\eps}{\varepsilon}
\renewcommand{\phi}{\varphi}
\newcommand{\ndiv}{\centernot\mid}

\MakePerPage{footnote}
\renewcommand*{\thefootnote}{\fnsymbol{footnote}}

\newcommand{\resetcntrs}{\setcounter{theorem}{0}\setcounter{definition}{0}
\setcounter{claim}{0}\setcounter{exercise}{0}}

\DeclareMathOperator{\aut}{aut}
\DeclareMathOperator{\cov}{cov}
\DeclareMathOperator{\argmin}{argmin}
\DeclareMathOperator{\argmax}{argmax}
\DeclareMathOperator*\lowlim{\underline{lim}}
\DeclareMathOperator*\uplim{\overline{lim}}
\DeclareMathOperator{\re}{Re}
\DeclareMathOperator{\im}{Im}

\frenchspacing


\begin{document}

\section*{Лекция 2. Эргодическая теорема I}
\addcontentsline{toc}{section}{Лекция 2. Эргодическая теорема II}
\resetcntrs

\section{Эргодические системы, теорема Биркгофа-Хинчина}

Рассматриваются системы вида $(G, (X, \mathcal{B}, \mu), T^t)$, с~конечной
мерой~$\mu(X) = 1$. Ограничимся также только дискретным временем $\mathbb{Z}$.

\begin{definition}
	$T$ называется \emph{эргодическим}, если $\forall A \in \mathcal{B}:
	0 < \mu(A) < \mu(X), TA \ne A \pmod 0$.
\end{definition}

Это значит, что в~$X$ нет разбиения на два инвариантных множества~$A, B$
ненулевой меры. Действие называется эргодическим, если $T^t$ эргодчино для
любого $t$.

\begin{theorem}[Эргодическая теорема Биркгофа-Хинчина]
	Если $T$~--- эргодическое, то $\underset{\text{п.н.}}{\forall} x \in X,
	\text{ограниченной измеримой } f \in L^\infty(X)$ выполнено
	$$\frac{1}{n} \sum\limits_0^{n-1} f(T^kx) \rightarrow const = \int\limits_X f
	d\mu$$
\end{theorem}

\begin{definition}
	$T$ называется \emph{перемешивающим} ($T \in Mix$), если $$\forall A, B \in
	\mathcal{B}: \mu(T^k A \cap B) \rightarrow \mu(A)\mu(B)$$.
\end{definition}

\begin{definition}
	$T$ называется \emph{слабо перемешивающим} ($T \in WMix$), если $$\forall A, B
	\in \mathcal{B}\,\exists \{k_j\}_1^\infty : \mu(T^k_j A \cap B) \rightarrow
	\mu(A)\mu(B)$$.
\end{definition}

\section{Оператор Купмана}

\begin{definition}
	Оператор Купмана $\hat T: f(x) \mapsto f(Tx)$.
\end{definition}

Для недискретного времени это будет представлением группы времени. Изучение
свойств этого линейного оператора приводит к~так называемой спектральной теории.

\begin{remark}
	Можно переформулировать все три данных определения:
	\begin{itemize}
		\item Эргодичность: $\hat T f_0 = f_0 \Rightarrow f_0 = const$.
		\item Перемешивание: $\left<T^k f, g\right> \rightarrow 0$, $\int f d\mu =
			\int g d\mu = 0$. Иначе, $\hat T^k \underset{w}{\rightarrow} \Theta =
			P_{\{const\}}$ (ортопроектор на константу).
		\item $WCl(\{\hat T^k\}) \ni \Theta$.
	\end{itemize}
\end{remark}

\begin{theorem}
	$T \in Mix \Rightarrow T$~--- эргодическое.
\end{theorem}
\begin{proof}
	От противного: пусть $\exists \xi \ne const \hat T \xi = \xi$.
	$\xi_0 = \xi - \Theta \xi = \xi - \mathbb{E}\xi = \xi - \overline{\xi}$
	($\Theta: f(x) \mapsto (x \rightarrow \int f d\mu)$. Обозначение $\Theta$
	похоже на 0, неслучайно: $\Theta A = A \Theta = \Theta$).

	$\Theta \xi_0 = 0, \int \xi_0 d\mu = 0, \xi_0 \ne const$.
	$\left< T^k \xi_0, \xi_0 \right> \rightarrow 0$, но $\left< T^k \xi_0, \xi_0
	\right> = \left< \xi_0, \xi_0 \right> > 0$, противоречие.
\end{proof}

\section{Семинарская часть}

\begin{exercise}
	Показать, что $[0; 1] \cong [0; 1] \times [0; 1]$ как пространства с~мерой, то
	есть построить измеримую биекцию, сохраняющую меру.
\end{exercise}
\begin{definition}
	Преобразование пекаря: $A\mid B \rightarrow \frac{B}{A}$.
\end{definition}

Формула для преобразования пекаря в~двоичном коде очень простая: $\ldots y_2
y_1 x_1 x_2 \ldots \Rightarrow \ldots y_2 y_1 x_1 x_2 x_3 \ldots$, почти как
левый сдвиг для случайных процессов.

\begin{definition}
	Подкова Смейла: $(x, y) \mapsto (\frac{x}{3}, 3y) \pmod 1$.
\end{definition}

\begin{definition}
	Сдвиг Бернулли: $\mathbb{A} = \{0, 1\}, p = (p_0, p_1), p_0 + p_1 = 1$.
	$\Sigma_2 = \mathbb{A}^\mathbb{Z} = \{ x : \mathbb{Z} \rightarrow \mathbb{A}
	\}$. Тогда для слова $w$: $P([w]) = p_0^{\text{\#нулей в~w}}
	p_1^{\text{\#единиц в~w}}$.
\end{definition}

\begin{exercise}
	Найти инвариантное множество для подковы Смейла.
	Показать, что канторовское множество изоморфно $[0; 1]$ как пространство
	с~мерой.
\end{exercise}

\begin{exercise}
	Попробовать устранить <<негладкость>> преобразования пекаря и~<<сингулярность
	подковы Смейла>>.
\end{exercise}

\begin{exercise}[$\star\star$]
	$T \in Mix \Leftrightarrow \forall A \in \mathcal{B} \rightarrow
	\mu(T^k A \cap A) \rightarrow \mu(A)^2$.
\end{exercise}

\end{document}
