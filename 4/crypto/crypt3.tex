\documentclass{article}
\usepackage[utf8x]{inputenc}
\usepackage[english,russian]{babel}
\usepackage{amsmath,amscd}
\usepackage{amsthm}
\usepackage{amsfonts}
\usepackage{amssymb}
\usepackage{cmap}
\usepackage{centernot}
\usepackage{enumitem}
\usepackage{perpage}
\usepackage{chngcntr}
%\usepackage{minted}
\usepackage[bookmarks=true,pdfborder={0 0 0 }]{hyperref}
\usepackage{indentfirst}
\hypersetup{
  colorlinks,
  citecolor=black,
  filecolor=black,
  linkcolor=black,
  urlcolor=black
}

\newtheorem*{conclusion}{Вывод}
\newtheorem{theorem}{Теорема}
\newtheorem{lemma}{Лемма}
\newtheorem*{corollary}{Следствие}

\theoremstyle{definition}
\newtheorem*{problem}{Задача}
\newtheorem{claim}{Утверждение}
\newtheorem{exercise}{Упражнение}
\newtheorem{definition}{Определение}
\newtheorem{example}{Пример}

\theoremstyle{remark}
\newtheorem*{remark}{Замечание}

\renewcommand{\le}{\leqslant}
\renewcommand{\ge}{\geqslant}
\newcommand{\eps}{\varepsilon}
\renewcommand{\phi}{\varphi}
\newcommand{\ndiv}{\centernot\mid}

\MakePerPage{footnote}
\renewcommand*{\thefootnote}{\fnsymbol{footnote}}

\newcommand{\resetcntrs}{\setcounter{theorem}{0}\setcounter{definition}{0}
\setcounter{claim}{0}\setcounter{exercise}{0}}

\DeclareMathOperator{\aut}{aut}
\DeclareMathOperator{\cov}{cov}
\DeclareMathOperator{\chos}{ch}
\DeclareMathOperator{\argmin}{argmin}
\DeclareMathOperator{\argmax}{argmax}
\DeclareMathOperator*\lowlim{\underline{lim}}
\DeclareMathOperator*\uplim{\overline{lim}}
\DeclareMathOperator{\re}{Re}
\DeclareMathOperator{\im}{Im}

\frenchspacing


\begin{document}

\section*{Лекция 3. Односторонняя перестановка и~генератор псевдослучайных
чисел}
\addcontentsline{toc}{section}{Лекция 3. Односторонняя перестановка и~генератор
псевдослучайных чисел}
\resetcntrs

\section{XOR-лемма Яо}

\begin{theorem}
	Если существует односторонняя перестановка $p: D_n \rightarrow D_n, D_n
	\subset \{0, 1\}^{k(n)} $, то существует генератор псевдослучайных чисел.
\end{theorem}
\begin{proof}
	Напоминание: схема доказательства теоремы:
	\begin{itemize}
		\item Односторонняя перестановка $f \mapsto$ односторонняя перестановка
			с~трудным битом (декодированиe списком кода Адамара и~дерандомизации
			с~помощью попарной независимости).
		\item Генератор $n \rightarrow n + 1$: $G(x) = g(x) b(x)$ (\text{XOR}-лемма
			Яо).
		\item Генератор $n \rightarrow p(n)$: $g(g(x))b(g(x))b(x),
			g(g(g(x)))b(g(g(x)))b(g(x))b(x), \ldots$ (hybrid argument).

	\end{itemize}
	Сначала сделаем второй шаг.

	\begin{definition}
		$b(x)$~--- трудный бит для $f(x)$, если он полиномиально вычислим
		и~$\forall p(\cdot) \forall \{P_n\}_{n=1}^{\infty} \exists N:
		\forall n \ge N \rightarrow \left| P(P_n(f(x)) = b(x)) -
		\frac{1}{2}\right| < \frac{1}{p(n)}$.
	\end{definition}
	\begin{lemma}
		$b(x)$~--- трудный бит для $f(x) \Rightarrow G(x) = f(x) b(x)$~--- генератор
		псевдослучайных чисел.
	\end{lemma}
	\begin{proof}
		Если существует отличитель для $G(x)$, то $\exists s(\cdot) \exists
		\{D_n\}_{n=1}^{\infty} \forall N \exists n > N: \left| P_x(D_n(f(x)b(x)) = 1)
		- P_y(D_n(y) = 1)\right| \ge \frac{1}{s(n)}$. Можно считать, что выражение
		под модулем положительно, так как для тех $n$, для которых это не так, можно
		инвертировать вывод $D_n$.

		Рассмотрим варианты для $D(f(x)0) = \alpha, D(f(x)1) = \beta$.
		\begin{itemize}
			\item $\alpha = \beta \Rightarrow$ значение предсказателя случайно.
			\item $\alpha = 0, \beta = 1 \Rightarrow$ предсказатель возвращает 1.
			\item $\alpha = 1, \beta = 0 \Rightarrow$ предсказатель возвращает 0.
		\end{itemize}
		Обозначим $A, B, C, D$~--- события для $00, 01, 10, 11$ соответственно.
		$A_0, A_1 \subset A B_0, B_1 \subset B \ldots$ разбиения по значениям
		трудного бита, $a_0, a_1, \ldots$~--- их вероятности.

		$P(D_n(f(x)b(x)) = 1) = b_1 + c_0 + d_0 + d_1$,

		$P(D_n(y) = 1) = \frac{b_0 + b_1}{2} + \frac{c_0 + c_1}{2} + d_0 + d_1$.

		Тогда разность $\Delta = \frac{b_0 + b_1}{2} + \frac{c_0 + c_1}{2} \ge
		\frac{1}{s(n)}$.

		Успех предсказателя: $\frac{a_0 + a_1}{2} + b_1 + c_0 + \frac{d_0 +
		d_1}{2} = \frac{1}{2} + \Delta \ge \frac{1}{2} + \frac{1}{s(n)}$.
	\end{proof}
	Почему \text{XOR}-лемма? Потому что $P(f(x)) = D(f(x)r) \oplus r \oplus 1$.

\section{Построение генератора любой длины}

	Теперь сделаем генератор $n \rightarrow q(n)$. Для начала рассмотрим $G(x) =
	f(f(x))b(f(x))b(x)$, что должно быть вычислительно неотличимо от $x r_1 r_2$.

	$x r_1 r_2 \sim f(x) r_1 r_2$, так как $x$ и~$f(x)$ одинаково распределены
	(так как $f$~--- перестановка). $f(x) r_1 r_2 \sim f(x) b(x) r_2$ по
	определению~$G$. Далее, $x r_2 \sim f(x)b(x) \Rightarrow x r_1 r_2 \sim
	f(f(x))b(f(x))b(x)$.

	Для любого константного увеличения можно сделать точно также. Для $n
	\rightarrow q(n)$ делаем так:
	$$ h_0(x) = x r_1 r_2 \ldots r_{q(n)} $$
	$$ \vdots $$
	$$ h_{q(n)}(x) = f^{q(n)}(x) b(f^{q(n)-1}(x)) \ldots b(x) $$

	Хотим доказать, что $h_{q(n)} \sim h_0(x)$. Если $P(D_n(h_{q(n)}(x)) = 1) -
	P(D_n(h_0(x)) = 1) \ge \frac{1}{s(n)}$, то $\exists m: P(D_n(h_m(x)) = 1) -
	P(D_n(h_{m-1}(x)) = 1) \ge \frac{1}{s(n)q(n)}$, что невозможно аналогично
	пункту $n \rightarrow n + 2$.

	\begin{theorem}[Левин-Голдрайх]
		Пусть~$f$~--- односторонняя перестановка, то $g(xy) = f(x)y$ тоже
		одностороняя перестановка, а~$b(xy) = x \odot y = \bigoplus\limits_{i=1}^{n}
		x_i y_i$~--- трудный бит для~$g$.
	\end{theorem}
	\begin{proof}
		Первая часть очевидна, если $f$~--- односторонняя пересатновка, то и~$g$
		тоже перестановка, легко вычисляется и~если $g$ можно обратить, то обратить
		можно и~$f$. Для доказательства второй части воспользуемся кодом Адамара.

		Код Адамара: $x \mapsto (x \odot z)_{z\in \{0, 1\}^n}$ слово длины~$n$
		превращает в~слово длины $2^n$. Его можно воспринимать как значение всех
		линейных функций на входе $x$ или как значение на всех входах линейной
		функции, заданной $x$.

		Пусть $\hat{f}(z)$ совпадает с~$f(z)$ на доле входов $z$ равной	$\frac{3}{4}
		+ \eps$. Тогда можно восстановить $f(z) = \hat{f}(z + r) + \hat{f}(r)$
		и~с~вероятностью $>\frac{1}{2}$ мы восстановим $f(z)$. Повторив много раз,
		можем узнать $f(e_i) = x_i$.

		Для доли повреждения $\frac{1}{2}$ декодировать уже не получится, но можно
		декодировать списком: имея доступ к~$\hat{f}(z)$ как к~оракулу, напечатать
		полиномиальный список в~котором с~вероятностью $\ge \frac{1}{2}$ находится
		вектор~$x$, определяющий~$f$.

		\begin{problem}
			В~шаре с~центром в~любой точке и~радиусом (в~смысле расстояния Хемминга)
			$\frac{1}{2} - \eps$ находится~$poly\left(\frac{1}{\eps}\right)$ кодовых
			слов.
		\end{problem}

		Запишем равенство: $f(z) = \hat{f}(z + r) + f(r)$, которое должо быть
		выполнено в~$\ge \frac{1}{2}$ случаев. Непонятно только, откуда взять
		$f(r)$.

		Идея попарной независимости: проведем процедуру выше для некоторого числа
		попарно независимых случайных~$r$. Возьмем случайные неависимые
		в~совокупности вектора~$u_1, \ldots, u_l$ и~вектора~$r_1, \ldots, r_{2^l -
		1}$ построим как~$r_a = a_1 u_1 + \ldots + a_l u_l$. Тогда $r_1, \ldots,
		r_{2^l - 1}$ попарно независимы. Алгоритм будет следующий:

		\begin{minted}[tabsize=4]{cpp}
	u_1, ..., u_l := random()
	for (f(u_1), ... f(u_l) in {{0,1}^n}^l) {
		for (int a = 1; a < 2^l - 1; ++a) {
			f(r_a) = a_1 f(u_1) + ... + a_l f(u_l)  // linearity
			f(e_i) = f_hat(e_i + r_a) + f(r_a)
		}
		choose f(e_i) as majority for all a
		add f(e_1), ..., f(e_l) in list
	}
		\end{minted}

		Утвержается, что по неравенству Чебышёва при большом числе
		повторений с~вероятностью больше, чем $\frac{1}{2}$ декодирование
		произведено верно.

		Теперь, если $g$~--- это односторонняя функция, $h(xy) = g(x)y, b(xy) = x
		\odot y = f(y)$ и~есть предсказатель $b$, то можно с~его помощью построить
		$\hat{f}(y)$, совпадающую на доле $\frac{1}{2} + \eps$, что можно
		декадировать списком $x_1, \ldots, x_m$ и~каждый $x$ проверить
		непосредственно.

		Несмотря на то, что $\hat{f}$ экспоненциально длинная, но нам нужно только
		занчение в~полиноме точек, которые мы и~запомним (или можно относиться
		к~$\hat{f}$ как к~оракулу).

	\end{proof}
\end{proof}

\end{document}
