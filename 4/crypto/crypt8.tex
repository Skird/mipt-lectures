\documentclass{article}
\usepackage[utf8x]{inputenc}
\usepackage[english,russian]{babel}
\usepackage{amsmath,amscd}
\usepackage{amsthm}
\usepackage{mathtools}
\usepackage{amsfonts}
\usepackage{amssymb}
\usepackage{cmap}
\usepackage{centernot}
\usepackage{enumitem}
\usepackage{perpage}
\usepackage{chngcntr}
%\usepackage{minted}
\usepackage[bookmarks=true,pdfborder={0 0 0 }]{hyperref}
\usepackage{indentfirst}
\hypersetup{
  colorlinks,
  citecolor=black,
  filecolor=black,
  linkcolor=black,
  urlcolor=black
}

\newtheorem*{conclusion}{Вывод}
\newtheorem{theorem}{Теорема}
\newtheorem{lemma}{Лемма}
\newtheorem*{corollary}{Следствие}

\theoremstyle{definition}
\newtheorem*{problem}{Задача}
\newtheorem{claim}{Утверждение}
\newtheorem{exercise}{Упражнение}
\newtheorem{definition}{Определение}
\newtheorem{example}{Пример}

\theoremstyle{remark}
\newtheorem*{remark}{Замечание}

\newcommand{\doublearrow}{\twoheadrightarrow}
\renewcommand{\le}{\leqslant}
\renewcommand{\ge}{\geqslant}
\newcommand{\eps}{\varepsilon}
\renewcommand{\phi}{\varphi}
\newcommand{\ndiv}{\centernot\mid}

\MakePerPage{footnote}
\renewcommand*{\thefootnote}{\fnsymbol{footnote}}

\newcommand{\resetcntrs}{\setcounter{theorem}{0}\setcounter{definition}{0}
\setcounter{claim}{0}\setcounter{exercise}{0}}

\DeclareMathOperator{\aut}{aut}
\DeclareMathOperator{\cov}{cov}
\DeclareMathOperator{\argmin}{argmin}
\DeclareMathOperator{\argmax}{argmax}
\DeclareMathOperator*\lowlim{\underline{lim}}
\DeclareMathOperator*\uplim{\overline{lim}}
\DeclareMathOperator{\re}{Re}
\DeclareMathOperator{\im}{Im}

\frenchspacing


\begin{document}

\section*{Лекция 8. Электронная подпись I}
\addcontentsline{toc}{section}{Лекция 8. Электронная подпись I}
\resetcntrs

\section{Электронная подпись с~закрытым ключом}

Схема электронной подписи с~закрытым ключом:
\begin{itemize}
	\item $K$~--- генератор закрытого ключа~$d$.
	\item $S(m, d) = s$~--- подпись.
	\item $V(m, d, s) \in \{0, 1\}$~--- верификатор.
	\item Корректность: $P(V(m, d, S(m, d)) = 1) \approx 1$.
	\item Атака происходит следующим образом: схема~$C$ несколько раз может
		сгенерировать сообщение $x_i$ и~получить подпись для него. После этого схема
		$D$ пытается сгенерировать новое слово $x$, не встречавшееся среди запросов
		и~подписать его так, чтобы верификатор принял подпись.

		$m_1 = C(), s_1 = S(m_1, d),\\
		 m_2 = C(s_2), s_2 = S(m_2, d),\\
		 \ldots,\\
		 m_k = C(s_1, \ldots, s_{k-1}), s_k = S(m_k, d),\\
		 m_{k+1} = C(s_1, \ldots, s_k), s_{k+1} = D(s_1, \ldots, s_k)$.

		Условие неподделываемости: $P(m_{k+1} \notin \{m_1, \ldots, m_k\},
		V(m_{k+1}, d, s_{k+1}) = 1) \approx 0$.
\end{itemize}

Если существует протокол подписи, то его можно использовать для задачи
идентификации: сервер посылает клиенту случайное сообщение~$m$ на подпись
и~проверяет подпись под ним. Несложно увидеть, что атака на такой протокол это и
есть атака на электронную подпись.

С~закрытым ключом алгоритм довольно простой: пусть $d$~--- идентификатор
псевдослучайной функции. Положим~$S(m, d) = f_d(m), V(m, d, s) = I(s = f_d(m))$.
Взломщик в~таком случае получает значение псевдослучайной функции на
полиномиальном числе входов. Однако семейство псевдослучайных функций
вычислительно неотличимо от семейства всех функции, поэтому подписывая значение
$m_{k+1}$ он угадает с~вероятностью, близкой к~вероятности угадать значение
случайной функции, которое не зависит от значений в~предыдущих точках, то есть
с~экспоненциально малой вероятностью.

\section{Электронная подпись с~открытым ключом}

\begin{itemize}
	\item $K$~--- генератор закрытого и~открытого ключей~$d, e$.
	\item $S(m, d) = s$~--- подпись.
	\item $V(m, e, s) \in \{0, 1\}$~--- верификатор.
	\item Корректность: $P(V(m, e, S(m, d)) = 1) \approx 1$.
	\item Атака происходит таким же образом, однако схемы $C, D$ дополнительно
		получают аргумент $e$.

		Условие неподделываемости: $P(m_{k+1} \notin \{m_1, \ldots, m_k\},
		V(m_{k+1}, e, s_{k+1}) = 1) \approx 0$.
\end{itemize}

Подпись одного бита можно сделать на базе любой односторонней функции~$f$. $d =
(x_0, x_1), e = (f(x_0), f(x_1))$ (нужно выбрать так, чтобы $f(x_0), \ne
f(x_1)$).

$m_1 = \sigma = C(y_0, y_1), s_1 = x_\sigma, m_2 = 1 - \sigma, s_2 =
D(y_0, y_1, x_\sigma)$. Нужно показать, что $P(f(s_2) = y_{1-\sigma}) \approx
0$.

Пусть существует алгоритм, взламывающий такую подпись, тогда построим алгоритм,
обращающий одностороннюю функцию. Обратитель $f$ будет действовать так:
\begin{itemize}
	\item Получает~$y$.
	\item Выбирает случайный~$x$ и~случайный бит~$\tau$.
	\item На всякий случай проверяет $f(x) \ne y$ (если равно, то обратил).
	\item $y' = f(x'), y_\tau = y', y_{1-\tau} = y$.
	\item $C(y_0, y_1) = \tau \Rightarrow $ можно вернуть $D(y_0, y_1, x')$.
	\item В~противном случае вернуть что угодно.
\end{itemize}

Так как вероятность последнего условия $\frac{1}{2}$, то вероятность обращения
не меньше $\frac{1}{2}$ вероятности подделки подписи.

Построим теперь подпись одного сообщения фиксированной длины. Для этого
переделаем схему подписи одного бита $(K, S, V)$ в~$(\hat K, \hat S, \hat V)$,
так что $\hat K$ запускает $k$ раз алгоритм~$K$, генерируя $(d_1, e_1), \ldots,
(d_k, e_k)$ то есть $d = (d_1, \ldots, d_k), e = (e_1, \ldots, e_k)$.

$\hat S(\sigma_1, \ldots, \sigma_k) = S(\sigma_1, d_1) \ldots S(\sigma_k, d_k)$.
Верификатор устроен очевидным образом.

Такая подпись работает только для фиксированной длины (нельзя даже брать слова
длины $\le k$, так как после любого сообщения можно корректно подписать любой
его префикс), притом подпись одноразовая, поскольку подпись сообщений $0\ldots0$
и~$1\ldots1$ позволяет подделать любую подпись.

Часть проблем решает беспрефиксным кодированием: например удваиваем каждый бит
и~дописываем в~конце $01$. Так, вместо длины ровно $k$ мы сможем подписывать
сообщения длины $\le k$.

\section{Снятие ограничений по длине и~одноразовости}

Идея: подпись под хеш-значением.

\begin{definition}
	Семейство хеш-функций с~трудно обнаружимыми коллизиями~--- $h: \{0, 1\}^{p(n)}
	\times \{0, 1\}^\ast \rightarrow \{0, 1\}^{q(n)}$, такая что для схемы, ищущей
	коллизии $C(s) = \{x_1, x_2\}, x_1 \ne x_2, h_s(x_1) = h_s(x_2)$ верятность
	успеха близка к~0 при слуайно выбранном~$s$.
\end{definition}

Если такое семейство существует, то достаточно взять хеш-значение сообщения
и~подписать только его. Атакующий должен будет сделать одну из двух вещей~---
либо найти коллизию хеш-функции, либо использовать соощение с~другим хешом, но
подделать его подпись. В~обеих ситуациях вероятность успеха мала.

Идея для многоразовой подписи: $S, V$ интерактивны и~хранят всю историю,
взломщик прослушивает канал и~может потом попытаться подделать подпись.
$S$ вместе с~подписью присылает новый открытый ключ для следующего раунда,
притом подпись распространяется и~на него.

Идею интерактивности можно использовать для подписи слова произвольной длины. $m
= \sigma_1 \ldots \sigma_k$, $(d_0, e_0)$ ключ для однократной
подписи~$k$-битов $e = (e_0, l)$.
Генерируются $(d_1, e_1), s_1$~--- подпись под~$h_l(\sigma_1, e_1)$ при помощи
$d$, $(d_2, e_2), s_2$~--- подпись под~$h_l(\sigma_2, e_2)$ при помощи $d_1$
и~так далее.

\begin{definition}
	Универсальное семейство односторонних хеш-функций~--- семейство, такое что
	трудно обнаружить коллизию для заранее заданного~$x$.
\end{definition}

Общая идея для многократной электронной подписи без внутренней памяти: вместо
следующего ключа подписываются 2 следующих ключа: какой из них используется
зависит от битов сообщения.

\end{document}
