\documentclass{article}
\usepackage[utf8x]{inputenc}
\usepackage[english,russian]{babel}
\usepackage{amsmath,amscd}
\usepackage{amsthm}
\usepackage{amsfonts}
\usepackage{amssymb}
\usepackage{cmap}
\usepackage{centernot}
\usepackage{enumitem}
\usepackage{perpage}
\usepackage{chngcntr}
%\usepackage{minted}
\usepackage[bookmarks=true,pdfborder={0 0 0 }]{hyperref}
\usepackage{indentfirst}
\hypersetup{
  colorlinks,
  citecolor=black,
  filecolor=black,
  linkcolor=black,
  urlcolor=black
}

\newtheorem*{conclusion}{Вывод}
\newtheorem{theorem}{Теорема}
\newtheorem{lemma}{Лемма}
\newtheorem*{corollary}{Следствие}

\theoremstyle{definition}
\newtheorem*{problem}{Задача}
\newtheorem{claim}{Утверждение}
\newtheorem{exercise}{Упражнение}
\newtheorem{definition}{Определение}
\newtheorem{example}{Пример}

\theoremstyle{remark}
\newtheorem*{remark}{Замечание}

\renewcommand{\le}{\leqslant}
\renewcommand{\ge}{\geqslant}
\newcommand{\eps}{\varepsilon}
\renewcommand{\phi}{\varphi}
\newcommand{\ndiv}{\centernot\mid}

\MakePerPage{footnote}
\renewcommand*{\thefootnote}{\fnsymbol{footnote}}

\newcommand{\resetcntrs}{\setcounter{theorem}{0}\setcounter{definition}{0}
\setcounter{claim}{0}\setcounter{exercise}{0}}

\DeclareMathOperator{\aut}{aut}
\DeclareMathOperator{\cov}{cov}
\DeclareMathOperator{\chos}{ch}
\DeclareMathOperator{\argmin}{argmin}
\DeclareMathOperator{\argmax}{argmax}
\DeclareMathOperator*\lowlim{\underline{lim}}
\DeclareMathOperator*\uplim{\overline{lim}}
\DeclareMathOperator{\re}{Re}
\DeclareMathOperator{\im}{Im}

\frenchspacing


\begin{document}

\section*{Лекция 1. Односторонние функции}
\addcontentsline{toc}{section}{Лекция 1. Односторонние функции}
\resetcntrs

\section{Введение}

5 миров Импальяццо.
\begin{itemize}
	\item Алгоритмика ($\textbf{P} = \textbf{NP}$).
	\item Эвристика ($\textbf{P} \ne \textbf{NP}$, но есть быстрый алгоритм
		в~среднем).
	\item Pessiland ($\textbf{P} \ne \textbf{NP}$, нет ни быстрых алгоритмов, ни
		односторонней функции).
	\item Миникрипт (есть односторонние функции, но нет односторонних функций
		с~секретом).
	\item Криптомания (есть односторонние функции с~секретом).
\end{itemize}

В~принципе, может статься еще что-то странное навроде $\textbf{P} =
\textbf{NP}$, но на практике эти алгоритмы очень долгие или $\textbf{P} \ne
\textbf{NP}$, но наоборот, есть какие-либо квазиполиномиальные быстрые
алгоритмы.

Сначала будут обсуждаться примитивы, односторонние функции, доказательства
с~нулевым разглашением и~прочее, потом на базе этого покажем, как построить
криптографические протоколы.

Литература: конспект лекций Верещагина <<Лекции по математической
криптографии>>, черновик, Glodreich <<Foundations of Cryptography>>, конспекты
Goldwasser.

\section{Односторонние функции}

В~криптографических задачах полиномиальность будет считаться от параметра
безопасности $n$ (неформально, длина ключа) для доказательства надёжности, и~от
длины шифруемого сообщения при шифровании.

\begin{definition}
	$\{f_n\}_{n=1}^{\infty}$~--- семейство односторонних функций, если:
	\begin{itemize}
		\item $f$ регулярны по длине: $f_n: \{0, 1\}^{k(n)} \rightarrow
			\{0, 1\}^{l(n)}$.
		\item $f$ вычислимы за полиномиальное время.
		\item $f$ труднообратима (4 варианта: в~сильном/слабом смысле, против
	равномерного/неравномерного обратителя).
	\end{itemize}
\end{definition}

Обратимость в~слабом смысле: вероятность неудачи обратителя больше, чем
некоторый обратный полином.

В~сильном смысле: вероятность успеха асимптотически меньше, чем любой обратный
полином.

Равномерный обратитель~--- полиномиальный вероятностный алгоритм.

Неравномерный обратитель~--- семейство схем полиномиального размера.

Задача обращения: по $f(x)$ найти $x': f(x') = f(x)$.

Кванторная запись определения труднообратимой функции в~сильном смысле: $\forall
p(\cdot) \forall \{R_n\}_{n=1}^{\infty} \exists N: \forall n > N \rightarrow
P_{x \sim U_k(n)} \{ f(R(f(x))) = f(x) \} < \frac{1}{p(n)}$.

$R$ в~определении пробегает по всем семействам схем или вероятностным
обратителям в~зависимости от вида обратителя. Определение в~слабом смысле
отличается первым квантором.

\begin{problem}
	Может ли семейство $f_n: |\text{Im} f_n| = poly(n) = s(n)$ быть труднообратимом
	в~слабом смысле?

	Неравномерный обратитель: можно <<зашить>> в~схему по одному прообразу от
	каждого класса.

	Равномерный: берёт случайный $x$, вычисляет $f(x)$, если $y = f(x)$, вернуть
	$x$. Можно подобрать такое число повторений $N$, чтобы вероятность ошибки была
	мала. Идея: классы бывают большие (размера $> 2^{l(n)} \cdot \eps$),
	и~маленькие. Вероятность неуспеха для больших классов не больше $(1 -
	\eps)^N$, а~для маленького класса можно оценить единицей. Тогда общая
	вероятность ошибки для слуйчайного $x$ не больше $s(n) \cdot \eps + (1 -
	\eps)^N$. Если $\eps$ взять как $\frac{1}{2s(n)q(n)}$, а~$N = \frac{n}{\eps}$,
	то сумма будет не больше $\frac{1}{q(n)}$ для любого полинома~$q(n)$.
\end{problem}
\begin{problem}
	$f$~--- односторонняя функция. Верно ли, что $g(x) = f(x)f(x)$ тоже
	одностороняя? Верно ли, что $h(x y) = f(x)f(y)$ будет односторонней?

	Если $g$ односторонняя, то $\exists R_g$, которая обращает $g$. Тогда $R_f(y) =
	R_g(yy)$ обращает $f$.

	Если $R_h$ обращает $h$, то можно построить такой обратитель $f$: берём
	случайный $y$, считаем $f(y)$ и~возвращаем первую часть $R_h(f(x)f(y))$, если
	всё нормально, иначе нужно повторить процедуру.

	Хорошие значения $x$ - это те, для которых доля пар $(x, y)$ больше или равна
	$\eps$. Остальных значений $x$ мало. Аналогичными предыдущей задаче
	рассуждениями можно получить оценку.

\end{problem}

\end{document}
