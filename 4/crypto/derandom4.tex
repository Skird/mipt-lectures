\documentclass{article}
\usepackage[utf8x]{inputenc}
\usepackage[english,russian]{babel}
\usepackage{amsmath,amscd}
\usepackage{amsthm}
\usepackage{mathtools}
\usepackage{amsfonts}
\usepackage{amssymb}
\usepackage{cmap}
\usepackage{centernot}
\usepackage{enumitem}
\usepackage{perpage}
\usepackage{chngcntr}
%\usepackage{minted}
\usepackage[bookmarks=true,pdfborder={0 0 0 }]{hyperref}
\usepackage{indentfirst}
\hypersetup{
  colorlinks,
  citecolor=black,
  filecolor=black,
  linkcolor=black,
  urlcolor=black
}

\newtheorem*{conclusion}{Вывод}
\newtheorem{theorem}{Теорема}
\newtheorem{lemma}{Лемма}
\newtheorem*{corollary}{Следствие}

\theoremstyle{definition}
\newtheorem*{problem}{Задача}
\newtheorem{claim}{Утверждение}
\newtheorem{exercise}{Упражнение}
\newtheorem{definition}{Определение}
\newtheorem{example}{Пример}

\theoremstyle{remark}
\newtheorem*{remark}{Замечание}

\newcommand{\doublearrow}{\twoheadrightarrow}
\renewcommand{\le}{\leqslant}
\renewcommand{\ge}{\geqslant}
\newcommand{\eps}{\varepsilon}
\renewcommand{\phi}{\varphi}
\newcommand{\ndiv}{\centernot\mid}

\MakePerPage{footnote}
\renewcommand*{\thefootnote}{\fnsymbol{footnote}}

\newcommand{\resetcntrs}{\setcounter{theorem}{0}\setcounter{definition}{0}
\setcounter{claim}{0}\setcounter{exercise}{0}}

\DeclareMathOperator{\aut}{aut}
\DeclareMathOperator{\cov}{cov}
\DeclareMathOperator{\argmin}{argmin}
\DeclareMathOperator{\argmax}{argmax}
\DeclareMathOperator*\lowlim{\underline{lim}}
\DeclareMathOperator*\uplim{\overline{lim}}
\DeclareMathOperator{\re}{Re}
\DeclareMathOperator{\im}{Im}

\frenchspacing


\begin{document}

\section*{Лекция 4. Амплификация}
\addcontentsline{toc}{section}{Лекция 4. Амплификация}
\resetcntrs

\section{Простые техники амплификации}

Хотим в~\textbf{RP} уменьшить ошибку с~$\frac{1}{2}$ до~$\frac{1}{2^k}$.
Стандартный метод: повторить~$k$ раз с~новыми случайными битами: время
увеличится в~$k$ раз, случайных битов нужно $mk$ вместо $m$.

Техника попарной независимости: время увеличено в~$2^k$ раз, но требуется $m +
k$ случайных битов.

\begin{claim}
	Пусть $X_1, \ldots, X_t$~--- попарно независимые СВ со значениями
	в~$\{0, 1\}$. $X = \frac{1}{t} \sum X_i$, $EX = \mu = \frac{1}{t} \sum \mu_i$.
	Тогда $P(|X - \mu| > \eps) < \frac{1}{t\eps^2}$.
\end{claim}
\begin{proof}
	$DX = E(X - \mu)^2 = \frac{1}{t^2} \left(\sum\limits_{i\ne j} cov(X_i, X_j)
	+ \sum DX_i \le \frac{1}{t}\right)$, значит по неравенству Чебышева
	утверждение доказано.
\end{proof}

\section{Амплификация экспандерами}

Экспандеры: время увеличино в~$k$ раз, требуется в~$mk$ случайных битов.

Идея: возьмём экспандер, в~нём случайную вершину, запустим случайное блуждание
длины~$t$ и~все вершины по дороге используем в~качестве случайных битов для
алгоритма.

Нужно показать, что для люого множества вершин, доля которого~$\le \frac{1}{2}$,
вероятность того, что всё блуждание останется внутри этого множества, будет
эскпоненциально малой.

\begin{theorem}
	Пусть~$G$~--- $d$-регулярный экспандер с~параметром $\lambda = 1 - \gamma$. $B
	\subset V(G), \frac{|B|}{|V(G)|} = \mu$. $v_1, \ldots, v_t$~--- случайное
	блуждание со стартом в~начальной вершине.

	Тогда $P(\forall i v_i \in B) \le (\mu + \lambda(1 - \mu))^t$.
\end{theorem}
\begin{proof}
	Будем считать, что любой вектор разложен на компоненты $v = v^\Vert + v^\bot,
	v^\Vert = \frac{\left<u, v\right>}{\left<u, u\right>}u, u = (\frac{1}{n},
	\ldots, \frac{1}{n}), v^\bot = v - v^\Vert$.

	Пусть~$M$~--- матрица блуждания. $vM = (v^\Vert + v^\bot) = v^\Vert M + v^\bot
	M = v^\Vert + v^\bot M$. Однако, $\left\Vert v^\bot M \right\Vert \le \lambda
	\left\Vert v^\bot \right\Vert$. Отметим, что для распределения вероятностей
	очевидно $v^\Vert = u$.

	Также рассмотрим матричное разложение: $vM = v^\Vert + v^\bot = \gamma v^\Vert
	+ (\lambda v^\Vert + v^\bot M) = \gamma v J + \lambda v E = v(\gamma J +
	\lambda E)$, где $J = \frac{1}{N} (1, \ldots, 1)^T (1, \ldots, 1)$~--- матрица
	из единиц.

	$vJ = v^\Vert, v^\Vert J = v^\Vert, v^\bot J = 0$. $E$ определена как
	остаточная матрица и~мы будем показывать про нее, что $\Vert vE \Vert \le
	\Vert v \Vert$.

	\begin{claim}
		Граф~--- экспандер с~параметром~$\lambda \Leftrightarrow M = \gamma J +
		\lambda E, \Vert E \Vert \le 1$.
	\end{claim}
	\begin{proof}
		$E = \frac{M - \gamma J}{\lambda}$. $uE = \frac{uM - \gamma uJ}{\lambda} =
		\frac{u(1 - \gamma)}{\lambda} = u$.

		Если $v \bot u$, то $vE = \frac{v^\bot - \gamma v^\bot J}{\lambda} =
		\frac{1}{\lambda} v^\bot M \Rightarrow \Vert v^\bot E \Vert =
		\frac{1}{\lambda} \Vert v^\bot M \Vert \le \Vert v^\bot \Vert$.
	\end{proof}

	Пусть $P = \text{diag}\{ \chi_B(i) \}, P(i, j) = I(i = j, i \in B)$. Тогда
	$P(v \in B) = |\pi P|_1$.

	\begin{claim}
		$P(v_1, \ldots, v_t \in B) = |u P(MP)^{t-1}|_1$.
	\end{claim}
	\begin{proof}
		Более того, $P(v_1, \ldots, v_{l+1} \in B, v_{l + 1} = i) =
		(uP(MP)^{l})_i$. Докажем индукцией по~$l$:

		База~$l = 0$ очевидна. Показываем переход: ясно, что $(uP(MP)^l \cdot M)_i =
		P(v_1, \ldots v_{l+1} \in B, v_{l+2} = i)$. Если еще раз умножить на $P$, то
		все координаты для $i \notin B$.
	\end{proof}

	$P^2 = P$, а~значит $uP(MP)^{t-1} = uP(PMP)^{t-1}$.

	\begin{claim}
		$\Vert PMP \Vert \le \mu + \lambda(1 - \mu)$.
	\end{claim}
	\begin{proof}
		$\Vert PMP \Vert = \Vert P(\gamma J + \lambda E) P \Vert = \gamma \Vert PJP
		\Vert + \lambda \Vert PEP \Vert \le \gamma \Vert PJP \Vert + \lambda$.

		$xPJP = yJP = N(yu^T)(uP) = (\sum y_i) (uP), \Vert xPJP \Vert = (\sum y_i)
		\Vert uP \Vert \le (\sqrt{\mu N} \Vert y \Vert) \sqrt{\frac{\mu}{N}} = \mu
		\Vert y \Vert \le \mu \Vert x \Vert \Rightarrow \Vert PJP \Vert \le \mu$.

		Итого, $\Vert PMP \Vert \le \gamma \mu + \lambda = \mu + \lambda(1 - \mu)$.
	\end{proof}

	Итого, $P(\forall i v_i \in B) \le |uP(MP)^{t-1}|_1 \le \sqrt{\mu N} \Vert
	uP(PMP)^{t-1} \Vert = \sqrt{\mu N} \Vert uP \Vert \Vert PMP \Vert^{t-1} \le \mu
	(\mu + (1 - \mu)\lambda)^{t-1} < (\mu + (1 - \mu) \lambda)^t$.
\end{proof}

Для \textbf{BPP} применима аналогичная техника.

\end{document}
