\documentclass{article}
\usepackage[utf8x]{inputenc}
\usepackage[english,russian]{babel}
\usepackage{amsmath,amscd}
\usepackage{amsthm}
\usepackage{mathtools}
\usepackage{amsfonts}
\usepackage{amssymb}
\usepackage{cmap}
\usepackage{centernot}
\usepackage{enumitem}
\usepackage{perpage}
\usepackage{chngcntr}
%\usepackage{minted}
\usepackage[bookmarks=true,pdfborder={0 0 0 }]{hyperref}
\usepackage{indentfirst}
\hypersetup{
  colorlinks,
  citecolor=black,
  filecolor=black,
  linkcolor=black,
  urlcolor=black
}

\newtheorem*{conclusion}{Вывод}
\newtheorem{theorem}{Теорема}
\newtheorem{lemma}{Лемма}
\newtheorem*{corollary}{Следствие}

\theoremstyle{definition}
\newtheorem*{problem}{Задача}
\newtheorem{claim}{Утверждение}
\newtheorem{exercise}{Упражнение}
\newtheorem{definition}{Определение}
\newtheorem{example}{Пример}

\theoremstyle{remark}
\newtheorem*{remark}{Замечание}

\newcommand{\doublearrow}{\twoheadrightarrow}
\renewcommand{\le}{\leqslant}
\renewcommand{\ge}{\geqslant}
\newcommand{\eps}{\varepsilon}
\renewcommand{\phi}{\varphi}
\newcommand{\ndiv}{\centernot\mid}

\MakePerPage{footnote}
\renewcommand*{\thefootnote}{\fnsymbol{footnote}}

\newcommand{\resetcntrs}{\setcounter{theorem}{0}\setcounter{definition}{0}
\setcounter{claim}{0}\setcounter{exercise}{0}}

\DeclareMathOperator{\aut}{aut}
\DeclareMathOperator{\cov}{cov}
\DeclareMathOperator{\argmin}{argmin}
\DeclareMathOperator{\argmax}{argmax}
\DeclareMathOperator*\lowlim{\underline{lim}}
\DeclareMathOperator*\uplim{\overline{lim}}
\DeclareMathOperator{\re}{Re}
\DeclareMathOperator{\im}{Im}

\frenchspacing


\begin{document}

\section*{Лекция 1. Задача $\textbf{UPATH}$}
\addcontentsline{toc}{section}{Лекция 1. Задача $\textbf{UPATH}$}
\resetcntrs

\section{Рандомизированный алгоритм для $\textbf{UPATH}$}

Главный вопрос: $\textbf{P} = \textbf{BPP}$? В~книжке <<Hardness and
randomness>> есть некоторые результаты на тему того, что из дерандомизации
может следовать $\textbf{P} \ne \textbf{NP}$.

Успешные примеры дерандомизации: проверка на простоту (алгоритм AKS), задача
$\textbf{UPATH}$ или $\textbf{S-T-CONN} = \{ (G, s, t): $ в~неорграфе $G$ есть
пусть из $s$ в~$t\}$.

\begin{theorem}
	$\textbf{UPATH} \in \textbf{RL}$ (randomized logspace).
\end{theorem}
\begin{proof}
	Запустим блуждание из $s$ на $N$ шагов. Если в~блуждании встретится $t$,
	сказать, что достижимо, иначе нет.

	Предельная частота (hitting time) ребра $P_{uv} = \lim_{n \rightarrow \infty}
	\frac{E\#\{(s_i, s_{i + 1}) = (u, v) \}}{n}$ (добавим петли, применим теорию
	марковских процессов).
	$$P_{u,v} = \frac{1}{\text{ожидаемое время первой встречи (u, v) после выхода
	из v}}$$
	Аналогично, существует предельная частота вершины.

	Так как блуждание равномерно, то $P_{uv} = \frac{1}{\deg u} P_u$ и~$P_u =
	\sum\limits_{t: (t, u) \in E} P_{tu}$. Тогда $P_{uv} = \frac{1}{\deg u}
	\sum\limits_{t: (t, u) \in E} P_{tu}$. Из этого следует, что все частоты
	одинаковы, так если есть максимальная частота, а~у~какого-то смежного меньше,
	то получается противоречие с~равентсвом. То есть $P_{uv} = \frac{1}{2m}, P_u =
	\frac{\deg u}{2m}$.

	Пусть $t_0 = s, t_1, \ldots, t_{k - 1}, t_k = t$~--- путь из $s$ в~$t$.
	Рассмотрим вершину~$t_0$. Среднее время возврата в~$t_0$ не зависит от истории
	блуждания, поэтому оно ровно такое, как в~пределе. Поэтому мы в~среднем не
	менее, чем за $\frac{2m}{\deg u}$ мы будем возвращаться в~$t_0$ и~рано или
	поздно пойдем по ребру $(t_0, t_1)$. Такими рассуждениями, по неравенству
	Маркова можно проделать $4km$ шагов, чтобы с~вероятностью $\ge \frac{1}{2}$
	прийти в~$t_k = t$.
\end{proof}

\begin{definition}
	Граф $d$-регулярный, если степени всех вершин равны $d$.
\end{definition}

\begin{claim}
Существует универсальная последовательность поворотов полиномиальной длины,
которая посещает все вершины.
\end{claim}

Идея доказательства состоит в~следующем: можно сделать случайное блуждание,
такое длинное, что доля графов, на которых оно не посещает все вершины крайне
мала. Тогда, так как таких графов не более $n^{dn}$, то можно сделать долю такой
маленькой, что найдется последовательность, удовлетворяющая всем графам.

\end{document}
