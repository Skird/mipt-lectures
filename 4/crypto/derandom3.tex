\documentclass{article}
\usepackage[utf8x]{inputenc}
\usepackage[english,russian]{babel}
\usepackage{amsmath,amscd}
\usepackage{amsthm}
\usepackage{amsfonts}
\usepackage{amssymb}
\usepackage{cmap}
\usepackage{centernot}
\usepackage{enumitem}
\usepackage{perpage}
\usepackage{chngcntr}
%\usepackage{minted}
\usepackage[bookmarks=true,pdfborder={0 0 0 }]{hyperref}
\usepackage{indentfirst}
\hypersetup{
  colorlinks,
  citecolor=black,
  filecolor=black,
  linkcolor=black,
  urlcolor=black
}

\newtheorem*{conclusion}{Вывод}
\newtheorem{theorem}{Теорема}
\newtheorem{lemma}{Лемма}
\newtheorem*{corollary}{Следствие}

\theoremstyle{definition}
\newtheorem*{problem}{Задача}
\newtheorem{claim}{Утверждение}
\newtheorem{exercise}{Упражнение}
\newtheorem{definition}{Определение}
\newtheorem{example}{Пример}

\theoremstyle{remark}
\newtheorem*{remark}{Замечание}

\renewcommand{\le}{\leqslant}
\renewcommand{\ge}{\geqslant}
\newcommand{\eps}{\varepsilon}
\renewcommand{\phi}{\varphi}
\newcommand{\ndiv}{\centernot\mid}

\MakePerPage{footnote}
\renewcommand*{\thefootnote}{\fnsymbol{footnote}}

\newcommand{\resetcntrs}{\setcounter{theorem}{0}\setcounter{definition}{0}
\setcounter{claim}{0}\setcounter{exercise}{0}}

\DeclareMathOperator{\aut}{aut}
\DeclareMathOperator{\cov}{cov}
\DeclareMathOperator{\chos}{ch}
\DeclareMathOperator{\argmin}{argmin}
\DeclareMathOperator{\argmax}{argmax}
\DeclareMathOperator*\lowlim{\underline{lim}}
\DeclareMathOperator*\uplim{\overline{lim}}
\DeclareMathOperator{\re}{Re}
\DeclareMathOperator{\im}{Im}

\frenchspacing


\begin{document}

\section*{Лекция 3. Вершинные экспандеры}
\addcontentsline{toc}{section}{Лекция 3. Вершинные экспандеры}
\resetcntrs

\section{Экспандеры и~их спектральные свойства}

Вершинный экспандер~--- двудольный граф, где любое не слишком большое
подмножество левой доли ($\le \frac{n}{3}$) хорошо расширяется (хотя бы
в~константу раз).

\begin{claim}
	Вершинный экспандер существует.
\end{claim}

По $D$-регулярному графу построим матрицу случайного блуждания
$M = \frac{A}{D}$, где $A$~--- матрица смежности.

\begin{itemize}
	\item $u = (\frac{1}{N}, \ldots, \frac{1}{N})$~--- собственый с~$\lambda = 1$.
	\item Все собственные значения $\le 1$ по модулю.
	\item Граф несвязен $\Leftrightarrow \lambda = 1$ имеет кратность $> 1$.
		В~одну сторону очевидно, в~другую нужно рассмотреть любой СВ, не
		пропорциональный~$(1, \ldots, 1)$ и~взять максимальную компоненту
		и~минимальную~--- это и~есть две компоненты связности.
	\item Пусть граф связен, тогда $\lambda = -1$~--- СЗ $\Leftrightarrow$ граф
		двудольный. В~одну сторону очевидно, в~другую нужно показать, что у~СВ с~СЗ
		$\lambda = -1$ максимальная компонента равна минус минимальной, далее
		аналогично предыдущему.
\end{itemize}

\begin{definition}
	$\lambda(G) = \max\limits_\pi \frac{\left| \pi M - u\right|}{\left| \pi - u
	\right|} = \max\limits_{x \bot u} \frac{|xM|}{|x|}$.
\end{definition}

\begin{claim}
	$\lambda(G)$~--- модуль второго СЗ матрицы~$M$.
\end{claim}
\begin{proof}
	$w = \alpha_2 v^2 + \ldots \alpha_n v^n \rightarrow wM = \alpha_2 \lambda_2
	v^2 + \ldots + \alpha_n \lambda_n v^n$.

	$|wM|^2 = \alpha_2^2 \lambda_2^2 + \ldots + \alpha_n^2 \lambda_n^2 \le
	\lambda_2^2 (\alpha_2^2 + \ldots + \alpha_n^2) = \lambda_2^2 |w|^2$.
\end{proof}

$|\pi M^t - u| \le \alpha(G)^t |\pi - u| \le \lambda(G)^t$, то есть
$\lambda(G)$~--- задает скорость сходимости распределения к~равномерному.

Утверждается, что если граф связный и~не двудольный, то $\lambda(G) <
1 - \frac{1}{N \cdot D \cdot \text{diam}(G)}$.

\begin{theorem}
	Если $\lambda(G) \le \lambda \Rightarrow \forall \alpha \rightarrow G$~---
	$(\alpha N, \frac{1}{\alpha + (1 - \alpha) \lambda^2})$-экспандер
\end{theorem}
\begin{proof}
	$CP(\pi) = |\pi|^2$~--- вероятность коллизии. $CP(\pi) = |\pi - u|^2 +
	\frac{1}{N}$. $CP(\pi) \ge \frac{1}{|\text{Supp} \pi|}$ по КБШ.

	$CP(\pi M) - \frac{1}{N} = |\pi M - u|^2 \le \lambda(G) |pi - u|^2 \le
	\lambda^2 (CP(\pi) - \frac{1}{N})$. Если $\pi$~равномерное на $S$, то $CP(\pi)
	= \frac{1}{|S|}$, а~$CP(\pi M) \ge \frac{1}{\text{Supp} \pi M} =
	\frac{1}{|N(s)|}$.

	Итого, $\frac{1}{|N(S)|} - \frac{1}{N} \le \lambda^2 (\frac{1}{|S|} -
	\frac{1}{N})$, подставляя $|S| \le \alpha N, \frac{1}{N} \le
	\frac{\alpha}{|S|}$, получаем требуемое.
\end{proof}

Спектральный разрыв: $\gamma(G) = 1 - \lambda(G)$.

Известно, что если граф $D$-регулярный и~является $\left(\frac{N}{2}, 1 +
\delta\right)$-экспандер, то $\gamma(G) =
\Omega\left(\frac{\delta}{D}^2\right)$.

\end{document}
