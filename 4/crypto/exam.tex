\documentclass{article}
\usepackage[utf8x]{inputenc}
\usepackage[english,russian]{babel}
\usepackage{amsmath,amscd}
\usepackage{amsthm}
\usepackage{amsfonts}
\usepackage{amssymb}
\usepackage{cmap}
\usepackage{centernot}
\usepackage{enumitem}
\usepackage{perpage}
\usepackage{chngcntr}
%\usepackage{minted}
\usepackage[bookmarks=true,pdfborder={0 0 0 }]{hyperref}
\usepackage{indentfirst}
\hypersetup{
  colorlinks,
  citecolor=black,
  filecolor=black,
  linkcolor=black,
  urlcolor=black
}

\newtheorem*{conclusion}{Вывод}
\newtheorem{theorem}{Теорема}
\newtheorem{lemma}{Лемма}
\newtheorem*{corollary}{Следствие}

\theoremstyle{definition}
\newtheorem*{problem}{Задача}
\newtheorem{claim}{Утверждение}
\newtheorem{exercise}{Упражнение}
\newtheorem{definition}{Определение}
\newtheorem{example}{Пример}

\theoremstyle{remark}
\newtheorem*{remark}{Замечание}

\renewcommand{\le}{\leqslant}
\renewcommand{\ge}{\geqslant}
\newcommand{\eps}{\varepsilon}
\renewcommand{\phi}{\varphi}
\newcommand{\ndiv}{\centernot\mid}

\MakePerPage{footnote}
\renewcommand*{\thefootnote}{\fnsymbol{footnote}}

\newcommand{\resetcntrs}{\setcounter{theorem}{0}\setcounter{definition}{0}
\setcounter{claim}{0}\setcounter{exercise}{0}}

\DeclareMathOperator{\aut}{aut}
\DeclareMathOperator{\cov}{cov}
\DeclareMathOperator{\chos}{ch}
\DeclareMathOperator{\argmin}{argmin}
\DeclareMathOperator{\argmax}{argmax}
\DeclareMathOperator*\lowlim{\underline{lim}}
\DeclareMathOperator*\uplim{\overline{lim}}
\DeclareMathOperator{\re}{Re}
\DeclareMathOperator{\im}{Im}

\frenchspacing


\begin{document}

\section*{Задача 1.}
а) Пусть $f(x)$~--- односторонняя и~вычислятеся за $O(n^k)$, тогда построим
функцию $g(x,y) = \left<f(x), y\right>$ для $y$ длины $|x|^k$.
Тогда её вычисление занимает квадратичное от размера входа время так как нужно
просто скоприровать число $x$ в~ответ, перенести число $x$ по ленте и~запустить
алгоритм вычисления $f(x)$. При этом она также будет односторонней, так
как, очевидно, с~помощью обратителя для $g$ легко построить обратитель для $f$.

b) Пользуясь пунктом а) покажем, что функция $f(M,x) = TM(M, x, n^3)$, которая
запускает машину $M$ на входе $x$ на $n^3$ шагов (и~возвращает результат или
$\bot$) односторонняя. Пусть вообще существует какая-то односторонняя
функция~$g(x)$, будем считать, что она вычисляется за $O(n^2)$ по пункту а).
Пусть $f$ не односторонняя, тогда есть машина $M_q$, и~многочлен $p$, такие что
$$P_{(M,x)\sim \{0,1\}^n}(M_q(f(M, x)) \in f^{-1}(f(M,x))) > \frac{1}{p(n)}$$
Потребуем, чтобы описание машины $M$ у~функции $f$ занимало первые $log|x|$ бит,
тогда с~вероятностью хотя бы $\frac{1}{n}$ в~качестве машины $M$ выберется
машина, вычисляющая $g$ (для достаточно большого $n$). Это значит, что найдется
обратитель для $g$, преуспевающий с~вероятностью не менее $\frac{1}{np(n)}$, что
противоречит тому, что она односторонняя.

\section*{Задача 2.}
а) Если $f$~--- односторонняя перестановка, а~функция $g = f^{n^c}$ обращается
за полином, то можно обратить $f$, просто вычислив $g^{-1} \circ f^{n^c - 1}$.
Вычисление работает за полиномиальное время и~вернёт корректный ответ c~той же
вероятностью, что и~обратитель $g$ в~силу того, что все функции биективны.

b) Пусть $f$~--- односторонняя, положим также, что $f: \{0,1\}^{n} \rightarrow
\{0,1\}^{n}$ сохраняет длину. Построим одностороннюю функцию $g$, такую, что
$g^2(x) = const$. Положим $g(xy) = 0^{n} f(x)$, где $\left|x\right| =
\left|y\right| = n$.  Эта функция односторонняя, так как её обратителем можно
обращать и~функцию $f$. Однако $g(g(xy)) = g(0^nf(x))= 0^n f(0^n) = const$, то
есть функция тривиально обратима.

\section*{Задача 4.}
a) Если длина строки есть $2n$, то количество строк, на которых генератор
возвращает $0$ есть хотя бы ${2n \choose n} \sim const \cdot
\frac{4^n}{\sqrt{n}}$. Тогда алгоритм, возвращающий $1$ на строке и~всех нулей
и~$0$ иначе различает ГСЧ и~случайную величину с~хорошей вероятностью.

b) Оценим долю строк, на которых выводы $G(s)$ и~$G''(s)$ отличаются.
$$\frac{{3n \choose n}}{2^{3n}} = \frac{(3n)!}{8^n n! (2n)!} \sim const \cdot
\frac{(3n)^{3n}}{8^n n^n (2n)^{2n}} = const \cdot
\left(\frac{27}{32}\right)^n.$$
Это значит, что вероятность любого алгоритма отличить выходы алгоритмов
на~случайной стороке пренебрежимо малы. Стало быть, $G''(s)$ вычислительно
неотличим от $G(s)$ и~является ГСЧ.

\section*{Задача 7.}

Изложим сразу алгоритм для $n$ студентов. Предположим, что у~них есть как общий
чат, так и~личные сообщения друг для друга. Тогда можно добиться линейного
количества пересланных бит. Обозначим за $c_i$ величину, отражающую, платил ли
студент $i$ за обед или нет, тогда им необходимо вычислить $c_1 \oplus \ldots
\oplus c_n$. Алгоритм такой:
\begin{itemize}
	\item Студенты $i$ и~$(i+1) \bmod n$ генерируют случайный бит $a_i$.
	\item В~общий чат студентом $i$ объявляется число $b_i = a_{(i-1) \bmod n}
		\oplus a_{i} \oplus c_i$.
	\item Сумма $\bigoplus\limits_{i=1}^n b_i$ равна $\bigoplus\limits_{i=1}^n
		c_i$, так как каждое $a_i$ встречается по 2 раза.
	\item Однако точно узнать, кто платил никто не может, так как если студент $j$
		говорит, что якобы платил студент $i$, то всегда возможен случай, что платил
		на самом деле $i+1$ или $i-1$, а~соответствующие случайные биты
		$a_i,a_{i-1}$ были другими (пользуемся тем, что студент $j$ не может быть
		одновременно левым и~правым соседом $i$ и~знать обе этих величины), так что
		значения $d_i$ были такие же.
\end{itemize}

\section*{Задача 8.}

a) Рассмотрим два случая:
\begin{itemize}
	\item Генерал честный, один из полковников предатель, другой нет. Полковник
		знает, какой приказ дал генерал, однако, второй говорит ему прямо
		противоположное.
	\item Генерал нечестный, оба полковника честные. Генерал выслал противоречивые
		приказы.
\end{itemize}

Ясно, что эти две ситуации неразличимы с~точки зрения честного полковника,
а~действовать он в~них должен по-разному. Значит искомого протокола нет.

b) Протокол такой: генерал высылает приказы, полковники пересылают их друг другу
и~выполняют тот приказ, которого больше.

Исполнительность: если генерал честный, все получили одинковые приказы, значит
у~всех честных полковников есть правильный приказ хотя бы в~двух экземплярах,
значит все честные выполнят приказ.

Согласованность: если генерал нечестный, то все командиры честные, они переслали
друг другу какие-то данные и~одинаковым образом выбрали то, что надо сделать.

\section*{Задача 9.}

В~общем случае действуем так:
\begin{itemize}
	\item Все получают приказ и~рассылают его всем остальным командирам. У~каждого
		получается $3m$ каких-то приказов. Будем рассматривать честных командиров.
		Каждый из них попадает одну из трёх категорий: те, что получили хотя бы $2m$
		приказов атаки, те, что получили хотя	бы $2m$ приказов отступления
		и~остальные. Заметим, что эти остальные точно знают, что генерал предатель,
		так как иначе бы они получили хотя бы $2m$ честных приказов от остальных.
	\item Не бывает двух честных командиров в~разной категории. Если генерал
		честный, то все честные командиры точно будут в~той категории, которая
		соответствует приказу генерала. Если же генерал нечестный, то честных
		полковников $2m+1$, значит, как бы ни были отданы им приказы, каких-то будет
		хотя бы $m+1$ (положим, приказов на атаку), значит либо все честные
		полковники будут в~наступательной категории, либо в~третей ($m+1$ приказ не
		даст другой категории взять большинство в~$\frac{2}{3}$).
	\item Каждый озвучивает каждому свою категорию. Если хотя бы $m+1$ человек
		считают, что генерал предатель, то из них был кто-то честный, значит
		генерал в~самом деле предатель. Аналогично, если сигнала какого-то действия
		было не меньше, чем $m+1$, то этот командир может быть уверен, что у~всех
		честных командиров есть либо информация об этом действии, либо информация
		о~том, что генерал нечестный.
	\item Путь честных командиров, которые считают, что генерал предатель, больше
		или равно $m+1$. Тогда все честные командиры это узнают. В~противном случае
		Есть хотя бы $m+1$ честный командир с~одним и~тем же приказом, тогда все
		узнают этот приказ. Осталось всем честным командиром обменяться своими
		знаниями и~различить эти две ситуации.
	\item Все пересылают всем своё мнение, честный ли генерал или нет. Если хотя
		бы $2m$ человек считают, что да, то полковник принимает одно фиксированное
		решение, например, отступать.
		В~противном случае можно быть уверенным, что все честные командиры в~одной
		категории, значит они сделают одно и~то же действие, которое, если генерал
		честный, будет его приказом.
\end{itemize}

\section*{Задача 10.}

Задача является одной из вариаций известной задачи
\href{https://en.wikipedia.org/wiki/Mental_poker}{Mental poker}. Классическое
решение состоит в~использовании коммутирующего шифрования:

\begin{itemize}
	\item Нам нужны функции $E(K, X), D(K, X)$.
	\item $D_K(E_K(X)) = X$ для всех возможных сообщений $X$ и~ключей $K$.
	\item $E_K(E_J(X)) = E_J(E_K(X))$ для всех возможных сообщений $X$ и~пар
		ключей $K, J$.
	\item $X \mapsto E(K,X)$ оносторонняя.
	\item Ключи неподменяемы, то есть по сообщениям $X, Y$ нельзя полиномиально
		быстро найти ключи $K, J$ такие что $E_K(X) = E_J(Y)$.
		такие
\end{itemize}

Существование такой криптосистемы (вроде) не следует из~существования
односторонней функции, но на практике такие системы использются. Используя это,
можем реализовать протокол раздачи карт так:

\begin{itemize}
	\item Первый игрок перемешивает карты $A_1, \ldots, A_4$ случайно. Он выбирает
		ключ $K$ и~посылает $E(K, A_1), \ldots, E(K, A_4)$ второму.
	\item Второй игрок выбирает из четырёх зашифрованных карт две, выбирает ключ
		$J$ и~пересылает $E(J, E(K, A_p)), E(K, A_q)$.
	\item Первый игрок забирает себе одну карту $A_q$, которую он узнал по $E(K,
		A_q)$ с~помощью расшифровки. Он также расшифровывает вторую карту, получая
		$E(J, A_p)$ и~отправляет это второму.
	\item Второй игрок расшифровывает свою карту $A_q$.
	\item Первый игрок гарантирует случайность выбора своей карты, так как он
		тасовал изначальную колоду.
	\item Второй игрок гарантирует случайность своей карты, так как он сам её
		случайно выбрал.
	\item После сыгранной игры в~покер они могут проверить, что жульничества не
		было, раскрыв оба ключа и~проверив все пересылки.
\end{itemize}

\end{document}
