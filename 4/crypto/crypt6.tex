\documentclass{article}
\usepackage[utf8x]{inputenc}
\usepackage[english,russian]{babel}
\usepackage{amsmath,amscd}
\usepackage{amsthm}
\usepackage{amsfonts}
\usepackage{amssymb}
\usepackage{cmap}
\usepackage{centernot}
\usepackage{enumitem}
\usepackage{perpage}
\usepackage{chngcntr}
%\usepackage{minted}
\usepackage[bookmarks=true,pdfborder={0 0 0 }]{hyperref}
\usepackage{indentfirst}
\hypersetup{
  colorlinks,
  citecolor=black,
  filecolor=black,
  linkcolor=black,
  urlcolor=black
}

\newtheorem*{conclusion}{Вывод}
\newtheorem{theorem}{Теорема}
\newtheorem{lemma}{Лемма}
\newtheorem*{corollary}{Следствие}

\theoremstyle{definition}
\newtheorem*{problem}{Задача}
\newtheorem{claim}{Утверждение}
\newtheorem{exercise}{Упражнение}
\newtheorem{definition}{Определение}
\newtheorem{example}{Пример}

\theoremstyle{remark}
\newtheorem*{remark}{Замечание}

\renewcommand{\le}{\leqslant}
\renewcommand{\ge}{\geqslant}
\newcommand{\eps}{\varepsilon}
\renewcommand{\phi}{\varphi}
\newcommand{\ndiv}{\centernot\mid}

\MakePerPage{footnote}
\renewcommand*{\thefootnote}{\fnsymbol{footnote}}

\newcommand{\resetcntrs}{\setcounter{theorem}{0}\setcounter{definition}{0}
\setcounter{claim}{0}\setcounter{exercise}{0}}

\DeclareMathOperator{\aut}{aut}
\DeclareMathOperator{\cov}{cov}
\DeclareMathOperator{\chos}{ch}
\DeclareMathOperator{\argmin}{argmin}
\DeclareMathOperator{\argmax}{argmax}
\DeclareMathOperator*\lowlim{\underline{lim}}
\DeclareMathOperator*\uplim{\overline{lim}}
\DeclareMathOperator{\re}{Re}
\DeclareMathOperator{\im}{Im}

\frenchspacing


\begin{document}

\section*{Лекция 6. Интерактивный протокол бросания монетки}
\addcontentsline{toc}{section}{Лекция 6. Интерактивный протокол бросания
монетки}
\resetcntrs

\section{Общая схема задачи}

Алиса и~Боб ненавидят друг друга. Неоходимо получить общий случайный бит
в~условиях полного недоверия. $A$ и~$B$ представляют собой два рандомизированных
полиномиальных алгоритма с~независимыми случайными битами. Они общаются
протоколу и~после завершения выдают по одному биту~$\sigma, \tau$.

Нужно, чтобы оказалось так, чтобы~$\sigma = \tau, P(\sigma = 0) \approx
\frac{1}{2}$. Для этого есть полиномиальный алгоритм~$J$, который получает
протокол и~возвращает~$A, B, \bot_A, \bot_B$~--- либо сторону-победителя, либо
сторону, которая первая нарушила протокол.

Требуемые свойства:
\begin{itemize}
	\item Корректность: если обе стороны используют предписанные алгоритмы, то
		$P(J = A) = P(J = B) = \frac{1}{2}$.
	\item Интересы Алисы: $\forall B^\ast \rightarrow P(J \in \{0, \bot_B\})
		\approx \frac{1}{2}$.
	\item Интересы Боба: $\forall A^\ast \rightarrow P(J \in \{1, \bot_A\})
		\approx \frac{1}{2}$.
\end{itemize}

Если бы можно было обмениваться сообщениями одновременно, то можно было бы
каждому послать случайный бит и~в~качестве результата взять их $\oplus$. Однако,
одновременных сообщений не предусмотрено, поэтому используется привязка к~биту
(bit commitment). Такие протоколы неформально <<запечатывают>> бит в~конверт
так, чтобы его уже нельзя было подменить, но и~нельзя посмотреть, не вскрыв
конверт.

\section{Неинтерактивный протокол привязки к~биту}

Неинтерактивная версия протокола подразумевает следующее: $\sigma$~---
запечатаныый бит, $r$~--- случайные биты, $c(\sigma, r)$~--- привязка,
$k(\sigma, r)$~--- ключ, $d(c, k) \in \{0, 1, \bot\}$~--- процедура вскрытия
(все алгоритмы полиномиальны и~детерменированы). Условия на протокол следующие:

\begin{itemize}
	\item Корректность: $d(c(\sigma, r), k(\sigma, r)) = \sigma$.
	\item Неразглашение: $c(0, r) \sim c(1, r)$.
	\item Неподменяемость: $\nexists c, k_0, k_1: d(c, k_0) = 0, d(c, k_1) = 1$.
\end{itemize}

\begin{remark}
	В~условии неразглашения требовать статистическую неотличимость не получится,
	так как третье условие говорит о~том, что привязка расшировывается однозначно
	либо в~1, либо в~0, то есть $c(0, r)$ и~$c(1, r)$ вообще распределены на
	разных множествах. Поэтому требуется вычислительная неотличимость, добиться
	которой можно.
\end{remark}

Функция привязки строится на базе односторонней перестановки~$f$, из которой по
теореме Левина-Голдрайха получается односторонняя перестановка~$g$ с~трудным
битом~$h$. А~именно~$c(\sigma, r) = (\sigma \oplus h(r), g(r))$.

Можно считать, что $k(\sigma, r) = \sigma r$, так как распаковщик сам может
провести все нужные вычисления. Каноническая процедура вскрытия может просто
вычислять $c(0, r), c(1, r)$, сравнивать его с~$c$ и~возвращать $0, 1$ или
$\bot$ в~зависимости от результата. То есть, формально $k(\sigma, r) = (\sigma,
r)$, $d((\tau, s); (\sigma, r)) = \tau \oplus h(r)$ если $g(r) = s$ и~$\sigma =
h(r)$, иначе $\bot$. Корректность такого алгоритма очевидна.

Трудность бита $h$ означает, что $(g(r), h(r)) \sim (g(r), \gamma)$, где
$\gamma$~--- случайный бит. Отсюда $(g(r), 1 \oplus h(r)) \sim (g(r),
1 \oplus \gamma) \sim (g(r), \gamma)$, то есть $(g(r), h(r)) \sim (g(r),
1 \oplus h(r))$, откуда следует неразглашение.

В~свою очередь, так как~$g$~--- инъекция, то $\forall s \exists \le 1 r: g(r) =
s$, то есть подменить ключ в~самом деле не получится.

\begin{remark}
	Практически, одностороннюю перестановку удобно иметь не на $\{0, 1\}^\ast$,
	а~на любой области~$D$. В~таком случае хочется сказать, что нужно уметь
	проверять условие $r \in D$, однако это не практично (например, если брать
	одностороннюю перестановку на базе RSA, мы не можем проверить, является ли это
	число произведением больших простых чисел). Поэтому уточнение такое: нужно
	уметь генерировать случайную величину, которая либо равномерно распределена на
	$D$, либо принимает фиктивное значение (пустую строку, к~примеру). В~таком
	случае, мы можем пытаться открывать конверт, пока он не откроется, то есть
	либо в~среднем за полиномиальное время, либо за гарантированный полином
	с~маленькой вероятностью ошибки.
\end{remark}

\section{Интерактивный протокол привязки к~биту}

Такой протокол действует в~две стадии, условно изображённые на схеме:
$$\begin{CD}
	\text{1 стадия: } @. \sigma @>>> S     @=    T\\
   @.              @.           @VVV         @VVV\\
	\text{2 стадия: } @.        @.   k @>>>  R(k, c) @>>> \sigma
\end{CD}$$
Где $c = c_{T,S(\sigma)}$~--- протокол общения между~$S, T$, а~алгоритм
верификации~$R$ проверяет его и~возвращает что-то $\{0, 1, \bot_S, \bot_T\}$.
Условия на протокол похожие с~поправкой на то, что жульничать могут обе стороны:

\begin{itemize}
	\item Корректность: $R(k_{T,S(\sigma)}, c_{T,S(\sigma)}) = \sigma$.
	\item Неразглашение: $\forall T^\ast \rightarrow c_{T^\ast, S(0)} \sim
		c_{T^\ast, S(1)}$.
	\item Неподменяемость: $\forall S^\ast$ с~пренебрежимо малой вероятностью
		могут произойти события:
		\begin{itemize}
			\item $\exists k_0, k_1 R(k_0, c_{T,S^\ast}) = 0,
				R(k_1, c_{T,S^\ast})= 1$.
			\item $\exists k_1: R(k_1, c_{T,S^\ast}) = \bot_T$.
		\end{itemize}
\end{itemize}

Конечно, неинтерактивный протокол можно без труда переделать в~интерактивный.
Однако, для существования интерактивного протокола будет достаточно меньшего
предположения, а~именно существования генератора~$G: \{0, 1\}^n \rightarrow
\{0, 1\}^{3n}$. Протокол устроен так:
\begin{itemize}
	\item $T$ посылается случайный вектор~$t \in \{0, 1\}^{3n}$.
	\item $S$ имеет собственный случайный вектор~$s \in \{0, 1\}^n$ и~посылает
		$m = G(s) \oplus (t \cdot \sigma)$.
	\item Верификатор~$R(t, m, s)$ выдает:
		\begin{itemize}
			\item $\bot_T$, если $|t| \ne 3n$.
			\item $\bot_S$, если $|m| \ne 3n$ или $|s| \ne n$ или $m \oplus G(s)
				\notin \{t, 0^{3m}\}$.
			\item $0$, если $m \oplus G(s) = 0^{3n}$.
			\item $1$, иначе.
		\end{itemize}
\end{itemize}

Корректность как всегда ясна (по модулю случая, когда $t = 0^{3n}$, его нужно
либо запретить с~небольшим перекосом в~распределении, либо допустить малую
вероятность некорректности).

$G(s)$ неотличимы от равномерных, значит какими бы ни были $t$, $G(s) \oplus t$
тоже неотличимы от равномерных, что даёт неразглашение.

С~неподменяемостью нужно разобрать некоторые случаи. Вероятность $\bot_T$ просто
равна~$0$. Остается только первое условие: в~нём множество плохих $t = G(s_0)
\oplus G(s_1)$ имеет размер не больше $2^{2n}$, значит вероятность выбрать
такое~$t$ экспоненциально мала, даже для любой константы~$2 + \eps$ вместо 3.

\section{Протокол бросания монетки}

Протокол выглядит так:
\begin{itemize}
	\item Алиса запукает протокол привязки к~своему случайному биту $\sigma$.
		После этого у~нее остается ключ~$k$, а~у~Боба появляется привязка~$c$.
	\item Боб отправляет случайный бит~$\tau$ прямым текстом.
	\item Алиса отправляет ключ к~привязке~$k$.
	\item Судья проверяет корректность всех действий и~выдаёт $\tau \oplus
		\sigma$.
\end{itemize}

Протокол очевидно корректен. Если $B^\ast$ отклоняется от протокола, то он может
использовать $T^\ast$ и~взять~$\tau$ в~зависимости от~$c$. Однако, так как
привязка к~0 и~привязка к~1 вычислительно неотличимы, то он не может существенно
повлиять на конечную вероятность, иначе он бы был отличителем привязок. Таким
образом, интересы Алисы следуют из неразглашения.

Интересы Боба же следуют из неподменяемости: так как Боб посылает реально
случайный бит~$\tau$, она не может жульничать с~самим битом~$\sigma$,
единственное, что она может попытаться сделать~--- это привязаться одновременно
к~0 и~1, что невозможно в~силу неподменяемости.

\end{document}
