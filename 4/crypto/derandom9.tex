\documentclass{article}
\usepackage[utf8x]{inputenc}
\usepackage[english,russian]{babel}
\usepackage{amsmath,amscd}
\usepackage{amsthm}
\usepackage{amsfonts}
\usepackage{amssymb}
\usepackage{cmap}
\usepackage{centernot}
\usepackage{enumitem}
\usepackage{perpage}
\usepackage{chngcntr}
%\usepackage{minted}
\usepackage[bookmarks=true,pdfborder={0 0 0 }]{hyperref}
\usepackage{indentfirst}
\hypersetup{
  colorlinks,
  citecolor=black,
  filecolor=black,
  linkcolor=black,
  urlcolor=black
}

\newtheorem*{conclusion}{Вывод}
\newtheorem{theorem}{Теорема}
\newtheorem{lemma}{Лемма}
\newtheorem*{corollary}{Следствие}

\theoremstyle{definition}
\newtheorem*{problem}{Задача}
\newtheorem{claim}{Утверждение}
\newtheorem{exercise}{Упражнение}
\newtheorem{definition}{Определение}
\newtheorem{example}{Пример}

\theoremstyle{remark}
\newtheorem*{remark}{Замечание}

\renewcommand{\le}{\leqslant}
\renewcommand{\ge}{\geqslant}
\newcommand{\eps}{\varepsilon}
\renewcommand{\phi}{\varphi}
\newcommand{\ndiv}{\centernot\mid}

\MakePerPage{footnote}
\renewcommand*{\thefootnote}{\fnsymbol{footnote}}

\newcommand{\resetcntrs}{\setcounter{theorem}{0}\setcounter{definition}{0}
\setcounter{claim}{0}\setcounter{exercise}{0}}

\DeclareMathOperator{\aut}{aut}
\DeclareMathOperator{\cov}{cov}
\DeclareMathOperator{\chos}{ch}
\DeclareMathOperator{\argmin}{argmin}
\DeclareMathOperator{\argmax}{argmax}
\DeclareMathOperator*\lowlim{\underline{lim}}
\DeclareMathOperator*\uplim{\overline{lim}}
\DeclareMathOperator{\re}{Re}
\DeclareMathOperator{\im}{Im}

\frenchspacing


\begin{document}

\section*{Лекция 9. Экстракторы I}
\addcontentsline{toc}{section}{Лекция 9. Экстракторы I}
\resetcntrs

\section{Общая идея}

Есть ряд процессов, результат которых довольно случайный, но никаких
гарантий, как можно использовать такую случайность, нет. Так появляется задача
получения независимых случайных бит из не совсем случайной последовательности.

\begin{example}
	Если есть независимые одинаково распределенные случайные биты, но вероятность
	единицы не равна $1\over2$, то можно получать случайные биты так двукратным
	бросанием: если выпало 01, то вернем 0, если 10, то 1, в~противном случае
	бросим еще раз.

	Если биты независимые, но вероятности разные на отрезке $[\delta; 1 -
	\delta]$, то $P(b_1 \oplus \ldots \oplus b_m = 1) \rightarrow {1\over2}$, так
	как $p(1 - \delta_m) + (1 - p)\delta_m$ это выпуклая комбинация.
\end{example}

Более общий класс источников это $k$-слабые источники случайности.
\emph{Мин-энтропия} это $H_\infty(B) = -\log_2\left( \max\limits_x P(\vec b = x)
\right)$.

$H_\infty(\vec b) \ge k \Leftrightarrow \forall x P(\vec b = x) \le
\frac{1}{2^k}$. В~этом случае $\vec b$ содержит хотя бы $k$ случайных битов.

\emph{Плоские распределения} на $K \subset \{0, 1\}^n$, $|K| = 2^k$~--- это
равномерные распределения.

\begin{theorem}
	Если $H_\infty(\vec b) \ge k$, то $\vec b$~--- выпуклая комбинация плоских
	источников.
\end{theorem}

\begin{definition}
	\emph{Seeded-экстрактор} $Ext: \{0, 1\}^n \times \{0, 1\}^d \rightarrow
	\{0, 1\}^m$ с~параметрами $(k, \eps)$ обладает свойством, что:
	$\forall\xi, H_\infty(\xi)
	\ge k \Rightarrow Ext(\xi, U_d)\ \eps$-близка к~$U_m$.
\end{definition}

Вероятностно можно показать существование seeded-экстрактора с~параметрами
$m = k + d - 2\log \frac{1}{\eps} - O(1)$ и~$d = \log(n - k) +
2\log\frac{1}{\eps} + O(1)$.

\begin{definition}
	\emph{Multisource-экстрактор} $MExt: \{0, 1\}^n \times \{0, 1\}^n \rightarrow
	\{0, 1\}^m$ с~параметрами $(k, \eps)$ обладает свойством, что:
	$\forall\xi, \eta: H_\infty(\xi), H_\infty(\eta) \ge k, \xi \bot \eta
	\Rightarrow Ext(\xi, \eta)\ \eps$-близка к~$U_m$.
\end{definition}

\end{document}
