\documentclass{article}
\usepackage[utf8x]{inputenc}
\usepackage[english,russian]{babel}
\usepackage{amsmath,amscd}
\usepackage{amsthm}
\usepackage{mathtools}
\usepackage{amsfonts}
\usepackage{amssymb}
\usepackage{cmap}
\usepackage{centernot}
\usepackage{enumitem}
\usepackage{perpage}
\usepackage{chngcntr}
%\usepackage{minted}
\usepackage[bookmarks=true,pdfborder={0 0 0 }]{hyperref}
\usepackage{indentfirst}
\hypersetup{
  colorlinks,
  citecolor=black,
  filecolor=black,
  linkcolor=black,
  urlcolor=black
}

\newtheorem*{conclusion}{Вывод}
\newtheorem{theorem}{Теорема}
\newtheorem{lemma}{Лемма}
\newtheorem*{corollary}{Следствие}

\theoremstyle{definition}
\newtheorem*{problem}{Задача}
\newtheorem{claim}{Утверждение}
\newtheorem{exercise}{Упражнение}
\newtheorem{definition}{Определение}
\newtheorem{example}{Пример}

\theoremstyle{remark}
\newtheorem*{remark}{Замечание}

\newcommand{\doublearrow}{\twoheadrightarrow}
\renewcommand{\le}{\leqslant}
\renewcommand{\ge}{\geqslant}
\newcommand{\eps}{\varepsilon}
\renewcommand{\phi}{\varphi}
\newcommand{\ndiv}{\centernot\mid}

\MakePerPage{footnote}
\renewcommand*{\thefootnote}{\fnsymbol{footnote}}

\newcommand{\resetcntrs}{\setcounter{theorem}{0}\setcounter{definition}{0}
\setcounter{claim}{0}\setcounter{exercise}{0}}

\DeclareMathOperator{\aut}{aut}
\DeclareMathOperator{\cov}{cov}
\DeclareMathOperator{\argmin}{argmin}
\DeclareMathOperator{\argmax}{argmax}
\DeclareMathOperator*\lowlim{\underline{lim}}
\DeclareMathOperator*\uplim{\overline{lim}}
\DeclareMathOperator{\re}{Re}
\DeclareMathOperator{\im}{Im}

\frenchspacing


\begin{document}

\section*{Лекция 2. Практические методы дерандомизации}
\addcontentsline{toc}{section}{Лекция 2. Практические методы дерандомизации}
\resetcntrs

\section{Задача \text{MAXCUT}}

\text{MAXCUT}: разбить вершины графа на 2 множества $S$, $T$, так чтобы между
ними было как можно больше ребер.

Если выбрать $S$ случайно, то ожидаемый размер разреза $\frac{1}{2}|E|$, то есть
легко можно посторить $\frac{1}{2}$-оптимальное приближение. Вопрос в~том, как
найти его, не используя случайность.

1й-способ: метод условных матожиданий: первую вершину кладем куда угодно, для
каждой следующей рассматриваем 2 ситуации: поместить её в~левую долю или
в~правую. Делаем это, максимизируя условное матожидание. Получается обычный
жадный алгоритм~--- поместить вершину так, чтобы было как можно больше ребер
между долями.

2й-способ: использование попарной независимости. Используем случайные биты, не
независимые в~совокупности, а~независимые попарно. Суть в~том, что обеспечание
попарной независимости требует только логарифмического количества случайных бит.

Матрица кода Адамара: $A$ размером $(2^l-1)\times l$, по строкам все ненулевые
вектора из нулей и~единиц. Тогда $y = A \cdot x$, где $x$ вектор случайных
величин длины $l$, будет вектор из равномерно распределенных попарно независимых
случайных величин.

Таким образом, если перебрать все случайные биты, мы можем выбрать из них
оптимальный и~затратить на это полином времени.

\section{Задача о~максимальном дизайне}

\begin{definition}
	$S_1, \ldots, S_m \subset \{1, \ldots, d\}$ есть $(m, d, l, a)$-дизайн,
	если $|S_i| = l$, а~$\forall i \ne j \rightarrow |S_i \cap S_j| < a$.
\end{definition}

\begin{claim}
	Если $d, l, a$~--- фискированные, то для $m = \frac{C^a_d}{(C_l^a)^2}$
	существует дизайн с~такими параметрами.
\end{claim}
\begin{proof}
	Рассмотрим случайный дизайн.
	$E_{S_i}(\#\{ j < i, |S_j \cap S_i| \ge a \}) =
	(i - 1) P(|S_j \cap S_i| \ge a) <
	m \frac{C^a_l C^{l-a}_{d-a}}{C_d^l} < 1$.

	Тогда найдется значение, равное 0.
\end{proof}

Отсюда $\forall \gamma > 0, l, m \in \mathbb{N} \rightarrow \exists
(m, d, l, a)$-дизайн, $a = \gamma \log m, d = o(\frac{l^2}{a})$. То есть
в~полиномиальную кастрюлю можно напихать экспоненциально много сарделек
с~пересечением в~какую-то константную долю, скажем 10\%.

Полученный результат можно дерандомизировать с~помощью метода условных
матожиданий.

\end{document}
