\documentclass{article}
\usepackage[utf8x]{inputenc}
\usepackage[english,russian]{babel}
\usepackage{amsmath,amscd}
\usepackage{amsthm}
\usepackage{amsfonts}
\usepackage{amssymb}
\usepackage{cmap}
\usepackage{centernot}
\usepackage{enumitem}
\usepackage{perpage}
\usepackage{chngcntr}
%\usepackage{minted}
\usepackage[bookmarks=true,pdfborder={0 0 0 }]{hyperref}
\usepackage{indentfirst}
\hypersetup{
  colorlinks,
  citecolor=black,
  filecolor=black,
  linkcolor=black,
  urlcolor=black
}

\newtheorem*{conclusion}{Вывод}
\newtheorem{theorem}{Теорема}
\newtheorem{lemma}{Лемма}
\newtheorem*{corollary}{Следствие}

\theoremstyle{definition}
\newtheorem*{problem}{Задача}
\newtheorem{claim}{Утверждение}
\newtheorem{exercise}{Упражнение}
\newtheorem{definition}{Определение}
\newtheorem{example}{Пример}

\theoremstyle{remark}
\newtheorem*{remark}{Замечание}

\renewcommand{\le}{\leqslant}
\renewcommand{\ge}{\geqslant}
\newcommand{\eps}{\varepsilon}
\renewcommand{\phi}{\varphi}
\newcommand{\ndiv}{\centernot\mid}

\MakePerPage{footnote}
\renewcommand*{\thefootnote}{\fnsymbol{footnote}}

\newcommand{\resetcntrs}{\setcounter{theorem}{0}\setcounter{definition}{0}
\setcounter{claim}{0}\setcounter{exercise}{0}}

\DeclareMathOperator{\aut}{aut}
\DeclareMathOperator{\cov}{cov}
\DeclareMathOperator{\chos}{ch}
\DeclareMathOperator{\argmin}{argmin}
\DeclareMathOperator{\argmax}{argmax}
\DeclareMathOperator*\lowlim{\underline{lim}}
\DeclareMathOperator*\uplim{\overline{lim}}
\DeclareMathOperator{\re}{Re}
\DeclareMathOperator{\im}{Im}

\frenchspacing


\begin{document}

\section*{Лекция 10. Экстракторы II}
\addcontentsline{toc}{section}{Лекция 10. Экстракторы II}
\resetcntrs

\emph{Про каждый из~описанных объектов есть вероятностное доказательство
существовования, которое не приводится}

\section{Комбинаторная интерпретация}

Seeded-экстрактор можно представить как двудольный граф с~долями $\{0, 1\}^n$
и~$\{0, 1\}^m$ и~рёбрами проведенными естественным образом. Тогда условие на
экстактор запишется как
$$ \forall S: |S| = 2^k \rightarrow \left|\frac{|E(S,T)|}{|S|D} -
\frac{|T|}{2^m} \right|$$.

Multisource-экстрактор удобно рассматривать как таблицу, раскрашенную в~один из
$\{0, 1\}^m$ цветов. Тогда условие на экстрактор будет выглядеть так: для
любых достаточно больших наборов столбцов и~строк $S$, $T$ число клеток~$x$,
покрашенных в~цвета из множества $Q \subset \{0, 1\}^m$ удовлетворяет следующему
неравенству:

$$\left| \frac{x}{|S||T|} - \frac{|Q|}{2^m} \right| < \eps.$$

\section{Некоторые усиления и~родственные объекты}

Если, например, взять multisource-экстрактор, и~испортить в~нём распределение
битов в~первой строке, то общее распределение пострадает не сильно. Поэтому
можно рассматривать экстраткторы в~сильном смысле.

\begin{definition}
	Multisource-экстрактор называется \emph{экстрактором в~сильном смысле}, если
	условные распределения~$MEXt(x,y) \mid y$ и~$Mext(x, y) \mid x$ тоже
	$\eps$-близки к~равномерному.
\end{definition}

\begin{definition}
	Seeded-экстрактор называется \emph{экстрактором в~сильном смысле}, если
	$(y, Ext(x, y))$~--- $\eps$-близко к~равномерному.
\end{definition}

\begin{definition}
	Двудольный граф, в~котором можно пошагово на запрос вершины в~левой доле
	говорить соседа в~правой доле (так, чтобы набор рёбер оставался
	парасочетанием), называется \emph{графом, допускающим online-парасочетание}.
\end{definition}

\begin{definition}
	\emph{Дисперсер} это функция $Disp(x, y)$ такая, что для $\forall \xi,
	H_\infty(\xi) \ge k, \eta \sim U_{2^d}, \eta \bot \xi \rightarrow
	Disp\left(\{0, 1\}^n \times \{0, 1\}^d\right)$ занимает $\ge 1 - \eps$ от
	$\{0, 1\}^m$.
\end{definition}

\section{Конструкции экстракторов}

Пусть $H: \{0, 1\}^n \rightarrow \{0, 1\}^m$~--- семейство хеш-функций, тогда
организуем экстрактор следующим образом: $Ext(x, h) = h(x)$.

\begin{lemma}[Leftover hash lemma]
	Если $m = k - 2\log\frac{1}{\eps}$, то полученный объект~--- сильный $\left(k,
	\frac{\eps}{2}\right)$-экстрактор.
\end{lemma}
\begin{proof}
	Нужно доказать, что $(h, h(x)) \sim U_d \times U_m$. Обозначим $D = 2^d, M =
	2^m, N = 2^n$, тогда $M = K\eps^2$.

	Оценим вероятность коллизии: $P_{x,h,x',h'} \{ Ext(x, h) = Ext(x', h') \land h
	= h'\} = P_{x,h,x',h'} \{ h = h' \land (x = x' \lor (x \ne x' \land h(x) =
	h(x')))\} \le \frac{1}{DK} + \frac{1}{DM} = \frac{1}{DK}(1 + \eps^2)$.

	Теперь оценим $L_2$ расстояние от нашего распределения до равномерного:
	$\Vert (h, h(\xi)) - U_d \times U_m \Vert^2 = \sum\limits_{z,t}
	(P(h-z,h(\xi)=t) - \frac{1}{DM})^2 = P(\text{коллизии}) - \frac{2}{DM}
	\sum\limits_{z,t}P(h=z,h(\xi)=t) - \frac{1}{DM} \le \frac{1}{DK} +
	\frac{1}{DM} - \frac{1}{DM} = \frac{\eps^2}{DM}$.

	Тогда $|(h, h(\xi)) - U_d \times U_m|_1 \le \eps$, а~значит статистическое
	расстояние не больше~$\frac{\eps}{2}$.
\end{proof}

Это довольно плохой экстрактор, однако, лучшие построенные ограничиваются
$O(\log^2 n)$ дополнительными чисто-случайными битами.

\end{document}
