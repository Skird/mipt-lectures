\documentclass{article}
\usepackage[utf8x]{inputenc}
\usepackage[english,russian]{babel}
\usepackage{amsmath,amscd}
\usepackage{amsthm}
\usepackage{mathtools}
\usepackage{amsfonts}
\usepackage{amssymb}
\usepackage{cmap}
\usepackage{centernot}
\usepackage{enumitem}
\usepackage{perpage}
\usepackage{chngcntr}
%\usepackage{minted}
\usepackage[bookmarks=true,pdfborder={0 0 0 }]{hyperref}
\usepackage{indentfirst}
\hypersetup{
  colorlinks,
  citecolor=black,
  filecolor=black,
  linkcolor=black,
  urlcolor=black
}

\newtheorem*{conclusion}{Вывод}
\newtheorem{theorem}{Теорема}
\newtheorem{lemma}{Лемма}
\newtheorem*{corollary}{Следствие}

\theoremstyle{definition}
\newtheorem*{problem}{Задача}
\newtheorem{claim}{Утверждение}
\newtheorem{exercise}{Упражнение}
\newtheorem{definition}{Определение}
\newtheorem{example}{Пример}

\theoremstyle{remark}
\newtheorem*{remark}{Замечание}

\newcommand{\doublearrow}{\twoheadrightarrow}
\renewcommand{\le}{\leqslant}
\renewcommand{\ge}{\geqslant}
\newcommand{\eps}{\varepsilon}
\renewcommand{\phi}{\varphi}
\newcommand{\ndiv}{\centernot\mid}

\MakePerPage{footnote}
\renewcommand*{\thefootnote}{\fnsymbol{footnote}}

\newcommand{\resetcntrs}{\setcounter{theorem}{0}\setcounter{definition}{0}
\setcounter{claim}{0}\setcounter{exercise}{0}}

\DeclareMathOperator{\aut}{aut}
\DeclareMathOperator{\cov}{cov}
\DeclareMathOperator{\argmin}{argmin}
\DeclareMathOperator{\argmax}{argmax}
\DeclareMathOperator*\lowlim{\underline{lim}}
\DeclareMathOperator*\uplim{\overline{lim}}
\DeclareMathOperator{\re}{Re}
\DeclareMathOperator{\im}{Im}

\frenchspacing


\begin{document}

\section*{Лекция 5. Шифрование с~открытым и~закрытым ключoм}
\addcontentsline{toc}{section}{Лекция 5. Шифрование с~открытым и~закрытым
ключoм}
\resetcntrs

\section{Принципиальная схема шифрования}

Шифрование с~закрытым ключом: есть $Encoder(m, d)$, который передает сообщение
$c$ полиномиальной длины~$Decoder(d, c) \rightarrow m$. Нужно чтобы перехватчик
$A(c)$ не мог восстановить~$m$.

Шифрование с~открытым ключом: $Encoder(m, e)$ передает $c$ программе $Decoder(c,
d) \rightarrow m$. Ключи $e, d$ у~них разные, и~перехватичик~$A(c, d)$ может
пользоваться одним из них.

Для закрытого ключа есть идеальная, но довольно бесполезная процедура: передать
$m \oplus d$, где~$d$~--- случайная строка. Есть две проблемы: ключ по длине
равен сообщению (если мы можем обменяться такими ключами, то почему не можем
обменяться сообщениями?), но даже если предположить, что мы заранее договорились
о~закрытом ключе, то остается проблема того, что шифр одноразовый: если известно
$m_1 \oplus d$ и~$m_2 \oplus d$, то можно узнать $m_1 \oplus m_2$, что может
быть полезной информацией.

\end{document}
