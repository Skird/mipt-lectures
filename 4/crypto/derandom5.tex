\documentclass{article}
\usepackage[utf8x]{inputenc}
\usepackage[english,russian]{babel}
\usepackage{amsmath,amscd}
\usepackage{amsthm}
\usepackage{mathtools}
\usepackage{amsfonts}
\usepackage{amssymb}
\usepackage{cmap}
\usepackage{centernot}
\usepackage{enumitem}
\usepackage{perpage}
\usepackage{chngcntr}
%\usepackage{minted}
\usepackage[bookmarks=true,pdfborder={0 0 0 }]{hyperref}
\usepackage{indentfirst}
\hypersetup{
  colorlinks,
  citecolor=black,
  filecolor=black,
  linkcolor=black,
  urlcolor=black
}

\newtheorem*{conclusion}{Вывод}
\newtheorem{theorem}{Теорема}
\newtheorem{lemma}{Лемма}
\newtheorem*{corollary}{Следствие}

\theoremstyle{definition}
\newtheorem*{problem}{Задача}
\newtheorem{claim}{Утверждение}
\newtheorem{exercise}{Упражнение}
\newtheorem{definition}{Определение}
\newtheorem{example}{Пример}

\theoremstyle{remark}
\newtheorem*{remark}{Замечание}

\newcommand{\doublearrow}{\twoheadrightarrow}
\renewcommand{\le}{\leqslant}
\renewcommand{\ge}{\geqslant}
\newcommand{\eps}{\varepsilon}
\renewcommand{\phi}{\varphi}
\newcommand{\ndiv}{\centernot\mid}

\MakePerPage{footnote}
\renewcommand*{\thefootnote}{\fnsymbol{footnote}}

\newcommand{\resetcntrs}{\setcounter{theorem}{0}\setcounter{definition}{0}
\setcounter{claim}{0}\setcounter{exercise}{0}}

\DeclareMathOperator{\aut}{aut}
\DeclareMathOperator{\cov}{cov}
\DeclareMathOperator{\argmin}{argmin}
\DeclareMathOperator{\argmax}{argmax}
\DeclareMathOperator*\lowlim{\underline{lim}}
\DeclareMathOperator*\uplim{\overline{lim}}
\DeclareMathOperator{\re}{Re}
\DeclareMathOperator{\im}{Im}

\frenchspacing


\begin{document}

\section*{Лекция 5. Экспандеры на основе зигзаг-произведения}
\addcontentsline{toc}{section}{Лекция 5. Экспандеры на основе
зигзаг-произведения}
\resetcntrs

Строим граф с~тремя параметрами $N$~--- число вершин, $D$~--- степень каждой
вершины, $\gamma = 1 - \lambda$~--- spectral gap.

Рассмотрим три операции, для которых оценим влияние на каждый параметр.

\section{Squaring}

Эта операция преобразует $G = (V, E) \mapsto (V, E') = G^2$, причем ребро
в~новом графе есть ребре $(u, w) \in E' \Leftrightarrow \exists v: (u, v) \in E,
(v, w) \in E$ с~кратностью, равной числу таких~$v$. В~матричном виде это
собственно возведение матрицы в~квадрат.

Тогда $\Vert xM^2 \Vert \le \lambda^2 \Vert x \Vert$. При такой операции $(N, D,
1 - \lambda) \mapsto (N, D^2, 1 - \lambda^2)$.

\section{Tensor product}

Принимает на вход $G_1 = (V_1, E_1), M_1$ с~параметрами~$D_1, \gamma_1$ и~$G_2 =
(V_2, E_2), M_2$ с~параметрами~$D_2, \gamma_2$. Результата $G_1 \otimes G_2 =
(V_1 \times V_2, E), D = D_1 D_2$. $(i, j)$ сосед пары $(v_1, v_2)$ есть пара
из~$i$-го соседа~$v_1$ и~$j$-го соседа~$v_2$.

Случайное блуждание по такому графу~--- это независимое одновременное случайное
блуждание по двум сомножителям.

\begin{claim}
	$\gamma(G) = \min\{\gamma(G_1), \gamma(G_2)\}$.
\end{claim}
\begin{proof}
	Покажем, что для $\forall x \in \mathbb{R}^{N_1 N_2}, x \bot u_{N_1 N_2}
	\rightarrow \Vert x M \Vert \le \lambda \Vert x \Vert$.

	$x = x^\Vert + x^\bot, x^\Vert \Vert u_{N_2}$ в~каждом облаке. $x^\Vert = y
	\times u_{N_2}$, где $y \bot u_{N_1}$.

	$\Vert xM^\Vert = y \otimes u_{N_2} = \Vert x^\Vert M + x^\bot M \Vert^2 =
	\Vert x^\Vert M\Vert^2 + \Vert x^\bot M\Vert^2 \le \lambda_1^2 \Vert x^\Vert
	\Vert^2 + \lambda_2^2 \Vert x^\bot \Vert$. Поясним, почему это так:

	$x^\Vert M = (y \otimes u_{N_2}) (M_1 \otimes M_2) = (yM_1) \otimes (u_{N_2}
	M_2) = yM_1 \otimes u_{N_2}.
	\Vert yM_1 \Vert \le \lambda_1 \Vert y \Vert
	\Rightarrow \Vert x^\Vert M \Vert \le \lambda_1 \Vert x^\Vert \Vert$.

	$x^\bot M = x^\bot (I_{N_1} \otimes M_2) (M_1 \otimes I_{N_2})$. Первое
	уменьшает норму в~$\lambda_2$ раз, второе нормы не уменьшает, поэтому $\Vert
	x^\bot M \Vert \le \lambda_2 \Vert x \Vert$.

	Из конструкции видно, что $x^\Vert M \bot x^\bot M$, значит утверждение
	доказано.
\end{proof}

\section{Zigzag product}

Принимает на вход $G = (N, D_1, \gamma_1), H = (D_1, D_2, \gamma_2)$ и~выдает $G
\textcircled{z} H$ с~параметрами~$(N D_1, D_2^2, \gamma = \gamma_1 \gamma_2^2),
\lambda \le \lambda_1 + 2 \lambda_2$.

$V = V_1 \times V_2, (u \in V_1, i \in V_2)$. Сосед $(u, i)$ с~номером~$(a,
b)$~--- это:
\begin{itemize}
	\item $i'$~--- $a$-й сосед~$i$ в~$H$.
	\item $v$~--- $i'$-й сосед $u$ в~$G$.
	\item $j'$~--- номер~$u$ среди соседей~$v$.
	\item $j$~--- $b$-й сосед $j'$ в~$H$.
	\item $(v, j)$~--- результат.
\end{itemize}

Если $H$~--- полный граф с~петлями, то $G \textcircled{z} H = G \otimes H$.

\begin{claim}
	Если~$A, B, M$~--- матрицы случайных блужданий графов~$G, H, G \textcircled{z}
	H$, то $M = \widetilde{B} \widehat{A} \widetilde{B}$, где $\widetilde{B} =
	I_{N_1} \otimes B$, $\widehat{A}_{(u, i), (v, j)} = 1$, если ребро $(u, v)$
	присутствует в~$G$, имеет номер~$i$ среди соседей~$u$ и~номер~$j$ среди
	соседей~$v$.
\end{claim}
\begin{proof}
	Следует из конструкции.
\end{proof}

$B = \gamma_2 J + (1 - \gamma_2) E$, где $J$ есть матрица из $\frac{1}{D_1}$,
а~$\Vert E \Vert \le 1$.

Тогда~$\widetilde{B} = \gamma_2 \widetilde{J} + (1 - \gamma_2) \widetilde{E}$.

$$B = (\gamma_2 \widetilde{J} + (1 - \gamma_2) \widetilde{E}) \widehat{A}
(\gamma_2 \widetilde{J} + (1 - \gamma_2) \widetilde{E} = \gamma_2^2
\widetilde{J} \widehat{A} \widetilde{J} + (1 - \gamma^2) F, \Vert F \Vert
\le 1.$$

При этом $\widetilde{J} \widehat{A} \widetilde{J} = A \otimes J$ так как $J$
соответствует полному графу. Тогда $M = \gamma_2^2 A \otimes J + (1 -
\gamma_2^2) F$.

$\Vert xM \Vert \le \gamma_2^2 \Vert x A \otimes J \Vert + (1 -
\gamma_2^2) \Vert xF \Vert \le (\gamma_2^2 (1 - \gamma_1) + (1 - \gamma_2^2))
\Vert x \Vert = (1 - \gamma_1 \gamma_2^2) \Vert x \Vert$.

\section{Конструкции экспандеров}

Первая конструкция. Пусть есть экспандер~$H$
с~параметрами~$(D^4, D, \frac{7}{8})$. Будем итерировать
процесс~$G_1 = H^2, G_{t+1} = G_t^2 \textcircled{z} H$.

$G_1$ тогда будет иметь параметры~$(D^4, D^2, \frac{63}{64})$. Если $G_t$ имеет
параметры $(N, D^2, 1 - \lambda)$, то у~$G_t^2$ они будут $(N, D^4, 1 -
\lambda^2)$. Тогда $G_{t+1}$ имеет параметры~$(ND^4, D^2, (1 - \lambda^2)
\frac{49}{64})$. Если $\lambda > \frac{1}{2}$ (что верно для $G_1$), то разрыв
сохраняется $>\frac{1}{2}$.

Вторая конструкция дает более быстрый рост графа. Если~$H$~--- экспандер
с~параметрами~$(D^8, D, \frac{7}{8})$, то $G_1 = H^2$, $G_{t+1} =
(G_t \otimes G_t)^2 \textcircled{z} H$.

$(G_t \otimes G_t)^2$ имеет параметры~$(N^2, D^8, >\frac{3}{4})$. $(G_t \otimes
G_t)^2 \textcircled{z} H$ тогда имеет параметры~$(N^2 D^8, D^2, >\frac{1}{2})$.

Если считать таким образом, то можно за полилог перечислить всех соседей
конкретной вершины.

\end{document}
