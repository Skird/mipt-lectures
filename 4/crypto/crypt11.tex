\documentclass{article}
\usepackage[utf8x]{inputenc}
\usepackage[english,russian]{babel}
\usepackage{amsmath,amscd}
\usepackage{amsthm}
\usepackage{amsfonts}
\usepackage{amssymb}
\usepackage{cmap}
\usepackage{centernot}
\usepackage{enumitem}
\usepackage{perpage}
\usepackage{chngcntr}
%\usepackage{minted}
\usepackage[bookmarks=true,pdfborder={0 0 0 }]{hyperref}
\usepackage{indentfirst}
\hypersetup{
  colorlinks,
  citecolor=black,
  filecolor=black,
  linkcolor=black,
  urlcolor=black
}

\newtheorem*{conclusion}{Вывод}
\newtheorem{theorem}{Теорема}
\newtheorem{lemma}{Лемма}
\newtheorem*{corollary}{Следствие}

\theoremstyle{definition}
\newtheorem*{problem}{Задача}
\newtheorem{claim}{Утверждение}
\newtheorem{exercise}{Упражнение}
\newtheorem{definition}{Определение}
\newtheorem{example}{Пример}

\theoremstyle{remark}
\newtheorem*{remark}{Замечание}

\renewcommand{\le}{\leqslant}
\renewcommand{\ge}{\geqslant}
\newcommand{\eps}{\varepsilon}
\renewcommand{\phi}{\varphi}
\newcommand{\ndiv}{\centernot\mid}

\MakePerPage{footnote}
\renewcommand*{\thefootnote}{\fnsymbol{footnote}}

\newcommand{\resetcntrs}{\setcounter{theorem}{0}\setcounter{definition}{0}
\setcounter{claim}{0}\setcounter{exercise}{0}}

\DeclareMathOperator{\aut}{aut}
\DeclareMathOperator{\cov}{cov}
\DeclareMathOperator{\chos}{ch}
\DeclareMathOperator{\argmin}{argmin}
\DeclareMathOperator{\argmax}{argmax}
\DeclareMathOperator*\lowlim{\underline{lim}}
\DeclareMathOperator*\uplim{\overline{lim}}
\DeclareMathOperator{\re}{Re}
\DeclareMathOperator{\im}{Im}

\frenchspacing


\begin{document}

\section*{Лекция 11. Безопасные многосторонние вычисления}
\addcontentsline{toc}{section}{Лекция 11. Безопасные многосторонние вычисления}
\resetcntrs

\section{Общий вид задачи}

Имеется $m$~--- участники, у~каждого есть вход~$x_i$ и~у~всех вместе есть
функционал $f: (x_1, \ldots, x_m) \mapsto (y_1, \ldots, y_m)$, возможно
вероятностный.

Нужно, чтобы всё, что узнали нечестные участники, они могли бы узнать лишь
исходя из своих $x_i, y_i$.

Получестная модель: все выполняют протокол, но могут анализировать промежуточные
результаты. Можно также рассматривать две нечестные модели: в~первой из них
любое количество участников могут быть нечестными и~отклоняться от протокола,
при этом нет защиты от прекращения общения; во второй модели нечестными могут
быть строго меньше половины участников и~есть защита от преждевременной
остановки.

Дополнительный аспект, проявляющийся, когда сторон становится больше двух,~---
это отличие между двухсторонними каналами и~широковещательными.

Также разумно поставить вопрос о~подслушивании и~адаптивном подслушивании (когда
противник может подслушать часть сообщений, а~потом перехватить контроль над
какой-то стороной по своему желанию).

\section{Получестная модель}

Функция $f$ вычисляется арифметической схемой. Пусть имеется $a, b$, хранящиеся
распределенно: $a = a_1 \oplus \ldots a_m$, $b = b_1 \oplus \ldots \oplus b_m$.
Хотим посчитать $c = a \land b$.

\begin{dmath*}
c = a \land b = \left(\sum a_i\right)\left(\sum b_j\right) =
\sum a_i b_i + \sum\limits_{1 \le i < j \le m} (a_i b_j + a_j b_i) =
\sum a_i b_i + \sum\limits_{1 \le i < j \le m} (a_i + a_j)(b_i + b_j) -
\sum\limits_{1 \le i < j \le m}(a_i b_i + a_j b_j) =
\sum (a_i + a_j)(b_i + b_j) + m \sum a_i b_i.
\end{dmath*}

Участники $i, j$ вычисляют $c_{ij}^i, c_{ij}^j$ такие, что $c_{ij}^i \oplus
c_{ij}^j = (a_i + a_j) \oplus (b_i \oplus b_j)$. $c_i = \sum\limits_{i \ne j}
c_{ij}^i + ma_ib_i$.

$c_{ij}^i$ вычисляются случайно, 4 варианта:
\begin{itemize}
	\item $d_{00} = c_{ij}^i \oplus a_i b_i$;
	\item $d_{01} = c_{ij}^i \oplus a_i (b_i \oplus 1)$;
	\item $d_{10} = c_{ij}^i \oplus (a_i \oplus 1) b_i$;
	\item $d_{11} = c_{ij}^i \oplus (a_i \oplus 1)(b_i \oplus 1)$.
\end{itemize}
Используется протокол пересылки вслепую.

Теперь можно смоделировать работу всей схемы:
\begin{itemize}
	\item Разделение секрета;
	\item Моделирование отдельных шагов;
	\item Восстановление ответа (все стороны присылают свои биты, соответствующие
		ответу данной стороны).
\end{itemize}

\section{Двухсторонние и~широковещательные каналы}

В~нечестных моделях может быть важен тип канала, однако мы покажем, что при
помощи односторонней перестановки с~секретом можно моделировать одно на базе
другого.

Чтобы смоделировать двусторонний канал с~помощью широковещательного используем
шифрование с~открытым ключом:
\begin{itemize}
	\item Каждая сторона генерирует пару ключей и~посылает всем открытый;
	\item Для отправки сообщения $i$-я~сторона может воспользоваться $j$-м
		открытым ключом и~послать шифрованное сообщение $j$-й стороне, которая
		сможет расшифровать его с~помощью своего закрытого ключа.
\end{itemize}

В~другую сторону чуть менее тривиально~--- используем так называемое
Византийское соглашение. Нам нужно, чтобы все честные участники получили одно и
то же, притом, если отправитель честный, то все получили то, что он и~послал.
\begin{itemize}
	\item Первая сторона рассылает $v_2, \ldots, v_m$ вместе с~подписями $s_2,
		\ldots, s_m$;
	\item Сторона $j$ пересылает полученное сообщение (с~подписью) вместе со
		своей~подписью;
	\item Следующие стадии происходят аналогично, если в~цепочке подписей не было
		подписи стороны $j$.
	\item В~конце каждая сторона проверяет исходное сообщение и~все подписи.
\end{itemize}

\section{Нечестные модели}

Для решения проблем с~нечестными участниками можно моделировать протокол
получестных вычислений. Для этого нужно сделать следующее:
\begin{itemize}
	\item Привязка ко входу;
	\item Генерация приватных, но проверяемо-случайных битов;
	\item Моделирование получестного протокола.
\end{itemize}

Такая модель никак не застрахована от подмены входа (от этого застраховаться
нельзя в~принципе), но также подверженна и~преждевременной остановке~--- сторона
может получить свой ответ и~отказаться посылать что-либо еще.

Чтобы решить проблему преждевременной остановки можно воспользоваться
продвинутым разделением секрета <<$t$ из $m$>>. Если $a$ исходный секрет, по
нему формируется $a_1, \ldots, a_m$ такие, что по любым $a_{i_1}, \ldots,
a_{i_t}$ можно изнуть $a$, а~по любому меньшему количеству невозможно.

Функция разделения секрета: выбирается простое число $p > m$ и~генерируется
случайный многочлен степени $t-1$ над $\mathbb{F}_p$ со свободным членом $a$.
$a_1, \ldots, a_m$ тогда можно положить равными заначению многочлена в~точках
$1, \ldots, m$. Тогда по любым $t$ точкам многочлен интерполируется, а~из любых
$t-1$ точек никакой информации про свободный член извлечь нельзя.

Тогда, если честные замечают, что кто-то прекратил действовать по
протоколу, они собираются вместе, рассекречивают исходный секрет и~продолжают
вычисления, эмулируя действия отключившейся стороны. Если же сторона отключилась
с~самого начала, то можно считать, например, что она подменила свой вход на~0.

\end{document}
