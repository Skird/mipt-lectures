\documentclass{article}
\usepackage[utf8x]{inputenc}
\usepackage[english,russian]{babel}
\usepackage{amsmath,amscd}
\usepackage{amsthm}
\usepackage{amsfonts}
\usepackage{amssymb}
\usepackage{cmap}
\usepackage{centernot}
\usepackage{enumitem}
\usepackage{perpage}
\usepackage{chngcntr}
%\usepackage{minted}
\usepackage[bookmarks=true,pdfborder={0 0 0 }]{hyperref}
\usepackage{indentfirst}
\hypersetup{
  colorlinks,
  citecolor=black,
  filecolor=black,
  linkcolor=black,
  urlcolor=black
}

\newtheorem*{conclusion}{Вывод}
\newtheorem{theorem}{Теорема}
\newtheorem{lemma}{Лемма}
\newtheorem*{corollary}{Следствие}

\theoremstyle{definition}
\newtheorem*{problem}{Задача}
\newtheorem{claim}{Утверждение}
\newtheorem{exercise}{Упражнение}
\newtheorem{definition}{Определение}
\newtheorem{example}{Пример}

\theoremstyle{remark}
\newtheorem*{remark}{Замечание}

\renewcommand{\le}{\leqslant}
\renewcommand{\ge}{\geqslant}
\newcommand{\eps}{\varepsilon}
\renewcommand{\phi}{\varphi}
\newcommand{\ndiv}{\centernot\mid}

\MakePerPage{footnote}
\renewcommand*{\thefootnote}{\fnsymbol{footnote}}

\newcommand{\resetcntrs}{\setcounter{theorem}{0}\setcounter{definition}{0}
\setcounter{claim}{0}\setcounter{exercise}{0}}

\DeclareMathOperator{\aut}{aut}
\DeclareMathOperator{\cov}{cov}
\DeclareMathOperator{\chos}{ch}
\DeclareMathOperator{\argmin}{argmin}
\DeclareMathOperator{\argmax}{argmax}
\DeclareMathOperator*\lowlim{\underline{lim}}
\DeclareMathOperator*\uplim{\overline{lim}}
\DeclareMathOperator{\re}{Re}
\DeclareMathOperator{\im}{Im}

\frenchspacing


\begin{document}

\subsection*{Задача 1}
	$g \in H(A) \Rightarrow g + A = A \Rightarrow g + A + B = A + B \Rightarrow g
	\in H(A + B)$.

\subsection*{Задача 2}
	Пусть $A < G$. Тогда если $g \in H(A) \Rightarrow g + A = A \Rightarrow g + 0
	\in A \Rightarrow g \in A$.

	Пусть $H(A) = A$. Рассмотрим $a \in H(A) = A, b \in A \Rightarrow a + A = A
	\Rightarrow a +	b \in A$. Теперь рассмотрим $-a$. В~силу того, что множество
	$A$ замкнуто по сложению и~конечно, приходим к~выводу, что $a$ имеет конечный
	порядок, то есть $k \cdot a = 0 \Rightarrow -a = a + \ldots + a \in A$.

\subsection*{Задача 3}
	По теореме Кнезера:
	$|A_1 + \ldots + A_h| \ge |A_1 + \ldots + A_{h-1}| + |A_h| - |H(A_1 + \ldots
	+ A_h)| \ge \ldots \ge |A_1| + \ldots + |A_h| - |H(A_1 + A_2)| - \ldots
	- |H(A_1 + \ldots + A_h)|$. Так как $H(A_1 + A_2) < \ldots < H(A_1 + \ldots +
	A_h)$, можем оценить последнее выражение как $\sum\limits_{i=1}^h |A_i| -
	(h-1) |H(A_1 + \ldots + A_h)|$.

\subsection*{Задача 4}
	Очевидно, что $|(A+B)/H| = |A/H| + |B/H| - 1 \Leftrightarrow |A + B + H| = |A +
	H| + |B + H| + |H|$. С~другой стороны $|A/H| = \frac{|A+H|}{|H|}$, значит
	$|(A+B)/H| = \frac{|A + H + B + H|}{|H|} = \frac{|A + B|}{|H|}$, стало быть
	$|A+B|$ делится на $|H|$.

	Пусть $|A + B| \le |A| + |B| - 1 \Rightarrow |A + B| \le |A + H| + |B + H| - 1$.
	По теореме Кнезера $|A + B| \ge |A + H| + |B + H| - |H|$ и, так как $|A + B|$
	кратно $H$, то $|A + B| = |A + H| + |B + H| - |H| \Rightarrow |(A + B)/H| =
	|A/H| + |B/H| - 1$.

\subsection*{Задача 5}
	Если $H = \{0\}$, то всё получаем требуемое по теореме Кнезера. Иначе $H$~---
	циклическая подгруппа, порожденная элементом $x$, притом $x$ делит $m$.
	Рассмотрим множество $B + H \supset B$. В~нём содержатся также элементы $x, 2x,
	\ldots, m - x$, которые не содержатся в~$B$, так как каждый элемент $B$, кроме 0
	взаимнопрост с~$m$. Отсюда $|B + H| \ge |B| + |H| - 1$. Применяя теорему
	Кнезера, получаем: $|A + B| \ge |A + H| + |B + H| - |H| \ge |A + H| + |B| + |H|
	- 1 - |H| \ge |A| + |B| - 1$.

\subsection*{Задача 6}
	Возьмём любые два множества $A, B$ и~рассмотрим $H = H(A + B)$. Применим
	неравенство к~$A + H, B + H: |A + H + B + H| \ge |A + H| + |B + H| - |H(A + H
	+ B + H)|$. Так как $A + B + H + H = A + B + H = A + B$, то получаем $|A + B|
	\ge |A + H| + |B + H| - |H(A + B)|$.

\subsection*{Задача 7}
	Выведем из каждого следующее:
	\begin{itemize}
		\item $|A + B| = |A||B| \Rightarrow \forall a_1 \ne a_2, b_1 \ne b_2
			\rightarrow a_1 + b_1 \ne a_2 + b_2 \Rightarrow a_1 - b_2 \ne a_2 - b_1
			\Rightarrow |A - B| = |A||B|$.
		\item $|A - B| = |A||B| \Rightarrow \forall a_1 \ne a_2, b_1 \ne b_2
			\rightarrow a_1 - b_1 \ne a_2 - b_2 \Rightarrow a_1 + b_2 \ne a_2 + b_1
			\Rightarrow$ для пары $(a_1, b_2)$ существует только одна пара $(x, y) =
			(a_1, b_2)$, такая что $a_1 + b_2 = x + y$, если $x \in A, y \in B$.
			Значит размер указанного множества равен $|A||B|$.
		\item Заметим, что $a_1 + b_1 = a_2 + b_2 \Leftrightarrow a_1 - b_2 = a_2 -
			b_1$, что даёт биекцию между множествами.
		\item Рассмотрим какой-то элемент $x = a_1 + b_1$. Если $a_2 = x - b_2$, то
			$a_2 = a_1 + b_1 - b_2 \Rightarrow a_2 - b_1 = a_1 - b_2$. Так как
			существует ровно одна такая четвёрка, то $a_2 = a_1, b_2 = b_1$, то
			элемент в~пересечении $|A \cap (x - B)|$ ровно один.
		\item Пусть для какого-то $y = a_1 - b_1 \in A - B$ это не так, то есть $|A
			\cap (B + a_1 - b_1)| > 1$, то есть существует $a_2, b_2: a_2 \ne a_1, b_2
			\ne b_1, a_2 = b_2 + a_1 - b_1 \Rightarrow a_1 = a_2 - b_2 + b_1$, то есть
			$|A \cap (B + y)| > 2$ для $y = a_2 - b_2$, противоречие.
		\item Пусть $0 \ne x \in (A - A) \cap (B - B), x = a_1 - a_2 = b_1 - b_2,
			a_1 \ne a_2, b_1 \ne b_2$. Тогда $|A \cap (B + y)| > 2$ для $y = a_2 -
			b_2$.
		\item Пусть $|A + B| < |A||B|$. Тогда $\exists (a_1, b_1) \ne (a_2, b_2):
			a_1 + b_1 = a_2 + b_2 \Rightarrow a_1 - a_2 = b_2 - b_1 = x$, притом $x
			\ne 0$. Значит $0 \ne x \in (A - A) \cap (B - B)$, противоречие.
	\end{itemize}

\subsection*{Задача 8}
	Если $|A + cB| < |A||B|$, то найдутся $(a_1, b_1) \ne (a_2, b_2): a_1 + cb_1 =
	a_2 + cb_2 \Rightarrow c = \frac{a_1 - a_2}{b_2 - b_1}$.

\subsection*{Задача 9}
	$|(c + dP)(c + dP)| = |c^2 + cdP + cdP + d^2P| = |c^2 + cdP + d^2P| = |cdP +
	d^2P|$. С~другой стороны это по условию $|c + dP|$. По задаче 8,
	$c$ представимо как $c = d\frac{p_1 - p_2}{p_3 - p_4} \in dP$.

\subsection*{Задача 10}
	Достаточность очевидна. Положим $|\mathbb{F}| = p^k$. Положим $A' = A - a_0,
	a_0 \in A$. Тогда $|A'| = |A| = |A + A| = |2a_0 + A' + A'| = |A' + A'|$.
	Однако $A' \subset A' + A' \Rightarrow A' = A' + A'$, то есть $A'$ есть
	смежный класс по $H$. Так как он содержит 0, то $A'$ есть подгруппа
	$\mathbb{F}$ по сложению.

	Если $0 \notin A$, то аналогичными рассуждениями получаем, что $A = cA''$, где
	$A''$ подгруппа $\mathbb{F^\ast}$ по умножению. Но тогда по теореме Лагранжа
	$|A|$ делит $|\mathbb{F}| = p^k$ и~$|\mathbb{F^\ast}| = p^k - 1$. Так как эти
	числа взаимнопросты, то $|A| = 1$, тогда все тривиально.

	Итак, $0 \in A$, значит $A \setminus \{0\}$~--- (возможно мультипликативно
	сдвинутая) подгруппа по умножению. То есть $A = cP$, где $P$~--- подкольцо
	с~единицей. Так как порядок всех элементов конечный, то если $a \in P$, то
	$\exists q: a^q = 1$. Так как $P \cdot P = P$, то $a^{-1} = a^{q-1} \in P$, то
	есть $P$~--- подполе, ч.т.д.

\subsection*{Задача 11}
	Рассмотрим двоичные записи чисел из $A + A$. Все числа вида $2^i + 2^j$ имеют
	две единицы в~двоичной записи (на позициях до $n$-й) за исключением тех, что
	имеют вид $2^i + 2^i = 2^{i+1}$. С~другой стороны каждое такое число легко
	получить, сложив нужные степени двойки. Стало быть $|A + A| \ge C_{n+1}^2$,
	значит и~$|A + A| = C_{n+1}^2$.

\end{document}
