\documentclass{article}
\usepackage[utf8x]{inputenc}
\usepackage[english,russian]{babel}
\usepackage{amsmath,amscd}
\usepackage{amsthm}
\usepackage{amsfonts}
\usepackage{amssymb}
\usepackage{cmap}
\usepackage{centernot}
\usepackage{enumitem}
\usepackage{perpage}
\usepackage{chngcntr}
%\usepackage{minted}
\usepackage[bookmarks=true,pdfborder={0 0 0 }]{hyperref}
\usepackage{indentfirst}
\hypersetup{
  colorlinks,
  citecolor=black,
  filecolor=black,
  linkcolor=black,
  urlcolor=black
}

\newtheorem*{conclusion}{Вывод}
\newtheorem{theorem}{Теорема}
\newtheorem{lemma}{Лемма}
\newtheorem*{corollary}{Следствие}

\theoremstyle{definition}
\newtheorem*{problem}{Задача}
\newtheorem{claim}{Утверждение}
\newtheorem{exercise}{Упражнение}
\newtheorem{definition}{Определение}
\newtheorem{example}{Пример}

\theoremstyle{remark}
\newtheorem*{remark}{Замечание}

\renewcommand{\le}{\leqslant}
\renewcommand{\ge}{\geqslant}
\newcommand{\eps}{\varepsilon}
\renewcommand{\phi}{\varphi}
\newcommand{\ndiv}{\centernot\mid}

\MakePerPage{footnote}
\renewcommand*{\thefootnote}{\fnsymbol{footnote}}

\newcommand{\resetcntrs}{\setcounter{theorem}{0}\setcounter{definition}{0}
\setcounter{claim}{0}\setcounter{exercise}{0}}

\DeclareMathOperator{\aut}{aut}
\DeclareMathOperator{\cov}{cov}
\DeclareMathOperator{\chos}{ch}
\DeclareMathOperator{\argmin}{argmin}
\DeclareMathOperator{\argmax}{argmax}
\DeclareMathOperator*\lowlim{\underline{lim}}
\DeclareMathOperator*\uplim{\overline{lim}}
\DeclareMathOperator{\re}{Re}
\DeclareMathOperator{\im}{Im}

\frenchspacing


\begin{document}

\subsection*{Задача 1}
	$g \in H(A) \Rightarrow g + A = A \Rightarrow g + A + B = A + B \Rightarrow g
	\in H(A + B)$.

\subsection*{Задача 2}
	Пусть $A < G$. Тогда если $g \in H(A) \Rightarrow g + A = A \Rightarrow g + 0
	\in A \Rightarrow g \in A$.

	Пусть $H(A) = A$. Рассмотрим $a \in H(A) = A, b \in A \Rightarrow a + A = A
	\Rightarrow a +	b \in A$. Теперь рассмотрим $-a$. В~силу того, что множество
	$A$ замкнуто по сложению и~конечно, приходим к~выводу, что $a$ имеет конечный
	порядок, то есть $k \cdot a = 0 \Rightarrow -a = a + \ldots + a \in A$.

\subsection*{Задача 3}
	По теореме Кнезера:
	$|A_1 + \ldots + A_h| \ge |A_1 + \ldots + A_{h-1}| + |A_h| - |H(A_1 + \ldots
	+ A_h)| \ge \ldots \ge |A_1| + \ldots + |A_h| - |H(A_1 + A_2)| - \ldots
	- |H(A_1 + \ldots + A_h)|$. Так как $H(A_1 + A_2) < \ldots < H(A_1 + \ldots +
	A_h)$, можем оценить последнее выражение как $\sum\limits_{i=1}^h |A_i| -
	(h-1) |H(A_1 + \ldots + A_h)|$.

\end{document}
