\documentclass{article}
\usepackage[utf8x]{inputenc}
\usepackage[english,russian]{babel}
\usepackage{amsmath,amscd}
\usepackage{amsthm}
\usepackage{mathtools}
\usepackage{amsfonts}
\usepackage{amssymb}
\usepackage{cmap}
\usepackage{centernot}
\usepackage{enumitem}
\usepackage{perpage}
\usepackage{chngcntr}
%\usepackage{minted}
\usepackage[bookmarks=true,pdfborder={0 0 0 }]{hyperref}
\usepackage{indentfirst}
\hypersetup{
  colorlinks,
  citecolor=black,
  filecolor=black,
  linkcolor=black,
  urlcolor=black
}

\newtheorem*{conclusion}{Вывод}
\newtheorem{theorem}{Теорема}
\newtheorem{lemma}{Лемма}
\newtheorem*{corollary}{Следствие}

\theoremstyle{definition}
\newtheorem*{problem}{Задача}
\newtheorem{claim}{Утверждение}
\newtheorem{exercise}{Упражнение}
\newtheorem{definition}{Определение}
\newtheorem{example}{Пример}

\theoremstyle{remark}
\newtheorem*{remark}{Замечание}

\newcommand{\doublearrow}{\twoheadrightarrow}
\renewcommand{\le}{\leqslant}
\renewcommand{\ge}{\geqslant}
\newcommand{\eps}{\varepsilon}
\renewcommand{\phi}{\varphi}
\newcommand{\ndiv}{\centernot\mid}

\MakePerPage{footnote}
\renewcommand*{\thefootnote}{\fnsymbol{footnote}}

\newcommand{\resetcntrs}{\setcounter{theorem}{0}\setcounter{definition}{0}
\setcounter{claim}{0}\setcounter{exercise}{0}}

\DeclareMathOperator{\aut}{aut}
\DeclareMathOperator{\cov}{cov}
\DeclareMathOperator{\argmin}{argmin}
\DeclareMathOperator{\argmax}{argmax}
\DeclareMathOperator*\lowlim{\underline{lim}}
\DeclareMathOperator*\uplim{\overline{lim}}
\DeclareMathOperator{\re}{Re}
\DeclareMathOperator{\im}{Im}

\frenchspacing


\begin{document}

\section{Ещё немного о~разных группах}

\begin{definition}
	Пусть есть последовательность $F_n$, тогда если $\frac{\lambda(F_n \oplus
	(\delta + F_n))}{\lambda(F_n)} \rightarrow 0$ для всех $z \in K$-компакта, то
	эти множества называются Фёльнеровскими.
\end{definition}

\begin{definition}
	Аменабельная группа~$G$~--- такая группа, в~которой есть
	<<последовательность>> Фёльнеровских множеств $F_n$.
\end{definition}

Утверждается, что если вероятность случайного блуждания вернуться в~1 за~$n$
шагов стремится к~0 очень быстро, то группа не аменабельна.

С~неаменабелностью $\text{SO}(3)$ связан парадокс Банаха-Тарского.

Насчёт автоматных групп: их можно представлять как некоторые преобразования
бинарного дерева. Необходимым условием обратимости, конечно, является
обратимость преобразования дерева.

Такие автоматы порождают 5 интересных групп, которые мы точно будем
рассматривать.

\begin{exercise}
	Дискретное преобразование Фурье в~$\mathbb{Z}_2, \mathbb{Z}_3, \mathbb{Z}_4,
	\mathbb{Z}_8$: спектр, СЗ, СВ, как все устроено.
\end{exercise}

\end{document}
