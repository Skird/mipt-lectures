\documentclass{article}
\usepackage[utf8x]{inputenc}
\usepackage[english,russian]{babel}
\usepackage{amsmath,amscd}
\usepackage{amsthm}
\usepackage{mathtools}
\usepackage{amsfonts}
\usepackage{amssymb}
\usepackage{cmap}
\usepackage{centernot}
\usepackage{enumitem}
\usepackage{perpage}
\usepackage{chngcntr}
%\usepackage{minted}
\usepackage[bookmarks=true,pdfborder={0 0 0 }]{hyperref}
\usepackage{indentfirst}
\hypersetup{
  colorlinks,
  citecolor=black,
  filecolor=black,
  linkcolor=black,
  urlcolor=black
}

\newtheorem*{conclusion}{Вывод}
\newtheorem{theorem}{Теорема}
\newtheorem{lemma}{Лемма}
\newtheorem*{corollary}{Следствие}

\theoremstyle{definition}
\newtheorem*{problem}{Задача}
\newtheorem{claim}{Утверждение}
\newtheorem{exercise}{Упражнение}
\newtheorem{definition}{Определение}
\newtheorem{example}{Пример}

\theoremstyle{remark}
\newtheorem*{remark}{Замечание}

\newcommand{\doublearrow}{\twoheadrightarrow}
\renewcommand{\le}{\leqslant}
\renewcommand{\ge}{\geqslant}
\newcommand{\eps}{\varepsilon}
\renewcommand{\phi}{\varphi}
\newcommand{\ndiv}{\centernot\mid}

\MakePerPage{footnote}
\renewcommand*{\thefootnote}{\fnsymbol{footnote}}

\newcommand{\resetcntrs}{\setcounter{theorem}{0}\setcounter{definition}{0}
\setcounter{claim}{0}\setcounter{exercise}{0}}

\DeclareMathOperator{\aut}{aut}
\DeclareMathOperator{\cov}{cov}
\DeclareMathOperator{\argmin}{argmin}
\DeclareMathOperator{\argmax}{argmax}
\DeclareMathOperator*\lowlim{\underline{lim}}
\DeclareMathOperator*\uplim{\overline{lim}}
\DeclareMathOperator{\re}{Re}
\DeclareMathOperator{\im}{Im}

\frenchspacing


\begin{document}

\section*{Лекция 3. Преоразование Фурье}
\addcontentsline{toc}{section}{Лекция 3. Преоразование Фурье}
\resetcntrs

\section{Общая конструкция преобразования Фурье}

Пусть есть топологическая группа~$G$. Определелим характер~$\gamma \in Hom(G,
S^1), S^1 = \{\lambda \in \mathbb{C} : |\lambda| = 1\})$, притом потребуем того,
что $\gamma$~--- непрерывен.

\begin{definition}
	Дуальная группа~$\hat G = \{\gamma\}$ определена поточечным умножением
	характеров.
\end{definition}

\begin{theorem}[Понтрягина о~двойственности]
Если $G$~--- абелева топологическая группа, тогда $\hat{\hat{G}} = G$.

При этом, если~$G$~--- компактна, то $\hat G$~--- дискретна. Если $G$~---
дискретна, то $\hat G$~--- компактна.
\end{theorem}

Топологические группы с~нестандартной топологией могут быть представлены как
стандратные топологии на смежных классах $G/H, H < G$.

\begin{exercise}
	Можно ли придумать нестандартную топологию на конечной группе, которая не
	встречатеся среди стандартных групповых топологий?
\end{exercise}

\begin{definition}
	Преобразование Фурье: $F: f(x) \mapsto \hat f(\gamma) = \int\limits_G
	f(x)\overline{\gamma(x)} d\mu$, где $\mu$~--- левая мера Хаара.
\end{definition}

Характеры $\mathbb{R}: \gamma_t(x) = \exp(2\pi itx), t \in \mathbb{R}$,
$\hat{\mathbb{R}} = \mathbb{R}$. Преобразование Фурье выглядит так:
$\hat f(t) = \int\limits_{-\infty}^{\infty} f(x) \exp(-2\pi itx) dx$.

Тор $\mathbb{T} = \mathbb{R} / \mathbb{Z}$, его характеры $\gamma_t = \exp(2\pi
itx), t \in \mathbb{Z}$, дуальная группа $\hat{\mathbb{T}} = \mathbb{Z}$.
Преобразование Фурье: $\hat f(j) = \int\limits_0^1 f(x) \exp(-2\pi ijx) dx$.

Соответственно $\hat{\mathbb{Z}} = \mathbb{T}$, так как достаточно задать
$\gamma(1) = \exp(2\pi i\alpha)$.

Для $\mathbb{Z}_n = \mathbb{Z} / n\mathbb{Z}$ характеры такие: $\gamma_j(x) =
\exp(2\pi i \frac{jx}{n})$, $\hat f(j) = \sum\limits_{x=0}^{n-1} f(x) \exp(-2\pi
i \frac{jx}{n})$.

\begin{exercise}
	Топология на $\mathbb{Z}_{(2)}$~--- односторонние двоичные последовательности
	со сложением. Проверить, что это компактная группа.
\end{exercise}
\begin{exercise}
	Преобразование Фурье на~$\mathbb{Z}_2^n$.
\end{exercise}

\end{document}
