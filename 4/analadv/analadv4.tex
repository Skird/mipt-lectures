\documentclass{article}
\usepackage[utf8x]{inputenc}
\usepackage[english,russian]{babel}
\usepackage{amsmath,amscd}
\usepackage{amsthm}
\usepackage{mathtools}
\usepackage{amsfonts}
\usepackage{amssymb}
\usepackage{cmap}
\usepackage{centernot}
\usepackage{enumitem}
\usepackage{perpage}
\usepackage{chngcntr}
%\usepackage{minted}
\usepackage[bookmarks=true,pdfborder={0 0 0 }]{hyperref}
\usepackage{indentfirst}
\hypersetup{
  colorlinks,
  citecolor=black,
  filecolor=black,
  linkcolor=black,
  urlcolor=black
}

\newtheorem*{conclusion}{Вывод}
\newtheorem{theorem}{Теорема}
\newtheorem{lemma}{Лемма}
\newtheorem*{corollary}{Следствие}

\theoremstyle{definition}
\newtheorem*{problem}{Задача}
\newtheorem{claim}{Утверждение}
\newtheorem{exercise}{Упражнение}
\newtheorem{definition}{Определение}
\newtheorem{example}{Пример}

\theoremstyle{remark}
\newtheorem*{remark}{Замечание}

\newcommand{\doublearrow}{\twoheadrightarrow}
\renewcommand{\le}{\leqslant}
\renewcommand{\ge}{\geqslant}
\newcommand{\eps}{\varepsilon}
\renewcommand{\phi}{\varphi}
\newcommand{\ndiv}{\centernot\mid}

\MakePerPage{footnote}
\renewcommand*{\thefootnote}{\fnsymbol{footnote}}

\newcommand{\resetcntrs}{\setcounter{theorem}{0}\setcounter{definition}{0}
\setcounter{claim}{0}\setcounter{exercise}{0}}

\DeclareMathOperator{\aut}{aut}
\DeclareMathOperator{\cov}{cov}
\DeclareMathOperator{\argmin}{argmin}
\DeclareMathOperator{\argmax}{argmax}
\DeclareMathOperator*\lowlim{\underline{lim}}
\DeclareMathOperator*\uplim{\overline{lim}}
\DeclareMathOperator{\re}{Re}
\DeclareMathOperator{\im}{Im}

\frenchspacing


\begin{document}

\section{Нормы на $\mathbb{Q}$}

\begin{theorem}
	Существуют следующие (мультипликативные) нормы на $\mathbb{Q}$:
	\begin{itemize}
		\item тривиальная
		\item стандартная: $|x| = x sgn(x)$
		\item $p$-адическая, $|x|_p = |\frac{a}{p^k}| = p^k$, $p$~--- простое.
	\end{itemize}
\end{theorem}

\begin{exercise}
	Если двигаться шагами по $2^k$ с~весом $2^{-k}$ от точки 0 к~точке $x \in
	\mathbb{Z}$, то чему равен вес кратчайшего пути?
\end{exercise}

\begin{exercise}
	$G = \left\{ \begin{bmatrix}a & b\\0 & \frac{1}{a}\end{bmatrix} \right\}$.
	Найти левую и~правую меру Хаара.
\end{exercise}

Если пополнить $p$-адические числа, получим $\mathbb{Q}_{(p)} =
[\mathbb{Q}]_{|\cdot|_p}$. Числа там имеют вид~$\sum\limits_{j=-\infty}^{\infty}
x_j p^j$. Можно выделить абелеву подгруппу $\mathbb{Z}_{(p)}$ с~числами, где нет
отрицательных~$j$.

\begin{exercise}
	$\mathbb{Z}_{(p)}$~--- компактно.
\end{exercise}

\begin{exercise}
	$\mathbb{Z}_{(p)}$~--- гомеоморфно $p$-ичному дереву и~канторовскому множеству.
\end{exercise}

\begin{exercise}
	Записать $-1, \frac{1}{2}$ как $p$-адическую дробь.
\end{exercise}

\begin{exercise}[$\ast\ast$]
	Исследовать в~$p$-адических числах $e^t = 1 + x + \frac{x^2}{2} + \ldots$.
\end{exercise}

\begin{exercise}
	Доказать, что $T: x \mapsto x + 1$ непрерывно, сохраняет меру Хаара, и~что все
	сдвиги на этой группе $R_a: x \mapsto x + a$ сводятся к~$T$.
\end{exercise}

\begin{exercise}
	Найти меру Хаара этой группы.
\end{exercise}

\begin{exercise}
	Проверить, что характеры~$\mathbb{Z}_{(p)}$~--- это
	$\gamma_{\frac{\alpha}{p^k}}(x) = \exp(2\pi i \frac{\alpha}{p^k}x)$.
\end{exercise}

\end{document}
