\documentclass{article}
\usepackage[utf8x]{inputenc}
\usepackage[english,russian]{babel}
\usepackage{amsmath,amscd}
\usepackage{amsthm}
\usepackage{mathtools}
\usepackage{amsfonts}
\usepackage{amssymb}
\usepackage{cmap}
\usepackage{centernot}
\usepackage{enumitem}
\usepackage{perpage}
\usepackage{chngcntr}
%\usepackage{minted}
\usepackage[bookmarks=true,pdfborder={0 0 0 }]{hyperref}
\usepackage{indentfirst}
\hypersetup{
  colorlinks,
  citecolor=black,
  filecolor=black,
  linkcolor=black,
  urlcolor=black
}

\newtheorem*{conclusion}{Вывод}
\newtheorem{theorem}{Теорема}
\newtheorem{lemma}{Лемма}
\newtheorem*{corollary}{Следствие}

\theoremstyle{definition}
\newtheorem*{problem}{Задача}
\newtheorem{claim}{Утверждение}
\newtheorem{exercise}{Упражнение}
\newtheorem{definition}{Определение}
\newtheorem{example}{Пример}

\theoremstyle{remark}
\newtheorem*{remark}{Замечание}

\newcommand{\doublearrow}{\twoheadrightarrow}
\renewcommand{\le}{\leqslant}
\renewcommand{\ge}{\geqslant}
\newcommand{\eps}{\varepsilon}
\renewcommand{\phi}{\varphi}
\newcommand{\ndiv}{\centernot\mid}

\MakePerPage{footnote}
\renewcommand*{\thefootnote}{\fnsymbol{footnote}}

\newcommand{\resetcntrs}{\setcounter{theorem}{0}\setcounter{definition}{0}
\setcounter{claim}{0}\setcounter{exercise}{0}}

\DeclareMathOperator{\aut}{aut}
\DeclareMathOperator{\cov}{cov}
\DeclareMathOperator{\argmin}{argmin}
\DeclareMathOperator{\argmax}{argmax}
\DeclareMathOperator*\lowlim{\underline{lim}}
\DeclareMathOperator*\uplim{\overline{lim}}
\DeclareMathOperator{\re}{Re}
\DeclareMathOperator{\im}{Im}

\frenchspacing


\begin{document}

\section{Подсчёт структур с~помощью экспоненциальных производящих функций}

Будем рассматривать функции вида $f(t) = \sum\limits_{n=0}^{\infty}
\frac{1}{n!} x_n t^n$. С~помощью них будем считать число каких-то интересных
множеств с~точностью до размера.

\begin{center}
	\begin{tabular}{|c|c|c|}
		\hline
		$1, 1, 1, \ldots$ & $f(t) = e^t$ & тривиальная, $P(A) = T$\\
		\hline
		$1, 0, 0, \ldots$ & $f(t) = 1$ & $P(A) = (A = \varnothing)$\\
		\hline
		$0, 1, 0, \ldots$ & $f(t) = x$ & $P(A) = (|A| = 1)$ \\
		\hline
		$0, \ldots, 0, 1, 0, \ldots$ & $f(t) = \frac{x^k}{k!}$ & $P(A)=(|A| = k)$\\
		\hline
	\end{tabular}
\end{center}

Можем складывать, если уверены в~дизъюнктности.

Умножение соответствует разбиению на два множества, каждое со своей структурой.

К~примеру две тривиальных функции: $e^t \cdot e^t = e^{2t} = \sum 2^n
\frac{t^n}{n!}$, количество подмножеств, что и~должно было получиться.

Числа Белла: $(e^t - 1)$~--- непустота, значит разбиения это $e^{e^t - 1}$.

Число перестановок: выбираем первый элемент, остальное должно иметь
упорядоченную структуру. То есть $t f(t) = f(t) - 1$ (минус один важно не
забыть, потому что нельзя выбрать один элемент из пустого множества). Итого
получаем $f(t) = \frac{1}{1-t}$.

Беспорядки: все есть сумма $\frac{1}{1-t} = f_0 + \ldots + f_n + \ldots$, где
$f_i$~--- число перестановок с~$i$ неподвижными точками. $f_k = \frac{x^k}{k!}
f_0$, так как перестановка с~$k$ неподвижными точками~--- это разбиение на $k$
точек и~беспорядок. Итого $f = \frac{e^{-t}}{1-t}$, вычет в~1 равен $e^{-1}$.

Логарифмирование. Рассмотрим $e^{L(t)} = \frac{1}{1-t}$. Перестановка
разбивается на циклы, число таких циклических упорядочиваний получается $L(t) =
-\ln(1-t) = -(-t-\frac{t^2}{2}-\frac{t^3}{3} + \ldots) = \sum\limits
\frac{t^n}{n!}(n-1)!$, как и~должно было получиться.

Производная соответствует удалению одного элемента. Например, $f'(t) = f \cdot
f = f^2$~--- удаление одного элемента из перестановки это тоже самое, что
разбиение на два множества~--- до и~после этого элемента.

\end{document}
