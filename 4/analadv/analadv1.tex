\documentclass{article}
\usepackage[utf8x]{inputenc}
\usepackage[english,russian]{babel}
\usepackage{amsmath,amscd}
\usepackage{amsthm}
\usepackage{mathtools}
\usepackage{amsfonts}
\usepackage{amssymb}
\usepackage{cmap}
\usepackage{centernot}
\usepackage{enumitem}
\usepackage{perpage}
\usepackage{chngcntr}
%\usepackage{minted}
\usepackage[bookmarks=true,pdfborder={0 0 0 }]{hyperref}
\usepackage{indentfirst}
\hypersetup{
  colorlinks,
  citecolor=black,
  filecolor=black,
  linkcolor=black,
  urlcolor=black
}

\newtheorem*{conclusion}{Вывод}
\newtheorem{theorem}{Теорема}
\newtheorem{lemma}{Лемма}
\newtheorem*{corollary}{Следствие}

\theoremstyle{definition}
\newtheorem*{problem}{Задача}
\newtheorem{claim}{Утверждение}
\newtheorem{exercise}{Упражнение}
\newtheorem{definition}{Определение}
\newtheorem{example}{Пример}

\theoremstyle{remark}
\newtheorem*{remark}{Замечание}

\newcommand{\doublearrow}{\twoheadrightarrow}
\renewcommand{\le}{\leqslant}
\renewcommand{\ge}{\geqslant}
\newcommand{\eps}{\varepsilon}
\renewcommand{\phi}{\varphi}
\newcommand{\ndiv}{\centernot\mid}

\MakePerPage{footnote}
\renewcommand*{\thefootnote}{\fnsymbol{footnote}}

\newcommand{\resetcntrs}{\setcounter{theorem}{0}\setcounter{definition}{0}
\setcounter{claim}{0}\setcounter{exercise}{0}}

\DeclareMathOperator{\aut}{aut}
\DeclareMathOperator{\cov}{cov}
\DeclareMathOperator{\argmin}{argmin}
\DeclareMathOperator{\argmax}{argmax}
\DeclareMathOperator*\lowlim{\underline{lim}}
\DeclareMathOperator*\uplim{\overline{lim}}
\DeclareMathOperator{\re}{Re}
\DeclareMathOperator{\im}{Im}

\frenchspacing


\begin{document}

\section{Введение}

Что будет затронуто:
\begin{itemize}
	\item Введение в~функциональный анализ
	\item Алгебраические структуры, геометрия графов
	\item Спектральная теория
	\item Гармонический анализ
	\item Приложения к~дискретной математике
\end{itemize}

\section*{Лекция 1. Алгебраические структуры}
\addcontentsline{toc}{section}{Лекция 1. Алгебраические структуры}
\resetcntrs

\section{Алгебраические структуры}

\begin{definition} Напоминание определений основных структур:

\begin{itemize}
	\item Полугруппа~--- множество с~ассоциативной операцией.
	\item Полугруппа с~единицей.
	\item Группа~--- множество с~обратимой ассоциативной операцией.

		В~том числе свободная группа и~группа, заданная соотношениями $G =
		\left< S \mid \mathcal{A} \right>$.

		Автоматные группы. Пусть задан конечный преобразователь~$F$ с~двумя
		состояниями~$\{a, b\}$. Несколько преобразователей можно комбинировать.
		Получился моноид.
		$G(\mathcal{A}) = \left< \mathcal{A}_a, \mathcal{A}_b \right>$,
		где~$\mathcal{A}$~--- обратимый преобразователь, $\mathcal{A}_x$~---
		преобразователь с~начальным состоянием~$x$.
\end{itemize}
\end{definition}

\section{Немного конечномерной линейной алгебры}

Рассмотрим вычисление аналитических функций от матриц. $f(z) =
\sum\limits_{k=0}^\infty a_k z^k$.

Метод: применение интерполяционных многочленов. Если оператор диагонализуем,
то все ясно, нужно знать только $f(\lambda_i)$. Утверждается, что всегда
работает следующее: для каждой Жорданового блока запишем $P(\lambda_1) =
f(\lambda_1), \ldots, P^{(r_1 - 1)}(\lambda_1) = f^{(r_1 - 1)}(\lambda_1)$,
где $r_1$~--- кратность~$\lambda_1$, интерполируем это и~вычислим~$P(A)$.

\end{document}
