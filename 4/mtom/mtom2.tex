\documentclass{article}
\usepackage[utf8x]{inputenc}
\usepackage[english,russian]{babel}
\usepackage{amsmath,amscd}
\usepackage{amsthm}
\usepackage{amsfonts}
\usepackage{amssymb}
\usepackage{cmap}
\usepackage{centernot}
\usepackage{enumitem}
\usepackage{perpage}
\usepackage{chngcntr}
%\usepackage{minted}
\usepackage[bookmarks=true,pdfborder={0 0 0 }]{hyperref}
\usepackage{indentfirst}
\hypersetup{
  colorlinks,
  citecolor=black,
  filecolor=black,
  linkcolor=black,
  urlcolor=black
}

\newtheorem*{conclusion}{Вывод}
\newtheorem{theorem}{Теорема}
\newtheorem{lemma}{Лемма}
\newtheorem*{corollary}{Следствие}

\theoremstyle{definition}
\newtheorem*{problem}{Задача}
\newtheorem{claim}{Утверждение}
\newtheorem{exercise}{Упражнение}
\newtheorem{definition}{Определение}
\newtheorem{example}{Пример}

\theoremstyle{remark}
\newtheorem*{remark}{Замечание}

\renewcommand{\le}{\leqslant}
\renewcommand{\ge}{\geqslant}
\newcommand{\eps}{\varepsilon}
\renewcommand{\phi}{\varphi}
\newcommand{\ndiv}{\centernot\mid}

\MakePerPage{footnote}
\renewcommand*{\thefootnote}{\fnsymbol{footnote}}

\newcommand{\resetcntrs}{\setcounter{theorem}{0}\setcounter{definition}{0}
\setcounter{claim}{0}\setcounter{exercise}{0}}

\DeclareMathOperator{\aut}{aut}
\DeclareMathOperator{\cov}{cov}
\DeclareMathOperator{\chos}{ch}
\DeclareMathOperator{\argmin}{argmin}
\DeclareMathOperator{\argmax}{argmax}
\DeclareMathOperator*\lowlim{\underline{lim}}
\DeclareMathOperator*\uplim{\overline{lim}}
\DeclareMathOperator{\re}{Re}
\DeclareMathOperator{\im}{Im}

\frenchspacing


\begin{document}

\section{Простейшая задача вариационного исчисления}

Небольшая известная мотивировочная задача:
\begin{problem}[1696, о~брахистохроне]
	Две точки: $(0, 0)$, $(a, b)$. Нужно найти кривую $y(x)$, по которой точка
	скатится под действием силы тяжести быстрее всего.

	Иными словами
	$\int dt = \int\limits_0^a \sqrt{\frac{1 + (y')^2}{2gy}}dx \rightarrow \min,
	y(0) = 0, y(a) = b$
\end{problem}

Итак, простейшая задача классического вариационного исчисления: $I(x) =
\int\limits_{t_0}^{t_1} L(t, x, \dot{x}) \rightarrow \inf, x(t_0) = x_0, x(t_1)
= x_1$.

\begin{definition}
	$\hat{x} \in C^1[t_0, t_1]$~--- слабый locmin $\Leftrightarrow \exists \delta
	> 0: \forall x: \norm{x - \hat{x}}_{C^1} < \delta \Rightarrow I(x) \ge
	I(\hat{x}), x(t_0) = x_0, x(t_1) = x_1$.

	Слабым он называется, потому что выбран довольно узкий функциональный класс
	$C^1$.
\end{definition}

\begin{theorem}[Необходимое условие слабого минимума]
	Пусть $\hat{x}$~--- слабый locmin, $L, L_x, L_{\dot{x}} \in C^1$. Тогда
	выполнено уравнение Эйлера-Лагранжа:
	$$ -\frac{d}{dt}\hat{L}_{\dot{x}} + \hat{L}_x = 0 $$
\end{theorem}
\begin{proof}
	Пусть $h \in C_0^1[t_0, t_1]$, то есть $h(t_0) = 0, h(t_1) = 0$. $I(\hat{x} +
	\lambda h) = \phi(\lambda)$. Рассмотрим $\phi'(0) = 0$:

	$$ \int\limits_{t_0}^{t_1} \frac{L(t, \hat{x} + \lambda h, \hat{\dot{x}}) -
	L(t, \hat{x}, \hat{\dot{x}})}{\lambda} dt = \int\limits_{t_0}^{t_1} (\hat{L}_x
	h + \hat{L}_{\dot{x}} \dot{h})dt = 0 \quad\forall h \in C_0^1$$

	\begin{lemma}[Дюбуа-Раймон]
		Пусть $a, b \in C[t_0, t_1]$ и~$\int\limits_{t_0}^{t_1} (a(t)h(t) +
		b(t)\dot{h}(t))dt = 0 \quad \forall h \in C_0^1$, тогда $b \in C^1[t_0,
		t_1], \dot{b} = a$.
	\end{lemma}
\end{proof}

\section{Интегралы уравнения Эйлера}

$-\frac{d}{dt}\hat{L}_{\dot{x}} + \hat{L}_x = 0$. Иногда можно указать первые
интегралы:
\begin{itemize}
	\item $L = L(t, x), L_x = 0$.
	\item $L = L(t, \dot{x}), L_{\dot{x}} = 0$, интеграл импульса
	\item $L = L(x, \dot{x}), H = \dot{x}L_{\dot{x}} - L$~--- интеграл энергии
\end{itemize}

С~помощью такого первого интеграла можем сократить себе работу по нахождению
брахистохроны, получим в~итоге $x = \frac{c}{2}(\tau - \sin\tau), y =
-\frac{c}{2} (1 - \cos \tau)$.

\begin{exercise}
	Сколько есть экстремалей у~уравнения Эйлера для брахистохроны? Как ведет себя
	найденное $\hat{y}$ в~окрестности 0? Даёт ли $\hat{y}$ минимум?
\end{exercise}

\begin{problem}[Задача Больца]
	$B(x) = \int\limits_{t_0}^{t_1} L(t, x, \dot{x}) dt + l(x(t_0), x(t_1))
	\rightarrow \min$.
\end{problem}

\begin{theorem}[Необходимое условие]
	Пусть $L, L_x, L_{\dot{x}}, l_{x(t_0}, l_{x(t_1)} \in C, \hat{x}$~--- слабый
	$locmin$, тогда выполнены условия:
	\begin{itemize}
		\item Уравнение Эйлера: $-\frac{d}{dt} \hat{L}_{\dot{x}} + \hat{L}_x = 0$
		\item Условие трансверсальности: $\hat{L}_{\dot{x}} =
			(-1)^j \hat{l}_{x(t_j)}$
	\end{itemize}
\end{theorem}
\begin{proof}
	$\phi(\lambda) = B(\hat{x} + \lambda h), h \in C^1[t_0, t_1]$. $\phi'(0) = 0
	\Rightarrow $ to be continued (no).
\end{proof}

\end{document}
