\documentclass{article}
\usepackage[utf8x]{inputenc}
\usepackage[english,russian]{babel}
\usepackage{amsmath,amscd}
\usepackage{amsthm}
\usepackage{mathtools}
\usepackage{amsfonts}
\usepackage{amssymb}
\usepackage{cmap}
\usepackage{centernot}
\usepackage{enumitem}
\usepackage{perpage}
\usepackage{chngcntr}
%\usepackage{minted}
\usepackage[bookmarks=true,pdfborder={0 0 0 }]{hyperref}
\usepackage{indentfirst}
\hypersetup{
  colorlinks,
  citecolor=black,
  filecolor=black,
  linkcolor=black,
  urlcolor=black
}

\newtheorem*{conclusion}{Вывод}
\newtheorem{theorem}{Теорема}
\newtheorem{lemma}{Лемма}
\newtheorem*{corollary}{Следствие}

\theoremstyle{definition}
\newtheorem*{problem}{Задача}
\newtheorem{claim}{Утверждение}
\newtheorem{exercise}{Упражнение}
\newtheorem{definition}{Определение}
\newtheorem{example}{Пример}

\theoremstyle{remark}
\newtheorem*{remark}{Замечание}

\newcommand{\doublearrow}{\twoheadrightarrow}
\renewcommand{\le}{\leqslant}
\renewcommand{\ge}{\geqslant}
\newcommand{\eps}{\varepsilon}
\renewcommand{\phi}{\varphi}
\newcommand{\ndiv}{\centernot\mid}

\MakePerPage{footnote}
\renewcommand*{\thefootnote}{\fnsymbol{footnote}}

\newcommand{\resetcntrs}{\setcounter{theorem}{0}\setcounter{definition}{0}
\setcounter{claim}{0}\setcounter{exercise}{0}}

\DeclareMathOperator{\aut}{aut}
\DeclareMathOperator{\cov}{cov}
\DeclareMathOperator{\argmin}{argmin}
\DeclareMathOperator{\argmax}{argmax}
\DeclareMathOperator*\lowlim{\underline{lim}}
\DeclareMathOperator*\uplim{\overline{lim}}
\DeclareMathOperator{\re}{Re}
\DeclareMathOperator{\im}{Im}

\frenchspacing


\begin{document}

\section{Введение}

Рассматриваем задачу $f_0(x) \rightarrow extr, x \in A \subset X, f_0(x): X
\rightarrow \mathbb{R}$.

Темы и~сюжеты, которые будут обсуждаться:
\begin{itemize}
	\item Конечномерный принцип Лагранжа
	\item Дифференцирование в~нормированных пространствах и~бесконечномерный
		принцип Лагранжа
	\item Условия первого порядка в~вариационном исчислении, простейшая задача,
		изопериметрическая задача, задача с~подвижными концами
	\item Условия второго порядка для простейших задач классического вариационного
		исчисления
	\item Задача оптимального управления и~принцип максимума Понтрягина
	\item Выпуклые задачи, алгоритмы.
\end{itemize}

\section{Конечномерный принцип Лагранжа}

Для начала рассмотрим случай $f_0: \mathbb{R}^n \rightarrow \mathbb{R}$
с~ограничениями-равенствами и~неравенствами. $f_j: \mathbb{R}^n \rightarrow
\mathbb{R}$, $f_0(x) \rightarrow \min, f_j(x) \le 0, j = 1, \ldots, m', f_k(x) =
0, k = m' + 1, \ldots, m$.

\begin{theorem}[Ферма, 1638]
	Пусть $X$~--- банахово, $f_0$ дифференцируема по Фреше. Тогда если $x_0$~---
	locmin функции $f_0$, то $f'(x_0) = 0$.
\end{theorem}
\begin{proof}
	$f(x_0 + h) = f(x_0) + f'(x_0)h + o(h)$. $f'(x_0) \ne 0 \Rightarrow \exists
	h_0: f'(x_0)h < 0 \Rightarrow $ для достаточно малого $\lambda > 0 \rightarrow
	f(x_0 + \lambda h_0) < f(x_0)$, противоречие.
\end{proof}

Соответственно при исследовании мы решали $f_0'(x) = 0$, получали множество
точек, считали Гессиан $\left<f_0''(x)h, h\right> = \left(\frac{\partial^2
f_0}{\partial x_i \partial x_j}\right)$. Необходимо, чтобы он был неотрицательно
определён, положительной определённости же достаточно для минимума.
Соответственно вопрос сводился к~линейной алгебре, где мы пользовались критерием
Сильвестра.

\begin{theorem}[Теорема Брауэра]
	Пусть в~конечномерном пространстве $f: B_r(x_0) \rightarrow B_r(x_0)$~---
	непрерывная. Тогда $\exists \hat{x} \in B_r(x_0), f(\hat{x}) = \hat{x}$.
\end{theorem}
\begin{corollary}[Теорема об $\eps$-сдвиге]
	Пусть $\phi: B_r(0) \rightarrow \mathbb{R}^n$ непрерывная, такая что $|\phi(y)
	- y| \le \eps \Rightarrow \forall a \in B_{r-\eps}(0) \rightarrow \exists
	\hat{y}: \phi(\hat{y}) = a$.
\end{corollary}
\begin{proof}
	$F(y) = a + y - \phi(y)$. Покажем, что $F(B_r(0)) \subset B_r(0)$. В~самом
	деле $|F(y)| \le |a| + |y - \phi(y)| \le r - \eps + \eps = r$. По теореме
	Брауэра $\exists \hat{y}: F(\hat{y}) = \hat{y} \Rightarrow a + \hat{y} -
	\phi(\hat{y}) = \hat{y} \Rightarrow a = \phi(\hat{y})$.
\end{proof}

\begin{theorem}[Правило множителей Лагранжа для задач с~равенством]
	Рассмотрим задачу $f_0(x) \rightarrow \min, f_j(x) = 0, j = 1, \ldots, m$.
	Пусть $x_0$~--- locmin, $x \in \mathbb{R}^n, f_j \in D(\mathbb{R}^n)$. Тогда
	$\exists \lambda \in \mathbb{R}^{m+1}: |\lambda| \ne 0, L_x'(x_0) = 0$, где
	$L(x, \lambda) = \sum\limits_{j=0}^m \lambda_j f_j(x)$.
\end{theorem}
\begin{proof}
	Пусть $x_0$~--- locmin, $f(x_0) = 0$. Рассмотрим множество $Y = \{(f_0'(x_0)h,
	\ldots, f_m'(x_0)h) \mid h \in \mathbb{R}^n\} \subset \mathbb{R}^{m+1}$~---
	линейное подпространство.	Рассмотрим два случая:

	1) $Y \ne \mathbb{R}^{m+1}$, тогда $\exists \lambda, |\lambda| \ne 0: \lambda
	\bot Y \Rightarrow \sum \lambda_j f_j'(x_0) = 0$.

	2) $\exists h_j: f_j'(x_0) h_k = \delta_{jk}$. Рассмотрим $\phi: B_r(0)
	\rightarrow \mathbb{R}^{m+1}$, $(y_0, \ldots, y_m) \mapsto (f_0(x_0 + \sum y_j
	h_j), \ldots, f_m(x_0 + \sum y_j h_j))$. Пусть $r > 0$. $f_k(x_0 + \sum y_j
	h_j) - y_k = 0 + \left<f_k'(x_0), y_k h_k \right> + o(y) - y_k = o(y)$.

	$\phi(y) - y = o(y), y \rightarrow 0, |\phi(y) - y| < \frac{r}{2} \forall y
	\in B_r(0)$, по лемме об $\eps$-сдвиге $\exists \hat{y} \in B_r(0):
	\phi(\hat{y}) = \left( -\frac{r}{2}, 0, 0\right)$, противоречие.
\end{proof}

\end{document}
