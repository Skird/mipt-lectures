\documentclass{article}
\usepackage[utf8x]{inputenc}
\usepackage[english,russian]{babel}
\usepackage{amsmath,amscd}
\usepackage{amsthm}
\usepackage{amsfonts}
\usepackage{amssymb}
\usepackage{cmap}
\usepackage{centernot}
\usepackage{enumitem}
\usepackage{perpage}
\usepackage{chngcntr}
%\usepackage{minted}
\usepackage[bookmarks=true,pdfborder={0 0 0 }]{hyperref}
\usepackage{indentfirst}
\hypersetup{
  colorlinks,
  citecolor=black,
  filecolor=black,
  linkcolor=black,
  urlcolor=black
}

\newtheorem*{conclusion}{Вывод}
\newtheorem{theorem}{Теорема}
\newtheorem{lemma}{Лемма}
\newtheorem*{corollary}{Следствие}

\theoremstyle{definition}
\newtheorem*{problem}{Задача}
\newtheorem{claim}{Утверждение}
\newtheorem{exercise}{Упражнение}
\newtheorem{definition}{Определение}
\newtheorem{example}{Пример}

\theoremstyle{remark}
\newtheorem*{remark}{Замечание}

\renewcommand{\le}{\leqslant}
\renewcommand{\ge}{\geqslant}
\newcommand{\eps}{\varepsilon}
\renewcommand{\phi}{\varphi}
\newcommand{\ndiv}{\centernot\mid}

\MakePerPage{footnote}
\renewcommand*{\thefootnote}{\fnsymbol{footnote}}

\newcommand{\resetcntrs}{\setcounter{theorem}{0}\setcounter{definition}{0}
\setcounter{claim}{0}\setcounter{exercise}{0}}

\DeclareMathOperator{\aut}{aut}
\DeclareMathOperator{\cov}{cov}
\DeclareMathOperator{\chos}{ch}
\DeclareMathOperator{\argmin}{argmin}
\DeclareMathOperator{\argmax}{argmax}
\DeclareMathOperator*\lowlim{\underline{lim}}
\DeclareMathOperator*\uplim{\overline{lim}}
\DeclareMathOperator{\re}{Re}
\DeclareMathOperator{\im}{Im}

\frenchspacing


\begin{document}

\section{Центральная турановская плотность}

Из оценки $t(b, k) \le \left( \frac{k-1}{b-1}\right)^{k-1}$ следует при $b = k +
1$, что $t(k + 1, k) \le \frac{1}{e}(1 + o(1))$.

\begin{theorem}
	$t(2k + 1, 2k) \le \frac{C_{2k}^k}{2^{2k}} =
	O\left(\frac{1}{\sqrt{k}}\right)$.
\end{theorem}
\begin{proof}
	Пусть $V = \{1, \ldots, n\}$, а~$i_1 < \ldots < i_{2k}$~--- упорядоченный
	набор чисел. Тогда объявляем $(i_1, \ldots, i_{2k})$ ребром, если во множестве
	$\{i_1 + 1, i_2 + 2, \ldots, i_{2k} + 2k\}$ ровно~$k$ чисел чётные.

	Обозначим за $\mathcal{A}$ множество таких наборов и~проверим, что
	$\mathcal{A}$~--- это $(n, 2k + 1, 2k)$-система. Будем делать это индукцией по
	$k$. База $k = 1$ следует из того, что среди любых трёх чисел $a_1 < a_2 <
	a_3$ найдутся два числа одной чётности, которые будут образовывать ребро.

	Пусть для $l \le k - 1$ всё доказано. Пусть $l = k, a_1 < \ldots < a_{2k+1}$.
	Если существует пара соседних чисел $a_j, a_{j+1}$ одной чётности, то удалим
	их из набора и~к~оставшимся применим индукцию. По её предположению найдётся
	поднабор $a_1' < \ldots < a_{2k-2}'$ такой, что при добавлении индексов среди
	них будет ровно $k-1$ чётных чисел.

	Теперь добавим к~этому поднабору удалённые числа $a_j, a_{j+1}$. Получаем
	$a_1' < \ldots < a_t' < a_j < a_{j+1} < a_{t+1}' < \ldots < a_{2k-2}'$. При
	добавлении $(1, 2, \ldots, 2k)$ получаем $a_1' + 1 < \ldots < a_t' + t < a_j +
	t + 1 < a_{j+1} + t + 2 < a_{t+1}' + t + 3 < \ldots < a_{2k-2}' + 2k$, где,
	очевидно, будет половина чётных и~половина нечётных чисел, так как $a_{j} + t
	+ 1$ и~$a_{j+1} + t + 2$ имеют разную чётность.

	Если же чётность постоянно меняется, то достаточно удалить $a_{k+1}$.

	Оценим теперь $|\mathcal{A}|$. Пусть $a_1 < \ldots < a_{2k}$~--- ребро из
	$\mathcal{A}$, $x_i \equiv a_i \pmod 2, x_i \in \{0, 1\}$. Положим $b_i =
	\frac{a_i - x_i}{2} \in [0; \frac{n}{2}]$. Тогда набор $b_1, \ldots, b_{2k}$
	может быть выбран $\le C_{\frac{n}{2}}^{2k} (1 + o(1))$ способами. Вектор
	$(x_1, \ldots, x_{2k})$ выбирается $C_{2k}^k$ способами. Значит $|\mathcal{A}|
	\le C_{\frac{n}{2}}^{2k}(1 + o(1)) C_{2k}^k = C_n^{2k} 2^{-2k} (1 + o(1))$,
	откуда $t(2k + 1, 2k) \le \lim\limits_n \frac{|\mathcal{A}|}{C_n^{2k}} \le
	C_{2k}^k 2^{-2k} = O\left( \frac{1}{\sqrt{k}}\right)$.
\end{proof}

Насколько хороша эта оценка? Известны такие результаты:
\begin{itemize}
	\item $0.409 \approx \frac{7 - \sqrt{21}}{6} \le t(3, 4) \le \frac{4}{9}
		\approx 0.44$. Предположение состоит в~том
	\item Сидоренко (1982, 1987) $\frac{1}{k} \le t(k+1,k) \le \frac{\ln k}{2k}(1
		+ o(1))$.
	\item для $b = k + a, a = const$ оценка Франкла-Рёдля (1985) $t(k+a,k) \le
		\frac{a(a + 4 + o(1)) \ln k}{C_k^a}$.
	\item Жиро (1997) $t(k+1,k) \ge \frac{2}{k\left(1 +
		\sqrt{\frac{k}{k+4}}\right)}$.
	\item для $b \ge k + \frac{k}{\log_2 k}$ оценка Сидоренко $t(b, k) \le
		\frac{(b - k + 1)(1 + o(1)) \ln C_b^k}{C_b^k}$.
\end{itemize}

\section{Нижние оценки турановской плотности}

\begin{claim}
	$T(n, b, k) \ge \frac{C_n^k}{C_b^k}$.
\end{claim}
\begin{proof}
	Любое $k$-подмножество представляет не более $C_{n-k}^{b-k}$ подмножеств. Если
	есть $(n, b, k)$ система, то все подмножества представлены, значит $|E(H)|
	C_{n-k}{b-k} \ge C_n^b \Rightarrow |E(H)| \ge \frac{C_n^k}{C_b^k}$.
\end{proof}

В~терминах турановской плотности $t(b,k) \ge \frac{1}{C_b^k} \approx b^{-k}$.

\begin{theorem}[Спенсер]
	Пусть $n \ge (b - 1) \frac{k}{k - 1}$. Тогда $T(n, b, k) \ge
	\left(\frac{n}{k}\right)^k \left( \frac{k-1}{b-1} \right)^{k-1}$.
\end{theorem}
\begin{proof}
	Пусть $H = (V, E)$~--- это $(n, b, k)$ система, то есть $\alpha(H) < b$.

	Пусть $X$~--- случайное подмножество, выбранное схемой Бернулли с~параметром
	$p$. $E|X| = np$. Пусть $Y$~--- это число рёбер, полностью вошедших в~$X$,
	$EY = |E| p^k$. Удалим по одной вершине из каждого ребра, полностью попавшего
	в~$X$, тогда останется $X^\ast$~--- независимое, значит $|X^\ast| \le b - 1
	\Rightarrow b - 1 \ge E|X^\ast| \ge E(|X| - Y) = np - |E|p^k \Rightarrow |E|
	\ge (np - b + 1)p^{-k}$. Максимизируя это по $p$, получаем, что $p =
	\frac{(b-1)k}{n(k-1}$, а~$|E| \ge \left( \frac{(b-1)k}{k-1} -
	(b-1)\right)\left( \frac{(b-1)k}{n(k-1)}\right)^{-k} = \left( \frac{n}{k}
	\right)^k \left( \frac{k-1}{b-1} \right)^{k-1}$.
\end{proof}

\begin{claim}
	Если $H$~--- $k$-однородный гиперграф, на $n$ вершинах, тогда
	$\alpha(H) \ge \frac{k-1}{k} \frac{n}{t(H)^{\frac{1}{k-1}}}$.
\end{claim}
\begin{proof}
	Пусть $\alpha(H) = b - 1$, тогда $H$~--- это $(n,b,k)$ система, тогда по
	теореме Спенсера
	\begin{equation*}
		\frac{n t(H)}{k} = |E(H)| \ge T(n, b, k) \ge
		\left(\frac{n}{k}\right)^k \left( \frac{k-1}{b-1} \right)^{k-1} \Rightarrow
		b - 1 \ge \frac{n}{k}(k-1) \frac{1}{t(H)^{\frac{1}{k-1}}}
	\end{equation*}
\end{proof}

\begin{claim}
	$t(b, k) \approx \frac{1}{b^{k-1}}$ при $k = const, b \rightarrow \infty$.
\end{claim}

\section{Аналоги теоремы Турана для разреженных графов и~гиперграфов}

\begin{theorem}[Айтаи, Комлош, Семереди, Ширер]
	Пусть $f(t) = \frac{t \ln t - t + 1}{(t-1)^2}, f(0) = 1, f(1) = \frac{1}{2}$.
	Тогда, если $G$~--- граф на $n$ вершинах без треугольников, то $\alpha(G) \ge
	n f(t(G))$.
\end{theorem}
\begin{proof}
	Заметим, что для $x \ge 0$ $f(x)$ непрерывна, кроме того $f(x) \in (0; 1)$ при
	$x > 0, f'(x) < 0, f''(x) \ge 0$. Также $f(x)$ является решением уравнения $(x
	+ 1) f(x) = 1 + (x - x^2)f'(x)$.

	Индукция по числу вершин в~$G$. Обозначим $t = t(G)$. Если $n \le
	\frac{t}{f(t)}$, то всё доказано, так как~соседи одной вершины являются
	независимыми.

	Пусть теперь $v$~--- произвольная вершина, $d_1$~--- её степень, $d_2$~---
	средняя степень её соседей. Покажем, что можно выбрать $v$ так, что
	\begin{equation}\tag{$\ast$}\label{eq:ast}
		(d_1 + 1)f(t) \le 1 + (td_1 + t - 2d_1d_2)f'(t)
	\end{equation}
	Для этого возьмём вершину случайно и~проверим это неравенство в~среднем.
	Пусть $Y$~--- сумма степеней соседей~$v$, $Y = d_1 d_2$.
	$$EY = \frac{1}{n} \sum\limits_v \sum\limits_{u: (u,v) \in E} \deg u =
	\frac{1}{n} \sum\limits_u \deg^2 u \ge t^2.$$

	В~левой части неравенства в~среднем стоит $(t+1)f(t)$, правая часть в~среднем
	не меньше $1 + (t^2 + t - 2t^2)f'(t) = 1 + (t - t^2)f'(t) \ge (t + 1) f(t)$
	в~силу свойств $f$. Значит неравенство в~самом деле выполнено в~среднем,
	значит существует вершина, для которой выполнено указанное требование.

	Удалим $v$ из графа вместе с~её соседями. Останется граф $G'$, с~$|V(G')| =
	n - 1 - d_1, |E(G')| = \frac{nt}{2} - d_1 d_2$, так как $G$ не содержит
	треугольников.

	$t' = t(G') = \frac{2|E(G')|}{|V(G')|} = \frac{nt -
	2d_1d_2}{n-1-d_1}$, значит по индукции $\alpha(G') \ge n' f(t')$. Значит
	$\alpha(G) \ge 1 + \alpha(G') \ge 1 + n' f(t')$. Оценим по формуле Тейлора
	$f(t') \ge f(t) + f'(t)(t' - t)$, так как $f''(t) > 0$, значит
	\begin{multline*}
		\alpha(G) \ge
		1 + n'(f(t) + f'(t)(t' - t)) = 1 + n'f(t) + n' f(t) t' - n' t f'(t) =\\
		1 + (n - 1 - d_1) f(t) + f'(t)(nt - 2d_1d_2) - (n - 1 - d_1)tf'(t) =\\
		1 + (n - 1 - d_1) f(t) + (t + td_1 - 2d_1d_2)f'(t) \ge \eqref{eq:ast} \ge\\
		(n - 1 - d_1) f(t) + (d_1 + 1) f(t) = n f(t).
	\end{multline*}
\end{proof}

Пусть $R(s, t)$~--- число Рамсея, то есть $\min\{n: \forall G = (V, E), |V| = n
\rightarrow \alpha(G) \ge t \vee \omega(G) \ge s\}$.

\begin{claim}
	$R(3, t) \le \frac{t^2}{\ln t} (1 + o(1))$.
\end{claim}
\begin{proof}
	Если $G = (V, E), |V| = n$. Если $\omega(G) \ge 3$, то есть треугольник, иначе
	$\omega(G) < 3$ и~$t(G) \ge t$, тогда $\alpha(G) \ge t(G) \ge t$. Если
	$\omega(G) < 3, t(G) < t$, то по теореме Ширера $\alpha(G) \ge nf(t) = \frac{n
	\ln t}{t} (1 + o(1)) \ge t$, если $n \ge \frac{t^2}{\ln t} (1 + o(1))$.
\end{proof}

Результат Кима (1995), улучшенный Бошаном, Кивашем, Гриффитсом (2013):
$R(3, t) \ge \frac{1}{4} \frac{t^2}{\ln t} (1 + o(1))$.

\end{document}
