\documentclass{article}
\usepackage[utf8x]{inputenc}
\usepackage[english,russian]{babel}
\usepackage{amsmath,amscd}
\usepackage{amsthm}
\usepackage{mathtools}
\usepackage{amsfonts}
\usepackage{amssymb}
\usepackage{cmap}
\usepackage{centernot}
\usepackage{enumitem}
\usepackage{perpage}
\usepackage{chngcntr}
%\usepackage{minted}
\usepackage[bookmarks=true,pdfborder={0 0 0 }]{hyperref}
\usepackage{indentfirst}
\hypersetup{
  colorlinks,
  citecolor=black,
  filecolor=black,
  linkcolor=black,
  urlcolor=black
}

\newtheorem*{conclusion}{Вывод}
\newtheorem{theorem}{Теорема}
\newtheorem{lemma}{Лемма}
\newtheorem*{corollary}{Следствие}

\theoremstyle{definition}
\newtheorem*{problem}{Задача}
\newtheorem{claim}{Утверждение}
\newtheorem{exercise}{Упражнение}
\newtheorem{definition}{Определение}
\newtheorem{example}{Пример}

\theoremstyle{remark}
\newtheorem*{remark}{Замечание}

\newcommand{\doublearrow}{\twoheadrightarrow}
\renewcommand{\le}{\leqslant}
\renewcommand{\ge}{\geqslant}
\newcommand{\eps}{\varepsilon}
\renewcommand{\phi}{\varphi}
\newcommand{\ndiv}{\centernot\mid}

\MakePerPage{footnote}
\renewcommand*{\thefootnote}{\fnsymbol{footnote}}

\newcommand{\resetcntrs}{\setcounter{theorem}{0}\setcounter{definition}{0}
\setcounter{claim}{0}\setcounter{exercise}{0}}

\DeclareMathOperator{\aut}{aut}
\DeclareMathOperator{\cov}{cov}
\DeclareMathOperator{\argmin}{argmin}
\DeclareMathOperator{\argmax}{argmax}
\DeclareMathOperator*\lowlim{\underline{lim}}
\DeclareMathOperator*\uplim{\overline{lim}}
\DeclareMathOperator{\re}{Re}
\DeclareMathOperator{\im}{Im}

\frenchspacing


\begin{document}

\section{Графы без больших клик}


Далее мы займёмя графами, не содержащими $K_n$ и~покажем, что $\alpha(G) \ge
c \frac{n \ln t}{t \ln \ln t}$, что является нетривиальной оценкой по сравнению
с~теоремой Турана. Для начала вспомним некоторые вещи:

\begin{definition}
	Пусть $X$~--- случайная величина или вектор с~плотностью $p$. Тогда
	энтропия это $H(X) = E\left[ -\log_2 p(x) \right]$.
\end{definition}

Простые свойства:
\begin{itemize}
	\item $H(X) \ge 0$, $H(X) = 0 \Leftrightarrow X = a_j$ п. н.
	\item $H(X) \le \log_2 n$
	\item $H(X_1, \ldots, X_m) \le \sum\limits_{j=1}^m H(X_j)$
	\item $h(x) = -x \log_2 x - (1 - x) \log_2 (1 - x)$~--- это энтропия
		распределения $Bern(p)$. Притом $h(x)$ выпукла вверх.
\end{itemize}

\begin{lemma}
	Пусть $G = (V, E), |V| = n, \omega(G) < r, r \ge 3$. Обозначим через $i(G)$
	число независимых множеств в~$G$, а~$\overline\alpha(G)$~--- их средний
	размер. Тогда
	$$ \overline\alpha(G) \ge c(r) \frac{\ln i(G)}{\ln \ln i(G)}, $$
	где $c(r) > 0$ зависит только от $r$.
\end{lemma}
\begin{proof}
	Пусть $S$~--- случайное (равномерно выбранное) независимое множество. Тогда
	$H(S) = \log_2 i(G)$. Считаем, что $V(G) = \{1, \ldots, n\}$. Обозначим $\phi
	= \frac{\overline\alpha(G)}{n}, \alpha = \alpha(G)$. Тогда выполнено
	следующее:
	\begin{itemize}
		\item $i(G) \le 2^{nh(\phi)}$
		\item $\overline\alpha(G) = n\phi$
		\item $i(G) \ge 2^\alpha$
	\end{itemize}
	Последние два пункта очевидны, докажем первый. Для $\forall v \in V(G)$
	положим $X_v = I\{v \in \}$. $\forall \beta \in \{0, 1\}^n, \beta = (\beta_1,
	\ldots, \beta_n)$. $P(X_1 = \beta_1, \ldots, X_n = \beta_n) = \frac{1}{i(G)}
	I(\beta \text{ определяет независимое множество})$. Значит
	$$\log_2 i(G) = H(X_1, \ldots, X_n) \le \sum\limits_{j=1}^n H(X_j) =
	\sum\limits_{j=1}^n h\left(\frac{i_j(G)}{i(G)}\right),$$
	где $i_j(G)$~--- число независимых множеств, содержащих вершину $j$. По
	неравенству Йенсена получаем: $$\log_2 i(G) \le n h\left(\frac{1}{n}
	\sum\limits_{j=1}^n \frac{i_j(G)}{i(G)}\right) = n h\left(\frac{1}{n}
	\overline\alpha(G)\right) = nh(\phi).$$

	Из первых двух неравенств $\overline\alpha(G) = n\phi = \frac{\phi}{h(\phi)} n
	h(\phi) \ge \frac{\phi}{h(\phi)} \log_2 i(G)$. Заметим, что $\frac{x}{h(x)}$
	возрастает по $x$. Из первого и~третьего неравенства получаем, что
	$nh(\phi) \ge \alpha$. По условию $\omega(G) < r, \alpha(G) < \alpha + 1$,
	значит $n < R(r, \alpha + 1) \le (\alpha + 1)^{r-1} \Rightarrow \alpha + 1 >
	n^\frac{1}{r-1}$. $h(\phi) \ge \frac{1}{n} \left( n^\frac{1}{r-1} - 1\right) =
	\frac{1 + o(1)}{n^\frac{r-2}{r-1}}$.

	Если $-x \ln x \ge \beta$, то $x \ge \frac{(1 + o(1)) \beta}{\ln \beta}$.
	Тогда $\phi \ge \frac{(1 + o(1)) \ln 2}{n^\frac{r-2}{r-1} \frac{r-2}{r-1}
	\ln n}$. Наконец, $\log_2 i(G) \ge \alpha \ge n^\frac{1}{r-1} - 1 \Rightarrow
	\log_2 \log_2 i(G) \ge \frac{1}{r-1} \log_2 n (1 + o(1))$.

	Подставляем всё в~нашу оценку:
	\begin{multline*}
		\overline\alpha(G) \ge \frac{\phi}{h(\phi)} \log_2 i(G) =
		\frac{1}{-\log_2 \phi (1 + o(1))} \log_2 i(G) \ge \frac{\log_2
		i(G)}{\frac{r-2}{r-1} \log_2 n} (1 + o(1)) \ge\\
		\frac{1}{r-2} \frac{\log_2 i(G)}{\log_2 \log_2 i(G)} (1 + o(1)) =
		\frac{1}{r-2} \frac{\ln i(G)}{\ln \ln i(G)} (1 + o(1)).
	\end{multline*}
\end{proof}

\begin{theorem}[Ширер]
	Пусть $G$~--- $d$-регулярный граф на~$n$ вершинах, не содержащий $K_r, r \ge
	4$. Тогда $\alpha(G) \ge c(r) \frac{n \ln d}{d \ln \ln d}$.
\end{theorem}
\begin{proof}
	Пусть $\mathcal{F}$~--- множество всех независимых множеств в~$G$, а~$S$~---
	его случайный элемент (с~равномерным распределением). Для $\forall v \in V(G)$
	положим $p_v = P(v \in S)$. $\overline p_v = \frac{1}{d} E \sum\limits_{u \in
	V} I\left\{ (u, v) \in E(G), u \in S\right\}$~--- среднее число соседей в~$S$.

	Обозначим $T_v$~--- множество соседей $v$. $H_v = G \setminus (\{v\} \cup
	T_v)$. Для $\forall U \subset T_v$ положим $f(U)$~--- вероятность того, что
	случайное независимое множество в~$H_v$ не имеет соседей в~$U$, но соединено
	со~всеми вершинами из $T_v \setminus U$. Пусть $I(U)$~--- число независимых
	множеств внутри $U$. Тогда
	$$p_v = \frac{i_v}{i(G)} = \frac{i_v}{i_v +
	\sum\limits_{U \subset T_v} I(U) f(U) i(H_v)} = \frac{1}{1 +
	\sum\limits_{U \subset T_v} f(U) I(U)}.$$

	Аналогично $$ \overline p_v = \frac{\sum\limits_{U \subset T_v} f(U) I(U)
	\overline\alpha(U)}{d\left(1 + \sum\limits_{U \subset T_v} f(U) I(U)\right)}.
	$$

	Заметим, что если $G$ не содержит $K_r$, то $T_v$ не содержит $K_{r-1}$,
	значит по лемме $\overline\alpha(U) \ge c(r - 1)
	\frac{\ln I(U)}{\ln \ln I(U)}$. Тогда для $\lambda > 0$ рассмотрим сумму
	$w = \sum\limits_{U \subset T_v, |I(U)| \ge \lambda} I(U) f(U)$.

	Положим $y = c(r - 1) \frac{\ln \lambda}{\ln \ln \lambda}$. Тогда
	$p_v \ge \frac{1}{1 + \lambda + w}$ и~$\overline p_v \ge \frac{yw}{(1 +
	\lambda + w) d}$. Первая оценка убывает, а~вторая возрастает, отсюда $p_v +
	\overline p_v \ge \max(p_v, \overline p_v) \ge \max\left( \frac{1}{1 + \lambda
	+ w}, \frac{yw}{(1 + \lambda + w)d}\right) \ge \frac{1}{1 + \lambda +
	\frac{d}{y}}$. Находим оптимальное $\lambda \sim \frac{d}{\ln d}$ и~получаем
	оценку
	$$p_v + \overline p_v \ge \frac{1}{1 + \frac{d}{\ln d} + \frac{d \ln \ln
	d}{c(r-1)\ln d} (1 + o(1))} = c(r-1) \frac{\ln d}{d \ln \ln d} (1 + o(1)).$$

	Осталось заметить, что $\overline\alpha(G) = E|S| = \sum\limits_{v \in V}
	p_v$, а~с~другой стороны $\overline\alpha(G) = E\sum\limits_{v \in V} I\{u
	\in S\} \frac{\deg u}{d} = E \sum\limits_{v \in V} \frac{1}{d}
	\sum\limits_{U \in T_v} I\{u \in S\} = \overline p_v$. Тогда
	$$\alpha(G) \ge \overline \alpha(G) = \frac{1}{2} \sum\limits_{v \in V}
	(p_v + \overline p_v) \ge \frac{c(r-1)}{2} \frac{n \ln d}{d \ln \ln d}
	(1 + o(1))$$.
\end{proof}

\begin{corollary}
	Пусть $G$~--- граф на~$n$ вершинах, не содержащий $K_r$ с~максимальной
	степенью $\Delta(G) = d$. Тогда $\alpha(G) \ge c(r)
	\frac{n \ln d}{d \ln \ln d}$.
\end{corollary}
\begin{proof}
	Пусть $G$~--- не $d$-регулярный. Рассмотрим $G'$~--- копию $G$, не
	пересекающуюся по вершинам. Соединим ребрами $v' \in G', v \in G$, если
	$v'$~--- копия $v$ и~$\deg_G v < d$. Продолжим процедуру, пока граф не станет
	$d$-регулярным. Тогда полученный граф $G^\ast$ не содержит $K_r$ и~$n^\ast =
	|V(G^\ast)| = 2^k n$, тогда по теореме $\alpha(G^\ast) \ge c(r) \frac{n^\ast
	\ln d}{d \ln \ln d}$. Но $\alpha(G^\ast) \le \alpha(G) 2^k$, значит $\alpha(G)
	\ge c(r) \frac{n \ln d}{d \ln \ln d}$.
\end{proof}

\begin{corollary}
	Пусть $G$~--- граф на~$n$ вершинах, не содержащий $K_r$ со~средней
	степенью $t(G) = t$. Тогда $\alpha(G) \ge c(r) \frac{n \ln t}{t \ln \ln t}$.
\end{corollary}
\begin{proof}
	Удалим из~$G$ все вершины степени больше $2t$. Для оставшегося графа $G'$
	максимальная степень не больше $2t$. При этом $|V(G')| = n' \ge \frac{n}{2}$.
	Тогда $\alpha(G) \ge \alpha(G') \ge c(r) \frac{n \ln 2t}{2t \ln \ln 2t} \ge
	c'(r) \frac{n \ln t}{t \ln \ln t}$.
\end{proof}

\begin{corollary}
	Пусть $H$~--- какой-то граф, а~$G$~--- граф на $n$ вершинах со~средней
	степенью вершины $t$, не содержащий подграфов, изоморфных $H$. Тогда
	$\alpha(G) \ge c(H) \frac{n \ln t}{t \ln \ln t}$.
\end{corollary}

Недоказанное пока что предположение состоит в~том, что повторный логарифм можно
убрать аналогично задаче о~графах без треугольников.

\end{document}
