\documentclass{article}
\usepackage[utf8x]{inputenc}
\usepackage[english,russian]{babel}
\usepackage{amsmath,amscd}
\usepackage{amsthm}
\usepackage{amsfonts}
\usepackage{amssymb}
\usepackage{cmap}
\usepackage{centernot}
\usepackage{enumitem}
\usepackage{perpage}
\usepackage{chngcntr}
%\usepackage{minted}
\usepackage[bookmarks=true,pdfborder={0 0 0 }]{hyperref}
\usepackage{indentfirst}
\hypersetup{
  colorlinks,
  citecolor=black,
  filecolor=black,
  linkcolor=black,
  urlcolor=black
}

\newtheorem*{conclusion}{Вывод}
\newtheorem{theorem}{Теорема}
\newtheorem{lemma}{Лемма}
\newtheorem*{corollary}{Следствие}

\theoremstyle{definition}
\newtheorem*{problem}{Задача}
\newtheorem{claim}{Утверждение}
\newtheorem{exercise}{Упражнение}
\newtheorem{definition}{Определение}
\newtheorem{example}{Пример}

\theoremstyle{remark}
\newtheorem*{remark}{Замечание}

\renewcommand{\le}{\leqslant}
\renewcommand{\ge}{\geqslant}
\newcommand{\eps}{\varepsilon}
\renewcommand{\phi}{\varphi}
\newcommand{\ndiv}{\centernot\mid}

\MakePerPage{footnote}
\renewcommand*{\thefootnote}{\fnsymbol{footnote}}

\newcommand{\resetcntrs}{\setcounter{theorem}{0}\setcounter{definition}{0}
\setcounter{claim}{0}\setcounter{exercise}{0}}

\DeclareMathOperator{\aut}{aut}
\DeclareMathOperator{\cov}{cov}
\DeclareMathOperator{\chos}{ch}
\DeclareMathOperator{\argmin}{argmin}
\DeclareMathOperator{\argmax}{argmax}
\DeclareMathOperator*\lowlim{\underline{lim}}
\DeclareMathOperator*\uplim{\overline{lim}}
\DeclareMathOperator{\re}{Re}
\DeclareMathOperator{\im}{Im}

\frenchspacing


\begin{document}

\section{Точность оценки теоремы Ширера}

\begin{lemma}
	Пусть $d = d(n)$~--- последовательность такая, что $4 \le d \le
	o(n^\frac{1}{4})$, тогда существует последовательность графов $G_n$, что
	$V(|G_n|) = n, g(G_n) > 3, t(G_n) \sim d, \alpha(G_n) \le \frac{2n \log
	d}{d}(1 + o(1))$.
\end{lemma}
\begin{proof}
	Рассмотрим случайный граф $G(n, p), p = \frac{d}{n}$.

	Если $X_n = |E(G(n,p))|$, то $EX_n = C_n^2 p \sum \frac{nd}{2},
	DX_n = C_n^2 p(1-p) = o(dn)$. Тогда $P(|X_n - EX_n| \ge n^\frac{2}{3}) \le
	\frac{DX_n}{n^\frac{4}{3}} \rightarrow 0$. Значит асимптотически почти
	наверное $C_n^2p - n^\frac{2}{3} \le X_n \le C_n^2p + n^\frac{2}{3}$, значит
	$X_n \sim \frac{nd}{2}$.

	Оценим вероятность того, что в~$G(n, p)$ есть независимое множество размера
	$\left\lceil \frac{2n \ln d}{d} \right\rceil = b$. $P(\alpha(G(n,p)) \ge b)
	\le C_n^b (1-p)^{C_b^2} \sim \frac{n^b}{b!}(1-p)^{C_b^2} \le \left(
	\frac{ne}{b} \right)^b \exp\left(-pC_b^2\right) = \left( \frac{ne}{b}
	\exp\left(-p\frac{b-1}{2}\right) \right)^b \le \left( \frac{\exp(1 + o(1))}{2
	\ln d}\right)^b \rightarrow b$, если $d \ge 4$.

	Пусть $Y$~--- число треугольников в~$G(n,p)$, $EY = C_n^3 p^3 <
	\frac{d^3}{6}$. $P(Y \ge d^3) \le \frac{1}{6}$.

	Значит, для достаточно больших $n$ существует $G_n'$ на $n$ вершинах, такой
	что число его рёбер есть $\frac{nd}{2}$, $\alpha(G_n') < \frac{2n\ln d}{d}$,
	а~число треугольников не больше $d^3$. Удалим по ребру из~каждого
	треугольника, получим граф $G_n$. $|E(G_n)| \ge |E(G_n')| - d^3 \sim
	\frac{nd}{2}$. $\alpha(G_n) \le \alpha(G_n') + d^3 \sim \frac{2n \ln d}{d}$,
	так как $d = o(n^\frac{1}{4})$.
\end{proof}

\section{Оценки внедиагональных чисел Рамсея}

Рассмотрим $R(s, t)$ в~случае фиксированного $s \ge 4$ и~растущего $t$.

\begin{lemma}\label{indlemma}
	Пусть $G$~--- на $n$ вершинах со средней степенью вершины $t$, числом рёбер
	$e$ и~числом треугольников $t$. Пусть $p \in (0, 1), np \ge 12$. Тогда $G$
	содержит индуцированный подграф $G'$ такой, что $n' > \frac{np}{2}, e' <
	3ep^2, h' < 3hp^3, t' < 6tp$.
\end{lemma}
\begin{proof}
	Нам подойдёт случайный подграф, индуцированный по схеме Бернулли.

	$P\left(n' \le \frac{np}{2}\right) < P\left(|n' - np| > \frac{np}{2}\right)
	\le \frac{np(1-p)}{\frac{n^2p^2}{4}} < \frac{4}{np} \le \frac{1}{3}$.

	$Ee' = ep^2 \Rightarrow P(e' \ge 3ep^2) \le \frac{1}{3}$ по неравенству
	Маркова.

	$Eh' = hp^3 \Rightarrow P(h' \ge 3hp^3) \le \frac{1}{3}$ по неравенству
	Маркова.

	$t' = \frac{2e'}{n'} < \frac{12ep^2}{np} = 6tp$.

	Значит с~положительной вероятностью искомый граф найдётся.
\end{proof}

\begin{lemma}\label{slemm}
	Пусть $\eps \in (0, 2)$, $G$~--- граф на $n$ вершинах со~средней степенью $t$
	и~числом треугольников $h < n t^{2-\eps}$. Тогда $\alpha(G) \ge c' \frac{n \ln
	t}{t}$, где $c'(\eps) > 0$ зависит только от $\eps$.
\end{lemma}
\begin{proof}
	Пусть $\gamma > 0$~--- абсолютная константа такая, что для любого графа без
	треугольников выполнено $\alpha(H) \ge \gamma \frac{|V(H)| \ln t(H)}{t(H)}$.
	Положим $c' = \frac{\eps \gamma}{168}$.

	Если $t < 12^\frac{2}{\eps}$, то все очевидно по теореме Турана. Пусть $t >
	12^\frac{2}{\eps}$. Применим к~$G$ лемму~\ref{indlemma} с~$p =
	t^\frac{\eps}{4} - 1$ и~рассмотрим найденный индуцированный подграф~$G'$.
	В~нём будет не более $3hp^3$ треугольников, то есть меньше, чем $3nt^{2-\eps}
	t^{\frac{1}{2}\eps - 2} p = 3np t^{-\frac{\eps}{2}} \le \frac{np}{4}$.

	Удалим из каждого тругольника $G'$ по вершине, получив $G''$ без
	треугольников.

	$n'' \ge n' - \frac{np}{4} \ge \frac{np}{4}$.

	$e'' \le e' < 3ep^2$.

	$t'' = \frac{2e''}{n''} \le \frac{24 ep^2}{np} = 12p \frac{2e}{n} =
	12t^\frac{\eps}{4}$.

	Тогда по теореме Ширера $\alpha(G) \ge \alpha(G'') \ge \gamma \frac{n'' \ln
	t''}{t''} \ge \gamma \frac{np}{4} \frac{\ln 12
	t^\frac{\eps}{4}}{12^\frac{\eps}{4}} = \frac{\gamma}{48} \frac{n}{t} \ln
	12t^\frac{\eps}{4} \ge \frac{\gamma \eps}{168} \frac{n \ln t}{t}$.
\end{proof}

\begin{theorem}
	Пусть $t \ge t_0(s)$, тогда $R(s, t) \le c^s \frac{t^{s-1}}{(\ln t)^{s-2}}$,
	где $0 < c \le 20000$~--- абсолютная константа.
\end{theorem}
\begin{proof}
	Из теоремы АКС известно, что $\gamma$ можно взять равной $\frac{1}{100}$.
	Выберем в~лемме~\ref{slemm} $\eps = \frac{1}{s - 2} 0.97 < \eps < \frac{1}{s -
	2} 0.99$. Индукция по~$s$ с~уже доказанной базой $s = 3$. Пусть для $s' < s$
	всё доказано, докажем теперь шаг.

	Рассмотрим произвольный граф $G$ на $n$ вершинах, $n \ge c^s
	\frac{t^{s-1}}{(\ln t)^{s-2}}$. Если $\omega(G) \ge s$, то очевидно.
	Если $\omega(G) < s$ и~$m = c^{s-1}\frac{t^{s-2}}{(\ln t)^{s-3}} \le
	\Delta(G)$, то среди соседей вершины максимальной степени также найдется либо
	$s - 1$-клика, либо $t$-независимое множество, то есть все выпонено. То есть
	считаем, что $\omega(G) < s, m > \Delta(G) \ge t(G)$.  Обозначим через $h$
	число треугольников в~$G$.

	Если $h < n m^{2-\eps}$, то по лемме~\ref{slemm} $\alpha(G) \ge c' \frac{n \ln
	m}{m} \ge 20000c' \frac{t}{\ln t} \ln m \ge \frac{50}{48} \eps \frac{t}{\ln t}
	(s - 2) \ln \frac{t}{\ln t} \ge \frac{0.97}{0.96} t \left( \frac{\ln t - \ln
	\ln t}{\ln t}\right) \ge t$ при всех достаточно больших $t$.

	Если $h \ge n m^{2 - \eps}$, то есть вершина, которая содержится в~$3
	m^{2-\eps}$ треугольниках. Пусть $G'$~--- это граф соседей $v$, тогда $|E(G')|
	\ge 3m^{2 - \eps}$, при том, что $|V(G')| < m$. Значит в~$G'$ есть вершина $w$
	степени хотя бы $6 m^{1-\eps}$. Индуцируем на соседях $w$ подграф $G''$.
	Осталось показать, что $|V(G'')| \ge R(s - 2, t)$. При достаточно больших $t$:

	\begin{multline*}
		6 m^{1-\eps} = 6 \cdot \left( 20000^{s-1} \frac{t^{s-2}}{(\ln
		t)^{s-3}}\right)^{1-\eps} \ge 6 \cdot 20000^{(s-1)\frac{s-3}{s-2}}
		\frac{t^{(s-2)-0.99}}{(\ln t)^{(s-3)-0.98\frac{s-3}{s-2}}}>\\
		6 \cdot 20000^{s-2} \frac{t^{s-3}}{(\ln t)^{s-4}} \frac{t^{0.01}}{(\ln t)^{1
		- 0.99 \frac{s-3}{s-2}} 20000^{\frac{1}{s-2}}} > 20000^{s-2}
		\frac{t^{s-3}}{(\ln t)^{s-4}} \ge R(s - 2, t).
	\end{multline*}
\end{proof}

С~помощью локальной леммы можно вывести нижнюю оценку $R(s, t) \ge c(s) \left(
\frac{t}{\ln t}\right)^\frac{s+1}{2}$, что сходится при $s = 3$, но существенно
расходится при больших $s$. В~других результатах улучшена степень логарифма, но
по степени $t$ результатов нет.

Далее мы займёмя графами, не содержащими, например, $K_4$ и~покажем, что
$\alpha(G) \ge c \frac{n \ln t}{t \ln \ln t}$, что является нетривиальной
оценкой по сравнению с~теоремой Турана. Предположение состоит в~том, что
повторный логарифм можно убрать, притом это верно для любых подграфов (не только
$K_4$).

\end{document}
