\documentclass{article}
\usepackage[utf8x]{inputenc}
\usepackage[english,russian]{babel}
\usepackage{amsmath,amscd}
\usepackage{amsthm}
\usepackage{amsfonts}
\usepackage{amssymb}
\usepackage{cmap}
\usepackage{centernot}
\usepackage{enumitem}
\usepackage{perpage}
\usepackage{chngcntr}
%\usepackage{minted}
\usepackage[bookmarks=true,pdfborder={0 0 0 }]{hyperref}
\usepackage{indentfirst}
\hypersetup{
  colorlinks,
  citecolor=black,
  filecolor=black,
  linkcolor=black,
  urlcolor=black
}

\newtheorem*{conclusion}{Вывод}
\newtheorem{theorem}{Теорема}
\newtheorem{lemma}{Лемма}
\newtheorem*{corollary}{Следствие}

\theoremstyle{definition}
\newtheorem*{problem}{Задача}
\newtheorem{claim}{Утверждение}
\newtheorem{exercise}{Упражнение}
\newtheorem{definition}{Определение}
\newtheorem{example}{Пример}

\theoremstyle{remark}
\newtheorem*{remark}{Замечание}

\renewcommand{\le}{\leqslant}
\renewcommand{\ge}{\geqslant}
\newcommand{\eps}{\varepsilon}
\renewcommand{\phi}{\varphi}
\newcommand{\ndiv}{\centernot\mid}

\MakePerPage{footnote}
\renewcommand*{\thefootnote}{\fnsymbol{footnote}}

\newcommand{\resetcntrs}{\setcounter{theorem}{0}\setcounter{definition}{0}
\setcounter{claim}{0}\setcounter{exercise}{0}}

\DeclareMathOperator{\aut}{aut}
\DeclareMathOperator{\cov}{cov}
\DeclareMathOperator{\chos}{ch}
\DeclareMathOperator{\argmin}{argmin}
\DeclareMathOperator{\argmax}{argmax}
\DeclareMathOperator*\lowlim{\underline{lim}}
\DeclareMathOperator*\uplim{\overline{lim}}
\DeclareMathOperator{\re}{Re}
\DeclareMathOperator{\im}{Im}

\frenchspacing


\begin{document}

\section{Введение}

Обзор курса: понятие информации, энтропия Шеннона, колмогоровская сложность,
коды, исправляющие ошибки, коммуникационная сложность.

Примерный адрес страницы курса: \url{/shad/base/Spring2017}.

\section{Тривиум}

Информация по Хартли (1928): текст из $n$ символов из алфавита $\Sigma$
кодируется $\log_2 |\Sigma|^n$ битами (далее логарифмы по умолчанию двоичные).
Определение незамысловатое, но уже полезное.

\begin{example}
	Известно, что $x \in A$, сказано, что $x \in B$. Сколько информации передано?
	Ясно, что было $\log |A|$ информации, стало $\log |A \cap B|$. Значит передано
	$\log \frac{|A|}{|A \cap B|}$ бит.
\end{example}

\begin{example}
	Имеем $n$ монет, одна из них фальшивая, легче остальных. Сколько нужно
	взвешиваний, чтобы её найти? Исходно не хватает $\log n$ информации, каждое
	взвешивание имеет 3 исхода, стало быть меньше, чем за $\frac{\log n}{\log 3}$
	взвешиваний найти не получится.
\end{example}

\begin{example}
	$x \in S_n$~--- перестановка. Можно сравнивать два элемента. Сколько нужно
	сравнений, чтобы найти перестановку?

	$\log n! = \log \sqrt{2\pi n}\left(\frac{n}{e}\right)^n (1 + o(1)) =
	n \log n - n \log e + \frac{1}{2} \log n + O(1)$. Можно ли асимптотически
	приблизиться к~этой границе?

	Естественный алгоритм: сортировка вставками (выглядит довольно оптимально по
	сравнениям, не учитываем сдвиги). Используется $\lceil \log 1 \rceil + \lceil
	\log 2 \rceil + \ldots + \lceil \log n - 1 \rceil \le (n - 1) + \log (n - 1)!
	= OPT(n) + n - 1 - \log n$.
\end{example}

\section{Бинарный поиск}

$A = [1, \ldots, m]$, нужно найти в~нём $y \in \{1, \ldots, m\}$ с~помощью
сравнения $x \overset{?}< y$. Ясно, что нужно $\lceil \log_2 m \rceil$ вопросов.
А~что будет, если оппонент может соврать $1$ раз? Легко придумать алгоритм,
который даёт $3 \log n$ сравнений и~$2 \log n$ (можно и~лучше). Нас будет
интересовать постановка, когда Responder (R) может соврать Questioner'у~(Q)
в~доле вопросов не более $\eps$.

Более формально, игра проходит с~объявлением числа раундов $n$ в~самом начале
игры и~не более $n \eps$ неверных ответов. Вопрос ставится так: при каких $n$
существует стратегия у~Q, которая гарантированно угадывает число? Утверждается,
что можно предъявить алгоритм, работающий за $c(\eps) \log n$ сравнений, чем мы
и~займёмся.

Ясно, что состояние бинарного поиска~--- это вершина бинарного дерева. Устроим
алгоритм не в~виде спуска по дереву, а~в~виде блуждания. Находясь в~вершине,
соответствующей числам $\{l, \ldots, r\}$, зададим вопросы $l \le x, x \le r$?
Если получен хотя бы один отрицательный ответ, пойдём вверх. Далее, кроме
случая, когда мы стоим в~листе, задаём вопрос $m \le x$ и~идём в~нужную сторону.

\begin{claim}
	Лист, в~который мы попадали чаще всего, есть ответ (при достаточно большой
	длине блуждания).
\end{claim}
\begin{proof}
	Подвесим дерево за лист $x$, тогда, если ориентировать рёбра к~этому листу, то
	против этого направления можно идти только если среди ответов на данном шаге
	была ложь. В~самом деле, разбор случаев помогает в~этом убедиться.

	Разделим все наши шаги на шаги вперёд $f$, назад $b$, $l_x$~--- шаги в~листе
	$x$, $l_{other}$~--- шаги в~других листах. Тогда можем утверждать, что $f \le
	b + \log m, b + l_{other} \le \eps n, f + b + l_x + l_{other} = cn, c \ge
	\frac{1}{3}$. Итого $l_x \ge cn - (b + \log n) - \eps n \le cn - \log m - 2
	\eps n$, а~$l_{other} \le \eps n$.

	$cn - \log m - 2 \eps n \ge \eps n \Rightarrow cn - \log m \ge 3 \eps n
	\Rightarrow n \ge \frac{\log m}{c - 3\eps}$.
\end{proof}

Константы, ясное дело, оценены грубо. Также про задачу известно, что при $\eps >
\frac{1}{2}$ всё плохо (ответ найти нельзя), при достаточно малых $\eps <
\frac{1}{10}$ всё совсем хорошо, при промежуточных можно получит вариации
(например, экспоненциальный рост). Известны точные ответы для небольшого
константного числа ошибок, и~для некоторых вариаций (например, оффлайн поиск).
Задача имеет связи с~кодами, исправляющими ошибки.

\end{document}
