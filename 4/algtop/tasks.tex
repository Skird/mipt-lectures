\documentclass{article}
\usepackage[utf8x]{inputenc}
\usepackage[english,russian]{babel}
\usepackage{amsmath,amscd}
\usepackage{amsthm}
\usepackage{amsfonts}
\usepackage{amssymb}
\usepackage{cmap}
\usepackage{centernot}
\usepackage{enumitem}
\usepackage{perpage}
\usepackage{chngcntr}
%\usepackage{minted}
\usepackage[bookmarks=true,pdfborder={0 0 0 }]{hyperref}
\usepackage{indentfirst}
\hypersetup{
  colorlinks,
  citecolor=black,
  filecolor=black,
  linkcolor=black,
  urlcolor=black
}

\newtheorem*{conclusion}{Вывод}
\newtheorem{theorem}{Теорема}
\newtheorem{lemma}{Лемма}
\newtheorem*{corollary}{Следствие}

\theoremstyle{definition}
\newtheorem*{problem}{Задача}
\newtheorem{claim}{Утверждение}
\newtheorem{exercise}{Упражнение}
\newtheorem{definition}{Определение}
\newtheorem{example}{Пример}

\theoremstyle{remark}
\newtheorem*{remark}{Замечание}

\renewcommand{\le}{\leqslant}
\renewcommand{\ge}{\geqslant}
\newcommand{\eps}{\varepsilon}
\renewcommand{\phi}{\varphi}
\newcommand{\ndiv}{\centernot\mid}

\MakePerPage{footnote}
\renewcommand*{\thefootnote}{\fnsymbol{footnote}}

\newcommand{\resetcntrs}{\setcounter{theorem}{0}\setcounter{definition}{0}
\setcounter{claim}{0}\setcounter{exercise}{0}}

\DeclareMathOperator{\aut}{aut}
\DeclareMathOperator{\cov}{cov}
\DeclareMathOperator{\chos}{ch}
\DeclareMathOperator{\argmin}{argmin}
\DeclareMathOperator{\argmax}{argmax}
\DeclareMathOperator*\lowlim{\underline{lim}}
\DeclareMathOperator*\uplim{\overline{lim}}
\DeclareMathOperator{\re}{Re}
\DeclareMathOperator{\im}{Im}

\frenchspacing


\begin{document}

\section{Задача 1}

Имеем: $f_{1,c}(x,y) = y^3 - x^2y + cx^3 + 1$, $f_2(x,y) = y^3 - x^3 + 2$.

Сделаем замену $p = \frac{y}{x}, q = x^3$. Тогда для вычисления $c$ получаем
систему $p^3q - pq + cq + 1 = 0, p^3q - q + 2 = 0$.

Отсюда $q = \frac{2}{1 - p^3}, c = \frac{-p^3 + 2p - 1}{2}$.

Исследуем полученный многочлен. Бифуркационное множество есть образы корней
производной: $\frac{d}{dx} \frac{-p^3+2p-1}{2} = 0 \Rightarrow p = \pm
\sqrt\frac{2}{3} \Rightarrow c = \pm \frac{2}{3}\sqrt\frac{2}{3} - \frac{1}{2}$.
Очень удачно, что два образы корней производной не совпадают, это значит, что
фундаментальная группа базы накрытия есть $F_2$.

Образующие группы монодромии можно получить поднятиями двух петель,
соответвующих образующим $F_2$. Эти образующие получаются двумя транспозициями,
так как над точками бифуркации склеиваются ровно два листа. Однако любые две
транспозиции порождают всё $S_3$, поэтому группа монодромии равна $S_3$.

\section{Задача 2}

В~системе 1 и~2 первое уравнение имеет носитель: $\{(6,0), (4,1), (0,3)\}$. Три
точки лежат на одной прямой, значит система приводима. В~общем положении~$x \ne
0, y \ne 0$, поэтому разделим на $y^3$ и~сделаем замену $p = \frac{x^2}{y}$.
Получим уравнение вида $a_3 p^3 + a_2 p^2 + a_0 = 0$, которое разрешимо по
известной формуле. Итак, $p$ выражается через коэффициенты в~радикалах, значит
можно выразить $y$ как $x^2 p$ и~подставить во второе уравнение.

В~первой системе после подстановки получится носитель $\{6, 5, 4, 2\}$, то есть
уравнение после сокращения на~$x^2$ получается четвертой степени и~решается по
известной формуле.

Во второй системе после подстановки получится носитель $\{6, 5, 4, 2, 1\}$,
который после сокращения на~$x$ превратится в~$\{5, 4, 3, 1, 0\}$. Уравнения
такого вида неразрешимы в~радикалах (неприводимо, невырождено, ожидаемое число
решений 5, есть две точки, отрезок между которыми не лежит в~границе выпуклой
оболочки).

Третья система имеет носители $A_1 = \{(0,2),(1,2),(1,1),(2,0),(1,0),(0,0)\}$ и
$A_2 = \{(2,1), (2,0), (0,0)\}$. Система с~такими носителями неприводима,
невырожденна, также в~первом носителе есть две точки, отрезок между которыми не
лежит в~границе выпуклой оболочки. Осталось найти смешанный объём Минковского,
который в~этом случае получается 6. Значит система в~радикалах неразрешима.

\end{document}
