\documentclass{article}
\usepackage[utf8x]{inputenc}
\usepackage[english,russian]{babel}
\usepackage{amsmath,amscd}
\usepackage{amsthm}
\usepackage{amsfonts}
\usepackage{amssymb}
\usepackage{cmap}
\usepackage{centernot}
\usepackage{enumitem}
\usepackage{perpage}
\usepackage{chngcntr}
%\usepackage{minted}
\usepackage[bookmarks=true,pdfborder={0 0 0 }]{hyperref}
\usepackage{indentfirst}
\hypersetup{
  colorlinks,
  citecolor=black,
  filecolor=black,
  linkcolor=black,
  urlcolor=black
}

\newtheorem*{conclusion}{Вывод}
\newtheorem{theorem}{Теорема}
\newtheorem{lemma}{Лемма}
\newtheorem*{corollary}{Следствие}

\theoremstyle{definition}
\newtheorem*{problem}{Задача}
\newtheorem{claim}{Утверждение}
\newtheorem{exercise}{Упражнение}
\newtheorem{definition}{Определение}
\newtheorem{example}{Пример}

\theoremstyle{remark}
\newtheorem*{remark}{Замечание}

\renewcommand{\le}{\leqslant}
\renewcommand{\ge}{\geqslant}
\newcommand{\eps}{\varepsilon}
\renewcommand{\phi}{\varphi}
\newcommand{\ndiv}{\centernot\mid}

\MakePerPage{footnote}
\renewcommand*{\thefootnote}{\fnsymbol{footnote}}

\newcommand{\resetcntrs}{\setcounter{theorem}{0}\setcounter{definition}{0}
\setcounter{claim}{0}\setcounter{exercise}{0}}

\DeclareMathOperator{\aut}{aut}
\DeclareMathOperator{\cov}{cov}
\DeclareMathOperator{\chos}{ch}
\DeclareMathOperator{\argmin}{argmin}
\DeclareMathOperator{\argmax}{argmax}
\DeclareMathOperator*\lowlim{\underline{lim}}
\DeclareMathOperator*\uplim{\overline{lim}}
\DeclareMathOperator{\re}{Re}
\DeclareMathOperator{\im}{Im}

\frenchspacing


\begin{document}

\section*{Лекция 3. }
\addcontentsline{toc}{section}{Лекция 3. }

\resetcntrs

\section{Разветвленные накрытия}

\begin{definition}
	Разветвленное накрытие~$\pi: E \rightarrow B$~--- это накрытие над~$B
	\setminus B'$, где $B' \subset B$~--- конечно (варианты: дискретно, нигде не
	плотно, имеет меньшую размерность. Все они равносильны, если~$B$ одномерно).

	Минимальное по включению такое $B'$ называется бифуркационным множеством
	разветвленного накрытия~$\pi$.
\end{definition}

\begin{definition}
	Пусть~$f(x)$~--- многозначная комплексная функция ($f(x) = \{y \mid F(x, y) =
	0\}$). Её разветвленным накрытием назовём $p: X \rightarrow \mathbb{C}, p(x,
	y) = x (X = \{(x, y) \mid F(x, y) = 0\})$.
\end{definition}

Хочется сказать, что бифуркационное множество многочлена содержится
в~множестве~$\{x \mid \exists y: (x, y) \in X, \frac{\partial F}{\partial y} =
0\}$. Однако, многочлен~$xy - 1$ имеет бифуркационное множество $\mathbb{C}
\setminus \{0\}$, хотя точек указанного вида нет. Проблема в~том, что у~точки~0
прообраза нет, однако проораз окрестности не пуст, поэтому повторить
доказательство для многочленов вида~$g(y) - x$ не выйдет. Поэтому нам нужно,
чтобы окрестности прообразов точки~$A$ содержали прообраз какой-то
окрестности~$A$, тогда все можно сделать по теореме об обратной функции.

То есть верное утверждение будем таким: бифуркационное множество $B$ функции
$f(x)$ лежит объединении в~$B_1 = \{x_0 \mid F(x_0, \cdot) < \deg_y F\}$ и~$B'$.

\begin{definition}
	Пара отображений~$f: E_1 \rightarrow E_2, g: B_1 \rightarrow B_2$ называется
	гомоморфизмом накрытий~$p_1: E_1 \rightarrow B_1$ и~$p_2: E_2 \rightarrow
	B_2$, если $p_2 \circ f = g \circ p_1$, то есть коммутирует диаграмма:

	$$\begin{CD}
	   E_1     @>f>>       E_2\\
	 @VV p_1 V          @VV p_2 V\\
		 B_1     @>g>>       B_2
	\end{CD}$$
\end{definition}

\begin{example}
	$$p_1: X \rightarrow \mathbb{C}, X = \{G = 0\} \subset \mathbb{C}^2$$
	$$p_2: X' \rightarrow \mathbb{C}, X' = \{f(y) = x\} \subset \mathbb{C}^2$$
\end{example}

Пусть $p: E \rightarrow \mathbb{C}$~--- разветвленное накрытие, $B_1 \supset B$,
где $B$~--- бифуркационное множество. Тогда:
\begin{itemize}
	\item $\pi_1(\mathbb{C} \setminus B) \overset{i_\ast}\leftarrow
		\pi_1(\mathbb{C} \setminus B_1)$, где~$i: \mathbb{C} \setminus B_1$
		вкладывает в~$\mathbb{C} \setminus B$, так как локальная связность не
		нарушается от выкидывания дискретного набора точек. $i_\ast$ является вместе
		с~тем эпиморфизмом.
	\item Пусть есть две группы монодромии $G_1 = G_{p\mid_{\mathbb{C} \setminus
		B}}$, $G_2 = G_{p\mid_{\mathbb{C} \setminus B_1}}$. $$G_1 =
		\psi_p(\pi_1(\mathbb{C} \setminus B)) = \psi_p(i_\ast(\pi_1(\mathbb{C}
		\setminus B_1))), G_2 = \psi_p(\pi_1(\mathbb{C} \setminus B_1))$$.

		Поскольку $i_\ast$ ничего не делает с~петлями с~точки зрения монодромии,
		хотя петель могло стать больше, то $G_1 = G_2$.
\end{itemize}

Таким образом группу монодромии разветвленного накрытия можно корректно
определелить как группу монодромии накрытия на $\mathbb{C} \setminus B$,
где~$B$~--- любое дискретное множество, содержащее бифуркационное.

\end{document}
