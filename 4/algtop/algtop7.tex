\documentclass{article}
\usepackage[utf8x]{inputenc}
\usepackage[english,russian]{babel}
\usepackage{amsmath,amscd}
\usepackage{amsthm}
\usepackage{mathtools}
\usepackage{amsfonts}
\usepackage{amssymb}
\usepackage{cmap}
\usepackage{centernot}
\usepackage{enumitem}
\usepackage{perpage}
\usepackage{chngcntr}
%\usepackage{minted}
\usepackage[bookmarks=true,pdfborder={0 0 0 }]{hyperref}
\usepackage{indentfirst}
\hypersetup{
  colorlinks,
  citecolor=black,
  filecolor=black,
  linkcolor=black,
  urlcolor=black
}

\newtheorem*{conclusion}{Вывод}
\newtheorem{theorem}{Теорема}
\newtheorem{lemma}{Лемма}
\newtheorem*{corollary}{Следствие}

\theoremstyle{definition}
\newtheorem*{problem}{Задача}
\newtheorem{claim}{Утверждение}
\newtheorem{exercise}{Упражнение}
\newtheorem{definition}{Определение}
\newtheorem{example}{Пример}

\theoremstyle{remark}
\newtheorem*{remark}{Замечание}

\newcommand{\doublearrow}{\twoheadrightarrow}
\renewcommand{\le}{\leqslant}
\renewcommand{\ge}{\geqslant}
\newcommand{\eps}{\varepsilon}
\renewcommand{\phi}{\varphi}
\newcommand{\ndiv}{\centernot\mid}

\MakePerPage{footnote}
\renewcommand*{\thefootnote}{\fnsymbol{footnote}}

\newcommand{\resetcntrs}{\setcounter{theorem}{0}\setcounter{definition}{0}
\setcounter{claim}{0}\setcounter{exercise}{0}}

\DeclareMathOperator{\aut}{aut}
\DeclareMathOperator{\cov}{cov}
\DeclareMathOperator{\argmin}{argmin}
\DeclareMathOperator{\argmax}{argmax}
\DeclareMathOperator*\lowlim{\underline{lim}}
\DeclareMathOperator*\uplim{\overline{lim}}
\DeclareMathOperator{\re}{Re}
\DeclareMathOperator{\im}{Im}

\frenchspacing


\begin{document}

\section*{Лекция 7. Системы уравнений I}
\addcontentsline{toc}{section}{Лекция 7. Системы уравнений I}

\section{Системы уравнений, их носитель, понятие разрешимости в~радикалах}

У~нас теперь есть система полиномиальных уравнений: $P_1 = \ldots = P_k = 0$,
притом $P_j \in \mathbb{C}[x_1, \ldots, x_k]$.

Для начала нужно понять, что вообще может играть роль степени многочлена для
системы. Подходы могут быть разные, мы рассмотрим только один из них.

Носитель многочлена $P$ от $k$ переменных $x_1, \ldots, x_k$ есть точки $A(P)
\subset \mathbb{Z}^k$. В~общем случае $P = \sum_{k \in \mathbb{Z}^k} c_k x^k$,
где $x^k = \prod x_j^{k_j}$, а~носитель это $A(P) = \left\{k \in \mathbb{Z}^k
\mid c_k \ne 0\right\}$. Положим также, что $|A(P)| < \infty$, чтобы у~нас был
многочлен, а~не ряд Лорана.

\begin{definition}
	Решением общей системы уравнений с~носителем $A$ называется функция $F =
	F_{A_1, \ldots, A_k}: \mathbb{C}^{A_1} \times \ldots \times
	{\mathbb{C}}^{A_k} \rightarrow 2^{{\mathbb{C}^\ast}^k}$.

	Для простоты, чтобы облегчить замены переменных, будем рассматривать решения
	только ненулевые.

	$F(c^1, \ldots, c^k) = \{x \in \mathbb{C}^k \mid P_{c^1}(x) = \ldots
	P_{c^k}(x) = 0\}, c^j \in \mathbb{C}^{A_j}$.
\end{definition}

\begin{definition}
	Нужно теперь как минимум определить многозначную вектор-функцию, выраженную
	в~радикалах. $g$ является таковой, если каждая её компонента является функцией,
	выраженной в~радикалах, то есть либо константа, либо корень из какой-то
	компоненты, либо сумма, произведение или композиция других выражений
	в~радикалах (частное подразумеваем как корень минус первой степени).

	Общая система с~носителями~$A_1, \ldots, A_k$ разрешима в~радикалах, если
	$\forall c_0 \in \mathbb{C}^A, |F(c_0)| < \infty\,\exists U(c_0) \subset
	\mathbb{C}^A, U(c_0)$~--- открытая по-Зарисски, такая что
	существует разрешимая в~радикалах функция $g: \mathbb{C}^{U(c_0)}$, такая,
	что $G \supset F$.
\end{definition}

\begin{remark}
	Топология Зарисского на множестве $\mathbb{F}^k$ это $\Omega = \{\mathbb{F}^k
	\setminus A \mid \exists l \in \mathbb{N}, P_1, \ldots, P_l \in
	\mathbb{F}[x_1, \ldots, x_k] \mid A = \{P_1 = \ldots = P_l = 0\}\}$.
\end{remark}

\end{document}
