\documentclass{article}
\usepackage[utf8x]{inputenc}
\usepackage[english,russian]{babel}
\usepackage{amsmath,amscd}
\usepackage{amsthm}
\usepackage{mathtools}
\usepackage{amsfonts}
\usepackage{amssymb}
\usepackage{cmap}
\usepackage{centernot}
\usepackage{enumitem}
\usepackage{perpage}
\usepackage{chngcntr}
%\usepackage{minted}
\usepackage[bookmarks=true,pdfborder={0 0 0 }]{hyperref}
\usepackage{indentfirst}
\hypersetup{
  colorlinks,
  citecolor=black,
  filecolor=black,
  linkcolor=black,
  urlcolor=black
}

\newtheorem*{conclusion}{Вывод}
\newtheorem{theorem}{Теорема}
\newtheorem{lemma}{Лемма}
\newtheorem*{corollary}{Следствие}

\theoremstyle{definition}
\newtheorem*{problem}{Задача}
\newtheorem{claim}{Утверждение}
\newtheorem{exercise}{Упражнение}
\newtheorem{definition}{Определение}
\newtheorem{example}{Пример}

\theoremstyle{remark}
\newtheorem*{remark}{Замечание}

\newcommand{\doublearrow}{\twoheadrightarrow}
\renewcommand{\le}{\leqslant}
\renewcommand{\ge}{\geqslant}
\newcommand{\eps}{\varepsilon}
\renewcommand{\phi}{\varphi}
\newcommand{\ndiv}{\centernot\mid}

\MakePerPage{footnote}
\renewcommand*{\thefootnote}{\fnsymbol{footnote}}

\newcommand{\resetcntrs}{\setcounter{theorem}{0}\setcounter{definition}{0}
\setcounter{claim}{0}\setcounter{exercise}{0}}

\DeclareMathOperator{\aut}{aut}
\DeclareMathOperator{\cov}{cov}
\DeclareMathOperator{\argmin}{argmin}
\DeclareMathOperator{\argmax}{argmax}
\DeclareMathOperator*\lowlim{\underline{lim}}
\DeclareMathOperator*\uplim{\overline{lim}}
\DeclareMathOperator{\re}{Re}
\DeclareMathOperator{\im}{Im}

\frenchspacing


\begin{document}

\section{Введение}

\begin{itemize}
	\item <<Теорема Абеля в~задачах и~решениях>>, Алексеев
	\item По алгебраической геометрии: Харцкорн, Шафаревич.
\end{itemize}

\section*{Лекция 1. Теорема Абеля-Руфини}
\addcontentsline{toc}{section}{Лекция 1. Теорема Абеля-Руфини}

\resetcntrs

\section{Полиномиальные уравнения, многозначные функции}

Задача обращения: $f: \mathbb{C} \rightarrow \mathbb{C}$, найти $g: \mathbb{C}
\rightarrow \mathbb{C}, f(g(y)) = y$, притом функция многозначная.

Если $f$~--- многочлен, то знаем формулу для~$\deg f \le 4$.

\begin{definition}[Многозначная функция]
	Многозначная функция~$f$~--- неявная функция $\mathbb{C} \rightarrow
	\mathbb{C}$, заданная полиномиальным уравнением~$\{F = 0\}, F: \mathbb{C}^2
	\rightarrow \mathbb{C}$.

	Либо $f = \{ x \in \mathbb{C}^2 \mid F(x) = 0\}$, либо $f: \mathbb{C}
	\rightarrow 2^\mathbb{C}, f(x) = \{y \in \mathbb{C} \mid F(x, y) = 0\}$.
\end{definition}

Если $f, g$~--- многозначные, то можно определить композицию~$h = g \circ f =
\{ (x, z) \mid \exists y: F(x, y) = G(y, z) = 0\}$. Определение пока что
некорректно, нужно как-то перейти к~одному уравнению.

\begin{definition}[Афинное алгебраическое многообразие]
	Афинное (в~смысле не проективное) алгебраическое многообразие $X \subset
	\mathbb{C}^n$~--- это множество, заданное системой полиномиальных уравнений.

	$X = \{ x \in \mathbb{C}^n \mid F_j(x) = 0, j = 1, \ldots, k\}, F_j:
	\mathbb{C}^n \rightarrow \mathbb{C}$~--- многочлены.
\end{definition}

Проблема: если взять афинное алгераическое многообразие, заданное двумя
уравнениями в~$\mathbb{C}^3$, то его проекция на~$(x, z)$ одним уравнением может
и~не задаваться.

\begin{theorem}[О~проекции афинного алгебраического многообразия]
	Пусть отображение $H: \mathbb{C}^n \rightarrow \mathbb{C}^n$ полиномиальное.
	Тогда $H(X) \subset \mathbb{C}^m$~--- афинное алгебраическое многообразие.
\end{theorem}

\begin{example}
	Многообразие~$X = \{ xy = yz = xz = 0 \}$ имеет коразмерность~2, хочется
	задать его двумя уравнениями, но можно показать, что это невозможно.
\end{example}

\begin{theorem}
	Любое алгебраическое многообразие размерности $n - 1$ в~$\mathbb{C}^n$ можно
	задать одним уравнением.
\end{theorem}

В~этом свете наша композиция определена корректна.

\begin{definition}[Сумма, произведение многозначных функций]
	Если $l: \mathbb{C}^2 \rightarrow \mathbb{C}$, то $l(f, g) =
	\{ (x, l(y_1, y_2) \mid F(x, y_1) = F(x, y_2) = 0 \}$. В~чатности, так можно
	определить сложение и~умножение.

	Так как это проекция многооразия $\{ (x, y_1, y_2, z) \mid F(x, y_1) = F(x,
	y_2) = z - l(y_1, y_2) = 0\}$, то полученный объект~--- это многозначная
	функция.
\end{definition}

\section{Теорема Абеля}

\begin{definition}[Выразимость в~радикалах]
	Функция~$f: \mathbb{C} \rightarrow \mathbb{C}$ выражена в~радикалах, если:
	\begin{itemize}
		\item $f(x) = c, f(x) = \sqrt[n]{x}\ (F(x, y) = x^n - y)$.
		\item Композиция функций, выраженных в~радикалах.
		\item $l(f, g)$, где $l$~--- многочлен, $f, g$ выражены в~радикалах.
	\end{itemize}
\end{definition}

\begin{definition}[Разрешимость в~радикалах]
	$f: \mathbb{C} \rightarrow \mathbb{C}$~--- разрешима в~радикалах, если
	существует $g$~--- многозанчная $\mathbb{C} \rightarrow \mathbb{C}$,
	выраженная в~радикалах, такая что $g(y) \supset f^{-1}(y)$.
\end{definition}

\begin{remark}
	$g(y) = f^{-1}(y)$ не получается, например, в~случае формулы Кардано 6 корней.
	Однако, нас в~принципе не очень смущает наличие побочных корней.
\end{remark}

Если мы можем выразить корни уравнения формулой в~радикалах, то можно разрешить
в~радикалах соответсвующий многочлен, просто подставив~$c_0 - y$ вместо
свободного члена~$c_0$.

\begin{theorem}[Теорема Абеля]
	Многочлен~$f$ общего положения $\deg f \ge 5$ неразрешим в~радикалах.
\end{theorem}

Говоря про общее положение подразумеваем, что это неверно лишь на нигде не
плотном множестве. Более того, в~нашем случае это нигде не плотное множество
будет алгебраическим многообразием меньшей размерности.

\section{Топологическая теория Галуа}

\begin{definition}[Накрытие]
	Накрытие~$\pi: E \rightarrow B$~--- это непрерывное отображение топологических
	пространств, такое, что существует~$F$~--- дискретное топологическое
	пространство, такое что~$\forall x \in B \rightarrow \exists U = U(x):
	\exists \phi_x: \pi^{-1}(U) \rightarrow U \times F$, такое что оно
	осуществляет гомеоморфизм, а~также $p_u(\phi_x(e)) = \pi(e)$, где $p: F \times
	U$~--- проектор на~$U$.
\end{definition}

\begin{remark}
	Если просто попросить, что $\pi^{-1}(U) \cong U \times F$, то накрытием будет
	отображение из интервала в~интервал, которое левую треть отображает в~левую
	половину линейно, правую в~правую, а~середину склеивает в~одну точку.
\end{remark}

$|f^{-1}(y)| = \deg f$, кроме некоторых точек, а~именно тех, где $f(x) = y,
f'(x) = 0$, то есть это верно для всех $y$ кроме так называемых критических
значений многочлена~$B'$.

\begin{claim}
	Пусть $B' = \{y \mid \exists x: f'(x) = 0, y = f(x)\}$. Тогда отображение
	$f\mid_{\mathbb{C} \setminus f^{-1}(B')}$ является накрытием над~$\mathbb{C}
	\setminus B'$.
\end{claim}
\begin{proof}
	Нужно взять окрестность некоторой точки $x \in \mathbb{C} \setminus B'$, взять
	все её прообразы, взять у~них по окрестности, пересечь их образы, позаботиться
	о~том, чтобы они не пересекались и~не содержали плохих точек и~применить
	теорему об обратной функции.
\end{proof}

\end{document}
