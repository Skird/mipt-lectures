\documentclass{article}
\usepackage[utf8x]{inputenc}
\usepackage[english,russian]{babel}
\usepackage{amsmath,amscd}
\usepackage{amsthm}
\usepackage{mathtools}
\usepackage{amsfonts}
\usepackage{amssymb}
\usepackage{cmap}
\usepackage{centernot}
\usepackage{enumitem}
\usepackage{perpage}
\usepackage{chngcntr}
%\usepackage{minted}
\usepackage[bookmarks=true,pdfborder={0 0 0 }]{hyperref}
\usepackage{indentfirst}
\hypersetup{
  colorlinks,
  citecolor=black,
  filecolor=black,
  linkcolor=black,
  urlcolor=black
}

\newtheorem*{conclusion}{Вывод}
\newtheorem{theorem}{Теорема}
\newtheorem{lemma}{Лемма}
\newtheorem*{corollary}{Следствие}

\theoremstyle{definition}
\newtheorem*{problem}{Задача}
\newtheorem{claim}{Утверждение}
\newtheorem{exercise}{Упражнение}
\newtheorem{definition}{Определение}
\newtheorem{example}{Пример}

\theoremstyle{remark}
\newtheorem*{remark}{Замечание}

\newcommand{\doublearrow}{\twoheadrightarrow}
\renewcommand{\le}{\leqslant}
\renewcommand{\ge}{\geqslant}
\newcommand{\eps}{\varepsilon}
\renewcommand{\phi}{\varphi}
\newcommand{\ndiv}{\centernot\mid}

\MakePerPage{footnote}
\renewcommand*{\thefootnote}{\fnsymbol{footnote}}

\newcommand{\resetcntrs}{\setcounter{theorem}{0}\setcounter{definition}{0}
\setcounter{claim}{0}\setcounter{exercise}{0}}

\DeclareMathOperator{\aut}{aut}
\DeclareMathOperator{\cov}{cov}
\DeclareMathOperator{\argmin}{argmin}
\DeclareMathOperator{\argmax}{argmax}
\DeclareMathOperator*\lowlim{\underline{lim}}
\DeclareMathOperator*\uplim{\overline{lim}}
\DeclareMathOperator{\re}{Re}
\DeclareMathOperator{\im}{Im}

\frenchspacing


\begin{document}

\section*{Лекция 8. Системы уравнений II}
\addcontentsline{toc}{section}{Лекция 8. Системы уравнений II}

\section{Основные соображения по упрощению}

Самый общий вопрос: при каких $A$ общая система разрешима в~радикалах. Первые
мысли:
\begin{itemize}
	\item Удобно рассматривать мономиальные замены переменных на комлексном торе
		${\mathbb{C}^\ast}^k$. $x^k = u^{Mk}$, где $M \in GL_k(\mathbb{Z})$.
	\item Случаи, которые можно свести к~более простым:
		\begin{itemize}
			\item $A$ называется \emph{невырожденной}, если $\forall i \rightarrow
				A_i \ni 0 \in \mathbb{Z}^k$, $\left< \bigcup\limits_{j=1}^k A_j
				\right>_\mathbb{Z} = \mathbb{Z}^k$. Вырожденные сводятся к~невырожденным.

				Если $\exists i: 0 \notin A_i$. $\mathbb{C}^{A_j}$ заменяется на
				$\mathbb{C}^{A_j - \{k_0\}}$.

				Если же $\left< \bigcup\limits_{j=1}^k A_j \right>_\mathbb{Z} \ne
				\mathbb{Z}^k$, то сведение делается (почти) мономиальной заменой
				$M: M^{-1} e_j = v_j \Rightarrow M^{-1} = (v_1 \ldots v_k)$. Почти
				потому что замена может быть необратимой.
			\item Пусть $\exists j_1 < \ldots < j_l: \dim \sum\limits_{p=1}^l A_{j_p}
				\le l < n$ (сумма Минковского). Тогда набор $A_1, \ldots, A_n$ называется
				\emph{приводимым}.

				Рассмотрим набор векторов $\alpha_1, \ldots, \alpha_l
				\in \mathbb{Z}^n$, которые порождают $\mathbb{Z}^n \cap \left<
				\sum\limits_{p=1}^l A_{j_p} \right>$. Чтобы сделать замену нам хочется
				достроить этот набор до базиса $\mathbb{Z}^n$ (это возможно не всегда,
				но можно достроить хотя бы просто до базиса подрешётки размерности $n$).

				Рассмотрим мономиальную замену $u^k = x^{Mk}$, где $M = (\alpha_1,
				\ldots, \alpha_n)$, переходя к~системе $P'_{j_s}(u) = P_{j_s}(x)$
				с~носителем $A'_{j_s} = A(P_{j_s}') = M^{-1}(A_{j_s})$. В~частности
				$M^{-1}(\alpha_j) = e_j$. Значит $A'_{j_s} \subset \mathbb{Z}^l \subset
				\left< e_1, \ldots, e_l \right>$.

				Тогда разрешимость системы сводится к~двум вопросам: разрешимость
				системы $l$ уравнений с~носителями $A'_{j_s}, s = 1, \ldots, l$ (кроме
				некоторых случаев, если она неразрешима, то и~большая тоже)
				и~разрешимость системы с~носителями $\{ A'_{j_s} / {\mathbb{Z}^l} \mid s
				> l\}$.
		\end{itemize}
\end{itemize}

Теперь можно сформулировать гипотезу.

\begin{claim}
	Если система невырождена и~неприводима, а~ожидаемое количество решений больше
	4, то она не разрешима в~радикалах.
\end{claim}

\section{Смешанный объём Минковского}

Нужно только уточнить, что понимается под <<ожидаемым числом решений>>.

\begin{theorem}[Бернштейн, Хованский] Для системы общего пололжения
	с~носителями $A_1, \ldots, A_n$ (невырожденной) количество решений
	в~$(\mathbb{C}^\ast)^n$ совпадает со смешанным объёмом Минковского
	$MV_n(\left<A_1\right>, \ldots, \left<A_n\right>)$.
\end{theorem}

Смешанный объём Минковского можно определить многими способами:
\begin{itemize}
	\item Конструктивно. Пусть выпуклые тела $A_1, \ldots, A_n \subset
		\mathbb{R}^n$, $F_A: (\mathbb{R}_+)^n \rightarrow \mathbb{R}$, $F(\lambda_1,
		\ldots, \lambda_n) = V_n(\sum\limits_{j=1}^n \lambda_j A_j)$. Можно
		показать, что $F$~--- гладкая, что даёт нам право рассмотреть
		$\frac{\partial^n F_A}{\partial \lambda_1 \ldots \partial \lambda_n}(0)$
		и~объявить это \emph{смешанным объёмом Минковского} $MV_n(A_1, \ldots,
		A_n)$.
	\item Некоторые свойства:
		\begin{itemize}
			\item $MV(A, \ldots, A) = n! V_n(A)$. В~самом деле~$F_A = V_n((\sum
				\lambda_j) A) = V_n(A) (\sum \lambda_j)^n \Rightarrow
				\frac{\partial^n F_A}{\partial \lambda_1 \ldots \partial \lambda_n} =
				n!$.
			\item $MV$~--- симметрична:
				$$MV_n(A_1, \ldots, A_n) = MV(A_{\sigma(1)}, \ldots, A_{\sigma(n)}),
				\sigma \in S_n.$$
			\item $MV$~--- полилинейна:
				\begin{multline*}
					MV_n(\lambda_1 A_1' + \lambda_2 A_1'', A_2, \ldots, A_n) =\\
					\lambda_1 MV_n(A_1', A_2, \ldots, A_n) + \lambda_2 MV_n(A_1'', A_2,
					\ldots, A_n).
				\end{multline*}
		\end{itemize}
	\item Предыдущих трёх свойств достаточно, чтобы определить функцию на
		множестве $(\Omega_n)^n$ ($\Omega_n$~--- множество выпуклых тел
		в~$\mathbb{R}^n$) однозначно. В~частности:

		$$MV_2(A_1, A_2) = V(A_1 + A_2) - V(A_1) - V(A_2).$$
	\item Предыдущая формула ведёт нас~к~явному определению:
		\begin{multline*}
			MV_n(A_1, \ldots, A_n) = \\
			V_n\left(\sum A_j\right) - \sum_{k=1}^n V_n\left(\sum_{j \ne k} A_j\right)
			+ \ldots + (-1)^{n-1}\sum_k V_n(A_k).
		\end{multline*}
\end{itemize}

\begin{example}
	Найдем ожидаемое число решений системы $P_1(x, y) = ax^3 + bxy + c =
	P_2(x, y) = dx + ey^2 + f$. Выпуклые оболочки носителей~--- два треугольника,
	посчитав площадь суммы и~суммы площадей, получаем 6.
\end{example}

\section{Критерий разрешимости системы в~радикалах}

\begin{theorem}
	Утверждение гипотезы верно для наборов $A_1, \ldots, A_n$, для которых
	$\exists j \exists k_1, k_2 \in A_j: [k_1; k_2] \not\subset \partial \left<
	A_j \right>$.
\end{theorem}

Частный случай такого препятствия~--- линейные уравнения ($A_j$~--- маленький
симплекс, который мономиальной заменой приводится к~стандартному), которые,
казалось бы, отметаются ограничением на невырожденность и~неприводимость,
однако, оказываются, бывают более сложные примеры таких многогранников.

\end{document}
