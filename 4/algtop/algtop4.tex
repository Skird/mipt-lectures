\documentclass{article}
\usepackage[utf8x]{inputenc}
\usepackage[english,russian]{babel}
\usepackage{amsmath,amscd}
\usepackage{amsthm}
\usepackage{mathtools}
\usepackage{amsfonts}
\usepackage{amssymb}
\usepackage{cmap}
\usepackage{centernot}
\usepackage{enumitem}
\usepackage{perpage}
\usepackage{chngcntr}
%\usepackage{minted}
\usepackage[bookmarks=true,pdfborder={0 0 0 }]{hyperref}
\usepackage{indentfirst}
\hypersetup{
  colorlinks,
  citecolor=black,
  filecolor=black,
  linkcolor=black,
  urlcolor=black
}

\newtheorem*{conclusion}{Вывод}
\newtheorem{theorem}{Теорема}
\newtheorem{lemma}{Лемма}
\newtheorem*{corollary}{Следствие}

\theoremstyle{definition}
\newtheorem*{problem}{Задача}
\newtheorem{claim}{Утверждение}
\newtheorem{exercise}{Упражнение}
\newtheorem{definition}{Определение}
\newtheorem{example}{Пример}

\theoremstyle{remark}
\newtheorem*{remark}{Замечание}

\newcommand{\doublearrow}{\twoheadrightarrow}
\renewcommand{\le}{\leqslant}
\renewcommand{\ge}{\geqslant}
\newcommand{\eps}{\varepsilon}
\renewcommand{\phi}{\varphi}
\newcommand{\ndiv}{\centernot\mid}

\MakePerPage{footnote}
\renewcommand*{\thefootnote}{\fnsymbol{footnote}}

\newcommand{\resetcntrs}{\setcounter{theorem}{0}\setcounter{definition}{0}
\setcounter{claim}{0}\setcounter{exercise}{0}}

\DeclareMathOperator{\aut}{aut}
\DeclareMathOperator{\cov}{cov}
\DeclareMathOperator{\argmin}{argmin}
\DeclareMathOperator{\argmax}{argmax}
\DeclareMathOperator*\lowlim{\underline{lim}}
\DeclareMathOperator*\uplim{\overline{lim}}
\DeclareMathOperator{\re}{Re}
\DeclareMathOperator{\im}{Im}

\frenchspacing


\begin{document}

\section*{Лекция 4. Разрешимость группы монодромии}
\addcontentsline{toc}{section}{Лекция 4. Разрешимость группы монодромии}

\resetcntrs

Список фактов с~подсказками:
\begin{itemize}
	\item Если группа транзитивна и~порождена транспозициями, то она есть $S_n$
		(комбинаторный факт)
	\item Группа монодромии транзитивна (нужно, чтобы $\mathbb{C} \setminus
		p^{-1}(B')$ было линейно связно, что верно, так как второе множество
		конечно, тогда все пути можно опустить, чтобы они стали петлями
		в~фундаментальной группе).
	\item Группа монодромии порождена транспозициями (нужно понять, как устроены
		петли в~$\mathbb{C} \setminus p^{-1}(B')$).
	\item Если $E_1$ вложено в~$E_2$ (то есть поднакрытие), то можно индуцировать
		эпиморфизм $i^\ast$ из группы монодромии $G_2 \rightarrow G_1$.
	\item Неразрешимая группа не может быть образом разрешимой при эпиморфизме.
	\item Группа монодромии накрытия, заданного функцей, выраженной в~радикалах,
		разрешима (ближайшая цель).
\end{itemize}

\section{Разрешимость группы монодромии накрытия функции, выраженной
в~радикалах}

Так или иначе, доказывать придется по индукции. База:
\begin{itemize}
	\item $g(x) = c$, $G = S_1$.
	\item $g(x) = \sqrt[n]{x}, X = \{(x, y) \mid x = y^n \}, p: X \rightarrow
		\mathbb{C}, p(x, y) = x, B' = \{0\}, x = f(y) = y^n$.

		$S_{\left\{\sqrt[n]{1}\right\}} \supset G_{p,1} =
		\psi(\underbrace{\pi_1(\mathbb{C} \setminus \{0\})}_{\cong \mathbb{Z}}
		)$.

		$G_{p,1} = \left< \psi([\phi]) \right>, \phi(t) = \exp(2\pi it),
		\psi([\phi])(z) = \widetilde{\phi}_z(1)$.

		Пусть $\widetilde{\phi}(t) =
		z\exp(2\pi i \frac{t}{n}), p \circ \widetilde{\phi}(t) = (z \exp(2\pi i
		\frac{t}{n}))^n = z^n \exp(2\pi it) = \exp(2\pi it) = \phi(t)$.

		Тогда $\widetilde{\phi}$~--- действительно поднятие $\phi$. Таким образом
		$\psi([\phi])(z) = \widetilde{\phi}_z(1) = z \exp(\frac{2\pi i}{n})$. Стало
		быть группа монодромии $\mathbb{Z}_n$~--- разрешима.
\end{itemize}

Для шага нужно две вещи: любой полином от двух разрешенных функций и~их
композиция.

\begin{definition}
	Пусть $p_1, p_2: E_1, E_2 \rightarrow B$~--- два разветвлённых накрытия. Тогда
	$p_1 \oplus p_2: E_3 \rightarrow B, E_3 = \{z_1, z_2 \mid p_1(z_1) =
	p_2(z_2)\} \subset E_1 \times E_2, p(z_1, z_2) = p_1(z_1) = p_2(z_2)$
	называется прямой суммой разветвлённых накрытий.
\end{definition}

Прямая сумма разветвлённых накрытий есть разветвлённое накрытие: нужно выяснить,
в~чём содержится бифуркационное множество.
\begin{claim}
	$B \subset B_1 \cup B_2$.
\end{claim}
\begin{proof}
	Пусть $b \in B \setminus (B_1 \cup B_2)$. Дано: $\exists U_1 \ni b, U_2 \ni
	b, \xi_1, \xi_2, \xi_j: p_j^{-1}(U_j) \rightarrow U_j \times F_j$.

	Рассмотрим тогда $U_3 = U_1 \cap U_2$. $(p_1 \oplus p_2)^{-1}(U_3) = \{(z_1,
	z_2) \in E_1 \times E_2 \mid p_1(z_1) = p_2(z_2) \in U_3\} \subset
	p_1^{-1}(U_3) \times p_2^{-1}(U_3) = V_3$.

	Нам нужно найти $\xi_3: V_3 \rightarrow U_3 \times F_1 \times F_2$. Определим
	её как~$\xi_3(z_1, z_2) = (p_1(z_1) = p_2(z_2), \xi_1(z_1)_2, \xi_2(z_2)_2)$.
	$z_j = \xi_j^{-1}(p_j(z_j), \xi_j(z_j)_2)$, значит это гомеоморфизм.

	Проекция $\xi_3$ на первый сомножитель и~есть $p_1 \oplus p_2$, поэтому
	корректность разветвлённого накрытия доказана.
\end{proof}

\begin{claim}
	$G_{p_1 \oplus p_2} \cong G < G_{p_1} \oplus G_{p_2}$.
\end{claim}
\begin{proof}
	Идея: сопоставить $\sigma = \psi_{p_1 \oplus p_2}([\phi]) \mapsto
	(\psi_{p_1}([\phi]), \psi_{p_2}([\phi])) = (\sigma_1, \sigma_2)$. Нужно
	показать, что отображение определено корректно.

	Утверждение:
	$\psi_{p_1 \oplus p_2}([\phi])(z_1, z_2) = (\sigma_1(z_1), \sigma_2(z_2))$,
	где $\sigma_j = \psi_{p_j}([\phi])$. То есть, $\sigma_3(z_1, z_2) =
	(\sigma_1(z_1), \sigma_2(z_2))$.

	В~самом деле
	$\widetilde{\phi}_{z_1,z_2}(t) = (\widetilde{\phi}_{z_1}(t),
	\widetilde{\phi}_{z_2}(t)) \in E_3$, так как $p_1(\widetilde\phi_{z_1}(t)) =
	\phi(t) = p_2(\widetilde\phi_{z_2}(t))$.

	$\psi_{p_1 \oplus p_2}([\phi])(z_1, z_2) = \widetilde\phi_{z_1,z_2}(1) =
	(\psi_{p_1}([\phi])(z_1), \psi_{p_2}([\phi])(z_2))$.

	Зная это, определеим $\chi: G_{p_1 \oplus p_2} \rightarrow G_{p_1} \oplus
	G_{p_2}$ по формуле $\chi(\sigma_3) = \sigma_{3,1} \oplus \sigma_{3,2}$.

	Из доказанного, это корректный гомоморфизм. Докажем, что это мономорфизм.
	В~самом деле, если образ какого-то элемента тривиален, то и~сам элемент есть
	тривиальная перестановка (обе компоненты тривиальны).
\end{proof}

\begin{lemma}
	Пусть $p_j$~--- накрытие многочлена $f_j$, $p_j: X_j \rightarrow \mathbb{C},
	p_3: X_3 \rightarrow \mathbb{C}$~--- накрытие $f_1 + f_2$. Тогда существует
	эпиморфизм накрытий~$h: E_3 \rightarrow X_3$, где $p_1 \oplus p_2: E_3
	\rightarrow \mathbb{C}, E_3 \subset X_1 \times X_2$.
\end{lemma}
\begin{proof}
	Положим $р((b, x_1), b(b, x_2)) = (b, x_1 + x_2)$. Легко видеть, что $p_3
	\circ h = p_1 \oplus p_2$, а~также, что $h$~--- непрерывна. Более того,
	$h$~--- сюрьекция. Значит~$h$~--- эпиморфизм.
\end{proof}

\begin{remark}
	Аналогичная лемма дословно верна для произведения.
\end{remark}

\begin{lemma}
	Пусть $h: E_1 \rightarrow E_2$ эпиморфизм накрытий $p_j: E_j \rightarrow B$.
	Тогда существует индуцированный эпиморфизм $h_\ast: G_{p_1} \rightarrow
	G_{p_2}$.
\end{lemma}

Из всего этого, $G_{p_1}, G_{p_2}$~--- разрешимы $\Rightarrow G_{f_1 + f_2},
G_{f_1 \cdot f_2}$ разрешимы.

\end{document}
