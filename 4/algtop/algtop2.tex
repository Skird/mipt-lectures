\documentclass{article}
\usepackage[utf8x]{inputenc}
\usepackage[english,russian]{babel}
\usepackage{amsmath,amscd}
\usepackage{amsthm}
\usepackage{amsfonts}
\usepackage{amssymb}
\usepackage{cmap}
\usepackage{centernot}
\usepackage{enumitem}
\usepackage{perpage}
\usepackage{chngcntr}
%\usepackage{minted}
\usepackage[bookmarks=true,pdfborder={0 0 0 }]{hyperref}
\usepackage{indentfirst}
\hypersetup{
  colorlinks,
  citecolor=black,
  filecolor=black,
  linkcolor=black,
  urlcolor=black
}

\newtheorem*{conclusion}{Вывод}
\newtheorem{theorem}{Теорема}
\newtheorem{lemma}{Лемма}
\newtheorem*{corollary}{Следствие}

\theoremstyle{definition}
\newtheorem*{problem}{Задача}
\newtheorem{claim}{Утверждение}
\newtheorem{exercise}{Упражнение}
\newtheorem{definition}{Определение}
\newtheorem{example}{Пример}

\theoremstyle{remark}
\newtheorem*{remark}{Замечание}

\renewcommand{\le}{\leqslant}
\renewcommand{\ge}{\geqslant}
\newcommand{\eps}{\varepsilon}
\renewcommand{\phi}{\varphi}
\newcommand{\ndiv}{\centernot\mid}

\MakePerPage{footnote}
\renewcommand*{\thefootnote}{\fnsymbol{footnote}}

\newcommand{\resetcntrs}{\setcounter{theorem}{0}\setcounter{definition}{0}
\setcounter{claim}{0}\setcounter{exercise}{0}}

\DeclareMathOperator{\aut}{aut}
\DeclareMathOperator{\cov}{cov}
\DeclareMathOperator{\chos}{ch}
\DeclareMathOperator{\argmin}{argmin}
\DeclareMathOperator{\argmax}{argmax}
\DeclareMathOperator*\lowlim{\underline{lim}}
\DeclareMathOperator*\uplim{\overline{lim}}
\DeclareMathOperator{\re}{Re}
\DeclareMathOperator{\im}{Im}

\frenchspacing


\begin{document}

\section*{Лекция 2. }
\addcontentsline{toc}{section}{Лекция 2. }

\resetcntrs

\section{Поднятие}

\begin{lemma}[О~поднятии]
	Для любого пути в~базе $\phi: [0; 1] \rightarrow B$ и~$v_0 \in
	\pi^{-1}(\phi(0))$ существует единственный путь-поднятие:
	$\overline{\phi}_v: [0; 1] \rightarrow E: \overline{\phi}_v(0) = v$
	и~$\phi = \pi \circ \overline{\phi}_v$.
\end{lemma}

\begin{proof}
	Для каждой точки отрезка $\forall t \in [0; 1]\ \exists U_t$~--- окрестность
	точки~$t$ с~таким свойством, что $\phi(U_t) \subset U(\phi(t))$,
	где~$U(\phi(t))$~--- тривиализующая окретсность: $\pi^{-1}(U(\phi(t)))
	\overset{k}\rightarrow U(\phi(t)) \times F$.

	В~силу компактности отрезка выберем конечное подпокрытие: $\exists t_0,
	\ldots, t_N: [0; 1] = \bigcup\limits_{j=0}^{N} U_{t_j}$. Более того, сузим все
	интервалы до отрезков, граничащих по концам: $[0; 1] =
	\bigcup\limits_{j=0}^{N-1} [t_j; t_{j+1}]$.

	Для каждого~$j\ \exists U_j \subset B: \phi([t_j; t_{j+1}]) \subset U_j, k_j:
	\pi^{-1}(U_j) \rightarrow U_j \times F$ (при этом $\pi = k_j \circ p_1$).

	Поднятие тогда определим так: $\overline{\phi} = k_j^{-1}(\phi(t), p_2 \circ
	k(\overline\phi(t_j))$. Это отображение непрерывно, так как непрерывно на
	каждом из замкнутых множеств, которые разбивают весь отрезок.

	$\pi \circ \overline\phi(t) = \pi(k^{-1}(\phi(t), \ldots)) = p_1(\phi(t),
	\ldots) = \phi(t)$.

	Единственность покажем по индукции: если $\overline\phi'$ тоже поднятие и~оно
	совпадает на нескольких первых отрезках, нужно показать, что оно совпадает
	и~на следующем.
\end{proof}
\begin{definition}[Действие фундаментальной группы]
	Действие группы~$\pi_1(B, b_0)$ на множестве~$\pi^{-1}(b_0)$ определим
	формулой $\psi: \pi_1(B, b_0) \rightarrow S(\pi^{-1}(b_0)), \psi([\phi])(v) =
	\overline\phi_v(1)$.
\end{definition}

\begin{claim}
	$\overline{(\phi_1 \phi_2)} = \overline\phi_{1,v_1} \overline\phi_{2,v_0}$,
	где~$v_1 = \overline\phi_{2,v_0}(1)$.
\end{claim}
\begin{proof}
	Отображение является поднятием какого-то пути, если удовлетворяет двум
	свойствам из определения. Первое свойство очевидно, второе: $\pi \circ
	(\overline\phi_{1,v_1}\overline\phi_{2,v_0} = \phi_2$ при $t \in [0;
	\frac{1}{2})$ и~$\phi_1(2t+1)$ иначе.
\end{proof}

Корректность определения:
\begin{itemize}
	\item Гомоморфизм: $\psi([\phi_1][\phi_2])(v) = \psi([\phi_1 \phi_2])(v) =
		\overline{(\phi_1 \phi_2)}_v(1) = \overline\phi_{1,v_1}
		\overline\phi_{2,v}(1) = \overline\phi_{1,v_1}(1) =
		\psi([\phi_1])(\overline\phi_{2,v}(1)) = \psi([\phi_1]) \circ
		\psi([\phi_2])(v)$ где, $v_1 = \overline\phi_{2,v}$
	\item Биективность: обратным будет отображение $\psi([\phi^{-1}])$.
	\item Если $[\phi_1] = [\phi_2]$, то $\overline\phi_{1,v}(1) =
		\overline\phi_{2,v}(1)$ (упражнение).
\end{itemize}

\section{Группа монодромии}

\begin{definition}[Группа монодромии]
	Группа монодромии~$G_{b_0}$ накрытия~$\pi: E \rightarrow B \ni b_0$~--- это
	образ $\psi(\pi_1(B, b_0))$.
\end{definition}

\begin{example}
	У~накрытия $\exp: \mathbb{R} \rightarrow S^1$ выполнено $\psi([\phi])(v) =
	v + 1$, то есть $G_{b_0} \cong \mathbb{Z} \cong \pi(B, b_0)$.
\end{example}

\begin{example}
	У~тривиального накрытия группа монодромии тривиальна, в~то время, как
	фундаментальная группа $\pi_1(B, b_0)$ может быть нетривиальна, то есть нельзя
	сказать, что $\psi$~--- изоморфизм.
\end{example}

\begin{example}
	У~накрытия $P_k: S^1 \rightarrow S^1, P_k(e^{2\pi it}) = e^{2\pi ikt}$
	выполнено $\psi([\phi])(v) = v e^{\frac{2\pi it}{k}}$, то есть $G_{b_0} \cong
	\mathbb{Z}_k$.
\end{example}

\begin{definition}[Изоморфизм накрытий]
	Накрытия~$\pi_i: E_i \rightarrow B_i$, $i \in \{0, 1\}$ изоморфны, если
	существует гомеоморфизмы~$f: E_1 \leftrightarrow E_2, g: B_1 \leftrightarrow
	B_2$, такие что~$\pi_2 \circ f \cong g \circ \pi_1$.
\end{definition}

\begin{remark}[Связь группы монодромии с~фундаментальной группой]
	Если $\phi: [0; 1] \rightarrow B, \phi(0) = b_0, \phi(1) = b_1$, то $G_{b_0}
	\cong G_{b_1}$.
\end{remark}
\begin{remark}[Изоморфизм групп монодромии]
	Если два накрытия изоморфны, то $G_{b_1} \cong G_{g(b_1)}$.
\end{remark}
\begin{proof}
	Рассмотрим два построения.

	Рассмотрим $[\phi] \in \pi_1(b_1, b_0)$ и~$[g(\phi)] \in \pi_1(B_2, b_2)$.
	Отображение определим как	$h: G_{b_1} \rightarrow G_{g(b_1)}, h(\psi([\phi]) =
	\psi(g_\ast([\phi]))$, где $g_\ast$~--- индуцированное отображением $g$
	отображение фундаментальных групп. Нужно показать, что если $[\phi_1] \ne
	[\phi_2]$, то $\psi([\phi_1]) \ne \psi([\phi_2])$.

	Пусть $\sigma \in G_{b_1} \subset S(\pi^{-1}(b_1))$. Определим отображение
	$h(\sigma)(v) = f(\sigma(f^{-1}(v)))$. Нужно показать корректность: $h(\sigma)
	\in G_{b_2}$.

	Если покажем, что эти построения это на самом деле одно и~то же, то оба будут
	корректны.

	Утверждается, что $\psi([g(\phi)])(v) = \overline{g(\phi)}_v(1) =
	f(\overline\phi_{f^{-1}(v)})(1)$. Для этого надо проверить свойства:
	1) $f(\overline\phi_{f^{-1}(v)})(0) = f(f^{-1}(v)) = v$; 2)
	$\pi_2(f(\overline\phi_{f^{-1}(v)})) = g(\pi_1(\overline\phi_{f^{-1}(v)})) =
	g(\phi)$.

	Тогда $\psi([g(\phi)]) = f(\overline\phi_{f^{-1}(v)})(1) =
	f(\psi([\phi])(f^{-1}(v))) = f(\sigma(f^{-1}(v)))$, что нам и~надо.
\end{proof}
\begin{remark}[Гомоморфизм накрытий]
	Эти рассуждения работают, если $f, g$~--- непрерывные отображения, такие что
	$\pi_2 \circ f = g \circ \pi_1$, а~также, что $f\mid_{P_i}$~--- биекция, где
	$P_i$~--- это $i$-й слой накрытия. В~этом случае не факт, что индуцированное
	отображение является изоморфизмом.

	Такие $f, g$ задают так называемый гомоморфизм накрытий.
\end{remark}

\end{document}
