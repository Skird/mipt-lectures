\documentclass{article}

\usepackage[utf8x]{inputenc}
\usepackage[english,russian]{babel}
\usepackage{amsmath}
\usepackage{amsthm}
\usepackage{amsfonts}
\usepackage{amssymb}
\usepackage{cmap}

\newtheorem{theorem}{Теорема}
\newtheorem{claim}{Утверждение}

\renewcommand{\le}{\leqslant}
\renewcommand{\ge}{\geqslant}
\newcommand{\eps}{\varepsilon}
\renewcommand{\phi}{\varphi}

\frenchspacing

\begin{document}

\section*{24.10.2015}

\begin{itemize}
  \item Графы~--- множество моделей теории с~парой предикатных символов.
  \item Группы~--- множество моделей теории с~чуть большим количеством
  предикатных и~функциональных символов.
\end{itemize}

Мысль первая: много моделей можно описать какими-то равенствами (кванторы
всеобщности можно убрать, ибо из~$\phi$ выводится $\forall \phi$. Также
всегда считаем, что всегда есть бинарный предикатный символ равенства.

Поэтому рассматриваем \emph{решётки}~--- $\mathcal{L} = (L \ne \varnothing,
\land, \lor, =)$. Что делает решётку решеткой:

\begin{itemize}
  \item[L1.] Коммутативность
  \item[L2.] Ассоциативность
  \item[L3.] Рефлексивность (можно получить из остальных)
  \item[L4.] Идемпотентность
\end{itemize}

Примеры
\begin{itemize}
  \item Булевы функции $(\mathbb{B}, \land, \lor)$
  \item Подмножества $(2^X, \cap, \cup)$
  \item $(\mathbb{N}_+, (\cdot, \cdot), [\cdot, \cdot])$
  \item Нормальные подгруппы $(N(G), \cap, \cdot)$
\end{itemize}

Порой (во всех предыдущих случаях) можно сопоставить решетке какой-то
порядок.

Пусть $(A, \le)$~--- ч.у.м. Говорим, что~$a$~--- верхняя (нижняя) грань~$P
\subset A$, если~$\forall x \in P x \le a (x \ge a)$. Ясно, что можно
определить~$\sup P$.

Говорим, что~$a \prec b$ (покрывает), если~$\forall c (a \le b \le c
\Rightarrow c = a \land c = b$).

Диаграммы Хасса~--- изображения порядков.

\begin{theorem}
  $(L, \land, \lor)$~--- решётка тогда и~только тогда, когда $\exists
  \le \subset L^2$~--- ч.у.м. и~$a \land b = \inf\{a, b\}, a \lor b =
  \sup\{a, b\}$.
\end{theorem}
\begin{proof}~\\
  $\Longrightarrow a \le b \overset{def}{=} a \land b = a$. Формально проверяем
  конструкцию.\\
  $\Longleftarrow$ Определим $a \land b = \inf\{a, b\}, a \lor b =
  \sup\{a, b\}$ и~снова всё проверим.
\end{proof}

Изоморфизм решёток:~$(L_1, \land_1, \lor_1) \overset{\alpha}{\cong} (L_2,
\land_2, \lor_2)$. При этом~$\alpha$~--- биекция, сохраняет~$\land, \lor$.

\begin{theorem}
  $L_1 \overset{\alpha}{\cong} L_2$ тогда и~только тогда, когда $L_1
  \overset{\alpha}{\sim} L_2$ и~$\alpha, \alpha^{-1}$~--- монотонные
  отображения, то есть являются изоморфизмами упорядоченных множеств.
\end{theorem}
\begin{proof}~\\
  $\Longrightarrow a \le_1 b \Rightarrow a = a \land_1 b \Rightarrow \alpha(a) =
  \alpha(a \land_1 b) = \alpha(a) \land_2 \alpha(b) \Rightarrow \alpha(a) \le_2
  \alpha(b)$. $\alpha^{-1}$~--- тоже изоморфизм.\\
  $\Longleftarrow$ Хотим: $\alpha(a \lor_1 b) = \alpha(a) \lor_2 \alpha(b)$.
  Пишем: $a \le_1 a \lor_1 b, b \le_1 a \lor_1 b \Rightarrow \alpha(a),
  \alpha(b) \le_2 \alpha(a \lor_1 b)$, то есть верхняя грань. $\alpha(a) \le_2
  c, \alpha(b) \le_2 c \Rightarrow a \le_1 \alpha^{-1}(c), b \le_1
  \alpha^{-1}(c) \Rightarrow \alpha(a \lor_1 b) \le_2 c$, что и~нужно.
\end{proof}

(Изомофрное) \emph{вложение} $\eta: \mathcal{L}_1 \rightarrow \mathcal{L}_2$,
причём $\eta$ сохраняет порядок и~инъективно.
\emph{Подрешётка}~--- когда носители вложены, причем $\eta = id_{L_1}: L_1
\rightarrow L_2$~--- вложение (NB: подмодель $\ne$ подрешётка).

\section*{31.10.2015}

Давайте посмотрим на~$\emph{дистрибутивные решётки}$. Это вот такие:
\begin{gather*}
D\land: x \land (y \lor z) = (x \land y) \lor (x \land z),\\
D\lor: x \lor (y \land z) = (x \lor y) \land (x \lor z).
\end{gather*}

\begin{theorem}
Достаточно любого из свойств выше.
\end{theorem}

Ещё бывают $\emph{модулярные решётки}$:
$$ M: x \le y \Rightarrow x \lor (y \land z) = y \land (x \lor z). $$

\begin{theorem}
  Любая дистрибутивная решётка модулярна.
\end{theorem}

Решётка~$M_5$ ($d \rightarrow a, b, c \rightarrow e$) недистрибутивна.
Решётка~$N_5$ ($d \rightarrow (a \rightarrow b), c \rightarrow e$) даже не
модулярна. В~некотром смысле она внезапно оказывается единственной немодулярной.

\begin{claim}
  $L \models x \le y \Rightarrow L \models x \lor (y \land z) \le y \land (x
  \lor z)$.
\end{claim}

\begin{theorem}[Дедекинд]
  $L \not\models M \Longleftrightarrow N_5$ вкладывается в~$L$.
\end{theorem}
\begin{proof}
  В~одну сторону проверено. Если решётка не модулярна, то есть~$a, b, c, a \le
  b: \underbrace{a \lor (b \land c)}_{a_1} < \underbrace{b \land (a \lor
  c)}_{b_1}$. Можно убедиться, что элементы~$a_1 \land c \rightarrow (a_1
  \rightarrow b_1), c \rightarrow a_1 \lor c$ образуют~$N_5$.
  Придётся попотеть, доказать все неравенства, а~ещё нужно, чтобы никакие два
  не совпадали.
\end{proof}

\begin{theorem}[Биркгоф]
  $L \not\models D\land \Longleftrightarrow N_5$ или~$M_5$ вкладывается в~$L$.
\end{theorem}
\begin{proof}
  В~одну сторону снова ясно. Если решётка не дистрибутивна, то~$\exists a, b, c:
  a \land (b \lor c) > (a \land b) \lor (a \land c)$, ибо нестрогое неравенство
  выполнено в~любой решётке, а~равенства нет. Тогда скажем, что
  \begin{gather*}
    d = (a \land b) \lor (a \land c) \lor (b \land c),\\
    e = (a \lor b) \land (a \lor c) \land (b \lor c),\\
    a_1 = (a \land e) \lor d,\\
    b_1 = (b \land e) \lor d,\\
    c_1 = (c \land e) \lor d.
  \end{gather*}
  Аккуратно проверим и~получим~$d \rightarrow a_1, b_1, c_1 \rightarrow e$
  изоморфно~$M_5$.
\end{proof}

\section*{07.11.2015}

Упорядоченное множество \emph{полно}, если у~любого подмножества есть супремум
и~инфимум. Решётка полна, если соответствующее упорядоченное множество полно.

\begin{theorem}
  Решётка полна тогда и~только тогда, когда у~любого подмножества существует
  супремум.
\end{theorem}
\begin{proof}
  Возьмем~$A$~--- множество нижних граней~$B$. У~него есть супремум, он будет
  инфимумом~$B$.
\end{proof}

Рассмотрим~$Eq(A)$~--- множество всех отношений эквивалентности на~$A$.

\begin{claim}
  $(Eq(A), \subseteq)$~--- полная решётка.
\end{claim}
\begin{proof}
  Легко показать, что~$\bigcap\limits_{i\in I}E_i = \bigwedge\limits_{i\in
  I}E_i$. То есть инфимумом будет пересечение.
\end{proof}

$\bigvee\limits_{i \in I}E_i = \bigcup \left\{ E_{i_1} \circ \ldots \circ
E_{i_k} \mid \left(i_1, \ldots, i_k\right) \in I^k, k \in \mathbb{N}
\right\}$, где~$\circ$ есть композиция отношений.

$C$~--- \emph{оператор замыкания}, если выпонены
\begin{itemize}
  \item[C1.] $X \subset C(X)$.
  \item[C2.] $C(C(X)) = C(X)$.
  \item[C3.] $X \subset Y \Rightarrow C(X) \subset C(Y)$.
\end{itemize}

Пример~--- замкнутые классы булевых функций, дедуктивное замыкание: $C(\Gamma) =
\{ \phi \in Formulas \mid \Gamma \vdash \phi\}$.

Есть ещё четвертое свойтсво (не входит в~определение): $C(X) = \bigcup\{C(Y)
\mid Y \overset{(fin)}\subset X\}$. Обратное включение верно всегда. Прямое
можно вывести из теоремы о~полноте, если она есть.

Закнутое множество совпадает со своим замыканием. Пересечение замыканий равно
замыканию пересечений. То есть замкнутые множества образуют полную решётку.
Следствие: множество замкнутых классов булевых функций образует полную решётку
(решётку Поста).

\begin{theorem}
  Любая полная решётка~$\mathcal{L} = (L, \le)$ изоморфна решётке~$(L_c,
  \subset)$ для некоторого оператора замыкания~$C$.
\end{theorem}
\begin{proof}
  $X \subset L, C(X) = \left\{ a \in L \mid a \le \bigvee X \right\}$. Тогда
  элементу~$a$ сопоставим замкнутый класс $\{b \in L \mid b \le a\}$.
\end{proof}

$a \in L$ \emph{компактен}, если $\forall A \subset L (\exists \vee A, a \le
\vee A \Rightarrow \exists A' \overset{fin} \subset A: a \le \vee A' \}$.

$b$ компактно порождённый, если $\exists C \subset L, b = \vee C, \forall x \in
C, x$~--- компактен.

Решётка~$L$~--- алгебраическая, если она полна и~любой её элемент компактно
порождён.

В~частности, конечная решётка алгебраическая и~решётка всех подмножеств
алгебраическая (компакты в~ней это в~точности конечные подмножества).

\section*{14.11.2015}

Замыкание: $C: P(A) \rightarrow P(A)$.
$$ (C4)~\forall X \subseteq A \rightarrow C(X) = \bigcup \{ C(Y) \mid Y
\overset{f} \subseteq X \} $$

(C1)~--- (C4) дают \emph{алгебраичность} оператора замыкания.

\begin{theorem}
  Если $C$~--- алгебраический оператор замыкания, то $(L_c = \{ X \subseteq A
  \mid X = C(X) \}, \subseteq)$ есть алгебраическая решётка, причём элементы
  есть в~точности $C(X)$, где~$X$~--- конечно.
\end{theorem}
\begin{proof}
  Пусть $X \overset{f}\subseteq A$, тогда докажем, что~$C(X)$ компактно:
  \begin{gather*}
    X = \{x_1, \ldots, x_k \}, X \subseteq C(X) \subseteq \bigvee\limits_{i\in
    I} C(A_i) = C(\bigcup\limits_{i\in I}A_i),\\
    \forall x_k \in X \rightarrow x_k \in C(A_{i_1} \cup \ldots \cup
    A_{i_{s_k}}),\\
    X \subset \bigcup\limits_{1 \le k \le n} C(A_{i_1} \cup \ldots
    \cup A_{i_{s_k}}),\\
    C(X) \subset C(\bigcup\limits_{1 \le k \le n} C(A_{i_1} \cup \ldots
    \cup A_{i_{s_k}})).
  \end{gather*}

  Далее, (C4): $C(X) = \bigcup\{C(Z) \mid Z \overset{f}\subseteq X\} \subseteq
  C(\bigcup\limits_{Z\overset{f}\subseteq X} C(Z)) \subset \bigvee\limits_{Z
  \overset{f}\subseteq X} C(Z) \subseteq C(\bigcup_f C(Z)) \subseteq C(\bigcup_f
  Z).$

  Алгебраичность: $C(X) \subseteq\{ C(Y) \mid Y \overset{f}\subseteq X \}
  \subseteq \bigvee\limits_{Y \overset{f}\subseteq X} C(Y) \subseteq C(X)$.
\end{proof}

\begin{theorem}
  Пусть~$L$~--- алгебраическая решётка. Тогда~$L \cong L_C$ для некоторго
  алгебрическаого оператора замыкания~$C$.
\end{theorem}
\begin{proof}
  Пусть~$A$~--- множество компактных элементов~$X \subseteq A$ решётки~$L$,
  а~$C(X)$ задано как~$C(X) = \{ b \in A \mid b \le \bigvee X \}$.

  Проверяем все свойства оператора замыкания и~отображаем элемент~$a \mapsto \{
  b \in A \mid b \le a \}$, то есть во множество компактов, которые его не
  превосходят.
\end{proof}

Рассмотрим структуру~$\mathcal{A} = (A \ne \varnothing, \mathcal{F})$, притом
$\forall f \in \mathcal{F} \rightarrow \exists n \in \mathbb{N}~f: A^n
\rightarrow A$. Это, собственно, алгебра.

Решётка~--- булева алгебра с~двумя функциями. Булева алгебра~--- дистрибутивная
решётка, в~которой есть~$0, 1$, с~тождетсвами~$x \land 0 = 0, x \lor 1 = 1, x
\land x' = 0, x \lor x' = 1$ (из этого всё выводится).

\end{document}
