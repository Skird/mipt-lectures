\documentclass{article}

\usepackage[utf8x]{inputenc}
\usepackage[english,russian]{babel}
\usepackage{amssymb}
\usepackage{amsmath}
\usepackage{amsthm}
\usepackage{amsfonts}
\usepackage{amssymb}
\usepackage{cmap}

\newtheorem{theorem}{Теорема}
\newtheorem{lemma}{Лемма}

\theoremstyle{remark}
\newtheorem*{remark}{Замечание}

\renewcommand{\le}{\leqslant}
\renewcommand{\ge}{\geqslant}
\newcommand{\eps}{\varepsilon}
\renewcommand{\phi}{\varphi}

\frenchspacing

\begin{document}

\section*{24.10.2015}
Аримфметика Пеано (NB: перечислимость множества аксиом важна).

Первая теорема Гёделя грубо: не получится вывести~$\exists k: T(n, x, k)$~---
проблема остановки.
Вторая теорема Гёделя о~построении арифметичного предиката выводимости.

Примитивно-рекурсивные функции ($\mathcal{PR}$)~--- наименьшее множество,
замкнутое относительно операций композиции, примитивной рекурсии
и~содержащее~$\{I_i^n, S, 0\}$.

Лемма о~разборе случаев:~$g I(h(y) = 0) + h (1 - I(h(y) = 0)) \in \mathcal{PR}$,
если $g, h \in \mathcal{PR}$.

Еще лемма:~$f(x, y) = \sum_{z \le x} g(z, y) \in \mathcal{PR}$, если $f \in
\mathcal{PR}$. Произведение, конечно, тоже.

Примитивно-рекурсивное отношение~--- то, у~которого характеристическая функция
примитивно-рекурсивна.

Еще лемма: если $R \in \mathcal{PR}$, то $\exists z \le x: R(z, y) \in
\mathcal{PR}$, а~также~$\forall z \le x: R(z, y) \in \mathcal{PR}$.

Конечно, примитивно-рекурсивны~$P \land Q, P \lor Q, \neg P$, также усечённая
минимизация, предыдущее число, равенство и~много ещё что.

Примитивно-рекурсивен оператор ограниченной минимизации~$f(x, y) = \mu z <
x.h(z, y) = 0$. То есть можно считать какие-нибудь обратные функции вроде корня.

\section*{31.10.2015}

Примитивно-рекурсивное кодирование пар $\left<x, y\right> = \frac{(x + y)(x + y
+ 1)}{2} + x$. Проекторы: $\pi^1\left<x,y\right> = x, \pi^2\left<x,y\right> =
y$.

Последовательности кодируем как $[a_1, \ldots, a_n] = p_1^{a_1 + 1} \ldots
p_n^{a_n+1}$. Говоря так, мы на самом деле работаем с~примитивно-рекурсивными
функциями $IsSeq(x), Len(x), Get(x, i)$ и~прочее.

С~помощью этого всего можно реализовать другую схему рекурсии, обращающуюся
ко всем предыдущим значениям (возвратная рекурсия).

\begin{theorem}
  Пусть~$f(\overline{x}, 0) = g(\overline{x}), f(\overline{x}, z + 1) = h(z,
  f'(\overline{x},z))$, где~$f'(\overline{x}, z) = \left[f(\overline{x}, 0),
    \ldots, f(\overline{x}, z)\right]$. Тогда, если $g, h \in \mathcal{PR}$, то
    $f \in \mathcal{PR}$.
\end{theorem}

Интресное: обратная функция Аккермана (было письмо).

\section*{07.11.2015}

Наша теория: предикатный символ равенства, функциональный символ последователя
$S^{(1)}$, нуля~$0^{(0)}$, а~также множество символов $Fnc$.

$Fnc^0 \ni 0, Fnc^1 \ni S, Fnc^n \supseteq \{ I^n_k \mid 0 \le k < n \}$, причём
$g_1, \ldots, g_m \in Fnc^n, h \in Fnc^m \Rightarrow Chg_1\ldots g_m) \in
Fnc^n$, а~также $g \in Fnc^n, h \in Fnc^{n+2} \Rightarrow Rgh \in Fnc^{n+1}$,
где $Rgh$ обозначает оператор примитивной рекурсии, $Fnc = \bigcup\limits_{i\in
\mathbb{N}} Fnc^i$.

Теперь надо закодировать весь наш язык первого порядка. Переменные $\#(v_k) =
\left<1, k\right>$. Логические символы $\{\neg, \rightarrow, \forall\}$ кодируем
как $\left<2, x\right>$, единственный предикатный символ как~$\left<3,
0\right>$. Осталось закодировать~$Fnc$.

$\#(0) = \left<4, \left<1, 0\right>\right>$.
$\#(S) = \left<5, \left<1, 1\right>\right>$.
$\#(I^m_k) = \left<6, \left<k, m\right>\right>$.
$\#(C) = \left<7, 0\right>$.
$\#(R) = \left<8, 0\right>$.

В~качестве кода какой-то программы~$p$ из $Fnc$ хотим взять код
последовательности номеров символов в~записи.

Хотим записать предикат <<быть кодом функционального терма с~$m$ переменными>>.
$FTm(x, n) = 1$ в~следующих случаях:
\begin{itemize}
  \item $Sq(x) \land Len(x) = 1 \land n = 0 \land (x)_0 = \#(0)$
  \item $\ldots, n = 1 \land (x)_0 = \#(S)$
  \item $\ldots, \exists 0 < k < n: (x)_0 = \left<6, \left<k, n\right>\right>$
  \item $Sq(x) \land Len(x) \ge 3 \land (x)_0 = \#(C) \land FTm((x)_1, Len(x)
    - 2) = 1 \land \forall i < Len(x) - 2 \rightarrow (FTm((x)_{i+2}, n))$
  \item $Sq(x) \land Len(x) = 3 \land n > 0 \land x > n + 1 \land (x)_0 = \#(R)
    \land FTm((x)_1, n - 1) \land (FTm((x)_2, n + 1))$
\end{itemize}

Кодирование терма $T(x)$:
\begin{itemize}
  \item $Sq(x) \land Len(x) = 1 \land \exists k < x: (x)_0 = \#(v_k)$
  \item $Sq(x) \land FTm((x)_0, Len(x) - 1) \land Len(x) > 0 \land \forall i <
    Len(x) - 1 \rightarrow (T((x)_i) = 1)$
\end{itemize}

Кодирование формул $F(x)$ ($\lceil \forall v_j \phi \rceil = [\#(\forall),
\#(v_j), \lceil \phi \rceil]]$):
\begin{itemize}
  \item $Sq(x) \land Len(x) = 3 \land (x)_0 = \#(=) \land T((x)_1) = 1 \land
    T((x)_2) = 1$
  \item $Sq(x) \land Len(x) = 3 \land (x)_0 = \#(\forall) \land \exists j < x:
    ((x)_1 = \left<1, j\right>) \land Fm((x)_2) = 1$
  \item $\ldots$
\end{itemize}

Следующая цель: функция подстановки $Sub(\lceil \phi \rceil, j, \lceil t \rceil)
= \lceil \phi(t \mid_{v_j}) \rceil$.

\section*{14.11.2015}

План построения~$Sub$ (считаем, что все формулы записываются через~$\forall,
\rightarrow, \bot$):
\begin{itemize}
  \item проверить, что первый аргумент~--- формула, если нет, то вернуть 0.
  \item если первый аргумент~--- код одной переменной, то осуществить простую
    подстановку.
  \item если первый аргумент начинается с~квантора по переменной, то
    нужно его проигнорировать в~случае, если переменная заменяемая и~осуществить
    подстановку под квантором в~противном случае.
  \item $\ldots$
\end{itemize}

Можно построить также двухаргументный~$Sub: Sub(\lceil \phi \rceil, \lceil t
\rceil) = sub(\lceil \phi \rceil, i, \lceil t \rceil), i = \mu j < \lceil \phi
\rceil: j$ входит свободно в~$\lceil \phi \rceil$.

Нумералы: $\overline 0 = 0, \overline{n + 1} = S\overline n$. Рассмотрим функцию
$\nu(n) = \lceil \overline n \rceil$ (примитивно-рекурсивную). Тогда $Sub(\lceil
\phi \rceil, \nu(n)) = \lceil \phi(\overline{n} / x) \rceil$. Такую функцию
назовем~$S(x, y)$.

Теперь выпишем предикатные аксиомы:
\begin{itemize}
  \item $\phi \rightarrow (\psi \rightarrow \phi)$
  \item $(\phi \rightarrow (\psi \rightarrow \theta)) \rightarrow ((\phi
    \rightarrow \psi) \rightarrow (\phi \rightarrow \theta))$
  \item $(\neg \psi \rightarrow \neg \phi) \rightarrow (\phi \rightarrow \psi)$
  \item $\forall x \phi \rightarrow \phi(t / x)$
  \item $\forall x (\phi \rightarrow \psi) \rightarrow (\forall x \phi
    \rightarrow \forall x \psi)$
  \item $\phi \rightarrow \forall x \phi, x \not\in FV(\phi)$
  \item Modus ponens, обобщение: $\frac{\phi}{\forall x \phi}$.
\end{itemize}

Примитивно-рекурсивная арифметика ($\mathcal{PRA}$):
\begin{itemize}
  \item $x = x$
  \item $x = y \rightarrow (A \rightarrow A(y / x))$
  \item $\neg(Sx = 0)$
  \item $Sx = Sy \rightarrow x = y$
  \item $I^n_k x_1, \ldots, x_n = x_k$
  \item Аксиомы композиции
  \item Аксиомы примитивной рекурсии
  \item Индукция по бекванторным формулам: $\phi(0) \rightarrow (\forall
    x(\phi(x) \rightarrow \phi(Sx)) \rightarrow \phi$, где в~$\phi$ нет
    кванторов.
\end{itemize}

Вывод в~$\mathcal{PRA}$ определяется как обычно: каждый шаг это либо логическая
аксиома, либо аксиома~$\mathcal{PRA}$, либо выводится из предыдущих по modus
ponens или обобщению. Стоит заметить, что множество аксиом
примитивно-рекурсивно.

Теперь можно написать предикат $Prf_{\mathcal{PRA}}(y, x) = $ <<$y$ вывод
в~$\mathcal{PRA}$ формулы с~кодом~$x$>>. $\mathcal{PRA}$ можно расширять
примитивно-рекурсивным множеством аксиом, сохраняя предикат доказательства.

Также можно построить предикат~$Pr_{\mathcal{PRA}}(x) = \exists y
Prf_{\mathcal{PRA}}(y, x)$ (квантор неограниченный!).

Если $T$ примитивно-рекурсивно расширяет~$\mathcal{PRA}$, то у~нас есть
$Prf_T$ и~$Pr_T$, причём первый из них примитивно рекурсивен. Притом так как
предикат~$Prf_T \in \mathcal{PR}$, то его характеристическая функция~---
какая-то примитивно-рекурсивная формула.

Обозначим $\lceil \phi(x \cdot) \rceil = S\overline{\lceil \phi
\rceil} x$.

\begin{theorem}[Гильберта-Бернайса-Лёба]
  Пусть~$T$~--- примитивно-рекурсивное расширение~$\mathcal{PRA}$.
  \begin{itemize}
    \item Если~$\phi$~--- замкнутая формула, то~$T \vdash \phi \Rightarrow
      \mathcal{PRA} \vdash Pr_T(\lceil \phi \rceil)$
    \item $\mathcal{PRA} \vdash Pr_T(\lceil \phi(\overline{x}\cdot) \rightarrow
      \psi(\overline{x}\cdot) \rceil) \rightarrow (Pr_T(\lceil \phi
      (\overline{x}\cdot) \rceil) \rightarrow Pr_T(\lceil
      \psi(\overline{x}\cdot)\rceil))$
    \item $\mathcal{PRA} \vdash Pr_T(\phi(\overline{x}\cdot) \rightarrow
      Pr_T(\lceil Pr_T(\lceil \phi(\overline{x}\cdot) \rceil))$
  \end{itemize}
\end{theorem}

Будем далее обозначать: $\square_T \phi = Pr_T(\lceil \phi \rceil)$, а~также
$[\phi(x_1, \ldots, x_n)] = \lceil \phi(x_1\cdot, \ldots, x_n\cdot) \rceil$.
Это терм, в~котором все переменные свободны.

\section*{12.12.2015}

\begin{theorem}[Лёба]
  Пусть~$T$ примитивно-рекурсивна и~расширяет~$\mathcal{PRA}$. Тогда~$\forall
  \phi$~--- предложения если~$T$ выводит $\square_T \phi \rightarrow \phi$, то
  выводит и~$\phi$.
\end{theorem}

\begin{theorem}[Тарский]
  Не существует формулы~$\psi$, такую что~$\forall \phi$~--- предложения
  выполнено, что в~$\mathbb{N}$ верно $\psi(\lceil \phi \rceil)
  \Leftrightarrow в~\mathbb{N}$ верно $\phi$.
\end{theorem}

\begin{theorem}[1-я~Гёделя о~непноте]
  Если $T$ примитивно-рекурсивное расширение $\mathcal{PRA}$, то
  (1) $T$ не выводит $\bot$.
  (2) Если $T$ $\Sigma_1$-корректна, то существует предложение~$\phi$, такое,
  что не выводится ни~$\phi$, ни~$\neg\phi$.
\end{theorem}

\begin{theorem}[2-я~Гёделя о~непноте]
  Если $T$ примитивно-рекурсивное расширение $\mathcal{PRA}$, $T$ не выводит
  $\bot$, $T$ является $\Sigma_1$-корректной. Тогда $T$ не выводит $Con(T)$
  и~$\neg Con(T)$ ($Con(T) = \neg \square_T \bot$).
\end{theorem}

\begin{theorem}[усиление Гёделя-Россера]
  Если $T$ примитивно-рекурсивное расширение $\mathcal{PRA}$, не выводит ложь,
  то существует предложение~$\phi$, такое что $T$ не выводит $\phi$ и~$\neg
  \phi$.
\end{theorem}

\begin{lemma}
  $\forall \phi(\overline{y}) \in \Delta_0 \rightarrow \exists \widetilde{f} \in
  \widetilde{\mathcal{PR}}: \mathcal{PRA} \vdash \phi(\overline{y})
  \leftrightarrow (\widetilde{f}\overline{y}=0)$.
\end{lemma}

Класс $\Delta_0$ с~ограниченными кванторами (?). $\Sigma_1, \Pi_1$ и~т.д.:
сначала обычные кванторы, потом~$\Delta_0$.

\begin{lemma}
  $\mathcal{PRA} \vdash \phi \leftrightarrow \exists y(Prf_T(y,\lceil \neg \phi
  \rceil) \land \forall z < y \neg Prf_T(z, \lceil \phi \rceil)$.
\end{lemma}
\begin{proof}
  Если $T \vdash \neg \phi$, то $\exists m \in \mathbb{N}: \mathbb{N} \vDash
  \forall Prf(\underline{m}, \lceil \neg \phi \rceil)$.
  $T \not\vdash \bot \Rightarrow \mathbb{N} \vDash \forall z < \underline{m}
  \neg Prf(z, \lceil \phi \rceil)$.

  Если $T \vdash \phi$, то $\vdash \exists y(Prf(y, \lceil \neg \phi \rceil)
  \land \forall z < y Prf(z, \lceil \phi \rceil))$, также $\mathbb{N} \vDash
  Prf(\underline{n}, \lceil \phi \rceil)$. По~$\Sigma_1$-полноте $T \vdash
  Prf(\underline{n}, \lceil \phi \rceil)$, так как $T \not\vdash \bot$, то $T
  \vdash \forall y \le \underline{n} \neg Prf(y, \lceil \neg \phi \rceil)$.
  $\mathcal{PRA} \vdash (y \le \underline{n}) \lor (\underline{n} < y)$.
\end{proof}

\begin{theorem}[формализованная теорема Лёба]
  Если $T$ примитивно-рекурсивное расширение $\mathcal{PRA}$, то $\mathcal{PRA}
  \vdash \square_T(\square_T \phi \rightarrow \phi) \rightarrow \square_T \phi$,
  то есть $\mathcal{PRA}$ знает о~теореме Лёба.
\end{theorem}

Если подставить ложь, то $\mathcal{PRA} \vdash \square_T Con(T) \rightarrow \neg
Con(T)$, то есть $\mathcal{PRA} \vdash Con(T) \rightarrow \neg \square_T
Con(T)$.

Если к~аксиомам добавить~$\neg Con(T)$, то $\mathcal{PRA} \vdash Con(T)
\rightarrow Con(T + \neg Con(T))$, что есть формализованная вторая тоерема
Гёделея.

Если $T \not \vdash \bot$ и~$T$~--- п.р. расширение~$\mathcal{PRA}$, то $T +
\neg Con(T) \not \vdash \bot$, но $\mathbb{N} \not \vDash T + \neg Con(T)$.
Получается пример непротиворечивой, но некорректной теории.

$Rf_n(T) = \{ \square_T \phi \rightarrow \phi \}$ по всем предложениям
$\phi$~--- локальная схема рефлексии для~$T$.

$RFN(T) = \{ \forall \overline{x} (\square_T [\phi] \rightarrow \phi) \}$ по
всем формулам $\phi(\overline{x})$~--- равномерная схема рефлексии.

$\{Con(T)\} \subset Rf_n(T) \subset RFN(T)$. Однако множество кодов $RFN(T)$
примитивно-рекурсивно, то есть его можно добавлять к~$T$.

\begin{lemma}
  Над $\mathcal{PRA}$ эквивалентны схемы
  \begin{itemize}
    \item $RFN(T)$
    \item $\{ \forall \overline{x} \square_T [\phi] \rightarrow \forall
      \overline{x} \phi\}$
    \item $\frac{\forall \overline{x} \square_T [\phi]}{\forall \overline{x}}
      \phi$ (правило Клини, более слабая вариация омега-правила).
  \end{itemize}
\end{lemma}
\begin{proof}
  $(1) \Rightarrow (2) \Rightarrow (3)$ понятно.

  $(3) \Rightarrow (1)$.
  \begin{lemma}[о~явной рефлексии]
    Выполнено утверждение и~его обобщение:
    \begin{itemize}
      \item $\forall m, n \in \mathbb{N}\ \forall \phi(x)\ T \vdash
        Prf_T(\underline{m}, \lceil \phi(\underline{n} / x) \rceil) \rightarrow
        \phi(\underline{n} / x)$.
      \item $\mathcal{PRA} \vdash \forall x, y \square_T[Prf_T(y, [\phi])
        \rightarrow \phi]$, то есть об этом знает~$\mathcal{PRA}$.
    \end{itemize}
  \end{lemma}
  \begin{proof}
    $\mathbb{N} Prf_T(\underline{m}, \lceil \phi(n) \rceil) \lor \neg
    Prf(\underline{m}, \lceil \phi(\underline{n}) \rceil)$.

    1 случай. $T \vdash \phi(\underline{n})$, тогда $T \vdash
    Prf_T(\underline{m}, \phi(\underline{n})) \rightarrow \phi(\underline{n})$.

    2 случай. $T \vdash \neg Ptf_t(\underline{m}, \lceil \phi(\underline{n})
    \rceil) \overset{\Sigma_1}\Rightarrow T \vdash \neg Prf_T(\underline{m},
    \lceil \phi(n) \rceil) \Rightarrow T \vdash Prf_T(\underline{m}, \lceil
    \phi(n) \rceil) \rightarrow \phi(\underline{m})$.

    Обобщение аналогично.
  \end{proof}

  Научимся получить произвольную формулу~$RFN(T)$ с~помощью правила Клини.
  $\mathcal{PRA} \vdash \forall x \forall y \square_T[Rf_T(y, [\phi])
  \rightarrow \phi(x)]$. Применим к~ней правило Клини:
  $\mathcal{PRA} \vdash \forall x \forall y (Prf_T(y, [\phi]) \rightarrow
  \phi(x) \vdash \forall x(\exists y Prf_T(y, [\phi]) \rightarrow \phi) \vdash
  \forall x(\square_T [\phi] \rightarrow \phi)$.
\end{proof}

\section*{12.12.2015}

$RFN_{\Pi_n}(T) = \{ \forall \overline{x} (\square_T [\phi] \rightarrow \phi)
\mid \phi(\overline{x}) \in {\Pi_n}\}$~--- ограниченная схема рефлексии
(аналогично по~$\Sigma_n$).

\begin{lemma}
  При $n \ge 1$ над $\mathcal{PRA}$ эквивалентны
  \begin{itemize}
    \item $RFN_{\Sigma_n}(T)$ и~$RFN_{\Pi_{n+1}}(T)$
    \item $RFN_{\Pi_1}(T)$, $Rfn_{\Pi_1}(T)$ и~$Con(T)$.
  \end{itemize}
\end{lemma}

\begin{theorem}[Частичное определение истинности]
  Пусть $n \ge 1$, тогда существует $Tr_{\Pi_n}(z) \in \Pi_n$, такая
  что~$\forall \phi(\overline{x}) \in \Pi_n$ выполнено~$\mathcal{PRA} \vdash
  \phi(\overline{x} \Leftrightarrow Tr_{\Pi_n}[\phi(\overline{x})]$.
\end{theorem}
\begin{remark}
  Под $\mathcal{PRA}$ $RFN(T)$ эквивалентна $\{RFN_{\Pi_n(T)} \mid n \ge 1\}$,
  каждая из которых конечно аксиоматизируема.
\end{remark}

Теория $U$~--- $\Gamma$-расширение теории~$T$, если $U \equiv T + \Phi, \Phi
\subset \Gamma$ (конечное расширение, $\Sigma_1$~расширение).

\begin{lemma}
  при $n \ge 1$
  \begin{itemize}
    \item
      $Rfn_{\Pi_n}(T)$ не содержится ни в~каком
      \begin{itemize}
        \item непротиворечивом
        \item конечном
        \item $\Sigma_n$
      \end{itemize}
      Расширении~$U$ теории~$T$.
    \item
      $Rfn_{\Pi_n}(T)$ не содержится ни в~каком непротиворечивом
      $\Sigma_n$-расширении~$U$ теории~$T$.
  \end{itemize}
\end{lemma}

\begin{remark}
  Следствия
  \begin{itemize}
    \item $Rfn(T)$ не содержится ни в~каком непротиворечивом конечном
      расширении~$T$.
    \item $RFN(T)$ не содержится ни в~каком непротиворечивом расширении~$T$
      ограниченной сложности.
    \item $T + Th_{\Pi_1}(\mathbb{N})$ (все истинные в~$\mathbb{N}\
      \Pi_1$-формулы) $\vdash Rfn(T)$.
  \end{itemize}
\end{remark}

\begin{theorem}
  Если $T$~--- $\Sigma_1$-корректна, то $T + Rfn(T) \not\vdash RFN(T)$.
\end{theorem}

\begin{remark}
  Факт: $\mathcal{PA} \vdash RFN(\mathcal{PRA})$.

  Следствие: $PA$ не содержится ни в~каком непротиворечивом расширении~$PRA$
  ограниченной сложности (в~частности, в~конечном).
\end{remark}

\begin{theorem}
  Пусть~$U$ перечислимоe непротиворечивое $\Sigma_n$ расширение~$T$. Тогда
  существует предложение~$\phi \in \Sigma_n$, такое что $T + \phi$
  непротиворечива и~$T + \phi \vdash U$.
\end{theorem}
\begin{remark}
  Факт (лемма Craig'a): Пусть~$\Gamma$~--- перечислимое множество формул. Тогда
  существует примитивно-рекурсивное множество формул $\Delta$, такое что
  $\mathcal{PRA} + \Delta = \mathcal{PRA} + T$.
\end{remark}

\begin{remark}
  Первое следствие из теоремы: $Rfn_{\Pi_n(T)}$ не содержится ни в~каком
  непротиворечивом перечислимом $\Sigma_n$-расширении теории~$T$.

  Второе: $Rfn$ не содержится ни в~каком
  непротиворечивом перечислимом расширении теории~$T$ ограниченной сложности.

  Из этого следует, что множество всех истинных~$\Pi_1$-формул
  $Th_{\Pi_1}(\mathbb{N})$ неперечислимо (иначе противоречие со вторым
  следствием).
\end{remark}

\section*{Можно почитать}

\begin{itemize}
  \item Смаринский <<Self-reference and modal logics>>
  \item Доказуемая $\Sigma_1$~--- полнота (Бухгольц?).
\end{itemize}

\end{document}
