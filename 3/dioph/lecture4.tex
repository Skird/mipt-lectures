\documentclass{article}
\usepackage[utf8x]{inputenc}
\usepackage[english,russian]{babel}
\usepackage{amsmath,amscd}
\usepackage{amsthm}
\usepackage{mathtools}
\usepackage{amsfonts}
\usepackage{amssymb}
\usepackage{cmap}
\usepackage{centernot}
\usepackage{enumitem}
\usepackage{perpage}
\usepackage{chngcntr}
%\usepackage{minted}
\usepackage[bookmarks=true,pdfborder={0 0 0 }]{hyperref}
\usepackage{indentfirst}
\hypersetup{
  colorlinks,
  citecolor=black,
  filecolor=black,
  linkcolor=black,
  urlcolor=black
}

\newtheorem*{conclusion}{Вывод}
\newtheorem{theorem}{Теорема}
\newtheorem{lemma}{Лемма}
\newtheorem*{corollary}{Следствие}

\theoremstyle{definition}
\newtheorem*{problem}{Задача}
\newtheorem{claim}{Утверждение}
\newtheorem{exercise}{Упражнение}
\newtheorem{definition}{Определение}
\newtheorem{example}{Пример}

\theoremstyle{remark}
\newtheorem*{remark}{Замечание}

\newcommand{\doublearrow}{\twoheadrightarrow}
\renewcommand{\le}{\leqslant}
\renewcommand{\ge}{\geqslant}
\newcommand{\eps}{\varepsilon}
\renewcommand{\phi}{\varphi}
\newcommand{\ndiv}{\centernot\mid}

\MakePerPage{footnote}
\renewcommand*{\thefootnote}{\fnsymbol{footnote}}

\newcommand{\resetcntrs}{\setcounter{theorem}{0}\setcounter{definition}{0}
\setcounter{claim}{0}\setcounter{exercise}{0}}

\DeclareMathOperator{\aut}{aut}
\DeclareMathOperator{\cov}{cov}
\DeclareMathOperator{\argmin}{argmin}
\DeclareMathOperator{\argmax}{argmax}
\DeclareMathOperator*\lowlim{\underline{lim}}
\DeclareMathOperator*\uplim{\overline{lim}}
\DeclareMathOperator{\re}{Re}
\DeclareMathOperator{\im}{Im}

\frenchspacing


\begin{document}
\section*{Лекция 4. Построения циркулем и~линейкой}
\addcontentsline{toc}{section}{Лекция 4. Построения циркулем и~линейкой}
\resetcntrs

\section{Представление чисел в~виде суммы двух квадратов}

\subsection{Одна из задач Эрдёша}

Одно из классических диофантовых уравнений второй степени записывается как~$x^2
+ y^2 = m, m \in \mathbb{N}$ и~ставит вопрос о~количестве целых точек на
окружности радиуса~$\sqrt{m}$. Одним из интересных приложений, мотивирующих
задачу, является открытая проблема, поставленная впервые Палом Эрдёшем.

\begin{problem}
  Пусть $P_n$~--- набор, состоящий из точек плоскости $p_1, \ldots, p_n$,
  а~$f(P_n)$ есть наибольшее количество одинаковых расстояний между какими-либо
  двумя точками. Какова асимптотическая скорость роста $f(P)$ при~$n
  \rightarrow~\infty$, если из всех конфигураций точек в~качестве~$P_n$ берется
  та, у~которой наибольшее значение~$f(P_n)$?
\end{problem}

Перебрав некоторые простые конструкции, легко получить примеры линейного роста
искомой величины. Содержательный же вопрос в~том, можно ли построить серию
конфигураций со сверхлинейным ростом. Наилучшую известную на сегодня конструкцию
построил сам Эрдёш. Его утверждение заключалось в~том, что на обычной квадратной
сетке~$\sqrt{n} \times \sqrt{n}$ можно найти расстояние~$m$ (зависящее от~$n$),
которое будет встречаться асимптотически чаще, чем $c \cdot n$ раз для
любого~$c > 0$. В~качестве такого значения~$m$ берется как раз то, для которого
на окружности радиуса~$\sqrt{m}$ лежит много целых точек. Эрдёш показал, что при
правильном выборе~$m$ в~зависимости от~$n$, можно найти асимптотически~$n \cdot
\frac{\log n}{\log \log n}$ расстояний, равных~$n$, сняв тем самым вопрос
о~возможности сверхлинейного роста. Остаётся, однако, открытым вопрос о~том,
можно ли получить рост быстрее, чем~$n^\alpha$ для какого-либо $\alpha > 1$.

\subsection{Решение задачи о~сумме двух квадратов}

\begin{claim}
  Для простого числа $p > 2$ следующие три утверждения равносильны:
  \begin{enumerate}[label=(\arabic*)]
    \item $p$ имеет вид $4k + 1, k > 0$.
    \item $p$ не является простым элементом кольца гауссовых
      чисел~$\mathbb{Z}[i]$.
    \item $p$ представляется в~виде суммы двух квадратов натуральных чисел,
      притом единственным образом.
  \end{enumerate}
\end{claim}
\begin{proof}
  Легче всего установить равносильность (2) и~(3). В~самом деле, если $p = x^2 +
  y^2$, то в~гауссовых числах можно записать $p = (x + yi)(x - yi)$.
  Поскольку~$x, y \in \mathbb{N}$, то оба элемента из правой части необратимы,
  то есть~$p$ разложим в~гауссовых числах и~не является простым элементом
  кольца.

  Если~$p = (a + bi)(c + di)$, то~$p = (a - bi)(c - di)$ и~$p^2 = (a^2 +
  b^2)(c^2 + d^2)$. Последнее равенство не выходит за пределы целых чисел,
  поэтому в~силу простоты~$p$ и~необратимости множителей, каждый из них
  равен~$p$. То есть $p = a^2 + b^2 = c^2 + d^2$. Каждое из чисел $a \pm bi, c
  \pm di$ является простым в~$\mathbb{Z}[i]$, так как имеет норму~$p$, поэтому
  из основной теоремы  арифметики получается, что либо~$(a + bi) \mid (a - bi),
  (c + di) \mid (c - di)$, либо $(a + bi) \mid (c - di), (c + di) \mid
  (a - bi)$.
  \begin{exercise}
    $(a + bi) \sim (a - bi) \Leftrightarrow (a + bi) \sim (1 + i)$.
  \end{exercise}
  Первый случай влечёт~$(a + bi) \sim (a - bi)$, то есть~$p = 2$, что
  противоречит условию. Второй случай влечет~$(a + bi) \sim (c + di)$, что
  означает, что разложения~$a^2 + b^2$ и~$c^2 + d^2$ совпадают. Аналогично,
  любое разложение~$p$ в~сумму двух квадратом совпадает с~$a^2 + b^2$.

  Так как~$p$ вида~$4k + 3$ не может быть суммой двух квадратов, то остается
  доказать, что любое число вида~$4k + 1$ раскладывается в~гауссовых числах.
  Для этого нужно доказать, что~$-1$ является квадратичным вычетом по
  модулю~$p$, что можно сделать, вычислив символ Лежандра или же
  воспользовавшись более общей теоремой.

  \begin{theorem}
    Пусть~$p - 1 = r \cdot l, r, l > 1$. Тогда~$0 \ne a \in \mathbb{Z}_p$
    является~$r$-й степенью ($\exists x: x^r \equiv a \pmod p$) тогда и~только
    тогда, когда~$a^l \equiv 1 \pmod p$.
  \end{theorem}

  Так как $p - 1 = 2r$, то~$-1$ является квадратичным вычетом тогда и~только
  тогда, когда $(-1)^r \equiv 1 \pmod p$, то есть~$r = 2k$ и~$p = 4k + 1$.
  Итак, $\exists x: p \mid (x^2 + 1)$, что в~гауссовых числах записывается как
  $p \mid (x + i)(x - i)$. Если бы~$p$ было простым числом, то из этого
  следовало бы, что~$p \mid (x + i)$ или~$p \mid (x - i)$. Легко видеть, что это
  противоречие при $p > 2$, так как если $p \mid (a + bi)$, то $p \mid a$ и~$p
  \mid b$. Итак, число вида~$4k + 1$ простым в~гауссовых числах быть не может,
  что завершает доказательство утверждения.
\end{proof}

\begin{exercise}
  Пусть $n = l^2 \cdot p_1 \cdot \ldots \cdot p_m$, где $p_i$~--- различные
  простые. Тогда~$n$ представимо в~виде суммы двух квадратов, если все~$p_i$
  имеют вид~$4k + 1$, притом количество разложений равно $2^{m-1}$.
\end{exercise}

В~связи с~тем, что по доказанному простые пары~$p$, $p+2$ не могут существовать
в~$\mathbb{Z}[i]$, можно поставить задачу о~простых близнецах в~гауссовых числах
по-другому.
\begin{exercise}[задача для исследования]
  Выяснить, конечно ли число пар простых вида~$(a \pm 1) + (b \pm 1)i$
  в~гауcсовых числах.
\end{exercise}

\section{Построения циркулем и~линейкой}

\subsection{Стандартная постановка задач на построение}

Задаваясь вопросом о~построимости того или иного геометрического объекта,
неоходимо предельно строго сформулировать задачу. К~примеру, если ставить вопрос
о~построимости отрезка длиной~$x^2$, если дан отрезок длины~$x$, необходимо
оговорить, дан ли единичный отрезок. Если, к~примеру, дан отрезок длины~1, то
важно оговорить, где лежат его конечные точки, потому что если один из его
концов имеет координаты $(\sqrt[3]{2}, 0)$, то отрезок длины~$\sqrt[3]{2}$ легко
построим, однако построить его не удастся, если единичный отрезок дан на оси
абсцисс с~одним из концов в~начале координат.

Устоявшаяся формулировка начальных условий и~список разрешённых действий
в~задачах на построение подразумевает следующие условия:
\begin{itemize}
  \item Даны координатные оси, перпендикулярные друг другу, и~точка их
    пересечения.
  \item Слова <<дан единичный отрезок>> трактуются как <<отмечена точка $(1,
    0)$>>.
  \item Если в~процессе построения используется произвольная точка (прямая), то
    координаты выбираемой точки (коэффициенты уравнения прямой) подразумеваются
    какими-либо рациональными числами. Это необходимо, так как в~результате
    выбора произвольной точки может быть получена, к~примеру, точка $(\pi, 0)$,
    построить которую в~стандартных условиях невозможно.
  \item Алгоритм построения конечен.
  \item Одним шагом алгоритма считается одно из следующих действий:
    \begin{itemize}
      \item Проведение прямой через две уже построенные точки.
      \item Проведение окружности с~одной из построенных точек в~качетсве центра
        через другую построенную точку.
      \item Взятие пересечения двух уже построенных прямых или окружностей или
        же взятие пересечения уже построенной прямой и~уже построенной
        окружности.
      \item Взятие <<произвольной>> точки или прямой в~смысле, оговоренном выше.
    \end{itemize}
\end{itemize}

Только формализовав таким образом круг возможных действий, можно перейти
к~доказательству содержательных теорем о~непостроимости. В~их числе:
непостроимость отрезка длины $\sqrt[3]{2}$, непостроимость углов
величиной~$\frac{\pi}{18}$ (что опровергает возможность трисекции угла в~30
градусов) и~$\frac{2\pi}{7}$ (опровергает возможность построения
правильного 7-угольника), непостроимость отрезка длины~$\pi$.

\begin{remark}
  Стоит оговориться, что поскольку построение любого отрезка длины~$x$
  эквивалентно построению его на оси абсцисс, то можно вести речь попросту
  о~<<построимости числа>>.

  Если рассматривать координатную плоскость как комплексную, то практически все
  задачи могут быть сформулированы как построение какого-то определенного
  числа~$z \in \mathbb{C}$ или же набора чисел.
\end{remark}

\subsection{Поле построимых чисел}

Уточнив постановку задачи, можно сформулировать несколько простых наблюдений.
Первое из них состоит в~том, что задача о~построении некоторой точки на
плоскости эквивалентна построению обеих её координат (проекций на оси или любых
отрезков, равных проекциям по длине). Второе же заключается в~том, что множество
чисел, построимых, например, на оси абсцисс замкнуто относительно операций
сложения, умножения и~обратных к~ним (для каждой операции можно поредъявить свое
несложное геометрическое построеное), то есть представляет собой поле. Эти два
наблюдения подитоживаются следующим предложением.

\begin{claim}
  Множество построимых комплексных чисел $\widetilde{P}$ является полем
  и~представимо как множество пар построимых вещественных чисел $\{(x,y) \mid x,
  y \in P\}$, которые также образуют поле.
\end{claim}

\begin{remark}
  Аналогичное утверждение можно сформулировать в~случае, если в~задаче на уже
  даны какие-то числа, отрезки или углы. Множество построимых чисел по-прежнему
  останется полем, структура которого, как выяснится, устроена похожим образом.
\end{remark}

\subsection{Расширения полей. Квадратичные расширения}

\begin{definition}
  Ситуацию, когда поле~$K_1$ является подполем поля~$K_2$, называют
  \emph{расширением}   полей. Одно или несколько последовательных
  расширений~$K_1 \subset \ldots \subset K_n$ называют \emph{башней} расширений.
\end{definition}

\begin{remark}
  Важное свойство расширения~$K_1 \subset K_2$ состоит в~том, что~$K_2$
  представляет собой линейное пространство над~$K_1$. Размерность такого
  линейного пространства называется \emph{степенью} расширения.
\end{remark}

\begin{definition}
  Расширение~$K_1 \subset K_2$ называется квадратичным, если его степень равна
  2 (пишут $[K_2\colon K_1] = 2$).
\end{definition}

\begin{claim}
  Пусть есть некоторое поле~$K$, являющееся для простоты подполем $\mathbb{C}$ и
  задано уравнение квадратное уравнение $x^2 = a$, которое не имеет решений
  в~$K$. Пусть $\sqrt{a}$~--- это какое-то из решений уравнения в~$\mathbb{C}$.
  Тогда минимальное поле $\widetilde{K}$, содержащее~$K$ и~$\sqrt{a}$
  (обозначается $K[\sqrt{a}]$), можно представить как~$\widetilde{K} = \{x +
  y\sqrt{a} \mid x, y \in K\}$, причём все такие линейные комбинации различны.
\end{claim}
\begin{proof}
  Очевидно, что все такие линейные комбинации должны лежать в~$K[\sqrt{a}]$
  в~силу того, что оно замкнуто и~содержит~$K$ и~$\sqrt{a}$. Достаточно
  непосредственно проверить, что $\widetilde{K}$ является полем, тогда по
  стандартному рассуждению оно и~будет минимальным.

  Если же какие-то из линейных комбинаций $x + y\sqrt{a}$ совпадают, то это бы
  значило, что~$\sqrt{a}$ выражается через элементы~$K$, то есть лежит в~$K$,
  что противоречит посылке.
\end{proof}

\begin{remark}
  Расширение~$K \subset K[\sqrt{a}]$ квадратично.
\end{remark}
\begin{remark}
  В~доказательстве мы использовали то, что $\frac{x + y\sqrt{a}}{p + q\sqrt{a}}
  \in \widetilde{K}$. Приём, использующийся для доказательства этого простого
  утверждения, называется домножением на <<сопряжённое>>.\footnote{В~общем
  случае операция сопряжения в~расширении $K_1 \subset K_2$~--- это
  автоморфизм~$K_2$, сохраняющий~$K_1$.

  Легко убедиться, что единственный нетривиальный автоморфизм в~расширении~$K
  \subset K[\sqrt{a}]$ переводит $x + y\sqrt{a}$ в~$x - y\sqrt{a}$.

  Как устанавливается в~теории Галуа, для широкого класса расширений количество
  таких автоморфизмов совпадает со степенью расширения (если она конечна), а~их
  группа, называемая группой Галуа, заключает в~себе много информации
  о~свойствах расширения. В~частности со свойствами группы Галуа связана
  выразимость корней уравнений высших степеней в~радикалах.}
\end{remark}

Следующее наблюдение состоит в~том, что любое квадратичное расширение поля~$K$
может быть получено добавлением квадратного корня некоторого числа $x \in K$.
В~самом деле, в~качестве базиса в~$L \supset K$ могут быть выбраны
числа~$1, x$, притом известно, что число~$(1 + x)(1 - x) = 1 - x^2 \in L$, что
означает, существует квадратное уравнение, имеющее~$x$ своим корнем. Тогда $x$
лежит в~$K[\sqrt{D}]$, где~$D$~--- дискриминант этого уравнения.

Имея число~$a$, простым геометрическим пострением можно получить
число~$\sqrt{a}$, поэтому любое число из любой башни квадратичных расширений
сторится циркулем и~линейкой.

С~другой стороны, в~соответствии с~описанием алгоритма построения циркулем
и~линейкой, получение новых точек на каких-то шагах алгоритма происходит
с~помощью пересечения прямых и~окружностей, параметры которых (угловые
коэффициенты, центры и~радиусы) лежат в~некотором поле чисел, которые можно
считать уже построенными (в~комплексной семантике это минимальное поле,
содержащее~$\mathbb{C}$ и~все точки, отмеченные в~ходе алгоритма до
рассматриваемого шага).

Самый простой случай~--- пресечение двух прямых. Легкая выкладка показывает, что
точка пересечения двух прямых, заданных уравнениями с~коэффициентами в~$K$,
лежит в~$K$. Пересечение прямой и~окружности же сводится к~решению квадратного
уравнения, корни которого лежат в~$K[\sqrt{D}]$. Наконец, пересечение двух
окружностей может быть сведено к~пересечению прямой и~окружности, так как
разность уравенений вида~$(x-a)^2 + (y-b)^2 = c^2$ будет линейным уравенинем.
Итого, точка, построенная на следующем шаге алгоритма либо лежит в~$K$, либо
в~квадратичном расширении~$K$.

Итого, сделанные наблюдения позволяют сформулировать следующее утверждение,
характеризующее поле построимых чисел.

\begin{claim}
  Поле построимых чисел~$P$ состоит из всех чисел $\alpha$, для которых $\exists
  K_0 \subset K_1 \subset \ldots \subset K_n, K_0 = \mathbb{Q}, [K_{i+1}:K_i] =
  2, \alpha \in K_n$.

  Иными словами, построимы только элементы, лежащие в~какой-либо башне
  квадратичных расширений.
\end{claim}

Следующее важное наблюдение описывает строение башен квадратичных расишрений.

\begin{theorem}
  Пусть $K \subset L \subset T$~--- двухэтажная башня конечных расширений полей,
  причем элементы $\{x_1, \ldots, x_{[L:K]}\}$ представляют собой базис~$L$
  над~$K$, а~$\{y_1, \ldots, y_{[T:L]} \}$~--- базис~$T$ над~$L$. Тогда
  расширение~$K \subset T$ конечно, имеет базис~$\{x_i y_j \mid 1 \le i \le
  [L:K], 1 \le j \le [T:L]\}$ и~степень~$[T:K] = [T:L] \cdot [L:K]$.
\end{theorem}

\begin{exercise}
  Доказать теорему.
\end{exercise}

\begin{remark}
  Простое, но очень важное следствие теоремы: если каждое расширение
  в~башне~$K_1 \subset \ldots \subset K_n$ конечно, то $K_1 \subset K_n$ тоже
  конечно.

  Несмотря на простоту, теорема представляет собой мощный инструмент: она
  позволяет по-другому доказать то, что построимые числа образуют поле, а~также,
  например, то, что полем являются все алгебриаические числа.
\end{remark}

Другим немедленным следствием теоремы является такое утверждение:

\begin{claim}
  Любое построимое число $\alpha \in P$ обладает
  свойством~$[\mathbb{Q}[\alpha]:\mathbb{Q}] = 2^r$ для некоторого
  натурального~$r$.
\end{claim}
\begin{proof}
  По характеристическому свойству построимых чисел получаем, что существует
  башня расширений~$K_0 \subset \ldots K_n$, $[K_{i+1}:K_i] = 2$, $\alpha \in
  K_n, K_0 = \mathbb{Q}$. Из того, что $\alpha \in K_n$ следует, что
  $\mathbb{Q}[\alpha] \subset K_n$, то есть имеет место башня
  расширений~$\mathbb{Q} \subset \mathbb{Q}[\alpha] \subset K_n$. По тоереме
  о~степени расширения, $2^n = [K_n:\mathbb{Q}] =
  [\mathbb{Q}[\alpha]:\mathbb{Q}] \cdot [K_n:\mathbb{Q}[\alpha]]$, откуда
  немедленно следует, что~$[\mathbb{Q}[\alpha]:\mathbb{Q}]$ является степенью
  двойки.
\end{proof}

\begin{remark}
  Утверждение можно использовать как инструмент для доказательства
  непостроимости каких-либо чисел. Так, если показать,
  что~$[\mathbb{Q}[\alpha]:\mathbb{Q}]$ не является степенью двойки, то
  из характеристического свойства и~утверждения выше немедленно получается, что
  $\alpha \notin P$.
\end{remark}

Теперь задача о~построимости числа~$\alpha$ практически свелась к~задаче
о~подсчёте степени расширения~$\mathbb{Q}[\alpha]$. Мощным инструментом для
поиска степени расширения оказывается следующая теорема:

\begin{theorem}
  Пусть~$K$~--- поле, $f(x) \in K[x]$~--- неразложимый многочлен, $\deg f = l,
  \alpha$~--- корень\footnote{В~этом месте стоит оговориться, откуда берётся
  $\alpha$. Например, если $K$ является подполем~$\mathbb{C}$, то~$\alpha$ можно
  брать из~$\mathbb{C}$. В~общем же случае можно показать, что существует
  конструкция поля, являющаяся расширением~$K$, в~котором у~$f$ есть один корень
  или даже все~$l$. Здесь и~далее неявно полагается, что $K \subset \mathbb{C}$,
  однако соответсвующие рассуждения можно провести и~в~общем случае}~$f(x)$.
  Тогда~$[K[\alpha]:K] = l$.
\end{theorem}
\begin{proof}
  Покажем, что~$K[\alpha] = \{a_0 + \ldots + a_{l-1}\alpha^{l-1}
  \mid a_0, \ldots, a_{l-1} \in K\}$. Тогда теорема будет доказана, так как
  такие выражения~$K[\alpha]$ содержать обязано. Осталось показать, что они
  образуют поле.

  Проверка замкнутости относительно сложения и~вычитания тривиальна. Для
  проверки умножения можно без потери общности считать, что старший коэффициент
  многочлена~$f$ равен единице (он не может быть нулём, так как $\deg f = l$).
  Пользуясь тем, что~$f(\alpha) = 0$, можно выразить~$\alpha^l$ как линейную
  комбинацию $1, \alpha, \ldots, \alpha^{l-1}$. Поэтому в~произведении двух
  линейных комбинаций вида $\sum a_i \alpha^i$ от всех степеней выше~$l$ можно
  избавиться.

  Осталось проверить наличие обратного по умножению элемента. Для этого как
  минимум нужно показать, что для никакая нетривиальная линейная комбинация~$1,
  \alpha, \ldots, \alpha^{l-1}$ не равна нулю. Такое равенство повлекло бы
  существование многочлена степени меньше~$l$, у~которого~$\alpha$ является
  корнем. Пусть~$g(x)$~--- многочлен минимальной степени среди всех многочленов,
  обнуляющих~$\alpha, \deg g < l$. Пусть также~$f$ даёт остаток~$\sigma$ при
  делении на~$g$: $f = gh + \sigma$. Но тогда, так как $f(\alpha) = 0, g(\alpha)
  = 0$, то~$\sigma(\alpha)  = 0$. Если $\sigma \not\equiv 0$, то $\deg
  \sigma < r$ по определению деления с~остатком, что противоречит
  определению~$g$. Значит~$\sigma \equiv 0$, что в~свою очередь влечёт
  противоречие с~неразложимостью~$f$.

  Таким образом, все линейные комбинации в~$K[x]$ различны. Осталось предъявить
  обратный по умножению элемент к~$h(\alpha) = v_0 + \ldots +
  v_{l-1}\alpha^{l-1}$. Пусть $h(x) = v_0 + \ldots + v_{l-1}x^{l-1}$, тогда,
  очевидно, что~$(f, h) = 1$, так как иначе~$f$ разложим. Но в~таком
  случае~$\exists g_1, g_2 \in K[x]: f(x)g_1(x) + h(x)g_2(x) \equiv 1$. При
  подставлении~$\alpha$ получается, что~$h(\alpha)g_2(\alpha) \equiv 1$, что
  означает, что $g_2(\alpha)$ и~будет обратным к~$h(\alpha)$ (если $\deg g_2 \ge
  l$, то от членов с~$\alpha$ в~степени выше, чем~$l-1$ можно избавиться
  стандартным способом).
\end{proof}

\subsection{Примеры непостроимых чисел}

\begin{claim}
  $x^3 - 2$ является неразложимым над~$\mathbb{Q}$ многочленом.
\end{claim}
\begin{proof}
  Пусть это неверно, тогда $x^3 - 2 = (x - a)h(x)$, тогда $a \in \mathbb{Q}$
  зануляет левую часть, то есть у~$x^3 - 2$ есть рациональный корень, что
  невозможно.
\end{proof}

Итак, $[\mathbb{Q}[\sqrt[3]{2}]:\mathbb{Q}] = 3$ и~число~$\sqrt[3]{2}$
непостроимо.

\begin{exercise}
  Записать минимальный многочлен для числа~$\alpha = \sin\frac{\pi}{18}$
  и~доказать его неразложимость.
\end{exercise}

\subsection{Построимость правильных многоугольников}
\subsubsection{Правильный 7-угольник}

Вопрос о~построимости правильного~$n$-угольника равносилен вопросу
о~построимости комплексного числа~$\xi = e^{\frac{2\pi}{n}}$ с~помощью циркуля
и~линейки. Несложно заметить, что~$2\cos\frac{2\pi}{n} = \xi + \xi^{-1}$.

Пусть~$n$ нечётно и~$\sigma_r = \xi^r + \xi^{-r} = 2\cos(\frac{2\pi r}{n})$
для~$r=1,\ldots,\frac{n-1}{2}$. Для решения вопроса о~построимости $\xi$ или,
что тоже самое, $\sigma_1$, нужно исследовать строение расширения $\mathbb{Q}
\subset \mathbb{Q}[\sigma_1]$.

Сперва можно исследовать арифметические свойства чисел $\sigma_n$ для~$n = 7$.
Как несложно посчитать, $\sigma_1^2 = \sigma_2 + 2, \sigma_2^2 = \sigma_4 + 2,
\sigma_4^2 = \sigma_1 + 2$. Вообще, таблица умножения в~$\mathbb{Q}[\sigma_1]$
выглядит следующим образом.

\begin{center}
  \begin{tabular}{c | c | c | c |}
    \centering
    & $\sigma_1$ & $\sigma_2$ & $\sigma_3$\\
    \hline
    $\sigma_1$ & $\sigma_2 + 2$ & $\sigma_1 + \sigma_3$ & $\sigma_2 +
    \sigma_3$\\
    \hline
    $\sigma_2$ & $\sigma_1 + \sigma_3$ & $\sigma_4 + 2$ & $\sigma_1 +
    \sigma_2$\\
    \hline
    $\sigma_3$ & $\sigma_2 + \sigma_3$ & $\sigma_1 + \sigma_2$ & $\sigma_1 +
    2$\\
    \hline
  \end{tabular}
\end{center}

Также~$\sigma_1 + \sigma_2 + \sigma_3 = -1$.

\begin{exercise}
  Пользуясь полученной таблицей, показать, что~$\sigma_1$ удовлетворяет
  уравнению~$\sigma_1^3 + \sigma_1^2 - 2\sigma_1 - 1 = 0$.
\end{exercise}

Таким образом вопрос о~построимости правильного семиугольника сведен
к~вопросу о~разложимости многочлена~$x^3 + x^2 - 2x - 1$. Однако рациональных
корней у~него нет (так как числитель рационального корня должен делить свободный
член, а~знаменатель~--- старший коэффициент), поэтому приводимым он быть не
может и~степень расширения $[\mathbb{Q}[\sigma_1]:\mathbb{Q}] = 3$ при~$n = 7$.
Итак, правильный семиугольник непосторим с~помощью циркуля и~линейки.

\subsubsection{Правильный 17-угольник}

Как было доказано Гауссом в~своё время, для правильного 17-угольника алгоритм
построения циркулем и~линейкой существует. Поэтому целью здесь будет являться не
просто нахождение степени расширения~$\mathbb{Q}[\sigma_1]$, а~построение
конкретной башни квадратичных расширений, последнее из которых
содержит~$\sigma_1$. Для начала стоит снова немного изучить арифметические
свойства чисел~$\sigma_i$.

Пусть~$\tau_1 = \sigma_1 + \sigma_2 + \sigma_4 + \sigma_8, \tau_2 = \sigma_3 +
\sigma_5 + \sigma_6 + \sigma_7, \tau_1 + \tau_2 = -1$.

Тогда~$\tau_1^2 = \tau_1 + 8 + 2(\sigma_1 + \sigma_3 + \sigma_3 + \sigma_5 +
\sigma_7 + \sigma_8 + \sigma_2 + \sigma_6 + \sigma_6 + \sigma_7 + \sigma_4 +
\sigma_5) = \tau_1 + 8 + 2(\tau_1 + 2\tau_2) = 8 + \tau_1 + 2\tau_1 + 4(-1 -
\tau_1) = 4 - \tau_1$. Значит, числа~$\tau_1$ и~$\tau_2$ строятся циркулем
и~линейкой (лежат в~$\mathbb{Q}[\sqrt{17}]$).

Далее нужно разбить~$\tau_1$ и~$\tau_2$ следующим образом:
\begin{gather*}
  \tau_1 = \underbrace{\sigma_1 + \sigma_4}_{\beta_1} + \underbrace{\sigma_2 +
  \sigma_8}_{\beta_2},\\
  \tau_2 = \underbrace{\sigma_3 + \sigma_5}_{\beta_3} + \underbrace{\sigma_6 +
  \sigma_7}_{\beta_4}.
\end{gather*}

Число~$\beta_1 + \beta_2$ уже построено, поэтому нужно понять, чему равно
произведение~$\beta_1 \beta_2 = (\sigma_1 + \sigma_4)(\sigma_2 + \sigma_8) =
\sigma_1 + \sigma_3 + \sigma_2 + \sigma_6 + \sigma_7 + \sigma_8 + \sigma_4 +
\sigma_5 = -1$. Тогда по теореме Виетта, числа~$\beta_1, \beta_2$ также будут
построимыми. Аналогично, $\beta_3 \beta_4 = -1$, поэтому все~$\beta_i$ будут
построимы, притом добавить надо числа $\sqrt{\tau_1^2 + 4}$ и~$\sqrt{\tau_2^2 +
4}$.

Осталось вычислить, что такое~$\sigma_1 \sigma_4 = \sigma_3 + \sigma_5 =
\beta_3$. Тогда сумма и~произведение чисел~$\sigma_1$ и~$\sigma_4$ оказываются
уже построены, то есть при расширении поля корнем~$\sqrt{\beta_1^2 - 4\beta_3}$,
получается поле, содержащее~$\sigma_1$.

Итак, алгоритм построения правильного 17-угольника полностью описан.

\end{document}
