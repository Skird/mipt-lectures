\documentclass{article}

\usepackage[utf8x]{inputenc}
\usepackage[english,russian]{babel}
\usepackage{amsmath,amscd}
\usepackage{amsthm}
\usepackage{mathtools}
\usepackage{amsfonts}
\usepackage{amssymb}
\usepackage{cmap}
\usepackage{centernot}
\usepackage{enumitem}
\usepackage{perpage}
\usepackage{chngcntr}
%\usepackage{minted}
\usepackage[bookmarks=true,pdfborder={0 0 0 }]{hyperref}
\usepackage{indentfirst}
\hypersetup{
  colorlinks,
  citecolor=black,
  filecolor=black,
  linkcolor=black,
  urlcolor=black
}

\newtheorem*{conclusion}{Вывод}
\newtheorem{theorem}{Теорема}
\newtheorem{lemma}{Лемма}
\newtheorem*{corollary}{Следствие}

\theoremstyle{definition}
\newtheorem*{problem}{Задача}
\newtheorem{claim}{Утверждение}
\newtheorem{exercise}{Упражнение}
\newtheorem{definition}{Определение}
\newtheorem{example}{Пример}

\theoremstyle{remark}
\newtheorem*{remark}{Замечание}

\newcommand{\doublearrow}{\twoheadrightarrow}
\renewcommand{\le}{\leqslant}
\renewcommand{\ge}{\geqslant}
\newcommand{\eps}{\varepsilon}
\renewcommand{\phi}{\varphi}
\newcommand{\ndiv}{\centernot\mid}

\MakePerPage{footnote}
\renewcommand*{\thefootnote}{\fnsymbol{footnote}}

\newcommand{\resetcntrs}{\setcounter{theorem}{0}\setcounter{definition}{0}
\setcounter{claim}{0}\setcounter{exercise}{0}}

\DeclareMathOperator{\aut}{aut}
\DeclareMathOperator{\cov}{cov}
\DeclareMathOperator{\argmin}{argmin}
\DeclareMathOperator{\argmax}{argmax}
\DeclareMathOperator*\lowlim{\underline{lim}}
\DeclareMathOperator*\uplim{\overline{lim}}
\DeclareMathOperator{\re}{Re}
\DeclareMathOperator{\im}{Im}

\frenchspacing


\begin{document}

\section*{Лекция 2. Великая теорема Ферма для малых показателей}
\addcontentsline{toc}{section}{Лекция 2. Великая теорема Ферма для малых
показателей}
\resetcntrs

\begin{theorem}[Ферма]
  Для любого $n > 2$ не существует отличных от нуля натуральных чисел
  $x, y, z$ таких, что $x^n + y^n = z^n$.
\end{theorem}

\section{Великая теорема Ферма при $n = 4$}

\subsection{Доказательство при $n = 4$}

Это самый простой случай теоремы, доказать его можно, не выходя за пределы
натуральных чисел. Доказательство, которое в~свое время придумал Пьер
Ферма\footnote{Это одно их тех редких его доказательств, что были записаны
и~дошли до нас}, основано на изобретенном им методе <<бесконечного спуска>>,
в~сущности, одном из видов индукции. В~первую очередь, как это часто бывает,
рассуждая по индукции, удобно перейти к~более общему уравнению $x^4+y^4=z^2$.
Шаг будет заключаться в~следующем: пусть из всех решений данного уравнения
некоторое (положительное, так как знаки переменных неважны) решение~$(x, y, z)$
имеет наименьший~$z$. Если по этому решению можно построить другое,
с~меньшим~$z$, то теорема будет доказана.

\begin{claim}
Числа~$x, y, z$ попарно взаимнопросты (пишут~$\ll\!x, y, z\!\gg = 1$, где под
записью~$\ll\!a_1, \ldots, a_n\!\gg$ понимается максимальный из попарных
наибольших общих делителей~$a_i$ и~$a_j, 1 \le i \ne j \le n$).
\end{claim}
\begin{proof}
  Пусть какие-либо два числа из~$x, y, z$ делятся на простое число~$p$.
  Тогда очевидно, что третье также делится на $p$. Тогда
  \begin{gather*}
    x = p \overline{x}, y = p \overline{y}, z = p \overline{z},\\
    p^4 (\overline{x}^4 + \overline{y}^4) = p^2 \overline{z}^2,\\
    p^2 (\overline{x}^4 + \overline{y}^4) = \overline{z}^2,\\
    p \mid \overline{z}^2 \Rightarrow p \mid \overline{z} \Rightarrow
    \overline{z} = p \overline{\overline{z}},\\
    \overline{x}^4 + \overline{y}^4 = \overline{\overline{z}}^2,
    \overline{\overline{z}} < z.
  \end{gather*}
  Противоречие с~минимальностью $z$.
\end{proof}

Далее, $(x^2,y^2,z)$~--- пифагорова тройка. Число~$z$ обязательно нечётно,
а~среди~$x$ и~$y$ чётным без потери общности можно считать~$x$. Тогда
\begin{gather*}
x = 2x_1,\\
4x_1^2 = 2mn, y^2 = m^2 - n^2, z = m^2 + n^2,\\
y^2 \bmod 4 = 1 \Rightarrow 2 \ndiv m, 2 \mid n \Rightarrow n = 2n_1,\\
x_1^2=m n_1
\end{gather*}
Далее, если $p \mid \gcd(m,n)$, то $p \mid \gcd(x,y) = 1$, значит~$m$ и~$n$
взаимно просты, поэтому $m$ и~$n_1$ тоже. Тогда $m = a^2, n_1 = b^2$ для
каких-то натуральных $a, b$.

Из того, что~$y^2 = m^2 - 4n_1^2 \Rightarrow y^2 + 4n_1^2 = m^2$ следует,
что~$(y, m, 2n_1)$~--- пифагорова тройка. Тогда для каких-то взаимно простых
натуральных~$q$ и~$r$ выполнено~$m = q^2 + r^2, 2n_1=2qr$. Итак, $n_1 = qr$
и~одновременно~$n_1 = b^2$. Так как~$q$ и~$r$ взаинопросты, то они должны
являться полными квадратами: $q = t^2, r = s^2$. Итого,
\begin{gather*}
  m = q^2 + r^2, m = a^2, q = t^2, r = s^2,\\
  a^2 = t^4 + s^4.
\end{gather*}
Так как $a = \sqrt{m} \le m \le m^2 < z$, то тройка $(t, s, a)$ даёт
необходимое противоречие.

\subsection{Роль случая $n = 4$ в~общей задаче}

Стоит отдельно отметить, что в~общем случае теоремы Ферма, если $n > 2$, то
либо~$4 \mid n$, либо~$n$ имеет нечётный простой делитель. Если $n = 4k$,
то соотношение $(x^k)^4 + (y^k)^4 = (z^k)^4$ противоречит только что проведённым
рассуждениям. Аналогичное рассуждение полностью сводит задачу к~рассмотрению
только простых значений $n$.

Куммер в~середине XIX~века доказал теорему для широкого класса регулярных
простых (предположительно их плотность в~натуральном ряде не превосходит
40\%, а~в~первой сотне нерегулярных простых только три:~37,~59,~67). Позже
с~помощью компьютера его доказательство было доработано для всех простых, не
превосходящих 2521 (1954~г.), а~позже 125,000 (1978~г.) и~4,000,000 (1993~г.).
Однако полностью решить задачу используя машинные вычисления практически
невозможно.

\begin{exercise}
  Если бы теорема Ферма была бы неверна и~$x^n + y^n = z^n$, то~$|x|,|y|,|z|>n$.
\end{exercise}

Это упражнение иллюстрирует, что компьютерный поиск контрпримера требовал бы
проведения операций с~числами порядка $n^n$, что для даже для~$n$
порядка~125,000 представляет вычислительно сложную задачу.

\section{Числа Эйзенштейна}

\subsection{Норма и~обратимые элементы}

Для решения задачи при $n = 3$ необходимо исследовать структуру
кольца~$\mathbb{Z}[\omega]$\footnote{$\omega = e^{\frac{2\pi}{3}i} =
-\frac{1}{2} + \frac{\sqrt{3}}{2}i$~--- примитивный корень из единицы степени
3.}~--- так называемых чисел Эйзенштейна.

\begin{exercise}
  Пусть $\xi \ne 1$~--- любой нетривиальный корень из единицы
  степени~$p$ ($p$~--- простое), тогда
  $$x^p + y^p = (x + y)(x + \xi y) \ldots (x + \xi^{p-1} y).$$
\end{exercise}

Таким образом, в~кольце $\mathbb{Z}[\omega]$ выражение $x^3 + y^3$
раскладывается на линейные множители, что значительно облегчает анализ.

Стоит напомнить, что в~общем случае $\mathbb{Z}[\xi_p]$ не является
факториальным кольцом и~основная теорема арифметики в~нем не выполнена (первый
такой пример при $p = 23$), что в~свое время помешало Ламе построить
доказательство для общего случая. Для некоторых простых можно доказать теорему
из других соображений, в~частности, Софи Жермен сделала это в~случае,
если~$p$ и~$2p + 1$ одновременно простые\footnote{Такие простые в~честь неё
названы простыми Софи Жермен.} и~$p \ndiv xyz$. Надо заметить, что
случай~$p \mid xyz$ сильно сложнее для анализа даже при $p = 3$
(в~доказательстве существенно используется то обстоятельство, что $3$~--- не
простое число в~$\mathbb{Z}[\omega]$).
\begin{exercise}[сложное]
  Найти разложения на простые множители $5$ в~$\mathbb{Z}[\xi_5]$ и~$p$
  в~$\mathbb{Z}[\xi_p]$\footnote{В~нефакториальных кольцах за определение
  простого берется свойство $p \mid ab \Rightarrow p \mid a$ или~$p \mid b$.}.
\end{exercise}

Итак, $\mathbb{Z}[\omega] = \{a + b \omega \mid a, b \in \mathbb{Z} \}$, так как
$\omega^2 = -1 - \omega$ (более строго, здесь сказано, что этот набор чисел
является кольцом и~что все числа такого вида лежат в~$\mathbb{Z}[\omega]$,
которое по определению есть минимальное кольцо,
содержащее~$\mathbb{Z}$ и~$\omega$). Невероятно удобная и~естественная
визуализация чисел Эйзенштейна~--- изображение их на комплексной плоскости,
где они формируют полное замощение правильными треугольниками.

\begin{definition}
  Нормой числа $z = a + b \omega$ называется $N(z) = a^2 - ab + b^2$. Легко
  убедиться, что $N(z) = a^2 + ab(\omega^2 + \omega) + b^2 \omega^3 =
  (a + b \omega)(a + b \omega^2) = (a + b \omega) (a + b \overline{\omega}) =
  z \overline{z}$, то есть~$N(z)$~--- это квадрат привычной комплексной нормы.
\end{definition}

Удобное свойство такой нормы~--- мультипликативность. В~самом деле, очевидно,
что $\forall a, b \in \mathbb{Z}[\omega] \rightarrow N(ab) = N(a) N(b)$. Стоит
отметить, что это ни в~коем случае не является аксимой нормы, и~в~других кольцах
это свойство может не быть выполнено.

При исследовании евклидового кольца первоочередная задача заключается в~описании
его мультипликативной группы ведь основная теорема арифметики верна с~точностью
до умножения на обратимые элементы.

\begin{claim}
    $a \in \mathbb{Z}[\omega]$~--- обратим $\Leftrightarrow N(a) = 1$.
\end{claim}
\begin{proof}
  $N(a) = 1 \Rightarrow a \overline{a} = 1 \Rightarrow \overline{a}$~---
  обратный к~$a$ элемент.

  Если $ab = 1$, то $ab \overline{ab} = 1 \Rightarrow N(a)N(b) = 1$.
  Произведение двух целых положительных чисел равно $1$ тогда и~только тогда,
  когда каждой из них равно $1$, то есть $N(a) = 1$.
\end{proof}

\begin{gather*}
  N(a + b \omega) = 1 \Leftrightarrow a^2 - ab + b^2 = 1,\\
  4 a^2 - 4ab + 4b^2 = 4,\\
  (2a - b)^2 + 3b^2 = 4.
\end{gather*}

Далее очевидно, что задача сводится к~перебору целых значений~$b$ от~$-1$
до~$1$. При фиксированном значении $b$ для $a$ существует не более двух
возможных значений.

\begin{exercise}
  Перебрав варианты, показать, что обратимыми в~кольце чисел Эйзенштейна
  являются элементы $1, -1, \omega, -\omega, 1 + \omega, -1 - \omega$.
\end{exercise}

Если записать эти элементы в~другом виде, то можно увидеть, что все они
представляют собой степени числа~$1 + \omega$, которое является примитивным
корнем степени шесть из единицы. Это также означает, что мультипликативная
группа кольца чисел Эйзенштейна изоморфна $\mathbb{Z}_6$.

\subsection{Число $\lambda$}

Дальнейшее исследование коснется важного числа $\lambda = 1 - \omega$. Первое
наблюдение состоит в~том, что~$N(\lambda) = 3$.

\begin{claim}
  Если норма числа $p \in \mathbb{Z}[\omega]$~--- простое число, то само
  число~$p$ тоже простое.
\end{claim}
\begin{proof}
  Если $p$ равно произведению двух неразложимых необратимых элементов~$a$ и~$b$,
  то $N(p)=N(a)N(b)$, то есть какая-то из норм равна $1$, а~в~этом случае
  или~$a$ или $b$~--- обратимый элемент, как было показано ранее\footnote{
  В~этом доказательстве использована как основная теорема арифметики, так
  и~мультипликативность нормы, поэтому в~произвольном кольце оно не проходит.
  Более того, можно убедиться, что в~произвольном кольце утверждение неверно.}.
\end{proof}

Более того, тот факт, что норма числа~$\lambda$ равна~$3$, автоматически
означает, что~$3 = \lambda \overline{\lambda} = (1 - \omega)(2 + \omega)$ в~этом
кольце не является простым числом. Это обстоятельство помогает разобрать важный
случай~$3 \mid xyz$, который в~общем случае ($n \mid xyz$) представляет
наибольшую трудность (в~частности, как было сказано выше, Софи Жермен удалось
доказать вариант теоремы для обширного класса простых, но только
при~$n \ndiv xyz$). С~точки зрения делимости,~$\lambda$, как и~любое другое
простое число, имеет ровно~12 делителей~--- 6~обратимых и~6~ассоциированных.

Естественным образом, решение уравнения Ферма (доказательство отсутствия
решений) в~числах Эйзенштейна автоматически решает задачу и~в~целых числах.
Используя внутренние симметрии кольца, можно заметить, что домножение~$x$,~$y$
или~$z$ на любой корень из единицы третьей степени (и~даже шестой, так от этого
меняется только знак соответсвующего слагаемого) не меняет множество решений.
Так, например, если одно из чисел~$x + y$, $x + y\omega$, $x + y\omega^2$
делится на какое-то число $q$, то без потери общности можно считать, что
это число~$x + y$, так как в~противном случае, домножая~$x$ и~$y$ на нужную
степень $\omega$, можно получить тройку, все еще являющуюся решением уравнения
Ферма и~удовлетворяющую нужному свойству.

\begin{claim}
  Пусть~$x \in \mathbb{Z}[\omega]$, $\lambda \ndiv x$. Тогда $x^3 \equiv \pm 1
  \pmod 9$.
\end{claim}
\begin{exercise}
  \leavevmode
  \begin{enumerate}[label=(\arabic*)]
    \item
      В~$\mathbb{Z}[\omega]$ существует ровно~$3$ класса вычетов по
      модулю~$\lambda$: $\{0, 1, -1\}$.
    \item
      В~$\mathbb{Z}[\omega]$ существует ровно~$N(p)$ классов вычетов по
      модулю~$p$.
  \end{enumerate}
\end{exercise}
\begin{proof}
  Если~$x$ не кратен~$\lambda$, то~$x = r\lambda \pm 1$. Тогда
  $$x^3 = r^3 \lambda^3 \pm 3 r^2 \lambda^2 + 3r\lambda \pm 1.$$
  Учитывая, что~$3\lambda^2 \equiv 0 \pmod 9$, необходимо показать,
  что~$r^3\lambda^3 + 3r\lambda$ делится на~$9$.
  $$r^3\lambda^3 + 3r\lambda = 3r\lambda - 3r^3\lambda\omega =
  3\lambda (r - r^3 \omega).$$
  В~свою очередь~$r = 0, 1$ или~$-1 \pmod \lambda$. Если~$\lambda \mid r$,
  то при вынесении $r$ выражение перед скобками делится на~$9$.
  Иначе~$r = q\lambda \pm 1$, в~этом случае~$r^2 = q^2 \lambda^2 \pm 2 q\lambda
  + 1$. Тогда~$\lambda \mid (r^2 - 1)$, то есть~$r^2 = \lambda s + 1$. Итого,
  $$3\lambda r(1 - r^2\omega) = 3\lambda r(1 - \omega - \lambda s \omega) =
  3\lambda r(\lambda - \lambda s \omega) = 3\lambda^2 (1 - s\omega)
  \equiv 0 \pmod 9$$
\end{proof}
\begin{remark}
  Геометрический смысл утверждения заключается в~том, что числа,
  кратные~$\lambda$, но не кратные~$\lambda^2$, в~кубе не могут попасть на
  расстояние меньше~$3$ от чисел, кратных~$\lambda^4$.
\end{remark}

\section{Великая теорема Ферма при $n = 3$}

\subsection{Основная теорема арифметики в~$\mathbb{Z}[\omega]$}

\begin{claim}
  $\pm x^3 \pm y^3 \pm z^3 \ne 0$ при~$\lambda \ndiv xyz$.
\end{claim}
\begin{proof}
  Если ни одно из чисел~$x, y, z$ не делится на $\lambda$, то их кубы дают
  остаток $\pm 1$ по модулю $9$, а~значит их сумма не сравнима с~$0$ по
  модулю~$9$, то есть не равна~$0$.
\end{proof}

Далее можно считать, что~$\ll\!x, y, z\!\gg = 1$, значит ровно одно число
делится на~$\lambda$. Исследовать этот случай можно методом <<бесконечного
спуска>>, рассмотрев более общее уравнение $\eps_1 x^3 + \eps_2 y^3 +
\eps_3 z^3 = 0$, где $\eps_1, \eps_2, \eps_3$~--- произвольные обратимые,
$xyz \ne 0$, $\ll\!x, y, z\!\gg = 1, \lambda \mid xyz$. Первое ключевое
рассуждение заключается в~следующем утверждении.

\begin{claim}
  $\lambda^2 \mid xyz$.
\end{claim}
\begin{proof}
  Пусть без потери общности $z = -\lambda \overline{z}$. Тогда
  $$ \eps_1 x^3 + \eps_2 y^3 = \eps_3 \lambda^3 \overline{z}^3,
  \lambda \ndiv \overline{z}.$$
  Тогда по модулю~$9$ левая часть есть $\pm \eps_1 \pm \eps_2$, а~правая не
  равна~$0$ и~делится на $\lambda^3$, но не на $\lambda^4$. По предыдущим
  утверждениям это противоречие.
\end{proof}

% Эту часть, вопреки ходу лекции, хочется поднять на раздел выше. Впрочем,
% практически не имеет значения.
Прежде чем предпринять следующий шаг, необходимо внести ясность в~вопрос об
основной теореме арифметики в~кольце чисел Эйзенштейна. Так как норма уже была
введена, осталось только привести правило деления с~остатком, согласующееся
с~нормой в~смысле определения евклидового кольца.

\begin{claim}
  Если $z_1, z_2 \in \mathbb{Z}[\omega]$, то $\exists \beta, \gamma \in
  \mathbb{Z}[\omega]\colon z_1 = z_2 \beta + \gamma$,
  причём~$N(\gamma) < N(z_2)$\footnote{Вообще говоря, норма нулевого элемента
  евклидового кольца не определена (для примера можно рассмотреть кольцо
  многочленов, которое тоже является евклидовым с~нормой, равной степени
  многочлена). Но, работая с~кольцами, аналогичными числам Эйзенштейна, как
  правило оставляют за нулевым элементом нулевую норму}.
\end{claim}
\begin{proof}
  $\frac{z_1}{z_2} = \frac{a + b\omega}{c + d\omega} = \alpha + \beta \omega,
  \alpha, \beta \in \mathbb{R}$.
  Пусть~$r, s$ являются округлёнными до ближайшего целого числами $\alpha,
  \beta$. Тогда для числа~$\gamma / z_2 = (\alpha - r) + (\beta - s) \omega$
  верно, что~$|\alpha - r|,|\beta - s| \le 0.5$ (здесь число $\gamma$ неявно
  определено через числа $\alpha, \beta, r, s, z_2)$). В~таком случае, сообразно
  с~формулой для нормы, получается
  $$N(\frac{\gamma}{z_2}) \le |\alpha - r|^2 + |\alpha - r||\beta - s| +
  |\beta - r|^2 \le \frac{3}{4} \Rightarrow N(\gamma) < N(z_2).$$

  Итого $z_1 = z_2 (r + s \omega) + \gamma$, $N(\gamma) < N(z_2)$. $\gamma$
  будет числом Эйзенштейна, так как все остальные числа в~равенстве лежат
  в~$\mathbb{Z}[\omega]$.
\end{proof}

\begin{exercise}[сложное]
  \label{ex_norm_5}
  Найти норму в~кольце $\mathbb{Z}[\xi_5]$.
\end{exercise}

\subsection{Доказательство при~$n = 3$}

Теперь, автоматически получив основную теорему арифметики для кольца чисел
Эйзенштейна, можно сформулировать второе ключевое утверждение, которое по сути
является шагом метода <<бесконечного спуска>>.

\begin{claim}
  \label{fermat_3}
  Если $\eps_1 x^3 + \eps_2 y^3 = \eps_3 \lambda^{3k} z^3, k \ge 2$,
  причём~$\lambda \ndiv z$, то существуют числа~$\overline{x}, \overline{y},
  \overline{z}, \sigma_1, \sigma_2, \sigma_3$, являющиеся решенем уравнения
  $\overline \sigma_1 \overline{x}^3 +
   \overline \sigma_2 \overline{y}^3 =
   \overline \sigma^3 \lambda^{3k-3} \overline{z}^3$,
   причём~$\lambda \ndiv \overline{z}$.
\end{claim}
\begin{proof}
  По модулю~$9$ данное уравнение обращается в~сравнение $\pm \eps_1 \pm \eps_2
  \equiv 0 \pmod 9$, что конечно, заменяется обычным равенством, то есть~$\eps_1
  = \pm \eps_2$. Итак,
  $$ \pm \eps_2 x^3 + \eps_2 y^3 = \eps_3 \lambda^{3k} z^3.$$

  Далее, поделив на $\eps_2$ и~меняя при необходимости знак~$x$, получаем
  \begin{gather*}
    x^3 + y^3 = u \lambda^{3k} z^3,\\
    (x + y)(x + y\omega)(x + y\omega^2) = u \lambda^{3k} z^3,\\
    (x + y) - (x - y\omega) = y \lambda, (x + y) - (x + y \omega^2) =
    y \lambda (1 + \omega)
  \end{gather*}

  Во-первых, невозможна ситуация, когда~$\lambda^2 \mid (x + y, x + y \omega)$,
  так как в~этом случае~$\lambda^2 \mid y\lambda \Rightarrow \lambda \mid y$,
  но в~то же время~$\lambda \mid z$, противоречие. Аналогично, наибольший общий
  делитель любой другой пары скобок не делится на~$\lambda^2$.

  Во-вторых, в~разложении~$(x + y, x + y \omega)$ не существует никаких простых
  множителей, кроме~$\lambda$. Если~$p \mid (x + y, x + y \omega)$,
  то~$p \mid y \lambda \Rightarrow p \mid y$. Однако, $p \mid (\omega(x + y) -
  (x + y \omega)) = -x \omega \Rightarrow p \mid x$, что невозможно, так
  как~$(x, y) = 1$. Так как $x + y \equiv x + y \omega \equiv x + y \omega^2
  \pmod \lambda$, то~$\ll\!x, y, z\!\gg \in \{1, \lambda \}$.

  Однако, так как правая часть уравнения делится на~$\lambda$, то и~левая тоже,
  значит каждая скобка делится на~$\lambda$. Более того, в~одну скобку~$\lambda$
  входит в~степени~$3k - 2 \ge 4$, а~в~остальные в~степени~$1$. Без потери
  общности $\lambda^{3k-2} \mid (x + y)$.

  \begin{gather*}
    x + y = \alpha \lambda^{3k-2}, x + y\omega = \beta \lambda,
    x + y\omega^2 = \gamma \lambda,\\
    \ll\! \alpha, \beta, \gamma \!\gg = 1,\\
    \alpha \beta \gamma = u z^3,\\
  \end{gather*}

  По основной теореме арифметики:
  \begin{gather*}
    x + y = \lambda^{3k-2} \alpha = \sigma_1 \lambda^{3k-2} \overline{x}^3,\\
    x + y\omega = \lambda \beta = \sigma_2 \lambda \overline{y}^3,\\
    x + y\omega^2 = \lambda \gamma = \sigma_3 \lambda \overline{z}^3.\\
  \end{gather*}

  Сложив с~коэффициентами $1, \omega, \omega^2$, получаем~$0$ в~левой, части,
  то есть
  \begin{gather*}
    0 = \sigma_1 \lambda^{3k-2} \overline{x}^3 +
        \sigma_2 \omega \lambda \overline{y}^3 +
        \sigma_3 \omega^2 \lambda \overline{z}^3,\\
    \overline \sigma_1 \lambda^{3k-3} \overline{x}^3 =
        \overline \sigma_2 \overline{y}^3 +
        \overline \sigma_3 \overline{z}^3,\\
    \ll\!\overline{x}, \overline{y}, \overline{z}\!\gg = 1.
  \end{gather*}

  Что в~точности и~нужно доказать.
\end{proof}

Таким образом, если существует решение уравнения Ферма со степенью вхождения
$\lambda$, равной $k$, то по доказанному можно за~$k-1$ шаг перейти к~решению со
степенью~$1$, а~таких решений нет.

\section{Великая теорема Ферма при $n = 5$}

\subsection{План доказательства при~$n = 5$}

Первый важный вопрос уже фигурировал раньше в~виде упражнения~\ref{ex_norm_5}.
Далее, необходимо немного исследовать структуру кольца $\mathbb{Z}[\xi_5]$.

\begin{exercise}[сложное]
  Найти мультипликативную группу $\mathbb{Z}[\xi_5]$.
  Сперва может быть полезно найти обратимые
  из~$\mathbb{Z}[\xi_5] \cap \mathbb{R}$.
\end{exercise}

\begin{exercise}
  Обобщить прием, использованный в~утверждении~\ref{fermat_3}, то есть найти
  способ скомбинировать уравнения
  \begin{gather*}
    x + y = \alpha_1 \lambda^{q},\\
    x + y \xi = \alpha_2 \lambda,\\
    \ldots,
  \end{gather*}
  чтобы снизить степень делимости на $\lambda$. Найти, чему в~этом случае
  равняется $\lambda$ и~найти разложение числа~$5$ на простые множители.
\end{exercise}

\begin{exercise}
  Доказать теорему Ферма при~$n = 5$.
\end{exercise}

\end{document}
