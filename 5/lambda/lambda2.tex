\documentclass{article}
\usepackage[utf8x]{inputenc}
\usepackage[english,russian]{babel}
\usepackage{amsmath,amscd}
\usepackage{amsthm}
\usepackage{amsfonts}
\usepackage{amssymb}
\usepackage{cmap}
\usepackage{centernot}
\usepackage{enumitem}
\usepackage{perpage}
\usepackage{chngcntr}
%\usepackage{minted}
\usepackage[bookmarks=true,pdfborder={0 0 0 }]{hyperref}
\usepackage{indentfirst}
\hypersetup{
  colorlinks,
  citecolor=black,
  filecolor=black,
  linkcolor=black,
  urlcolor=black
}

\newtheorem*{conclusion}{Вывод}
\newtheorem{theorem}{Теорема}
\newtheorem{lemma}{Лемма}
\newtheorem*{corollary}{Следствие}

\theoremstyle{definition}
\newtheorem*{problem}{Задача}
\newtheorem{claim}{Утверждение}
\newtheorem{exercise}{Упражнение}
\newtheorem{definition}{Определение}
\newtheorem{example}{Пример}

\theoremstyle{remark}
\newtheorem*{remark}{Замечание}

\renewcommand{\le}{\leqslant}
\renewcommand{\ge}{\geqslant}
\newcommand{\eps}{\varepsilon}
\renewcommand{\phi}{\varphi}
\newcommand{\ndiv}{\centernot\mid}

\MakePerPage{footnote}
\renewcommand*{\thefootnote}{\fnsymbol{footnote}}

\newcommand{\resetcntrs}{\setcounter{theorem}{0}\setcounter{definition}{0}
\setcounter{claim}{0}\setcounter{exercise}{0}}

\DeclareMathOperator{\aut}{aut}
\DeclareMathOperator{\cov}{cov}
\DeclareMathOperator{\chos}{ch}
\DeclareMathOperator{\argmin}{argmin}
\DeclareMathOperator{\argmax}{argmax}
\DeclareMathOperator*\lowlim{\underline{lim}}
\DeclareMathOperator*\uplim{\overline{lim}}
\DeclareMathOperator{\re}{Re}
\DeclareMathOperator{\im}{Im}

\frenchspacing


\begin{document}

Литература:
\begin{itemize}
	\item ???
\end{itemize}

\section{Лямбда-исчисление}

Безтиповое лямбда-исчисление: термы, $\alpha$-конверсия, $\beta, \eta$-редукция.

\begin{definition}
	$V = \{v_0, \ldots, v_1\} \sim \mathbb{N}$~--- алфавит. Определим $\lambda$-терм:
	\begin{itemize}
		\item $x \in V \Rightarrow x \in \Lambda$
		\item $M, N \in \Lambda \rightarrow (MN) \in \Lambda$
		\item $x \in V, M \in \Lambda \Rightarrow (\lambda x.M) \in \Lambda$
	\end{itemize}
\end{definition}

\begin{remark}
	Примечания: аппликация (применение) левоассоциативно, абстракция~--- наоборот.

	Конверсии бывают некорректные. Можно сформулировать правило, определяющее корректную подстановку.
	Но мы будем далее считать, что в~любом контексте ни одна переменная не имеет одновременно
	сводобных и~связанных вхождений. Для этого примерним автоматические переименование
	в~невстречающиеся: $\lambda x. M \rightarrow_\alpha \lambda y.(M[x \coloneqq y]), y \notin V(M)$.
\end{remark}

\begin{definition}
	Определим абстрактные редукции. $R \subset \Lambda \times \Lambda$~--- редукция. Условия
	совместимости:
	\begin{enumerate}
		\item $MRN \Rightarrow M \rightarrow_R N$
		\item $M \rightarrow_R N \Rightarrow LM \rightarrow_R LN, ML \rightarrow_R NL, \lambda x.M
			\Rightarrow_R \lambda x.N$
	\end{enumerate}

	Также замкнём нашу редукцию: $\doublearrow_R$ и~построим соответствующее отношение эквивалентности
	$=_R$.
\end{definition}

\begin{definition}
	Терм $R$-нормален, если $\nexists N M \rightarrow_R N$. Терм $N$ является нормальной формой терма
	$M$, если он $R$-нормальный и~$M =_R N$.

	Терм нормализуемый, если у~него есть нормальная форма и~сильно нормализуемый, если любая цепочка
	редукций обрывается (на $R$-нормальной форме).
\end{definition}

\begin{lemma}[О~ромбе]
	Пусть $R \in \{\beta, \beta\eta\}$. Тогда если $M \doublearrow N_1, M \doublearrow N_2$, то
	$\exists L: N_1 \doublearrow L, N_2 \doublearrow L$.
\end{lemma}

\begin{theorem}[Чёрча-Россера]
	Если $M =_R N \Rightarrow \exists L: M \doublearrow L, N \doublearrow L$.
\end{theorem}
\begin{corollary}
	Нормальная форма (если существует) единственна.
\end{corollary}

Нормальная форма существует не всегда. Не каждая цепочка преобразований ведёт к~нормальной форме.

\begin{claim}
	На базе лямбда-термов можно построить логику высказываний и~примитивно-рекурсивную арифметику,
	нужным образом закодировав нужные функции.
\end{claim}

Комбинатор неподвижной точки: $Y = \lambda f. (\lambda x.f(xx))(\lambda x.f(xx))$ обладает
интересным свойством $F(YF) = YF$.

\begin{example}
	Хотим найти комбинатор $M$, заданный условием $MXY = XM(MM)Y(MMM)$. Рассмотрим $F = \lambda mxy.
	xm(mm)y(mmm)$ и~положим $M = YF$. Тогда все получится.

	Можно находить и~совместные неподвижные точки: $X = FXY, Y = GXY$. Возьмём $\Phi = \lambda
	p.\left< F(\pi_1 p)(\pi_2 p), G(\pi_1 p)(\pi_2 p)\right>$. Тогда $Z = Y\Phi$ неподвижная точка,
	значит $Z = \Phi Z = \left< F(\pi_1 Z)(\pi_2 Z), G(\pi_1 Z)(\pi_2 Z)\right>$. Тогда остаётся взять
	$X = \pi_1 Z, Y = \pi_2 Z$.
\end{example}

Используя эту машинерию, можно построить все примитивно-рекурсивные функции, а~также другие виды
рекурсии, а~также и~минимизацию.

\begin{theorem}
	Следующие условия равносильны:
	\begin{itemize}
		\item $f$ представимо лямбда-термом
		\item $f$ вычислимо на машинах Тьюринга
		\item $f$ частично-рекурсивна (примитивная рекурсия + минимизация, возможны не тотальные
			функции)
	\end{itemize}
\end{theorem}

\end{document}
