\documentclass{article}
\usepackage[utf8x]{inputenc}
\usepackage[english,russian]{babel}
\usepackage{amsmath,amscd}
\usepackage{amsthm}
\usepackage{amsfonts}
\usepackage{amssymb}
\usepackage{cmap}
\usepackage{centernot}
\usepackage{enumitem}
\usepackage{perpage}
\usepackage{chngcntr}
%\usepackage{minted}
\usepackage[bookmarks=true,pdfborder={0 0 0 }]{hyperref}
\usepackage{indentfirst}
\hypersetup{
  colorlinks,
  citecolor=black,
  filecolor=black,
  linkcolor=black,
  urlcolor=black
}

\newtheorem*{conclusion}{Вывод}
\newtheorem{theorem}{Теорема}
\newtheorem{lemma}{Лемма}
\newtheorem*{corollary}{Следствие}

\theoremstyle{definition}
\newtheorem*{problem}{Задача}
\newtheorem{claim}{Утверждение}
\newtheorem{exercise}{Упражнение}
\newtheorem{definition}{Определение}
\newtheorem{example}{Пример}

\theoremstyle{remark}
\newtheorem*{remark}{Замечание}

\renewcommand{\le}{\leqslant}
\renewcommand{\ge}{\geqslant}
\newcommand{\eps}{\varepsilon}
\renewcommand{\phi}{\varphi}
\newcommand{\ndiv}{\centernot\mid}

\MakePerPage{footnote}
\renewcommand*{\thefootnote}{\fnsymbol{footnote}}

\newcommand{\resetcntrs}{\setcounter{theorem}{0}\setcounter{definition}{0}
\setcounter{claim}{0}\setcounter{exercise}{0}}

\DeclareMathOperator{\aut}{aut}
\DeclareMathOperator{\cov}{cov}
\DeclareMathOperator{\chos}{ch}
\DeclareMathOperator{\argmin}{argmin}
\DeclareMathOperator{\argmax}{argmax}
\DeclareMathOperator*\lowlim{\underline{lim}}
\DeclareMathOperator*\uplim{\overline{lim}}
\DeclareMathOperator{\re}{Re}
\DeclareMathOperator{\im}{Im}

\frenchspacing


\begin{document}

\section{Типизированное лямбда-исчисление}

Типы: пропозициональные формулы с~импликацией.

\begin{definition}
	Счетный алфавит типов

	$A, B \in Tp, (A \rightarrow B) \in Tp$

	$A_1 \rightarrow (A_2 \rightarrow \ldots \rightarrow (A_n \rightarrow B)) = A_1 \rightarrow \ldots
	\rightarrow A_n \rightarrow B$
\end{definition}

\begin{definition}
$PdTm$~--- псевдотермы.

$X \in Var \Rightarrow X \in PdTm$

$M, N \in PdTm \Rightarrow (MN) \in PdTm$

$x \in Var, A \in Tp, M \in PdTm \Rightarrow (\lambda x^A.M) \in PdTm$.
\end{definition}

Псевдотермы, потому что не все из них будут считаться корректными.

Контекст $\Gamma = \{x_1: A_1, \ldots, x_n: A_n\}$.

Вывод типов: $\vdash \subset Context \times Statement$. Правила

\begin{itemize}
	\item $\Gamma, x: A \vdash x: A$
	\item $\frac{\Gamma \vdash M: A \rightarrow B, \Gamma \vdash N: A}{\Gamma \vdash MN: B}$
	\item $\frac{\Gamma, x: A \vdash M: B}{\Gamma \vdash (\lambda x^A.M): A \rightarrow B}$
\end{itemize}

Каждое выведение может быть получено единственным образом (лемма об обращении).

\begin{lemma}[О~подстановке]~
	\begin{itemize}
		\item $\Gamma \vdash M: A \Rightarrow \Gamma[\alpha:= B] \vdash M:A[\alpha:=B]$
		\item $\frac{\Gamma, x:A \vdash M: B, \Gamma \vdash N: A}{\Gamma \vdash M[x:=N]:B}$
	\end{itemize}
\end{lemma}

\begin{lemma}[subject reduction]
	$\Gamma \vdash M: A, M \doublearrow_{\beta\eta} N \Rightarrow \Gamma \vdash N:A$
\end{lemma}

\begin{claim}
	$\Gamma \vdash M: A, \Gamma \vdash M: B \Rightarrow A = B$
\end{claim}
\begin{proof}
	Разберём один из случаев: $\Gamma \vdash \lambda x^C.N:A \Rightarrow A = C \rightarrow D, \Gamma,
	x: C \vdash N: D$. Если $B = C \rightarrow E$, то аналогично $\Gamma, x: C \vdash N: E$. По
	индукции $D = E$, значит $A = B$.
\end{proof}

\begin{lemma}[о~ромбе]
	Если $\Gamma \vdash M,N_1,N_2: A$, притом $M \doublearrow N_1, M \doublearrow N_2$, тогда $\exists
	L: \Gamma \vdash L: A, N_1 \doublearrow L, N_2 \doublearrow L$.
\end{lemma}

\begin{theorem}[Чёрча-Россера]
	Пусть $\Gamma \vdash M, N: A, M =_{\beta\eta} N$, тогда $\exists L: \Gamma \vdash L: A, M
	\doublearrow L, N \doublearrow L$.
\end{theorem}

Более того, верно, что у~любого терма нормальная форма существует, единственна и~любая
последовательность редукций конечна. В~частности, терму $\omega_2 = \lambda x.xx$ нельзя приписать
никакой тип.

\begin{example}
	$\lambda x^Ay^B.x: A \rightarrow (B \rightarrow A)$.

	Берём контекст $\Gamma = \{x: A, y: B\}$. Выводим:

	$\Gamma \vdash x: A$

	$x: A \vdash \lambda y^B.x: B \rightarrow A$

	$\vdash \lambda x^A y^B.x:A \rightarrow B \rightarrow A$
\end{example}

Теперь можно определить <<корректные>> псевдотермы:
$\Lambda_\rightarrow^\Gamma(A) = \{ M \in PdTm \mid \Gamma \vdash M: A\}$.

$\Lambda_\rightarrow(A) = \bigcup\limits_{\Gamma} \Lambda^\Gamma_\rightarrow(A)$

$\Lambda_\rightarrow = \bigcup\limits_{A \in Tb} \Lambda_\rightarrow(A)$

\begin{theorem}[сильная нормализуемость]
	Если $M \in \Lambda_\rightarrow$, то любая цепочка $\beta\eta$-редукций конечна и~может быть
	прододжена до нормальной формы.
\end{theorem}

Что можно вычислить в~типизированном лямбда-исчислении?

$Int = (\alpha_0 \rightarrow \alpha_0) \rightarrow (\alpha_0 \rightarrow \alpha_0)$~---
целочисленный тип, взятый из привычных нумералов Чёрча: $\underline{n} = \lambda f^{\alpha_0
\rightarrow \alpha_0} x^\alpha_0. f^n x$.

\begin{claim}
	$\forall n$ выполнено $\underline{n} \in \Lambda_\rightarrow(Int)$.
	$\vdash \lambda f^{\alpha_0 \rightarrow \alpha_0} x^\alpha_0.f^nx$.
\end{claim}

Аналогично перенесём понятие представимости функций $f: \mathbb{N}^k \rightarrow \mathbb{N}$.

Заметим, что $Sc = \lambda kfx.f(kfx)$ типизуем типом $Int \rightarrow Int$. Точно также типизуются
сложением, умножение. Но с~возведением в~степень проблемы: $E = \lambda kl.lk$ не типизуется.

Можно показать, что при таком кодировании чисел получится представить только эти операции,
проекторы, проверку на 0 и~все их композиции.

\begin{definition}~
	\begin{itemize}
		\item $PVar = \{p_0, \ldots, p_1\} \sim \mathbb{N}$
		\item $p_i \in PVar \Rightarrow p_i \in IFm$
		\item $A, B \in IFm \Rightarrow (A \rightarrow B) \in IFm$
	\end{itemize}

\end{definition}

Натуральный вывод: помеченное дерево с~правилами элиминации импликации (modus ponens) и~добавлением
импликации. Такой вывод двойственнен выводу типа.

Одинаковых меток на разных гипотезах быть не может.

Элиминация импликации: $\frac{A, A\rightarrow B}{B}$, где $A, A \rightarrow B$ имеют какой-то вывод.

Добавление импликации: $[A^x] \Rightarrow \frac{B}{A\rightarrow B} I[x]$.

Контекст соответветствует гипотезам, метки~--- переменным в~лямбда-термах, типы~--- функциям, терм
на самом деле кодирует доказательство формулы.

Интуиционистская имликация: $f$ <<доказательство>> импликации $A \rightarrow B$, если $f$ вычислима
и~преобразует любое доказательство $A$ в~доказательство $B$. В~этом смысле типизированные функции
интерпретируются как описания доказательств (изоморфизм Карри-Ховарда).

\end{document}
