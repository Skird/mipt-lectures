\documentclass{article}
\usepackage[utf8x]{inputenc}
\usepackage[english,russian]{babel}
\usepackage{amsmath,amscd}
\usepackage{amsthm}
\usepackage{amsfonts}
\usepackage{amssymb}
\usepackage{cmap}
\usepackage{centernot}
\usepackage{enumitem}
\usepackage{perpage}
\usepackage{chngcntr}
%\usepackage{minted}
\usepackage[bookmarks=true,pdfborder={0 0 0 }]{hyperref}
\usepackage{indentfirst}
\hypersetup{
  colorlinks,
  citecolor=black,
  filecolor=black,
  linkcolor=black,
  urlcolor=black
}

\newtheorem*{conclusion}{Вывод}
\newtheorem{theorem}{Теорема}
\newtheorem{lemma}{Лемма}
\newtheorem*{corollary}{Следствие}

\theoremstyle{definition}
\newtheorem*{problem}{Задача}
\newtheorem{claim}{Утверждение}
\newtheorem{exercise}{Упражнение}
\newtheorem{definition}{Определение}
\newtheorem{example}{Пример}

\theoremstyle{remark}
\newtheorem*{remark}{Замечание}

\renewcommand{\le}{\leqslant}
\renewcommand{\ge}{\geqslant}
\newcommand{\eps}{\varepsilon}
\renewcommand{\phi}{\varphi}
\newcommand{\ndiv}{\centernot\mid}

\MakePerPage{footnote}
\renewcommand*{\thefootnote}{\fnsymbol{footnote}}

\newcommand{\resetcntrs}{\setcounter{theorem}{0}\setcounter{definition}{0}
\setcounter{claim}{0}\setcounter{exercise}{0}}

\DeclareMathOperator{\aut}{aut}
\DeclareMathOperator{\cov}{cov}
\DeclareMathOperator{\chos}{ch}
\DeclareMathOperator{\argmin}{argmin}
\DeclareMathOperator{\argmax}{argmax}
\DeclareMathOperator*\lowlim{\underline{lim}}
\DeclareMathOperator*\uplim{\overline{lim}}
\DeclareMathOperator{\re}{Re}
\DeclareMathOperator{\im}{Im}

\frenchspacing


\begin{document}

\section{Параметры унициклических компонент}

\begin{corollary}
	Если $c \in (0, 1)$, то сложных компонент в~графе нет.
\end{corollary}

\begin{theorem}
	Пусть $np = c > 0, k \ge 3$. Обозначим за $U_k$~--- число унициклических компонент размера $k$
	в~$G(n, p)$. Тогда $U_k \overset{d}\rightarrow Pois(\lambda)$, $\lambda = \frac{1}{2k} (c
	e^{-c})^k \sum\limits_{j=0}^k \frac{k!}{j!}$.
\end{theorem}
\begin{proof}
	Подсчёт по методу моментов.
\end{proof}

\begin{theorem}
	Пусть $k_1, \ldots, k_s$~--- различные числа. Тогда $(U_{k_1}, \ldots, U_{k_s})
	\overset{d}\rightarrow (Z_1, \ldots, Z_s)$, где $Z_i$ независимые $Pois(\lambda_i)$, $\lambda_i =
	\frac{1}{2k_i} (ce^{-c})^{k_i} \sum\limits_{j=0}^{k_i} \frac{k_i^j}{j!}$.
\end{theorem}

\begin{theorem}
	Пусть $U$~--- общее число вершин в~унициклических компонентах $G(n, p)$, притом $c < 1$. Тогда
	$EU_n \underset{n\rightarrow \infty}\rightarrow \frac{1}{2} \sum_{k=3}^\infty (ce^{-c})^k
	\sum_{j=0}^{k-3} \frac{k^j}{j!}$, а~также
	$DU_n \underset{n\rightarrow \infty}\rightarrow \frac{1}{2} \sum_{k=3}^\infty (ce^{-c})^k
	\sum_{j=0}^{k-3} \frac{k^j}{j!}$.
\end{theorem}
\begin{proof}
	$U_n(k)$~--- число унициклических компонент размера $k$. $U_n = \sum\limits_{k=3}^n kU_n(k)$.
	Отсюда $EU_n = \sum\limits_{k=3}^n kC_n^k C(k, k) p^k (1-p)^{C_k^2 - k + (n - k)k}$.

	Каждое слагаемое сходится туда, куда нужно. Нужно показать, что сходимость равномерная. А~именно,
	проверим, что $\exists \gamma > 0: \forall k \le n\ kC_n^k C(k, k) p^k (1-p)^{C_k^2 - k + (n -
	k)k} \le \exp(-\gamma k)$.

	$C(k, k) = \frac{(k-1)!}{2} \sum\limits_{j=0}^{k-3} \frac{k^j}{j!} \le \frac{(k-1)!}{2} e^k$.

	$C_n^k = o\left(\frac{1}{k!} \left(\frac{n}{e}\right)^n \sqrt{n} \left(\frac{e}{n-k}\right)^{n-k}
	\frac{1}{\sqrt{n-k}}\right) = o\left(\frac{1}{k!} e^{-k} \left(\frac{n}{n-k}\right)^{n - k +
	\frac{1}{2}} n^k\right)$.

	$(1-p)^{C_k^2 - k + (n - k)k} = o\left((1-p)^{nk - \frac{k^2}{2}}\right)$.

	Итого, слагаемое $S_k$ равно
	$o\left( \left(\frac{n}{n-k}\right)^{n-k+\frac{1}{2}} c^k (1-p)^{nk - \frac{k^2}{2}} \right) =
	o\left( \sqrt\frac{n}{n-k} e^{f(\beta)} k\right)$, где $\beta = \frac{k}{n}$.

	Заметим, что $\sqrt\frac{n}{n-k}$ не мешает экспоненциальной скорости сходимости по $k$ и~выкинем
	это.

	$1 - p = 1 - \frac{c}{n}, \ln p = -p + O(p^2)$. Тогда $f(\beta) = \ln c - c + \frac{c\beta}{2} +
	\ln(1 - \beta)^{-1} \left( \frac{1}{\beta} - 1 \right) = -\frac{1 - \beta}{\beta} \ln(1 - \beta) +
	\ln c - c + \frac{c\beta}{2}$.

	Хотим показать, что $\forall \beta \in [0, 1] f(\beta) \le -\gamma$ для $\gamma > 0$. $f(0) = 1 +
	\ln c - c < 0$ для $c < 1$. Тогда $\exists \beta_0 > 0: \forall \beta \le \beta_0\ f(\beta) \le
	\frac{f(0)}{2} < 0$.

	Рассмотрим $g(\beta) = \beta f(\beta) = -(1 - \beta)\ln(1 - \beta) + \ln c \beta - \beta c +
	\frac{\beta^2 c}{2}$.

	$g'(\beta) = \ln(1 - \beta) + 1 + \ln c - c + \beta c$.

	$g''(\beta) = -\frac{1}{1 - \beta} + c$. Это равно 0 при $\beta = 1 - \frac{1}{c}$.

	Если $c < 1$, то $g''(\beta) < 0$ на $[0, 1]$. $g'(1 - \frac{1}{c}) = 0$. То есть $g$ будет
	убывать на $[0, 1]$.

	Даже если $c > 1$, $1 - \frac{1}{c}$~--- точка максимума $g'(\beta)$. Но $g'(1 - \frac{1}{c}) = 0
	\Rightarrow g'(\beta) \le 0$ на $[0, 1]$, то есть так или иначе $g(\beta)$ убывает на $[0, 1]$ для
	всех $\beta > \beta_0$.

	$f(\beta) = \beta^{-1}g(\beta) \le g(\beta) \le g(\beta_0) = \gamma' < 0$. Тогда взяв $\gamma =
	\min\{-\frac{f(0)}{2}, -\gamma'\}$, получаем $f(\beta) \le -\gamma$. Тем самым, равномерная
	сходимость доказана, значит $EU_n$ сходится к~сумме пределов.

	Заметим также, что при фиксированных $k_1, k_2$ $E U_n(k_1) U_n(k_2) \sim EU_n(k_1) EU_n(k_2)$ при
	$k_1 \ne k_2$. А~$EU_n(k)(U_n(k)-1) \sim EU_n(k)^2$.

	Тогда $EU_n^2 = E\left(\sum\limits_{k=3}^n kU_n(k)\right)^2 \sim \sum\limits_{k_1 \ne k_2} k_1 k_2
	EU_n(k_1) EU_n(k_2) + \sum\limits_{k=3}^n k^2 EU_n^2(k) \sim (EU_n)^2 + \sum\limits_{k=3}^n k^2
	EU_n(k) \Rightarrow DU_n \rightarrow \frac{1}{2} \sum\limits_{k=3}^\infty k(ce^{-c})^k
	\sum\limits_{j=0}^{k-3} \frac{k^j}{j!}$.
\end{proof}

\begin{corollary}
	Общее число вершин в~унициклических компонентах ограничено по вероятности.
\end{corollary}

\section{Теорема о~гигантской компоненте}

Ветвящиеся процессы Гальтона-Ватсона: $\{ \xi_k^{(n)} \}$~--- н. о. р., $\xi \in \mathbb{Z}_+$.

$x_0 = 1, x_n = \sum_{k=1}^{X_n-1} \xi_k^{(n)}$.

$\phi_\xi(z) = Ez^\xi = \sum\limits_{k=0}^\infty z^k P(\xi = k)$. $q = P(\exists n: X_n = 0)$.

\begin{claim}
	$q = \phi_\xi(q)$.
\end{claim}

\begin{theorem}
	$\mu = E\xi, P(\xi = 1) < 1$. Тогда
	\begin{enumerate}
		\item $\mu \le 1 \Rightarrow q = 1$ и~других решений нет
		\item $\mu > 1 \Rightarrow q = q_0 \in [0; 1]$ и~решений ровно два: $q_0$ и~$1$.
	\end{enumerate}
\end{theorem}

\begin{example}
	$\xi \sim Pois(c)$, $\phi_\xi(z) = \sum\limits_{k=0}^\infty z^k \frac{c^k}{k!} e^{-c} =
	\exp((z-1)c)$.

	$q = e^{(q-1)c}, \beta = 1 - q$~--- вероятность невырождения. $\beta + \exp(-\beta c) = 1$.
\end{example}

\begin{theorem}
	Пусть $np = c > 1$. Положим $\beta = \beta(c)$~--- решение уравнения $\beta + \exp(-\beta c) = 1$
	из $(0, 1)$. Тогда с~вероятностью, стремящейся к~1 $G(n, p)$ содержит гигантскую компоненту, чей
	размер при делении на $n$ стремится к~$\beta$ по вероятности.

	Все остальные компоненты при этом имеют размер не более $\frac{16c}{(c-1)^2} \ln c$.
\end{theorem}
\begin{proof}
	Обозначим $k_- = \frac{16c}{(c-1)^2} \ln n, k_+ = n^\frac{2}{3}$.

	Для всех вершин $v$ запустим процесс набора её компоненты связности.

	Для $\forall t=0, 1, \ldots$ введем тройку $(C_t, A_t, U_t)$, где $C_t$~--- рассмотренные вершины,
	$A_t$~--- активные вершины, $U_t$~--- неактивные вершины.

	$C_0 = \varnothing, A_0 = \{v\}, U_0 = V \{ v \}$.

	При тройке $(C_t, A_t, U_t)$ на шаге $t+1$:
	\begin{itemize}
		\item берем первую вершину $v_t$ из $A_t$.
		\item $C_{t+1} = C_t \cup \{v_t\}$
		\item Пусть $X_{t+1}$~--- множество соседей $v_t$ в~$U_t$
		\item $A_{t+1} = A_t \setminus \{v_t\} \cup X_{t+1}$.
		\item $U_{t+1} = U_t \setminus X_{t+1}$.
	\end{itemize}

	Процесс останавливается когда либо $A_t = \varnothing, U_t = \varnothing$, компонента при этом
	есть $C_t$.

	Покажем, что с~вероятностью, стремящейся к~1 выполнена следующая альтернатива:
	\begin{enumerate}
		\item процесс закончился ко времени $k_-$
		\item для $\forall t \in [k_-, k_+]\ |A_t| \ge \left(\frac{c-1}{2}\right)t$
	\end{enumerate}

	Пусть это не так. Тогда $\exists t \in [k_-; k_+]: |A_t| < \left(\frac{c-1}{2}\right)t$, притом
	$|A_i| > 0$ при $i \le t - 1$. Заметим, что $|A_t| = \sum\limits_{k=1}^t Y_k - t + 1$, где
	$Y_k = |X_k|$.

	$P(|A_t| < \frac{c-1}{2}t) = P(\sum\limits_{k=1}^t Y_k < \frac{c+1}{2}t - 1)$. На любом шаге у~нас
	есть не меньше, чем $n - \frac{c+1}{2}k_+$ неактивных вершин. Тогда $P(|A_t| < \frac{c-1}{2}t) \le
	P(\sum\limits_{k=1}^t Z_k < \frac{c+1}{2}t - 1)$, где $Z_1, \ldots, Z_t$ независимые $Bin(n -
	\frac{c + 1}{2}k_+, p)$.

	Это равно $P(\sum\limits_{k=1}^t (Z_k - EZ_k) < \frac{c-1}{t} - 1 + \frac{c+1}{2}k + pt) \le
	\exp\left(-\frac{(\frac{c-1}{2}t + 1 - \frac{c+1}{2}k + pt)^2}{2(ct - \frac{c+1}{2}k + pt)}\right)
	= \exp\left(-\frac{(c-1)^2}{8c} t(1 + o(1))\right)$.

	Так как $t \ge k = \frac{16c}{(c-1)^2} \ln n$, $P(|A_t| < \frac{c-1}{2}t) \le \exp(-2\ln n(1 +
	o(1))) = n^{-2 + o(1)}$.

	Суммируя по всем $v \in V$ ($n$ штук) и~$t \in [k-, k+]$ получаем следующее: $P(\text{альтернатива не
	выполнена}) \le n^\frac{5}{3} n^{-2+o(1)} \rightarrow 0$.

	Назовём компоненты размера $\ge k$ большими. Пусть $v, w$~--- две вершины из разных больших
	компонент. Для них обоих выполнена вторая часть алтернативы. Значит в~любой момент времени между
	их множествами активных вершин рёбер нет. Но в~момент времени $k_+$ в~этих множествах хотя бы
	$\frac{c-1}{2}k_+$ активных вершин.

	$P(v, w \text{ лежат в~разных компонентах} \mid (C_{k_+},
	A_{k_+}, U_{k_+}), (C'_{k_+}, A'_{k_+}, U'_{k_+})) \le (1 - p)^{|A_{k_+}||A'_{k_+}|} \le (1 -
	p)^{\left( \frac{c-1}{2}\right)^2 k_+^2} = \left(1 - \frac{c}{n}\right)^{\frac{(c-1)^2}{4}
	n^\frac{4}{3}} \rightarrow 0$.

	Будем называть вершины из большой компоненты большими, а~остальные~--- маленькими. Пусть $v$~---
	вершина $G(n, p)$. Тогда

	$P(v \text{ маленькая}) \le \rho(n - k_-, p)$, где $\rho(n, p)$~--- вероятность вырождения
	ветвящегося процесса с~законом $Bin(n, p)$. $\rho(n, p) \rightarrow 1 - \beta$.

	С~другой стороны $$P(v \text{ маленькая}) \ge P(\text{процесс выродился, набрав } Y \le k_-).$$

	Так как $k_- \rightarrow \infty$, то $P(v \text{ маленькая}) \ge P(Y \le \infty) = 1 -
	\beta$.

	Значит $P(v \text{ маленькая}) = (1 - \beta)(1 + o(1))$. Пусть $X_n$~--- общее число маленьких
	вершин в~$G(n, p)$, тогда $EX_n = n(1 - \beta)(1 + o(1))$.

	$EX_n(X_n-1) \le n(1 - \beta)(1 + o(1))k_- + (n-k_-)(1 - \beta)(1 + o(1)) \sim (EX_n)^2$.

	По неравенству Чебышева $\forall \eps > 0$
	$$ P\left(\left|\frac{X_n - (1 - \beta)n}{n} \right| \ge \eps \right) \le \frac{DX_n}{(n\eps)^2} =
	o\left(\frac{(EX_n)^2}{n^2}\right) = o(1)$$

	Тогда $\frac{X_n}{n} \overset{P}\rightarrow 1 - \beta$, что и~требовалось доказать.
\end{proof}

\end{document}
