\documentclass{article}
\usepackage[utf8x]{inputenc}
\usepackage[english,russian]{babel}
\usepackage{amsmath,amscd}
\usepackage{amsthm}
\usepackage{amsfonts}
\usepackage{amssymb}
\usepackage{cmap}
\usepackage{centernot}
\usepackage{enumitem}
\usepackage{perpage}
\usepackage{chngcntr}
%\usepackage{minted}
\usepackage[bookmarks=true,pdfborder={0 0 0 }]{hyperref}
\usepackage{indentfirst}
\hypersetup{
  colorlinks,
  citecolor=black,
  filecolor=black,
  linkcolor=black,
  urlcolor=black
}

\newtheorem*{conclusion}{Вывод}
\newtheorem{theorem}{Теорема}
\newtheorem{lemma}{Лемма}
\newtheorem*{corollary}{Следствие}

\theoremstyle{definition}
\newtheorem*{problem}{Задача}
\newtheorem{claim}{Утверждение}
\newtheorem{exercise}{Упражнение}
\newtheorem{definition}{Определение}
\newtheorem{example}{Пример}

\theoremstyle{remark}
\newtheorem*{remark}{Замечание}

\renewcommand{\le}{\leqslant}
\renewcommand{\ge}{\geqslant}
\newcommand{\eps}{\varepsilon}
\renewcommand{\phi}{\varphi}
\newcommand{\ndiv}{\centernot\mid}

\MakePerPage{footnote}
\renewcommand*{\thefootnote}{\fnsymbol{footnote}}

\newcommand{\resetcntrs}{\setcounter{theorem}{0}\setcounter{definition}{0}
\setcounter{claim}{0}\setcounter{exercise}{0}}

\DeclareMathOperator{\aut}{aut}
\DeclareMathOperator{\cov}{cov}
\DeclareMathOperator{\chos}{ch}
\DeclareMathOperator{\argmin}{argmin}
\DeclareMathOperator{\argmax}{argmax}
\DeclareMathOperator*\lowlim{\underline{lim}}
\DeclareMathOperator*\uplim{\overline{lim}}
\DeclareMathOperator{\re}{Re}
\DeclareMathOperator{\im}{Im}

\frenchspacing


\begin{document}

\section{Предельные теоремы для $X_G$}

\begin{proof}[Доказательство пуассоновской предельной теоремы]
	Воспользуемся методом моментов. Факториальные моменты $Y \sim Pois(\lambda)$
	равны $E(Y)_k = EY(Y-1)\ldots(Y-k+1) = \lambda^k$. Достаточно показать, что
	$E(X_G)_k \rightarrow \lambda^k$ при $n \rightarrow \infty$.

	Пусть $G_1, \ldots, G_N$~--- копии $G$ в~$K_n$, $I_{G_i} = I\{G_i \subset G(n,
	p)\}$. Тогда $X_G = \sum\limits_{i=1}^N I_{G_i}$ и
	$$(X_G)_k = \sum\limits_{i_1,\ldots,i_k} I_{G_{i_1}} \ldots I_{G_{i_k}}.$$

	$$ E(X_G)_k = \sum\limits_{i_1\ldots,i_k} E I_{G_{i_1}} \ldots I_{G_{i_k}} =
	E_k' + E_k'',$$
	где $E'_k$~--- сумма по тем наборам, где все $G_{i_k}$ попарно не имеют общих
	вершин, $E''_k$~--- остальные слагаемые.

	\begin{multline*}
		E'_k = (p^{e_G})^k \sum 1 = (p^{e_G})^k C_n^{v_G} \frac{v_G!}{\aut(G)}
		C_{n-v_G}^{v_G} \frac{v_G!}{\aut{G}} \ldots C_{n-(k-1)v_G}^{v_G}
		\frac{v_G!}{\aut{G}} \sim \\
		(p^{e_G} \frac{n^{v_G}}{\aut(G)})^k \rightarrow
		(\frac{c^{v_G}}{\aut(G)})^k = \lambda^k
	\end{multline*}

	Нужно показать, что $E_k'' = o(1)$. Для каждого $t$ рассмотрим $e(t) =
	\min\{|E(G_1 \cup \ldots \cup G_k)| \mid |V(G_1 \cup \ldots G_k)| = t\}$.

	\begin{claim}
		Пусть $k \ge 2, 2 \le t < kv_G$, тогда $e(t) > t m(G)$.
	\end{claim}
	\begin{proof}
		Пусть $F$~--- любой граф. Положим $f_F = m(G)v_F - e_F$. Тогда $f_G = 0$
		и~$f_H > 0$ для любого собственного подргафа $H \subset G$.

		Покажем, что если $F = G_1 \cup \ldots G_k$, то $f_F < 0$. Заметим, что
		$f_{F_1 \cup F_2} = f_{F_1} + f_{F_2} - f_{F_1 \cap F_2}$. Если $k = 2$, то
		$F = G_1 \cup G_2$ и~$|V(G_1 \cap G_2)| > 0$. Тогда $f_{G_1 \cup G_2} =
		f_{G_1} + f_{G_2} - f_{G_1 \cap G_2} = 0 + 0 - f_{G_1 \cap G_2} < 0$.

		Работаем по индукции: пусть $F' = G_1 \cup \ldots \cup G_{k-1}$ и~считаем,
		что $f_{F'} < 0$. Тогда $f_{G_1 \cup \ldots G_k} = f_{F'} + f_{G_k} - f_{F'
		\cap G_k} < 0$.

		Это и~означает, что $|E(G_1 \cup \ldots G_k)| > tm(G)$.
	\end{proof}

	Применим утверждение к~оценке $E''_k$. Если $A(k, t)$~--- это число способов
	разместить $k$ копий на $t$ вершинах.
	\begin{multline*}
		E''_k \le \sum\limits_{t=k}^{kv_G - 1} C_n^t A(k, t) p^{e(t)} =
		o\left(\sum\limits_{t=k}^{kv_G - 1} n^t p^{e(t)}\right) =\\
		o\left(\sum\limits_{t=k}^{kv_G - 1} (n^t p^{tm(G)}) p^{e(t) - tm(G)} \right)
		\rightarrow 0
	\end{multline*}
\end{proof}

\begin{theorem}[Многомерный случай]
	Пусть $G_1, \ldots, G_s$~--- различные строго сбалансированные графы одной
	и~той же плотности $m = m(G_i)$. Тогда если $np^m \rightarrow c > 0$, то
	$(X_{G_1}, \ldots, X_{G_s}) \overset{d}\rightarrow (Z_1, \ldots, Z_s)$, где
	$Z_j$~--- независимые случайные величины, $Z_j \sim Pois(\lambda_j), \lambda_j
	= \frac{c^{V_{G_j}}}{\aut(G)}$.
\end{theorem}

\begin{example}
	Всюду $m(G) = 1$, $np \rightarrow c > 0$
	\begin{itemize}
		\item $G = C_3 \sqcup C_3$~--- два неперсекающихся треугольника. Тогда $X_G
			\overset{d}\rightarrow \frac{1}{2}Z(Z-1)$, где $Z \sim
			Pois(\frac{c^3}{6})$. Тогда $P(X_G = 0) \rightarrow (1 +
			\frac{c^3}{6})\exp(-\frac{c^3}{6})$
		\item $G = C_3 \sqcup C_4$. $X_G \overset{d}\rightarrow Z_1 Z_2$, $Z_i$~---
			независимые, $Z_1 \sim Pois\left(\frac{c^3}{6}\right),
			Z_2 \sim Pois\left(\frac{c^4}{8}\right)$.
			Тогда $P(X_G = 0) \rightarrow 1 - (1 - e^{-\frac{c^3}{6}})(1 -
			e^{-\frac{c^4}{8}})$
		\item $G$~--- треугольник с~висячей вершиной. Тогда $X_G
			\overset{d}\rightarrow \sum_{i=1}^W Z_i$, где $Z_i$~--- независимые
			$Pois(3c)$, $W$~--- независима с~ними, $W\sim
			Pois\left(\frac{c^3}{6}\right)$.
			$P(X_G = 0) \rightarrow \exp\left(-(1 - e^{-3c})\frac{c^3}{6}\right)$
	\end{itemize}
\end{example}

Итого, ясно, что $np^{m(G)} \rightarrow 0 \Rightarrow X_G \overset{d}\rightarrow
0$ и~$np^{m(G)} \rightarrow c \Rightarrow X_G \overset{d}\rightarrow Pois$.
Утверждение состоит в~том, что в~случае, если $np^{m(G)} \rightarrow \infty
\Rightarrow X_G \overset{d}\rightarrow N$.

\begin{theorem}[ЦПТ для $X_G$]
	Пусть $G$~--- непустой фиксированный граф, $np^{m(G)} \rightarrow \infty$,
	$n^2(1-p) \rightarrow \infty$. Тогда
	$$ \frac{X_G - EX_G}{\sqrt{DX_G}} \overset{d}\rightarrow N(0, 1). $$
\end{theorem}
\begin{proof}
	Работаем по методу моментов. Вспомним, что если $Y \sim N(0, 1)$, то $EY^k =
	(k-1)!!$ при чётных $k$ и~0 при нечётных.

	Пусть $G_1, \ldots, G_N$~--- копии $G$ в~$K_n$, $I_{G_i}$~--- соответствующие
	индикаторы. Тогда $X_G = \sum I_{G_i}$ и~обозначим $T(G_{i_1},\ldots,G_{i_k})
	= E \prod\limits_j (I_{G_{i_j}} - EI_{G_{i_j}})$. Тогда $E(X_G - EX_G)^k =
	\sum\limits_{i_1, \ldots, i_k} T(G_{i_1}, \ldots, G_{i_k})$.

	Для набора копий $(G_1, \ldots G_K)$ введём граф $L(G_1, \ldots G_k)$
	с~вершинами $\{1, \ldots, k\}$ и~$(j, m)$~--- ребро $\Leftrightarrow G_{i_j}$
	и~$G_{i_m}$ имеют общее ребро. Тогда сумму перепишем как:
	$$ E(X_G - EX_G)^k = \sum\limits_{L \subset K_k} \sum\limits_{i_1, \ldots,
	i_k} T(G_{i_1}, \ldots, G_{i_k}). $$

	Разбираем три случая. Если $L$~--- совершенное паросочетание. Вспомним, что
	$Dx_G = \sum\limits_{H \subset G, e_H > 0} C_n^{v_H} C_{n-v_H}^{v_G-v_H}
	C_{n-v_G}^{v_G - v_H} A(G, H) \cdot (p^{2e_G - e_H} - p^{2e_G}) = d(n, p)$.
	Положим рёбра $L$ равными $\{(1, 2), \ldots, (k - 1, k)\}$, $k$~--- чётное.

	\begin{multline*}
	  \sum T = \sum\limits_{i_1,\ldots,i_k}
		\prod\limits_{j=1}^{\frac{k}{2}} \cov(I_{G_{2j-1}}, I_{G_{2j}}) \\
		\le \sum \prod\limits_{j=1}^{\frac{k}{2}} \sum\limits_{G_{2j-1} \not\cap
		G_{2j}} \cov(I_{G_{2j-1}}, I_{G_{2j}}) = (DX_G)^\frac{k}{2}.
	\end{multline*}
	С~другой стороны
	\begin{multline*}
		\sum T \ge \sum \prod\limits_{j=1}^\frac{k}{2}
		\cov(I_{G_{2j-1}},I_{G_{2j}})\\
		= \sum\limits_{i_1, i_2} \cov(G_{i_1}, G_{i_2}) \sum\limits_{G_{i_3}
		\cup G_{i_4} \not\cap G_{i_1} \cup G_{i_2}} \cov(G_{i_3}, G_{i_4}) \sum
		\ldots \\
		\ge d(n, p) d(n - 2v_G, p) \ldots \sim (d(n, p))^\frac{k}{2} =
		(DX_G)^\frac{k}{2}
	\end{multline*}

	Таким образом, первый случай даёт вклад $(k-1)!! (DX_G)^\frac{k}{2}$.

	Если $L$ имеет изолированную вершину, то $T = E(I_{G_{i_1}} - EI_{G_{i_1}})
	\ldots = 0$, то есть вклад таких слагаемых равен 0.

	В~противном случае в~$L$ строго меньше, чем $\frac{k}{2}$ компонент связности.
	Пронумеруем его так, чтобы компоненты имели вид $\{1, \ldots, r_1\}, \{ r_1 +
	1, \ldots, r_2 \}, \ldots$. Пусть также число компонент равно $c(L) <
	\frac{k}{2}$, а~также $\forall i \notin \{1, r_1 + 1, \ldots, r_{c-1}+1\}
	\exists j: (j, i) \in E(L)$.

	Пусть $G_{i_1}, \ldots, G_{i_k}$~--- набор копий, такой что $L(G_{i_1},
	\ldots, G_{i_k}) = L$. Обозначим $G^{(j)} = \bigcup\limits_{s=1}^j G_{i_s}$,
	$F_j = G^{(j-1)} \cap G_{i_j}$. $e_{F_j} = 0 \Leftrightarrow j \in \{1, r_1 +
	1, \ldots, r_{c-1} + 1\}$.

	Если $p \le \frac{1}{2}$, то
	$$|T| \le E \prod\limits_{j=1}^k (I_{G_{i_j}} + EI_{G_{i_j}}) \le 2^k
	EI_{G_{i_1}} \ldots I_{G_{i_k}} = 2^k p^{e_{G^{(k)}}}.$$

	Если же $p > \frac{1}{2}$, то оставим в~каждой компоненте по одному множителю.
	\begin{multline*}
	|T| \le E \prod\limits_{s=1}^c |I_{G_{i_{r_s}}} - EI_{G_{i_{r_s}}}| =
	(E|I_{G_1} - EI_{G_1}|)^c =\\
		(2(1-p)^{e_G}p^{e_G})^c \le (2e_G(1-p))^c
	\end{multline*}

	Итого, $|T(G_{i_1}, \ldots, G_{i_k})| = o(p^{e_{G^{(k)}}} (1 - p)^c)$. Далее
	$e_{G^{(k)}} = k e_G - \sum\limits_{j=1}^k e_{F_j}$.

	Тогда при заданных графах $F_1, \ldots, F_k$ число наборов $(G_{i_1}, \ldots,
	G_{i_k})$ с~условием $G^{(j-1)} \cap G_{i_j} \cong F_j$ в~$K_n$ есть
	$o(n^{kv_G - \sum\limits_{j=1}^k v_{F_j}})$.

	$$\sum\limits_{i_1, \ldots, i_k, L(\ldots) = L, F_1, \ldots F_k \text{~---
	фикс}} T = O\left(n^{kv_G - \sum\limits_{j=1}^k v_{F_j}} p^{ke_G -
	\sum\limits_{j=1}^k e_{F_j}} (1-p)^c\right).$$

	Если $e_{F_j} = 0$, то $n^{v_{F_j}} p^{e_{F_j}} = n^{v_{F_j}} \ge 1$. Таких
	$F_j$ ровно $c$. Остальные $F_j$ имеют рёбра, значит $n^{v_{F_j}} p^{e_{F_j}}
	\ge E X_{F_j} \ge \Phi_G$.

	Значит
	$$\sum T = O\left((n^{v_G} p^{e_G})^k \frac{(1-p)^c}{(\Phi_G)^{k-c}}\right) =
	O\left((DX_G)^\frac{k}{2} \frac{(1-p)^{c-\frac{k}{2}}}{(\Phi_G)^{\frac{k}{2}
	- c}}\right).$$

	Осталось показать, что $((1-p)\Phi_G)^{c-\frac{k}{2}} \rightarrow 0$, но $c -
	\frac{k}{2} < 0$, то есть $(1 - p)\Phi_G \rightarrow +\infty$.

	Если $p \le \frac{1}{2}$, то $\Phi_G(1-p) \asymp \Phi_G$, но по условию
	$np^{m(G)} \rightarrow \infty \Rightarrow \Phi_G \rightarrow \infty$.

	Если же $p > \frac{1}{2}$, то $\Phi_G \asymp \min\limits_{H \subset G, e_H >
	0} n^{v_H} p^{e_H} \asymp \min\limits_{H \subset G, e_H > 0} n^{v_H} = n^2$.

	По условию $n^2(1-p) \rightarrow \infty \Rightarrow \Phi_G \rightarrow
	\infty$.

	Итого, по методу моментов, теорема доказана.
\end{proof}

\end{document}
