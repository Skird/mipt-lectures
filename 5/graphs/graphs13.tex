\documentclass{article}
\usepackage[utf8x]{inputenc}
\usepackage[english,russian]{babel}
\usepackage{amsmath,amscd}
\usepackage{amsthm}
\usepackage{amsfonts}
\usepackage{amssymb}
\usepackage{cmap}
\usepackage{centernot}
\usepackage{enumitem}
\usepackage{perpage}
\usepackage{chngcntr}
%\usepackage{minted}
\usepackage[bookmarks=true,pdfborder={0 0 0 }]{hyperref}
\usepackage{indentfirst}
\hypersetup{
  colorlinks,
  citecolor=black,
  filecolor=black,
  linkcolor=black,
  urlcolor=black
}

\newtheorem*{conclusion}{Вывод}
\newtheorem{theorem}{Теорема}
\newtheorem{lemma}{Лемма}
\newtheorem*{corollary}{Следствие}

\theoremstyle{definition}
\newtheorem*{problem}{Задача}
\newtheorem{claim}{Утверждение}
\newtheorem{exercise}{Упражнение}
\newtheorem{definition}{Определение}
\newtheorem{example}{Пример}

\theoremstyle{remark}
\newtheorem*{remark}{Замечание}

\renewcommand{\le}{\leqslant}
\renewcommand{\ge}{\geqslant}
\newcommand{\eps}{\varepsilon}
\renewcommand{\phi}{\varphi}
\newcommand{\ndiv}{\centernot\mid}

\MakePerPage{footnote}
\renewcommand*{\thefootnote}{\fnsymbol{footnote}}

\newcommand{\resetcntrs}{\setcounter{theorem}{0}\setcounter{definition}{0}
\setcounter{claim}{0}\setcounter{exercise}{0}}

\DeclareMathOperator{\aut}{aut}
\DeclareMathOperator{\cov}{cov}
\DeclareMathOperator{\chos}{ch}
\DeclareMathOperator{\argmin}{argmin}
\DeclareMathOperator{\argmax}{argmax}
\DeclareMathOperator*\lowlim{\underline{lim}}
\DeclareMathOperator*\uplim{\overline{lim}}
\DeclareMathOperator{\re}{Re}
\DeclareMathOperator{\im}{Im}

\frenchspacing


\begin{document}

\section{k-связность}

\begin{definition}
	Граф $G$ называется $k$-связным, если при удалении $k-1$ вершины нельзя получить несвязный или
	пустой граф.

	Минимальное $k$, такое, что $G$ $k$-связен~--- вершинная связность, $k(G)$.
\end{definition}

Аналогично определяется рёберная $k$-связность.

Графовый случайный процесс: $\tilde{G} = (\tilde{G}(n,m), m = 0, \ldots, C_n^2)$.

$\tilde{G}(n, 0)$~--- пустой. $\tilde{G}(n,m+1)$ получен из $\tilde{G}{n}{m}$ добавлением случайного
недобавленного ребра.

Для возрастающего свойства $Q$ введём $\tau(Q) = \min \{m: \tilde{G}(n, m) \models Q\}$.

$\tau_k(\delta) = \tau(\delta(G) \ge k), \tau_k(\kappa) = \tau(k(G) \ge k), \tau_k(\lambda) =
\tau(\lambda(G) \ge k)$.

Цель: показать, что при фиксированном $k \in \mathbb{N}$ выполнено $P(\tau_k(\delta) =
\tau_k(\kappa)) \rightarrow 1$.

\begin{definition}
	Пусть $G$~--- граф, подмножество $S \subset V$ называется \emph{сепаратором}, если $V \setminus S$
	представимо в~виде $W_1 \sqcup W_2, |W_1| \le |W_2|$ и~между $W_1$ и~$W_2$ нет рёбер.

	Сепаратор называется тривиальным, если $|W_1| = 1$.

	$S$ называется $s$-сепаратором, если $|S| = s$.
\end{definition}

\begin{lemma}[О сепараторах]
	Пусть $k \in \mathbb{N}$ и $w(n) \rightarrow \infty, w(n) \le \ln \ln \ln n$. Пусть $p = \frac{\ln
	n + k \ln \ln n - w(n)}{n}$, тогда вероятность того, что $G(n, p)$ содержит нетривиальный
	сепаратор стремится к 0.
\end{lemma}
\begin{proof}
	$A_r = \{G(n, p) \text{ имеет нетривиальный $k$-сепаратор с $|W_1|=r$}\}$. Будем рассматривать
	$r_0 = \max(k+1,10), r_1 = \lfloor n^\frac{5}{9} \rfloor, r_2 = \lfloor \frac{n-k}{2} \rfloor$

	Хотим $P\left(\bigcup\limits_{r=2}^{r_2}A_r\right) \rightarrow 0$. Сумму разбиваем на три, каждую
	будем оценивать отдельно.

	\begin{enumerate}
		\item $r \le r_0$. Из условия на $p$ получаем $P(\delta(G(n,p))=k) \rightarrow 1$. Тогда в $W_1$
			найдутся две вершины степени $\le k + r$ на расстоянии не более 2 (либо рёбер внутри $W_1$
			нет, тогда оно соединено со всем $S$, либо есть). Оценим вероятность этого события:
			\begin{multline*}
				\sum\limits_{i,j=k}^{k+r} \sum\limits_{k=1}^2 C_n^2 C_{n-2}^{l-1} p^l C_n^{i-1} C_n^{j-1}
				p^{i+j-2} (1-p)^{2n-i-j+1} \le \\
				\sum\limits_{i,j=k}^{k+r}\sum\limits_{r=1}^2i n^{l+1+i+j-2} p^{i+j-2+l} (1-p)^{2n-i-j+1} \le
				\\
				\sum\limits_{i,j=k}^{k+r} \sum\limits_{l=1}^2 n (np)^{i+j+l-2} e^{-2np + o(1)}
				\le o\left((\ln n)^{2(k+r)} e^{-2np} n\right) = \\
				o\left( (\ln n)^{2k + 2r} \exp(-2\ln n - 2k \ln \ln n + 2w) n \right) = o\left((\ln n)^r
				e^{2w(n)} \frac{1}{n} \right) \rightarrow 0
			\end{multline*}
		\item $r_0 \le r \le r_1$. Заметим, что на множестве $S \cup W_1$ должно быть не меньше, чем
			$\frac{rk}{2}$ рёбер.
			\begin{multline*}
				C_{C_{r+k}^2}^{\frac{rk}{2}} p^\frac{rk}{2} \le \left(
				\frac{\exp(\frac{1}{4}(r+k)^2)}{\frac{rk}{2}}\right)^\frac{rk}{2} p^\frac{rk}{2} \le
				\left(\frac{4er^2p}{rk}\right)^\frac{rk}{2} = \left(\frac{4erp}{k}\right)^\frac{rk}{2} \le\\
				\left(n^{-\frac{4}{9} + o(1)}\right)^\frac{rk}{2} < n^{-\frac{3}{14}rk}
			\end{multline*}
			Отсюда 
			\begin{multline*}
				P\left( \bigcup\limits_{r=r_0+1}^{r_1} A_r \right) \le o(1) +
				\sum\limits_{r=r_0+1}^{r_1} C_n^r C_{n-r}^k n^{-\frac{3}{14}rk} (1-p)^{r(n-r-k)} \le \\
				o(1) + \sum\limits_{r=r_0+1}^{r_1} n^{r+k - \frac{3}{14}rk} \exp(-pr(n-r-k)) \le \\
				o(1) + \sum\limits_{r=r_0+1}^{r_1} n^{k - \frac{3}{14}rk + \frac{r^2}{n}}
			\end{multline*}
			% \exp(-p(n -r - k)r) = e^{-r ln n} - kr\ln \ln n + w(n)r + o(1) + r^2p} 
			Так как $r \ge 11$, то сумма стремится к 0.
		\item $r > r_1$. Между $W_1$ и $W_2$ нет рёбер (и они большие).
			\begin{multline*}
				P\left( \bigcup\limits_{r=r_1+1}^{r_2} A_r \right) \le \sum\limits_{r=r_1+1}^{r_2} C_n^r
				C_{n-r}^k (1 - p)^{r(n-r-k)} \le \sum\limits_{r=r_1+1}^{r_2} \left(\frac{en}{r}\right)^r n^k
				e^{-pr(n-r-k)} \le \\
				\sum\limits_{r=r_1+1}^{r_2} \left( \frac{en}{r} \right)^r n^k e^{-pr \frac{n-k}{2}} =
				\sum\limits_{r=r_1+1}^{r_2} n^k \left( \frac{en}{r} e^{-p \frac{n-k}{2}}\right)^r \le
				\sum\limits_{r=r_1+1}^{r_2} n^k e^{r(\frac{4}{9} - \frac{1}{2})} \ln n (1 + o(1)) \le \\
				\sum\limits_{r=r_1+1}^{r_2} n^k n^{-\frac{r}{20}} \rightarrow 0
			\end{multline*}
	\end{enumerate}
\end{proof}

\begin{theorem}
	В графовом случайном процессе для фиксированного $k$ $P(\tau_k(\delta) = \tau_k(\kappa))
	\rightarrow 1$.
\end{theorem}
\begin{proof}
	Пусть $k = 1$. Ясно, что $\tau_k(\kappa) \ge \tau_k(\delta)$. Положим $w(n) = \ln \ln \ln n$.

	$m_1 = \lfloor \frac{n}{2} (\ln n - w(n)) \rfloor$

	$m_2 = \lfloor \frac{n}{2} (\ln n + w(n)) \rfloor$

	В силу асимп. эквивалентности моделей
	
	$P(\tilde{G}(n,m_1) \text{ связен}) \rightarrow 0$

	$P(\tilde{G}(n,m_2) \text{ связен}) \rightarrow 1$

	Тем самым $P(m_1 < \tau_k(\delta) \le m_2) \rightarrow 1$. Граф $\tilde{G}(n,m_1)$ с большой
	вероятностью состоит из гигантской компоненты и изолированных вершин, причем число последних $\le
	\ln n$. Обозначим это событие через $\mathcal{B}$. Тогда $\tau_1(\kappa) > \tau_1(\delta)$, если
	в моменты времени $\{m_1 + 1, \ldots, m_2\}$ одно из рёбер было проведено между изолированными.

	$ P(\tau_1(\kappa) > \tau_1(\delta) \mid \mathcal{B}) \le \frac{(m_2 - m_1) (\ln n)^2}{C_n^2 -
	m_2} = o\left(\frac{nw(n) (\ln n)^2}{n^2}\right) \rightarrow 0$.

	Пусть $k \ge 2$. По лемме о сепараторах при $m_0 = \lfloor \frac{n}{2}(\ln n + (k-1)\ln\ln n - \ln
	\ln \ln n)\rfloor$

	$P(\tilde{G}(n,m_0) \text{ содержит нетривиальный $(k-1)$-сеператор}) \rightarrow 0$.

	Кроме того $P(\delta(\tilde{G}(n,m_0))=k-1) \rightarrow 1$. Тогда $\forall m \ge m_0$
	$\tilde{G}(n,m)$ тоже не содержит нетривиальных $(k-1)$-сепараторов. Снова $\tau_k(\kappa) \ge
	\tau_k(\delta)$ и если $\delta(\tilde{G}(n,m)) = k$, то $\tilde{G}(n, m)$ не будет вершинно
	$k$-связным только при наличии нетривиального $(k-1)$-сепаратора. Но их нет, значит
	$\tau_k(\delta) = \tau_k(\kappa)$.
\end{proof}

\begin{exercise}
	Доказать, что при фиксированном $k$ $P(\tau_k(\delta) = \tau_k(\lambda)) \rightarrow 1)$.
\end{exercise}

\begin{corollary}
	Если $p \le \frac{\ln n + k \ln \ln n}{n}$, $k$~--- фиксированное, то
	$P(\delta(G(n,p))=k(G(n,p))=\lambda(G(n,p))) \rightarrow 1$.
\end{corollary}
\begin{corollary}
	Пусть $p = \frac{\ln n + k \ln \ln n + x + o(1)}{n}$, $k \in \mathbb{Z}_{+}, x \in \mathbb{R}$, то

	$P(k(G(n,p))=\lambda(G(n,p))=k) \rightarrow 1 - \exp\left(-\frac{e^{-x}}{k!}\right)$,

	$P(k(G(n,p))=\lambda(G(n,p))=k+1) \rightarrow \exp\left(-\frac{e^{-x}}{k!}\right)$.
\end{corollary}

\section{Совершенные паросочетания}

Пороговая вероятность наличия совершенного паросочетания в $G(n, p)$?

\begin{theorem}[Холл]
	Пусть $G$~--- двудольный с равными долями $X, Y$. Тогда в~$G$ есть совершенное паросочетание тогда
	и только тогда, когда $\forall W \subset X |N(W)| \ge |W|$
\end{theorem}

В случайном графе сложно: он не двудольный, а также в~нём достаточно много <<плохих>> вершин.
Оказывается, все зависит от наличия конфигурации <<вишня>>: пара вершин степени 1 с общим соседом.

\end{document}
