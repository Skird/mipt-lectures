\documentclass{article}
\usepackage[utf8x]{inputenc}
\usepackage[english,russian]{babel}
\usepackage{amsmath,amscd}
\usepackage{amsthm}
\usepackage{amsfonts}
\usepackage{amssymb}
\usepackage{cmap}
\usepackage{centernot}
\usepackage{enumitem}
\usepackage{perpage}
\usepackage{chngcntr}
%\usepackage{minted}
\usepackage[bookmarks=true,pdfborder={0 0 0 }]{hyperref}
\usepackage{indentfirst}
\hypersetup{
  colorlinks,
  citecolor=black,
  filecolor=black,
  linkcolor=black,
  urlcolor=black
}

\newtheorem*{conclusion}{Вывод}
\newtheorem{theorem}{Теорема}
\newtheorem{lemma}{Лемма}
\newtheorem*{corollary}{Следствие}

\theoremstyle{definition}
\newtheorem*{problem}{Задача}
\newtheorem{claim}{Утверждение}
\newtheorem{exercise}{Упражнение}
\newtheorem{definition}{Определение}
\newtheorem{example}{Пример}

\theoremstyle{remark}
\newtheorem*{remark}{Замечание}

\renewcommand{\le}{\leqslant}
\renewcommand{\ge}{\geqslant}
\newcommand{\eps}{\varepsilon}
\renewcommand{\phi}{\varphi}
\newcommand{\ndiv}{\centernot\mid}

\MakePerPage{footnote}
\renewcommand*{\thefootnote}{\fnsymbol{footnote}}

\newcommand{\resetcntrs}{\setcounter{theorem}{0}\setcounter{definition}{0}
\setcounter{claim}{0}\setcounter{exercise}{0}}

\DeclareMathOperator{\aut}{aut}
\DeclareMathOperator{\cov}{cov}
\DeclareMathOperator{\chos}{ch}
\DeclareMathOperator{\argmin}{argmin}
\DeclareMathOperator{\argmax}{argmax}
\DeclareMathOperator*\lowlim{\underline{lim}}
\DeclareMathOperator*\uplim{\overline{lim}}
\DeclareMathOperator{\re}{Re}
\DeclareMathOperator{\im}{Im}

\frenchspacing


\begin{document}

\begin{theorem}[ЦПТ для размера гигантской компоненты]
	Пусть $c > 1, p = \frac{c}{n}$, $N(n, p)$~--- размер гигантской компоненты $G(n, p)$. Тогда
	$$ \sqrt{n} \left( \frac{N(n, p)}{n} - \beta \right) \overset{d}\rightarrow N(0, \sigma^2),$$
	где $\sigma^2 = \frac{\beta(1 - \beta)}{(1 - c(1 - \beta))^2}$
\end{theorem}

\section{Случай $np \sim 1$}

При $np < 1$ и~при $np > 1$ картина ясна. Высяним промежуточную картину, наблюдаемую при $np \sim
1$.

Введем параметризацию $G(\lambda) = G(n, p)$, где $p = \frac{1}{n} + \frac{\lambda}{n^\frac{4}{3}}$.

\begin{definition}
	Компонента связности называется \emph{$l$-компонентой}, если число рёбер равно числу вершин плюс
	$l$.
\end{definition}

Введем такие обозначения:
\begin{itemize}
	\item $X(n, l)$~--- число $l$-компонент
	\item $Y(n, l)$~--- общее число вершин во всех $l$-компонентах
	\item $X(n, k, l)$~--- число $l$-компонент на $k$ вершинах
	\item $C(k, k + l)$~--- число связных графов на $k$ вершинах
\end{itemize}

\begin{claim}~
	\begin{enumerate}
		\item $X(n, l) = \sum\limits_{k=1}^n X(n, k, l)$
		\item $Y(n, l) = \sum\limits_{k=1}^n kX(n, k, l)$
		\item $EX(n, k, l) = C_n^k C(k, k + l) p^{k+l} (1-p)^{C_k^2 - k - l + k(n - k)}$
	\end{enumerate}
\end{claim}

\begin{claim}~
	\begin{enumerate}
		\item Если $a = o(b^\frac{3}{4})$, то $\frac{b(b-1)\ldots(b-a+1)}{b^2} = (1 + O(\frac{a}{b}) +
			O(\frac{a^4}{b^3})) \exp\left(-\frac{a^2}{2b} - \frac{a^3}{6b^2}\right)$.
		\item $\exists c > 0: \forall a \le b\ \frac{(b)_a}{b^a} = O(\exp\left(-\frac{a^2}{2b} -
			\frac{a^3}{6b^2} - c\frac{a^4}{b^3}\right))$.
	\end{enumerate}
\end{claim}

\begin{lemma}
	В~модели $G(\lambda)$ для фиксированного $l \ge -1$
	\begin{enumerate}
		\item если $k = o(n^\frac{3}{4})$, то $EX(n, k, l) = (1 + l\lambda n^{-\frac{1}{3}} +
			O(n^{-\frac{2}{3}}) + O(\frac{k}{n}) + O(\frac{k^4}{n^3})) n^{-l} c(k, k + l)
			\frac{e^{-k}}{k!} \exp(-F(x_k))$
		\item для всех $k$ $EX(n, k, l) = O(n^{-l} C(k, k + l) \frac{e^{-k}}{k!} \exp(-F(x_k))$,
	\end{enumerate}
	где $x_k = \frac{k}{n^\frac{2}{3}}, F(x) = \frac{1}{6}(x^3 - 3x^2\lambda + 3x\lambda^2) =
	\frac{1}{6}((x - \lambda)^3 + \lambda^3)$.
\end{lemma}
\begin{proof}
	$C_n^k = \frac{n^k}{k!} \frac{(n)_k}{n^k} = \frac{n^k}{k!}(1 + O(\frac{k}{n})
	+ O(\frac{k^4}{n^3})) \exp(-\frac{k^2}{2n} - \frac{k^3}{6n^2})$.

	$p^{k+l} = n^{-(k+l)} (1 + \lambda n^{-\frac{1}{3}})^{k+l} = n^{-(k+l)} (1 + l \lambda
	n^{-\frac{1}{3}} + O(n^{-\frac{2}{3}}))(1 + \lambda n^{-\frac{1}{3}})^k = n^{-k - l}(1 +
	l\lambda n^{-\frac{1}{3}} + o(n^{-\frac{2}{3}})) \exp(\lambda k n^{-\frac{1}{3}} -
	\frac{\lambda^2 k}{2 n^\frac{2}{3}} + O(\frac{n}{k}))$.

	Осталось рассмотреть $(1 - p)$.

	$(1 - p)^{C_k^2 - k - l + k(n - k)} = (1 + O(\frac{k}{n})) (1 - p)^{\frac{k^2}{2} + kn - k^2} = (1
	+ O(\frac{k}{n})) \exp(\frac{k^2}{2}p - pkn + O(\frac{k}{n})) = (1 + O(\frac{k}{n}))
	\exp(\frac{k^2}{2n} - k + \frac{\lambda k^2}{2 n^\frac{4}{3}} - \lambda k n^{-\frac{1}{3}})$.

	Собирая все вместе и~упрощая, получаем требуемое. Аналогично доказывается и~второй пункт.
\end{proof}

\begin{lemma}~
	\begin{enumerate}
		\item $EY(n, -1) = n - n^\frac{2}{3}(f_{-1}(\lambda) + O(n^{-\frac{1}{3}}))$
		\item $EY(n, 0) = f_0(\lambda) n^\frac{2}{3} + O(n^\frac{1}{3})$,
	\end{enumerate}
	где $f_{-1}(\lambda) = \frac{1}{\sqrt{2\pi}} \int\limits_0^\infty x^{-\frac{3}{2}} (1 -
	e^{-F(x)}) dx + \lambda$, $f_0(\lambda) = \frac{1}{4} \int\limits_0^\infty e^{-F(x)} dx$.
\end{lemma}
\begin{proof}
	\begin{enumerate}
		\item Разобъем сумму для $EY(n, -1)$ на две части: для $k \le n^\alpha$ и~$K > n^\alpha$ для
			какого-то фиксированного $\alpha \in (\frac{2}{3}; \frac{3}{4})$.

			$\sum\limits_{k \le n^\alpha} kEX(n, k, l) = \sum\limits_{k \le n^\alpha}
			\frac{k^{k-1}e^{-k}}{k!} n e^{-F(X_k)} (1 - \lambda n^{-\frac{1}{3}} + O(n^{-\frac{2}{3}})) +
			R_n$, где $R_n = O\left(\sum\limits_{k \le n^\alpha} \frac{k^{k-1} e^{-k}}{k!} k e^{-F(x_k)} +
			\sum\limits_{k \le n^\alpha} \frac{k^{k-1} e^{-k}}{k!} \frac{k^4}{n^2} e^{-F(x_k)}\right)$.

			Далее $\sum\limits_{k \le n^\alpha} \frac{k^k e^{-k}}{k!} e^{-F(x_k)} = O\left(\sum\limits_{k
			\le n^\alpha} \frac{1}{\sqrt{k}} e^{-F(x_k)}\right) = O\left( \sum\limits_{k \le n^\alpha}
			\frac{1}{\sqrt{x_k}} n^{-\frac{1}{3}} e^{-F(x_k)}\right) = O\left( n^\frac{1}{3}
			\sum\limits_{k \le n^\alpha} \frac{1}{\sqrt{x_k}} e^{-F(x_k)} \delta x_k \right)$, где $x_k =
			\frac{k}{n^\frac{2}{3}}, \delta x_k = \frac{1}{n^\frac{2}{3}}$.

			Это равно $O\left(n^\frac{1}{3} \int\limits_0^\infty \frac{1}{\sqrt{x}} e^{-F(x)} dx\right) =
			O\left(n^\frac{1}{3}\right)$

			Оцениваем вторую часть суммы ($F(x) \ge \frac{x^3}{7}$ для больших $x$):
			$\sum\limits_{k > n^\alpha} kE(x, n, -1) = O\left(\sum\limits_{k > n^\alpha} \frac{k^{k-1}
			e^{-k}}{k!} n e^{-F(x_k)}\right) = O\left(n \sum\limits_{k > n^\alpha} \frac{k^{k-1}
			e^{-k}}{k!} \exp(-\frac{x_k^3}{7})\right) = O\left(n \exp(-\frac{n^{\alpha - \frac{2}{3}}
			\cdot 3} {7} \sum\limits_{k > n^\alpha} \frac{k^{k-1} e^{-k}}{k!} )\right) =
			O\left(n e^{-\frac{1}{7} n^{3\alpha - 2} \sum\limits_k \frac{k^{k-1} e^{-k}}{k!} }\right) =
			O\left(n^\frac{1}{3}\right)$.

			Итого $EY(n, -1) = \sum\limits_{k \le n^\alpha} \frac{k^{k-1} e^{-k}}{k!} n
			e^{-F(x_k)} (1  - \lambda n^{-\frac{1}{3}} + O(n^{-\frac{2}{3}})) + O(n^\frac{1}{3})$.

			$\sum\limits_{k=1}^\infty \frac{k^{k-1} e^{-k}}{k!} e^{-F(x_k)} = 1 - \sum\limits_{k=1}^\infty
			\frac{k^{k-1} e^{-k}}{k!} (1 - e^{-F(x_k)})$, так как $\sum\limits_{k=1}^\infty \frac{k^{k-1}
			e^{-k}}{k!} = 1$ (это можно увидеть, сопоставив с~формулой для числа вершин, занимаемых
			древесными компонентами).

			По формуле Стирлинга: $\frac{k^{k-1} e^{-k}}{k!} (1 - e^{-F(x_k)}) = \frac{1}{\sqrt{2\pi x_k^3
			n^2}}(1 - e^{-F(x_k)})(1 + O(\frac{1}{k}))$.

			Тогда $EY(n, -1) = (1 - \lambda n^{-\frac{1}{3}}) (-1) \sum\limits_{k \le n^\alpha}
			\frac{1}{\sqrt{2\pi x_k^3}} \delta x_k (1 - e^{-F(x_k)}) \cdot n^\frac{2}{3} + n - \lambda
			n^\frac{2}{3} + O(n^\frac{1}{3})$.

			С~ростом $n$ это асимптотически эквивалентно $n - n^\frac{2}{3}(\lambda + \int\limits_0^\infty
			\frac{1}{\sqrt{2\pi}} x^{-\frac{3}{2}} (1 - e^{-F(x)}) dx) + O(n^\frac{1}{3})$.
		\item Доказывается аналогично
	\end{enumerate}
\end{proof}

\begin{corollary}
	$EY(n, \ge 1) = n^\frac{2}{3}(f_{-1}(\lambda) - f_0(\lambda)) + O(n^\frac{1}{3})$
\end{corollary}
\begin{corollary}
	В~модели $G(\lambda)$ размер наибольшей недревесной компоненты есть $O_P(n^\frac{2}{3})$.
\end{corollary}

\begin{lemma}
	В~модели $G(\lambda)$ размер наибольшей древесной компоненты есть $O_P(n^\frac{2}{3})$.
\end{lemma}
\begin{proof}
	Пусть $w(n) \rightarrow \infty$. $P(\exists T: |T| \ge n^\frac{2}{3} w(n)) \le \sum\limits_{k >
	n^\frac{2}{3} w(n)} EX(n, k, -1) = O\left(\sum\limits_{k > n^\frac{2}{3} w(n)} n \frac{k^{k-2}
	e^{-k}}{k!} e^{-F(x_k)}\right) = O\left(e^{-\frac{w^3}{7}} n \sum\limits_{k > w(n) n^\frac{2}{3}}
	k^{-\frac{-5}{2}}\right) = O\left(e^{-\frac{w^3}{7} w(n)^{-\frac{3}{2}}}\right) \rightarrow 0$.
\end{proof}

\begin{corollary}
	В~модели $G(\lambda)$ размер наибольшей компоненты есть $O_P(n^\frac{2}{3})$.
\end{corollary}

\section{Поведение сложных компонент}

\begin{theorem}[Багаев]
	$C(k, k + 1) \sim \frac{5}{24} k^{k+1}$.
\end{theorem}

\begin{theorem}[Райт, 1980]
	Для $l \ge 2$ и~$l = o(k^\frac{1}{3})$ выпонено
	$$C(k, k + l) = \gamma_l k^{k + \frac{3l - 1}{2}} \left(1 +
	O\left(\sqrt{\frac{l^3}{k}}\right)\right),$$
	где
	$\gamma_l = \frac{\sqrt{\pi} 3^l (l - 1) \delta_l}{2^\frac{5l-1}{2} \Gamma(\frac{l}{2})}$,
	$\delta_1 = \delta_2 = \frac{5}{36}, \delta_{l+1} = \delta_l + \sum\limits_{h=1}^{l-1}
	\frac{\delta_h \delta_{l-h}}{(l+1)C_l^h}$.
\end{theorem}

\begin{theorem}[Боллобаш]~
	\begin{enumerate}
		\item Если $1 \le l \le k$, то $C(k, k + l) \le \left(\frac{c_1}{l}\right)^{\frac{l}{2}}
			k^{k + \frac{3l - 1}{2}}$
		\item Если $k \le l \le C_k^2 - k$, то $C(k, k + l) \le c_2 e^{-\frac{l}{2}} k^{k + \frac{3l -
			1}{2}}$
	\end{enumerate}
\end{theorem}

\begin{lemma}
	Пусть $l \ge 1$~--- фиксировано, $w(n) \rightarrow \infty$. Тогда с~вероятностью, стремящейся к~1,
	$G(\lambda)$ не содержит $l$-компонент размер $\le \frac{n^\frac{2}{3}}{w(n)}$.
\end{lemma}
\begin{proof}
	Положим $k_1 = \frac{n^\frac{2}{3}}{w(n)}$, тогда $P(\exists H: |H| \le
	\frac{n^\frac{2}{3}}{w(n)}) \le \sum\limits_{k \le k_1} EX(n, k, l) = O\left( e^{-\frac{l}{2}}
	k^{k + \frac{3l - 1}{2}} \frac{e^{-k}}{k!} n^{-l} e^{-F(x_k)}\right) = O\left(\sum \limits_{k \le
	k_1} k^{\frac{3l}{2} - 1} n^{-l} e^{-F(x_k)}\right) = O\left( \sum\limits_{k \le k_1}
	x_k^{\frac{3l}{2} - 1} n^{l - \frac{2}{3}} n^{-l} e^{-F(x_k)}\right) = O\left( \sum\limits_{k \le
	k_1} x_k^{\frac{3l}{2} - 1} \delta x_k e^{-F(x_k)}\right) = O\left( \int\limits_0^\frac{1}{w(n)}
	x^{\frac{3l}{2}-1} e^{-F(x)} dx\right) \rightarrow 0$
\end{proof}

\end{document}
