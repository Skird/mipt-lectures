\documentclass{article}

\usepackage[utf8x]{inputenc}
\usepackage[english,russian]{babel}
\usepackage{amsmath,amscd}
\usepackage{amsthm}
\usepackage{amsfonts}
\usepackage{amssymb}
\usepackage{cmap}
\usepackage{centernot}
\usepackage{enumitem}
\usepackage{perpage}
\usepackage{chngcntr}
\usepackage{minted}
\usepackage[bookmarks=true,pdfborder={0 0 0 }]{hyperref}
\usepackage{indentfirst}
\hypersetup{
  colorlinks,
  citecolor=black,
  filecolor=black,
  linkcolor=black,
  urlcolor=black
}

\newtheorem*{conclusion}{Вывод}
\newtheorem{theorem}{Теорема}
\newtheorem{lemma}{Лемма}
\newtheorem*{corollary}{Следствие}

\theoremstyle{definition}
\newtheorem*{problem}{Задача}
\newtheorem{claim}{Утверждение}
\newtheorem{exercise}{Упражнение}
\newtheorem{definition}{Определение}
\newtheorem{example}{Пример}

\theoremstyle{remark}
\newtheorem*{remark}{Замечание}

\renewcommand{\le}{\leqslant}
\renewcommand{\ge}{\geqslant}
\newcommand{\eps}{\varepsilon}
\renewcommand{\phi}{\varphi}
\newcommand{\ndiv}{\centernot\mid}

\MakePerPage{footnote}
\renewcommand*{\thefootnote}{\fnsymbol{footnote}}

\newcommand{\resetcntrs}{\setcounter{theorem}{0}\setcounter{definition}{0}
\setcounter{claim}{0}\setcounter{exercise}{0}}

\DeclareMathOperator{\aut}{aut}
\DeclareMathOperator{\cov}{cov}
\DeclareMathOperator{\argmin}{argmin}
\DeclareMathOperator{\argmax}{argmax}
\DeclareMathOperator*\lowlim{\underline{lim}}
\DeclareMathOperator*\uplim{\overline{lim}}
\DeclareMathOperator{\re}{Re}
\DeclareMathOperator{\im}{Im}

\frenchspacing

\begin{document}

\title{Задание 4 по случайным графам}
\author{Дмитрий Иващенко}
\date{\today}
\maketitle

\section*{Задача 1}

Запишем первопорядковую формулу для данного свойтсва. Она будет состоять из
\begin{itemize}
	\item Кванторы по $n$ вершинам $\exists v_1, \ldots, \exists v_n$, где $n = |V(G)|$
	\item Условия различности всех вершин: $\bigwedge\limits_{i \ne j} v_i \ne v_j$
	\item Условие того, что проведены нужные ребра: $\bigwedge\limits_{(u,v) \in E(G)} u \sim v$
	\item Если дополнительно требуется, чтобы подграф был индуцированным, то добавим условие на то,
		чтобы лишних рёбер не было: $\bigwedge\limits_{(u, v) \notin E(G)} u \not\sim v$
\end{itemize}

Ясно, что если формула выполнена при каких-то значениях $v_1, \ldots, v_n$, то нужный подграф есть,
иначе его нет.

\section*{Задача 2}

Пускай для какого-то свойства $P$ закон нуля или единицы в~$G(n, p)$ не выполнен, при этом $p = 1 -
n^{-\alpha}$ при иррациональном $\alpha$. Рассмотрим тогда	свойтво $P'$, полученное из $P$ заменой всех
подформул вида $x \sim y$ на $\neg (x \sim y)$. Тогда в~дополнении графа $G(n, p)$ это свойство
выполнено тогда и~только тогда, когда в~$G(n, p)$ выполнено $P$. Значит вероятности их выполнения
совпадают. Однако для дополнения $G(n, p)$ выполнен закон нуля или единицы по теореме Спенсера
и~Шелаха. Противоречие.

\section*{Задача 3}

Рассмотрим свойство полного расширения $\phi_k$, состоящее в~том, что
$\forall S_1, S_2 \subset V, |S_1| +
|S_2| \le k - 1 \rightarrow \exists v$ такая, что она смежна со всеми вершинами в~$S_1$ и~несмежна
ни с~одной из $S_2$. Заметим, что так как суммарный размер $S_1$ и~$S_2$ фиксирован, то это свойство
записывается формулой как раз с~$k$ кванторами ($k - 1$ общности и~$1$ существования). Теперь
покажем, что $G(n, p)$ асимптотически почти наверное обладает $\phi_k$.

\begin{multline*}
	P(G(n, p) \not\models \phi_k) \le \sum\limits_{l_1 + l_2 \le k - 1} n^{l_1 + l_2} (1 - p^{l_1} (1 -
	p)^{l_2})^{n-k} \le \\
	\le \sum\limits_{l_1 + l_2 \le k - 1} n^{k-1} (1 - p^{k-1} (1 - p)^{k-1})^{n-k} \le \\
	\le k^2 n^{k-1} \exp\left( (k - n) p^{k-1} (1 - p)^{k-1} (1 + o(1))\right) \le \\
	\le k^2 n^{k-1} \exp\left( (k - n p^{k-1} (1-p)^{k-1} ) (1 + o(1)) \right)
\end{multline*}

Оценим логарифм выражения (отбрасывая константы):
\begin{multline*}
	(k - 1) \ln n + (k - n p^{k-1} (1 - p)^{k-1})(1 + o(1)) =\\
	= (k - 1)\ln n + k - n^{1 - \alpha (k-1)} \exp(-(k-1)n^{-\alpha} + o(1)) + o(1)
\end{multline*}

Видим, что так как $1 - \alpha (k - 1) > 0$ и~$\exp(-(k-1)n^{-\alpha}) \rightarrow 0$, то главным
слагаемым является последнее и~оно есть $\Theta(-n^\eps)$ для какого-то положительного $\eps$.
Значит все вместе стремится к~$-\infty$, то есть вероятность стремится к~0. Таким образом, свойство
асимтотически почти наверное выполнено. Но на графах с~таким свойством в~$k$ раундах выигрывает
консерватор, то есть на них выполнены одни и~те же формулы. Значит, вероятность $\lim\limits_{n
\rightarrow \infty}P(G(n, p) \models \phi) \in \{0, 1\}$ для всех свойств $\phi$ кванторной
глубины не более $k$.

\section*{Задача 5}

Оценим вероятность того, что какие-то две вершины не обладают ни одним общим соседом. Она не больше,
чем

$$ \sum\limits_{i \ne j} (1 - p^2)^{n-2} \le n^2 (1 - p^2)^{n-2} $$

Логарифм этого выражения есть $2 \ln n + (n - 2) \ln (1 - p^2) = 2\ln n - (n - 2) p^2 (1 + o(1))$.
То есть при $p = \omega\left(\sqrt\frac{\ln n}{n}\right)$ выражение стремится к~$-\infty$,
а~вероятность к~0.

Пусть теперь $p = o\left(\sqrt\frac{\ln n}{n}\right)$. Разобьем вершины произвольным образом на
пары. Тогда вероятность, что одна пара <<хорошая>> равна $1 - (1 - p^2)^{n-2}$, а~вероятность
пересечения этих событий равна $(1 - (1 - p^2)^{n-2})^{\frac{n}{2}}$. Оценим логарифм
этого выражения:

\begin{multline*}
	\frac{n}{2} \ln(1 - (1 - p^2)^{n-2}) \le -\frac{n}{2} (1 - p^2)^{n-2} \le -\frac{1}{2} e^{\ln n -
	(n - 2) o\left( \frac{ln n}{n}\right)} = -\frac{1}{2} e^{\ln n (1 - o(1))} \rightarrow -\infty
\end{multline*}

Значит вероятность $p = \sqrt\frac{\ln n}{n}$~--- пороговая.

\section*{Задача 6а}

Рассмотрим свойство <<содержать полный двудольный подграф $K_{k, k}$>>. Оно является свойством
первого порядка, записывается формулой с~$2k$ кванторами. Плотность такого графа есть $d(K) =
\frac{k^2}{2k}=\frac{k}{2}$, он является строго сбаласированным (так как при $m \le n < k,
n + m < 2k$ выполено $\frac{nm}{n+m} \le \frac{n}{2} < \frac{k}{2}$). Пороговая вероятность
наличия такого графа есть $n^{-\frac{2}{k}}$, а~по Пуассоновской предельной теореме при ровно такой
вероятности ребра вероятность наличия такого графа стремится к~нетривиальному пределу, то есть
закон 0 или 1 не выполнен.

\section*{Задача 7}

Приведем конструкцию сторго сбалансированного графа с~$n+1$ вершиной и~$m$ рёбрами, $n < m < 2n$.
Пусть $m = n + k, 1 < k < n$. Тогда возьмём цикл длины $n$ и~вершину $n + 1$ соединим с~множеством
вершин $R = \{t \mid \lfloor \frac{(t-1)k}{n} \rfloor < \lfloor \frac{tk}{n} \rfloor \}$, то есть
с~такими номерами, на которых целая часть дроби $\frac{tk}{n}$ увеличивается. Легко проверить по
формуле, что таких перескоков будет ровно $k$. Основное свойство такого почти равномерного разбиения
состоит в~том, что среди $q$ последовательных чисел по модулю $n$ меньше, чем $\frac{qr}{n} + 1$
попадают в~это множество.

Теперь рассмотрим <<дефицит>> подграфа $H \subset G$:
$$f(H) = \frac{|E(G)|}{|V(G)|-1} (|V(H)| - 1) - |E(H)|.$$
Наша цель состоит в~том, чтобы показать, что у~всех собственных подграфов неотрицательный
дефицит. Если это так, то для всех подграфов $H$ выполнено $\frac{|E(G)|}{|V(G)|-1} \ge
\frac{|E(H)|}{|V(H)|-1}$. Это значит, что $E(G)V(H) - E(H)V(G) \ge E(G) - E(H) > 0$, то есть
$\frac{E(G)}{V(G)} > \frac{E(H)}{V(H)}$.

Заметим, что по формуле включения-исключения $f(H_1 \cup H_2) = f(H_1) + f(H_2) - f(H_1 \cap H_2)$.
Поэтому если в~подграфе $H$ отрицательный дефицит, а~также есть какая-то точка сочленения, то есть
разбиение на два связных графа $H_1, H_2$, у~которых ровно одна точка пересечения, то в~формуле
последнее слагаемое нулевое, то есть у~какого-то из двух меньших графов отрицательный дефицит.
Поэтому достаточно проверить для графов, которые такого разбиения не допускают.

В~нашем случае это может быть цикл на вершинах $1, \ldots, n$, чей дефицит равен $\frac{n + k}{n}
(n - 1) - n = \frac{nk - n - k}{n^2} \ge 0$, так как $k \ge 2$. В~противном случае это множество
вершин, содержащее точку $n + 1$ и~несколько (скажем $q$) точек цикла, притом они идут
последовательно, а~также первая и~последняя связаны с~$n + 1$. Тогда вершине $n + 1$ инцедентно
максимум $\frac{qk}{n} + 1$ ребро, поэтому мы можем оценить дефицит такого графа:
$$ f(H) = \frac{n+k}{n} q - |E(H)| > \frac{n+k}{n}q - (q - 1 + \frac{qk}{n}+1) = \frac{qk + qn - qk
- qn}{n} = 0.$$
Таким образом мы доказали, что цикл с~центральной вершиной, от которой почти равномерно идут ребра
во все стороны, есть строго сбалансированный граф. Его плотность можно сделать любым рациональным
числом $\rho \in (1; 2)$.

\end{document}
