\documentclass{article}
\usepackage[utf8x]{inputenc}
\usepackage[english,russian]{babel}
\usepackage{amsmath,amscd}
\usepackage{amsthm}
\usepackage{mathtools}
\usepackage{amsfonts}
\usepackage{amssymb}
\usepackage{cmap}
\usepackage{centernot}
\usepackage{enumitem}
\usepackage{perpage}
\usepackage{chngcntr}
%\usepackage{minted}
\usepackage[bookmarks=true,pdfborder={0 0 0 }]{hyperref}
\usepackage{indentfirst}
\hypersetup{
  colorlinks,
  citecolor=black,
  filecolor=black,
  linkcolor=black,
  urlcolor=black
}

\newtheorem*{conclusion}{Вывод}
\newtheorem{theorem}{Теорема}
\newtheorem{lemma}{Лемма}
\newtheorem*{corollary}{Следствие}

\theoremstyle{definition}
\newtheorem*{problem}{Задача}
\newtheorem{claim}{Утверждение}
\newtheorem{exercise}{Упражнение}
\newtheorem{definition}{Определение}
\newtheorem{example}{Пример}

\theoremstyle{remark}
\newtheorem*{remark}{Замечание}

\newcommand{\doublearrow}{\twoheadrightarrow}
\renewcommand{\le}{\leqslant}
\renewcommand{\ge}{\geqslant}
\newcommand{\eps}{\varepsilon}
\renewcommand{\phi}{\varphi}
\newcommand{\ndiv}{\centernot\mid}

\MakePerPage{footnote}
\renewcommand*{\thefootnote}{\fnsymbol{footnote}}

\newcommand{\resetcntrs}{\setcounter{theorem}{0}\setcounter{definition}{0}
\setcounter{claim}{0}\setcounter{exercise}{0}}

\DeclareMathOperator{\aut}{aut}
\DeclareMathOperator{\cov}{cov}
\DeclareMathOperator{\argmin}{argmin}
\DeclareMathOperator{\argmax}{argmax}
\DeclareMathOperator*\lowlim{\underline{lim}}
\DeclareMathOperator*\uplim{\overline{lim}}
\DeclareMathOperator{\re}{Re}
\DeclareMathOperator{\im}{Im}

\frenchspacing


\begin{document}

\section{Связь в~обратную сторону}

\begin{lemma}
	Пусть $Q$~--- монотонное свойтво, $a \in [0; 1]$. Если $\forall p = p(n)$
	такой, что $p = \frac{m}{N} + o(\sqrt\frac{m(N-m)}{N^3})$ выполнено, что
	$P(\Gamma(n, p) \models Q) \rightarrow a$, то $P(\Gamma(n, m) \models Q)
	\rightarrow a$.
\end{lemma}

Докажем только ослабленный вариант, где $a = 0$ или $a = 1$.

\begin{lemma}
	Пусть $Q$~--- монотонное свойство, $m = m(n), m(n) \rightarrow +\infty$
	и~$\overline{\lim\limits_{n\rightarrow \infty}} \frac{m}{N} < 1$. Тогда если
	$P(\Gamma(n, \frac{m}{N}) \models Q) \rightarrow 1$, то $P(\Gamma(n, m)
	\models Q) \rightarrow 1$.
\end{lemma}
\begin{proof}~\\
	\begin{enumerate}
		\item Если $Q$~--- возрастающее свойство, то
			\begin{align*}
				P(\Gamma(n, \frac{m}{N}) \models Q) = \sum\limits_{k=0}^N P(\Gamma(n,
				\frac{m}{N}) \models Q \mid |\Gamma(n, \frac{m}{N})| = k) P(|\Gamma(n,
				\frac{m}{N})| = k) \le \\
				\sum\limits_{k=0}^N P(\Gamma(n, k) \models Q) P(|\Gamma(n, \frac{m}{N})|
				= k) \le \sum\limits_{k=0}^m + \sum\limits_{k>m+1} \le\\
				P(\Gamma(n, m) \models Q) P(|\Gamma(n, \frac{m}{N})| \le m) + P(|\Gamma(n,
				\frac{m}{N})| > m)
			\end{align*}
			По ЦПТ (условие на скорость роста $m(n)$ позволяет ею воспользоваться),
			получаем, что
			$$1 \le \frac{1}{2} \lowlim\limits_n P(\Gamma(n, m) \models Q) +
			\frac{1}{2} $$
			Значит $\exists\lim\limits_n P(\Gamma(n, m) \models Q) = 1$.
		\item
			Если $Q$~--- убывающее, то $P(\Gamma(n, \frac{m}{N}) \models Q) \le
			P(|\Gamma(n, m)| > m) P(\Gamma(n, m) \models Q) + P(|\Gamma(n, m)| \le
			m)$. Далее, все тоже самое.
	\end{enumerate}
\end{proof}

\begin{corollary}
	То же самое верно и~для $a = 0$.
\end{corollary}

\begin{corollary}[Асимптотическая эквивалентность моделей]
	Пусть $Q$~--- возрастающее свойство, $m = m(n) \rightarrow +\infty$,
	$\uplim\limits_n \frac{m}{N} \le 1 - \delta$. Тогда
	\begin{enumerate}
		\item $P(\Gamma(n, \frac{m}{N}) \models Q) \rightarrow 1 \Rightarrow
			P(\Gamma(n, m) \models Q) \rightarrow 1$.
		\item $P(\Gamma(n, \frac{m}{N}) \models Q) \rightarrow 0 \Rightarrow
			P(\Gamma(n, m) \models Q) \rightarrow 0$.
		\item $P(\Gamma(n, m) \models Q) \rightarrow 1 \Rightarrow \forall \eps > 0
			\,P(\Gamma(n, \frac{m}{N}(1 + \eps)) \models Q) \rightarrow 1$.
		\item $P(\Gamma(n, m) \models Q) \rightarrow 0 \Rightarrow \forall \eps > 0
			\,P(\Gamma(n, \frac{m}{N}(1 - \eps)) \models Q) \rightarrow 0$.
	\end{enumerate}
\end{corollary}
\begin{proof}
	Первые два~--- это лемма и~следствие. Положим $\frac{m}{N}(1 + \eps) = p(n)$.
	Тогда если $m'(n) = NP + O(\sqrt{Npq}) = (1 + \eps)m + O(\sqrt{m})$, то $m'(n)
	\ge m(n)$ начиная с~какого-то момента, значит в~силу возрастания $Q$
	$P(\Gamma(n, m') \models Q) \ge P(\Gamma(n, m) \models Q) \rightarrow 1$.
	Значит $P(\Gamma(n, m') \models Q) \rightarrow 1$, то есть по лемме
	$P(\Gamma(n, \frac{m}{N}(1 + \eps)) \models Q) \rightarrow 1$. Аналогично
	следует последний пункт.
\end{proof}

\section{Пороговые вероятности}

Мы доказали эквивалентность моделей только в~случае вероятности, стремящейся к~0
или к~1. Однако, это самый важный случай, так как имеет место эффект
<<пороговой вероятности>>.

\begin{definition}
	Пусть $Q$~--- возрастающее свойство подмножеств $\Gamma(n)$. Функция $\hat{p}
	= \hat{p}(n)$ называется \emph{пороговой вероятностью} для $Q$, если выполнено
	$\lim\limits_{n\rightarrow\infty} P(\Gamma(n, p) \models Q) = 1$ при $p =
	\omega(\hat{p})$ и~0, если $p = o(\hat{p})$.
\end{definition}
\begin{definition}
	Если $Q$~--- возрастающее свойство, то функция $\hat{m} = \hat{m}(n)$
	называется \emph{пороговой функцией} для $Q$, если выполнено
	$\lim\limits_{n\rightarrow\infty} P(\Gamma(n, m) \models Q) = 1$ при $m =
	\omega(\hat{m})$ и~0 при $m = o(\hat{m})$.
\end{definition}

\begin{remark}
	Для убывающих свойств все то же самое, с~точностью до наоборот.
\end{remark}
\begin{remark}
	$\hat{m}$~--- пороговая вероятность $\Leftrightarrow \hat{p} =
	\frac{\hat{m}}{N}$~--- пороговая функция.
\end{remark}

\begin{example}
	\begin{itemize}
		\item $\Gamma(n) = \{1, \ldots, n\}$, $Q = \{\text{внутри есть
			3-прогрессия}\}$. Тогда $\hat{p} = n^{-\frac{2}{3}}$~--- пороговая
			вероятность, $\hat{m} = n^{\frac{1}{3}}$~--- пороговая функция.
		\item $\Gamma(n)$~--- рёбра $K_n$, $Q = \{\text{есть }\Delta\}$. Тогда
			$\hat{p} = \frac{1}{n}$~--- пороговая вероятность.
	\end{itemize}
\end{example}

\begin{claim}
	Пусть $Q$~--- нетривиальное возрастающее свойство подмножеств $\Gamma(n)$.
	Тогда функция $f(p) = P(\Gamma(n, p) \models Q)$ является непрерывной, строго
	возрастающей на $[0; 1], f(0) = 0, f(1) = 1$.
\end{claim}
\begin{proof}
	Возрастание следует из предыдущих лемм.
	$$f(p) = \sum\limits_{A \in Q} P(\Gamma(n, p) = A) =
	\sum\limits_{A \in Q} p^{|A|}(1 - p)^{N - |A|}.$$
	Это многочлен, строго возрастающая непрерывная функция.
\end{proof}

\begin{definition}
	Если $Q$~--- возрастающее свойство, то $\forall a \in (0, 1)$ положим $p(a, n)
	= f_n^{-1}(a)$. Введём также $m(a, n) = \min \{m: P(\Gamma(n, m) \models Q)
	\ge a\}$.
\end{definition}

\begin{lemma}
	Пусть $Q$~--- возрастающее свойство, тогда $\hat{p} = \hat{p}(n)$ является
	пороговой вероятностью для $Q$ $\Leftrightarrow \forall a \in
	(0, 1)$ выполнено $\hat{p} \asymp p(a, n)$. И~$\hat{m}$~--- пороговая
	вероятность для $Q \Leftrightarrow \forall a \in (0, 1)$ выполнено $\hat{m}
	\asymp m(a, n)$.
\end{lemma}
\begin{proof}
	Докажем для равномерной модели. Пусть $\hat{m}$~--- пороговая, но $\exists a:
	\in (0, 1)$ такое, что $\hat{m} \not\asymp m(a, n)$. Тогда существует
	подпоследовательность $\hat{m}_{n_k}$ такая, что отношение
	$\frac{\hat{m}_{n_k}}{m(a, n_k)} \rightarrow 0$ или $+\infty$.

	Пусть предел нулевой. Тогда $m' = m(a, n_k) - 1$ есть $\omega(\hat{m})$.
	В~таком случае $\lim\limits_{k\rightarrow\infty} P(\Gamma(n, m'(n_k)) \models
	Q) = 1$.  Но $P(\Gamma(n, m'(n_k)) \models Q) \le a < 1$, противоречие.

	Если же предел равен $+\infty$, то $m(n_k) = o(\hat{m})$. Тогда
	$\lim\limits_k P(\Gamma(n, m(n_k)) \models Q) = 0$. Но для любого $k$
	выполнено $P(\Gamma(n, m(n_k)) \models Q) \ge a > 0$, противоречие.

	В~обратную сторону: пусть $\hat m = \omega(\hat{m})$. Тогда $\forall a \in (0,
	1) \,m = \omega(m(a, n))$, значит в~силу возрастания $Q$
	$P(\Gamma(n, m) \models Q) \ge P(\Gamma(n, m(a, n)) \models
	Q) \Rightarrow \lowlim\limits_n P(\Gamma(n, m) \models Q) \ge \lowlim\limits_n
	P(\Gamma(n, m(a, n)) \models Q) \ge a$, то есть $\exists \lim\limits_n
	P(\Gamma(n, m) \models Q) = 1$.

	Если $m = o(\hat{m})$, то все аналогично.
\end{proof}

\begin{theorem}
	Каждое монотонное свойство имеет пороговую вероятность.
\end{theorem}
\begin{proof}
	Считаем, что $Q$~--- возрастающее свойство. Надо показать, что все функции
	$p(a, n)$ имеют один и~тот же порядок. Возьмём $\eps \in (0, \frac{1}{2})$
	и~такое $m$, что $(1 - \eps)^m \le \eps$. Рассмотрим $\Gamma^{(1)}(n, p(\eps,
	n)), \ldots, \Gamma^{(m)}(n, p(\eps, n_))$~--- н.о.р. случайные подмножества
	$\Gamma(n)$. Тогда
	$$\tilde\Gamma = \Gamma^{(1)}(n, p(\eps, n)) \cup \ldots \cup
	\Gamma^{(m)}(n, p(\eps, n)) \overset{d}= \Gamma(n, p'),$$
	где $p' = 1 - (1 - p(\eps, n))^m \le m p(\eps, n)$.

	$P(\tilde\Gamma \models Q) = P(\Gamma(n, p') \models Q) \le P(\Gamma(n,
	mp(\eps, n)) \models Q)$.

	С~другой стороны $P(\tilde\Gamma \not\models Q) \le P(\forall i \Gamma^{(i)}(n,
	p(\eps, n)) \not\models Q) = P^m(\Gamma^{(1)}(n, p(\eps, n)) \not\models Q) =
	(1 - \eps)^m \le \eps$. Тогда $P(\tilde\Gamma \models Q) \ge 1 - \eps) =
	P(\Gamma(n, p(1 - \eps, n)) \models Q)$.

	Значит $\forall n\, mp(\eps, n) \ge p(1 - \eps, n)$. Итого $p(\eps, n) \le
	p(\frac{1}{2}, n) \le p(1 - \eps, n) \le mp(\eps, n)$. Значит по лемме,
	$p(\frac{1}{2}, n) = \hat{p}$~--- пороговая вероятность для $Q$.
\end{proof}

\begin{corollary}
	Для $\forall$монотонного свойства $\exists$ пороговая функция $\hat{m}$.
\end{corollary}

\begin{definition}
	Пусть $Q$~--- выпуклое свойство. Тогда функции $\hat{p_1} \le \hat{p_2}$
	называются \emph{пороговыми} для $Q$, если...
\end{definition}

\begin{example}
	$\Gamma(n)$~--- рёбра $K_n$.
	\begin{itemize}
		\item $Q = \{\text{обхват} = 4\}$, $\hat{p_1} = \hat{p_2} = \frac{1}{n}$
		\item $Q = \{\text{кликовое число}  = 4\}$, $\hat{p_1} = n^{-\frac{2}{3}}$,
			$\hat{p_2} = n^{-\frac{1}{2}}$
	\end{itemize}
\end{example}

\begin{definition}
	Пусть $Q$~--- возрастающее. Тогда $\hat{p} = \hat{p}(n)$ называется
	\emph{точной пороговой вероятностью} для $Q$, если $\forall \eps > 0$
	выполнено $\lim\limits_{n\rightarrow\infty} P(\Gamma(n, p) \models Q) = 1$ при
	$p \ge (1 + \eps) \hat{p}$ и~0 при~$p \le (1 - \eps)\hat{p}$.
\end{definition}

\begin{example}
	$\Gamma(n)$~--- рёбра $K_n$.
	\begin{itemize}
		\item $Q = \{\text{связность}\}$, $\hat{p} = \frac{\ln n}{n}$~--- точная
			пороговая вероятность
		\item $Q = \{\text{есть } \Delta\}$, $\hat{p} = \frac{1}{n}$~--- пороговая
			вероятность, но точной пороговой вероятности нет
		\item $Q = \{\text{ацикличность}\}$, $\hat{p} = \frac{1}{n}$~--- пороговая
			вероятность для $Q$, но точна она только с~одной стороны
	\end{itemize}
\end{example}

\begin{theorem}[Фридгут]
	Пусть $Q$~--- монотонное свойство графов, $\hat{p}$~--- пороговая и~она не
	точная. Тогда существует конечное разбиение $N_j, j = 1, \ldots, k$ множества
	$\mathbb{N}$ и~рациональные числа $\alpha_1, \ldots, \alpha_k > 0$ такие, что
	$\forall n \in N_j$ выполнено $\hat{p}(n) \asymp n^{-\alpha_j}$.
\end{theorem}

\end{document}
