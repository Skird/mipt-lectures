\documentclass{article}
\usepackage[utf8x]{inputenc}
\usepackage[english,russian]{babel}
\usepackage{amsmath,amscd}
\usepackage{amsthm}
\usepackage{mathtools}
\usepackage{amsfonts}
\usepackage{amssymb}
\usepackage{cmap}
\usepackage{centernot}
\usepackage{enumitem}
\usepackage{perpage}
\usepackage{chngcntr}
%\usepackage{minted}
\usepackage[bookmarks=true,pdfborder={0 0 0 }]{hyperref}
\usepackage{indentfirst}
\hypersetup{
  colorlinks,
  citecolor=black,
  filecolor=black,
  linkcolor=black,
  urlcolor=black
}

\newtheorem*{conclusion}{Вывод}
\newtheorem{theorem}{Теорема}
\newtheorem{lemma}{Лемма}
\newtheorem*{corollary}{Следствие}

\theoremstyle{definition}
\newtheorem*{problem}{Задача}
\newtheorem{claim}{Утверждение}
\newtheorem{exercise}{Упражнение}
\newtheorem{definition}{Определение}
\newtheorem{example}{Пример}

\theoremstyle{remark}
\newtheorem*{remark}{Замечание}

\newcommand{\doublearrow}{\twoheadrightarrow}
\renewcommand{\le}{\leqslant}
\renewcommand{\ge}{\geqslant}
\newcommand{\eps}{\varepsilon}
\renewcommand{\phi}{\varphi}
\newcommand{\ndiv}{\centernot\mid}

\MakePerPage{footnote}
\renewcommand*{\thefootnote}{\fnsymbol{footnote}}

\newcommand{\resetcntrs}{\setcounter{theorem}{0}\setcounter{definition}{0}
\setcounter{claim}{0}\setcounter{exercise}{0}}

\DeclareMathOperator{\aut}{aut}
\DeclareMathOperator{\cov}{cov}
\DeclareMathOperator{\argmin}{argmin}
\DeclareMathOperator{\argmax}{argmax}
\DeclareMathOperator*\lowlim{\underline{lim}}
\DeclareMathOperator*\uplim{\overline{lim}}
\DeclareMathOperator{\re}{Re}
\DeclareMathOperator{\im}{Im}

\frenchspacing


\begin{document}

Литература:
\begin{itemize}
	\item Boloobas <<Random graphs>>
	\item Janson, Lucak, Rucinski <<Random graphs>>
\end{itemize}

\section{Модели случайных графов}

\begin{definition}
	\emph{Случайный граф}~--- случайный элемент со значениями в~некотором конечном
	множестве графов.
\end{definition}

\begin{definition}
	\emph{Равномерная модель}. $K_n$~--- полный граф, $0 \le m \le C_n^2$,
	$\mathcal{G}_m$~--- множество всех остовных подграфов $K_n$, имеющих ровно $m$
	рёбер. Случайный граф в~этой модели~--- случайный элемент с~равномерным
	распределением на $\mathcal{G}_m$.

	$$ P(G(n, m) = F) = \frac{1}{C_{C_n^2}^m} \,\forall F \in \mathcal{G}_m $$
\end{definition}

Фиксировано число рёбер, но другие характеристики выглядят посложнее, скажем
$\deg v$ имеет гипергеометрическое распределение.

\begin{definition}
	\emph{Биномиальная модель}. $\mathcal{G}$~--- множество всех остовных
	подграфов $K_n$, $p \in [0, 1]$. Случайный граф в~этой модели~--- случайный
	элемент на $\mathcal{G}$ со следующим распределением:

	$$ P(G(n, p) = F) = p^{|E(F)|} (1 - p)^{C_n^2 - |E(F)|} \, \forall F \in
	\mathcal{G}$$
\end{definition}

Много независимых событий, из-за чего многие характеристики имеют удобное
распределение, например $\deg v \sim B(n - 1, P)$. Число рёбер, впрочем,
случайно.

Другие модели:
\begin{itemize}
	\item Граф $G$, схема Бернулли на его рёбрах. Скажем, $G = K_{n,m}$~---
		случайный двудольный граф.
	\item Равномерное распределение на какой-то совокупности графов $\mathcal{F}$.
		Например, случайный $d$-регулярный граф
		\begin{itemize}
			\item $d = 1$~--- случайное совершенное паросочетание
			\item $d = 2$~--- случайный набор циклов
			\item $d = 3$~--- можно показать, что а.п.н. это гамильтонов цикл плюс
				какое-то совершенное паросочетание
		\end{itemize}
	\item Случайный процесс на графе
		\begin{itemize}
			\item С~дискретным временем: $\tilde{G} = (\tilde{G}(n,m), m = 0 \ldots
				C_n^2)$, в~котором на каждом шаге появляется новое случайное равномерно
				выбранное ребро. $\tilde{G}(n,m) \overset{d}= G(n, m)$. Можно смотреть
				случайные моменты
				\begin{itemize}
					\item $\tau_1(n) = \min\{m: \delta(\tilde{G}(n,m)) \ge 1\}$
					\item $\sigma_1(n) = \min\{m: \tilde{G}(n,m)\text{ связен}\}$
				\end{itemize}
				\begin{theorem}[Баллобаш, Томасон]
					$$ P\left(\tau_1(n) = \sigma_1(n)\right) \rightarrow 1, n \rightarrow
					\infty $$
				\end{theorem}
			\item С~непрерывным временем: пусть для каждого ребра $e$ графа $K_n$
				задана случайная величина $T_e$. Тогда для $\forall t > 0$ можно
				рассмотреть процесс:
				$$ \tilde{G}_T = \{e \mid T_e \le t\} $$
				Если все $T_e$ распределены одинаково, $\tilde{G}_T(n,t) \overset{d}=
				G(n, p)$, где $p = P(T_e \le t)$.
			\item Triangle-free process. На каждом шаге включаем одно случайное ребро
				так, чтобы не возникало треугольников. Можно показать, что в~результате
				такого процесса $\alpha(\text{итогового графа}) = O(\sqrt{n \ln n})$.
				Следствие: оценка на число Рамсея $R(3, t) \ge c \frac{t^2}{\ln t}$.
		\end{itemize}
\end{itemize}

\section{Общая теория случайных подмножеств}

Пусть $\Gamma$~--- конечное множество, $|\Gamma| = N$.
\begin{itemize}
	\item $\Gamma(p)$~--- схема Бернулли на $\Gamma$.
	\item $\Gamma(n)$~--- случайное подмножество размера $n$ с~равномерным
		распределением
	\item $\tilde\Gamma(m)$~--- случайный процесс, включающий элементы
		последовательно
\end{itemize}

В~асимптотиских утверждениях $\Gamma = \Gamma_n, n \in \mathbb{N}$~---
последовательность, притом $N=N(n)$.

\section{Монотонные и~выпуклые свойства}

\begin{definition}
	$Q$~--- семейство подмножеств $\Gamma$ называется \emph{возрастающим}, если
	$A \in Q, A \subset B \rightarrow B \in Q$, \emph{убывающим}, если $A \supset
	B \rightarrow B \in Q$, \emph{монотонным}, если оно возрастающее или
	убывающее.
\end{definition}

Ясно, что $Q$~--- возрастающее тогда и~только тогда, когда $\overline{Q} =
2^\Gamma \setminus Q$~--- убывающее. Будем обозначать $\Gamma(p) \models Q
\Leftrightarrow \Gamma(p) \in Q$ (<<обладает свойством Q>>).

\begin{example}
	$\Gamma$~--- рёбра $K_n$. Возрастающие свойства:
	\begin{itemize}
		\item связность
		\item содержит какой-то подграф
		\item $\delta(G) \ge k$
	\end{itemize}
	Убывающие свойтва:
	\begin{itemize}
		\item планарность
		\item $\chi(G) \le k$
		\item ацикличность
	\end{itemize}
\end{example}

\begin{lemma}
	Пусть $Q$~--- возрастющее свойство. Тогда $\forall p_1 \le p_2, m_1 \le m_2$:
	$$ P(\Gamma(p_1) \models Q) \le P(\Gamma(p_2) \models Q) $$
	$$ P(\Gamma(m_1) \models Q) \le P(\Gamma(m_2) \models Q) $$
\end{lemma}
\begin{proof}~\
	\begin{itemize}
		\item $P(\Gamma(m_1) \models Q) = P(\tilde\Gamma(m_1) \models Q) \le
			P(\tilde\Gamma(m_2) \models Q) = P(\Gamma(m_2) \models Q)$
		\item Пусть $\Gamma(p') \bot \Gamma(p'')$~--- два независимых подмножества.
			Тогда $\Gamma(p') \cup \Gamma(p'') \overset{d}= \Gamma(p)$, где $p = p' +
			p'' - p' p''$. Тогда можно положить $p' = \frac{p_2 - p_1}{1 - p_1}$,
			а~также, что $\Gamma(p') \bot \Gamma(p_1)$. Тогда
			$$P(\Gamma(p_1) \models Q) \le P(\Gamma(p_1) \cup \Gamma(p') \models Q) =
			P(\Gamma(p_2) \models Q).$$
	\end{itemize}
\end{proof}

\begin{definition}
	Свойство $Q$ называется \emph{выпуклым}, если $A \subset C \subset B \in Q
	\Rightarrow C \in Q$
\end{definition}

\begin{example}~\
	\begin{itemize}
		\item все монотонные выпуклы
		\item $\chi(G) = k$
	\end{itemize}
\end{example}

\section{Асимптотическая эквивалентность моделей}

Хотим установить какую-то связь между моделями $\Gamma(p)$ и~$\Gamma(m)$ при $pN
\sim m$. Для этого введём следующий контекст:

\begin{itemize}
	\item $\Gamma(n), n \in \mathbb{N}$~--- последовательность конечных множеств
	\item $N = N(n) \rightarrow +\infty$
	\item $Q = Q(n)$
	\item $p = p(n) \in [0, 1]$
	\item $m = m(n) \in \{0, \ldots, N\}$
	\item $\Gamma(n, p), \Gamma(n, m)$~--- случайные подмножества $\Gamma(n)$
\end{itemize}

\begin{lemma}
	Пусть $Q$~--- свойство $\Gamma(n)$. Пусть $p = p(n) \in [0, 1]$~--- некоторая
	функция. Если для любой последовательности $m = m(n)$, такой что
	$$ m = Np + O(\sqrt{Npq}), q = 1 - p$$ выполнено
	$$P(\Gamma(n, m) \models Q) \rightarrow a, n \rightarrow \infty$$
	то
	$$P(\Gamma(n, p) \models Q) \rightarrow a, n \rightarrow \infty.$$
\end{lemma}
\begin{proof}
	Пусть $C > 0$~--- большая константа и~положим $M(C) = \{m \mid |m - Np| \le
	C\sqrt{Npq}\}$. Обозначим
	$$ m_\ast = \underset{m \in M(C)}\argmin P(\Gamma(n, m) \models Q) $$
	$$ m^\ast = \underset{m \in M(C)}\argmax P(\Gamma(n, m) \models Q) $$
	По формуле полной вероятности:
	\begin{align*}
		P(\Gamma(n, p) \models Q) = &\sum\limits_{m=0}^N P(\Gamma(n,p) \models Q \mid
		& |\Gamma(n, p)| = m) P(|\Gamma(n, p)| = m) = \\
		& \sum\limits_{m=0}^N P(\Gamma(n,m) \models Q) P(|\Gamma(n,p)|=m) \ge \\
		& \sum\limits_{m \in M(C)} P(\Gamma(n,m) \models Q) P(|\Gamma(n,p)|=m) \ge \\
		& P(\Gamma(n, m_\ast) \models Q) P(|\Gamma(n, p)| \in M(C)|)
	\end{align*}
	Но $|\Gamma(n, p)| \sim Bin(N, p), E|\Gamma(n, p)| = Np, D|\Gamma(n, p)| =
	Npq$. По неравенству Чебышева:
	$$ P(||\Gamma(n, p)| - Np| > C\sqrt{Npq}) \le \frac{Npq}{C^2Npq} =
	\frac{1}{C^2} $$
	Значит $P(\Gamma(n, p) \models Q) \ge P(\Gamma(n,m_\ast) \models Q)\left(
	1 - \frac{1}{C^2}\right)$.

	Аналогично
	\begin{multline*}
		P(\Gamma(n, p) \models Q) \le
		\sum\limits_{m \in M(C)} P(\Gamma(n,m) \in Q) P(|\Gamma(n,p)| = m) +
		\sum\limits_{m \notin M(C)} P(|\Gamma(n,p)| = m) \\
		\le P(\Gamma(n,m^\ast) \in Q) + \frac{1}{C^2}
	\end{multline*}

	Значит $\overline{\lim\limits_{n\rightarrow \infty}} P(\Gamma(n, p) \models Q)
	\le a + \frac{1}{C^2}$.

	Также $\underline{\lim\limits_{n \rightarrow \infty}} P(\Gamma(n, p) \models Q)
	\ge a(1 - \frac{1}{C^2})$.

	Это верно для любого $C > 0$. Тогда $\exists \lim P(\Gamma(n, p) \models Q) =
	a$.
\end{proof}

\end{document}
