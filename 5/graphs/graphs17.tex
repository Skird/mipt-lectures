\documentclass{article}
\usepackage[utf8x]{inputenc}
\usepackage[english,russian]{babel}
\usepackage{amsmath,amscd}
\usepackage{amsthm}
\usepackage{mathtools}
\usepackage{amsfonts}
\usepackage{amssymb}
\usepackage{cmap}
\usepackage{centernot}
\usepackage{enumitem}
\usepackage{perpage}
\usepackage{chngcntr}
%\usepackage{minted}
\usepackage[bookmarks=true,pdfborder={0 0 0 }]{hyperref}
\usepackage{indentfirst}
\hypersetup{
  colorlinks,
  citecolor=black,
  filecolor=black,
  linkcolor=black,
  urlcolor=black
}

\newtheorem*{conclusion}{Вывод}
\newtheorem{theorem}{Теорема}
\newtheorem{lemma}{Лемма}
\newtheorem*{corollary}{Следствие}

\theoremstyle{definition}
\newtheorem*{problem}{Задача}
\newtheorem{claim}{Утверждение}
\newtheorem{exercise}{Упражнение}
\newtheorem{definition}{Определение}
\newtheorem{example}{Пример}

\theoremstyle{remark}
\newtheorem*{remark}{Замечание}

\newcommand{\doublearrow}{\twoheadrightarrow}
\renewcommand{\le}{\leqslant}
\renewcommand{\ge}{\geqslant}
\newcommand{\eps}{\varepsilon}
\renewcommand{\phi}{\varphi}
\newcommand{\ndiv}{\centernot\mid}

\MakePerPage{footnote}
\renewcommand*{\thefootnote}{\fnsymbol{footnote}}

\newcommand{\resetcntrs}{\setcounter{theorem}{0}\setcounter{definition}{0}
\setcounter{claim}{0}\setcounter{exercise}{0}}

\DeclareMathOperator{\aut}{aut}
\DeclareMathOperator{\cov}{cov}
\DeclareMathOperator{\argmin}{argmin}
\DeclareMathOperator{\argmax}{argmax}
\DeclareMathOperator*\lowlim{\underline{lim}}
\DeclareMathOperator*\uplim{\overline{lim}}
\DeclareMathOperator{\re}{Re}
\DeclareMathOperator{\im}{Im}

\frenchspacing


\begin{document}

\section{Неравенство FKG}

$f(x_1, \ldots, x_n): \mathbb{R}^n \rightarrow \mathbb{R}$

\begin{definition}
	Функция $f$ называется возрастающей, если не убывает по каждой переменной. Аналогично определяется
	убывающая функция.
\end{definition}

\begin{theorem}[FKG-неравенство]
	Пусть $X_1, \ldots, X_n$~--- независимые с.в., $f(x_1, \ldots, x_n), g(x_1, \ldots, x_n)$~---
	возрастающие (убывающие) функции. Тогда
	$$cov(f(X_1, \ldots, X_n), g(X_1, \ldots, X_n)) \ge 0.$$
\end{theorem}
\begin{proof}
	Индукция по $n$. Пусть $n = 1$, тогда $(f(x) - f(y))(g(x) - g(y)) \ge 0$. Пусть $Y$ независима
	с~$X_1$ с.в. с~тем же распределением, $Y \overset{d}= X_1$. Тогда
	$$ (f(X_1) - f(Y))(g(X_1) - g(Y)) \ge 0.$$
	Берем мат. ожидание:
	\begin{multline*}
		E(f(X_1) - f(Y))(g(X_1) - g(Y)) =\\
		= 2Ef(X_1)g(X_1) - 2Ef(X_1) \cdot Eg(X_1) = 2\cov(f(X_1), g(X_1) \ge 0.
	\end{multline*}

	Докажем теперь шаг. Рассмотрим $f'(x_1, \ldots, x_{n-1}) = Ef(X_1, \ldots, X_n) \mid X_1 = x_1,
	\ldots, X_{n-1} = x_{n-1}) = Ef(x_1, \ldots, x_{n-1},X_n), g'(x_1, \ldots, x_{n-1}) = Eg(x_1,
	\ldots, x_{n-1}, X_n)$.

	Это тоже возрастающие функции от $n - 1$ переменной. По предположению индукции $\cov(f'(x_1,
	\ldots, x_{n-1}), g'(x_1, \ldots, x_{n-1})) \ge 0$. Аналогично, $\tilde{f}_{x_1, \ldots,
	x_{n-1}}(x_n) = f(x_1, \ldots, x_n), \tilde{g}_{x_1, \ldots, x_{n-1}}(x_n) = g(x_1, \ldots,
	x_n)$~--- возрастающие функции 1 переменной, то есть $\cov(\tilde{f}, \tilde{g}) \ge 0$.

	В~итоге
	\begin{multline*}
		Ef(X_1, \ldots, X_n)g(X_1, \ldots, X_n) = E(E(f(X_1, \ldots, X_n)g(X_1, \ldots, X_n) \mid X_1,
		\ldots X_n)) =\\
		= \int\limits_{\mathbb{R}^n} E(f(x_1, \ldots, x_{n-1},X_n) g(x_1, \ldots, x_{n-1},x_n))
		P_{X_1}(dx_1) \ldots P_{X_n}(dx_n) \ge\\
		\ge \int\limits_{\mathbb{R}^n} E\tilde{f}_{x_1, \ldots, x_{n-1}}(x_n) \tilde{g}_{x_1, \ldots,
		x_{n-1}}(x_n) P_{X_1}(dx_1) \ldots P_{X_n}(dx_n) =\\
		= \int\limits_{\mathbb{R}^n} Ef'(x_1, \ldots, x_{n-1}) g'(x_1, \ldots, x_{n-1}) = Ef'(X_1,
		\ldots, X_{n-1}) g'(X_1, \ldots, X_{n-1}) \ge\\
		\ge Ef'(X_1, \ldots, X_{n-1}) g'(X_1, \ldots, X_{n-1}) = Ef(X_1, \ldots, X_n)g(X_1, \ldots,
		X_n)
	\end{multline*}
\end{proof}

\section{Неравенство Янсона}

Пусть $\Gamma$~--- конечное множество мощности $N$. $\Gamma(p_1, \ldots, p_N)$~--- биномиальная
модель случайного подмножества. Введем для $A \subset \Gamma$ величину $I_A = I(A \subset
\Gamma(p_1, \ldots, p_N))$.

\begin{theorem}[неравенство Янсона]
	Пусть $S$~--- система подножеств $\Gamma$, $X = \sum\limits_{A \in S} I_A$, $\lambda = EX$,
	$\overline{\Delta} = \sum\limits_{A,B\in S, A\cap B\ne\varnothing} EI_A I_B$. Тогда $\forall 0 \le
	t \le EX$ выполнено
	$$ P(X \le EX - t) \le \exp\left(-\frac{t^2}{2\overline{\Delta}}\right).$$
\end{theorem}
\begin{proof}
	Рассмотрим $\psi(s) = Ee^{-sX}, s \ge 0$. Тогда
	$$-\psi'(s) = EXe^{-sX} = \sum\limits_{A \in S} E(I_A e^{-sX}).$$

	Для $A \in S$ введем $Y_A = \sum\limits_{B \in S, A \cap B \ne \varnothing} I_B, Z_A =
	\sum\limits_{B \in S, A \cap B = \varnothing} I_B$.

	Заметим, что $Y_A, Z_A$~--- возрастающие функции от $\{\xi_i = I(i \in \Gamma(p_1, \ldots, p_N)):
	i \notin A \}$. Тогда

	\begin{multline*}
		E(I_A e^{-sX}) = p(A) E(e^{-sX} \mid I_A = 1) = p(A) E(e^{-sY_A} e^{-sZ_a} \mid I_A = 1) \ge\\
		\ge [\text{FKG}] \ge p(A) E(e^{-sY_a} \mid I_A = 1) E(e^{-sZ_A} \mid I_A = 1) = [Z_A \bot I_A] =
		\\ = p(A) E(e^{-sY_a} \mid I_A = 1) Ee^{-sZ_a} \ge p(A) E(e^{-sY_A} \mid I_A = 1) \psi(s)
		\\ \Rightarrow  (-\ln \psi(s))' \ge \sum\limits_{A \in S} p(A) E(e^{sY_a} \mid I_A = 1) \ge
		[\text{неравенство Йенсена}] \ge\\\ge \sum\limits_{A \in S} p(A) e^{-E(sY_a \mid I_A = 1)}
	\end{multline*}

	\begin{multline*}
		-\ln \psi(s) = \int\limits_0^s (-\ln \psi(u))' du \ge \int\limits_0^s (\lambda
		e^{-\frac{u}{\lambda}\overline{\Delta}}) du = \frac{\lambda^2}{\overline{\Delta}}(1 -
		e^{-\frac{s\overline{\Delta}}{\lambda}}) \ge\\\ge [1 - e^{-x} \ge x - \frac{x^2}{2}]  \ge
		s\lambda - \frac{s^2\overline{\Delta}}{2}
	\end{multline*}

	В~итоге, $P(X \le \lambda - t) = P(e^{-sX} \ge e^{-s(\lambda - t)}) \le
	\frac{Ee^{-sX}}{e^{-s(\lambda - t)}} = e^{\ln\psi(s)} e^{s\lambda - st} \le e^{-s\lambda +
	\frac{s^2\overline{\Delta}}{2} + s\lambda - st}$

	Минимум при $s = \frac{t}{\overline{\Delta}}$, значит $P(X \le EX - t) \le
	e^{-\frac{t^2}{2\overline{\Delta}}}$.
\end{proof}

\begin{corollary}
	В~условиях неравенства Янсона $P(X = 0) \le e^{-\frac{\lambda^2}{2\overline{\Delta}}}$.
\end{corollary}

\section{Неравенство Азумы-Хёффдинга}

\begin{definition}
	$(X_n, n=0,\ldots,)$~--- мартингал относительно фильтрации $(F_n, n=0,\ldots)$, если
	\begin{itemize}
		\item $X_n$ измерим относительно $F_n$
		\item $E|X_n| < +\infty$
		\item $\forall n > k E(X_n \mid F_k) = X_k$
	\end{itemize}
\end{definition}

\begin{exercise}
	Последнее условие эквивалентно $E(X_n \mid F_{n-1}) = X_{n-1}$.
\end{exercise}

Если в~последнем условии неравенство вместо равенства, то это субмартингал или супермартингал.

\begin{theorem}[неравенство Азумы-Хёффдинга]
	Пусть $(X_k,k=0,\ldots,n)$~--- мартингал относительно $F_k$, причём $X_0 = EX_n$ и~для некоторых
	констант $c_1, \ldots, c_n$ выполнено $|X_k - X_{k-1}| \le c_k\ \forall k=1,\ldots,n$.

	Тогда $\forall t \ge 0$ выполнено
	$$P(X_n \ge EX_n + t), P(X_n \le EX_n - t) \le e^{-\frac{t^2}{2\sum\limits_{k=1}^n c_k^2}}$$
\end{theorem}
\begin{proof}
	Обозначим $Y_k = X_k - X_{k-1}, k = 1, \ldots, n$, тогда $X_n - X_0 = \sum\limits_{k=1}^n Y_k$.
	Обозначим $S_j = \sum\limits_{k=1}^j Y_k$, тогда $\forall u > 0$ выполнено
	\begin{multline*}
		P(X_n \ge EX_n + t) = P(S_n \ge t) = P(e^{uS_n} \ge e^{ut}) \le [\text{Марков}] \le e^{-ut}
		Ee^{uS_n} =\\= e^{-ut}E(E(e^{uS_n}\mid F_{n-1})) = e^{-ut} E(e^{uS_{n-1}}E(e^{uY_n}\mid
		F_{n-1})).
	\end{multline*}

	Далее, $|Y_n| \le c_n$ и~$E(Y_n \mid F_{n-1}) = 0$. Тогда в~силу выпуклости функции $e^{ux}$:
	$$e^{uY_n} \le \frac{c_n + Y_n}{2c_n} e^{uc_n} + \frac{c_n - Y_n}{2c_n} e^{-uc_n}.$$

	Берем УМО относительно $F_{n-1}$:
	$$
		E(e^{uY_n} \mid F_{n-1}) \le \frac{1}{2}(e^{uc_n} + e^{-uc_n}) = \chos(uc_n) =
		\sum\limits_{m=0}^{+\infty} \frac{(uc_n)^{2m}}{(2m)!} \le e^{\frac{(uc_n)^2}{2}}
	$$

	Отсюда $P(X_n \ge EX_n + t) \le e^{-ut} e^\frac{(uc_n)^2}{2} Ee^{uS_{n-1}} \le [\text{рекурсия}]
	\le e^{-ut} e^{\frac{u^2}{2} \sum\limits_{k=1}^n c_k^2}$.

	Минимум достигается при $u = \frac{t}{\sum\limits_{k=1}^n c_k^2}$, что нам и~нужно.
\end{proof}

Пусть $X$~--- случайная величина, а~$(F_0, \ldots, F_n)$~--- фильтрация, такая что $F_0$~---
тривиальная и~$X$ измерим относительно $F_n$. Тогда $X_k = E(X \mid F_k)$~--- мартингал, почти
подходящий под неравенство Азумы-Хёффдинга.

\begin{corollary}
	Пусть $Z_1, \ldots, Z_n$~--- независимые случайные вектора размерность $Z_i$ равна $k_i$.
	Пусть $f(x_1, \ldots, x_n)$ такова, что $\forall i=1,\ldots,n \forall x_i, y_i \in
	\mathbb{R}^{k_i}$ выполнено
	$$|f(z_1, \ldots, x_i, \ldots, z_n) - f(z_1, \ldots, y_i, \ldots, z_n)| \le c_i.$$

	Обозначим $X = f(Z_1, \ldots, Z_n)$, тогда $\forall t \ge 0$ выполнено:
	$$ P(X \ge EX + t), P(X \le EX - t) \le e^{-\frac{t^2}{2\sum\limits_{k=1}^n c_k^2}} $$
\end{corollary}
\begin{proof}
	Пусть $F_k = \sigma(Z_1, \ldots, Z_k)$. Тогда $F_0, \ldots, F_n$~--- фильтрация, $X_k = E(X \mid
	F_k)$~--- мартингал, притом $X_0 = EX$, $X_n = X$. Тогда для неравенства Азумы-Хёффдинга
	необходимо проверить, что $\forall k |X_k - X_{k-1}| \le c_k$.

	Но для $\forall z_1, \ldots, z_k$ выполнено
	\begin{multline*}
		|E(X \mid Z_1 = z, \ldots, Z_{k-1}=z_{k-1}) - E(X \mid Z_1 = z_1, \ldots, Z_k=z_k)| =
		[\text{нез}] =\\=
		|Ef(z_1, \ldots, z_{k-1}, Z_k, \ldots, Z_n) - Ef(z_1,\ldots,z_k,Z_{k+1},\ldots,Z_n)| \le c_k
	\end{multline*}
\end{proof}

\section{Мартингалы реберного и~вершинного типа}

$Z_1, \ldots, Z_N$~--- индикаторы появления ребер в~$G(n, p), N = C_n^2$. $Z_j =
I(E_j \in G(n, p))$.

Таким образом, нас интересуют характеристики, которые слабо меняются при добавлении и~удалении
ребра.
\begin{itemize}
	\item число ребер
	\item макс. и~мин. степень
	\item число компонент связности
\end{itemize}

Нарямую применять его тяжеловато из-за того, что $N$ достаточно большое. Можно перейти к~равномерной
модели (если $p$ какое-то небольшое и~среднее число ребер, например, линейное). Другой путь:
мартингалы вершинного типа.

$v_1, \ldots, v_n$~--- вершины $K_n$, $Z_j = \{I\{(v_i, v_j) \in E(G(n, p))\}: i < j\}$~---
случайный вектор из $(j-1)$-го индикатора. Такой процесс неформально <<проявляет>> вершины по одной.

Таким образом, нас интересуют характеристики, которые слабо меняются при добавлении и~удалении
вершины:

\begin{itemize}
	\item хроматическое число
	\item число независимости, кликовое число
\end{itemize}

\begin{exercise}
	$\forall \delta > 0$ выполнено
	$$\frac{\alpha(G(n, p)) - E\alpha(G(n, p))}{n^{\frac{1}{2}+\delta}} \overset{P}\rightarrow 0.$$
\end{exercise}

\end{document}
