\documentclass{article}
\usepackage[utf8x]{inputenc}
\usepackage[english,russian]{babel}
\usepackage{amsmath,amscd}
\usepackage{amsthm}
\usepackage{mathtools}
\usepackage{amsfonts}
\usepackage{amssymb}
\usepackage{cmap}
\usepackage{centernot}
\usepackage{enumitem}
\usepackage{perpage}
\usepackage{chngcntr}
%\usepackage{minted}
\usepackage[bookmarks=true,pdfborder={0 0 0 }]{hyperref}
\usepackage{indentfirst}
\hypersetup{
  colorlinks,
  citecolor=black,
  filecolor=black,
  linkcolor=black,
  urlcolor=black
}

\newtheorem*{conclusion}{Вывод}
\newtheorem{theorem}{Теорема}
\newtheorem{lemma}{Лемма}
\newtheorem*{corollary}{Следствие}

\theoremstyle{definition}
\newtheorem*{problem}{Задача}
\newtheorem{claim}{Утверждение}
\newtheorem{exercise}{Упражнение}
\newtheorem{definition}{Определение}
\newtheorem{example}{Пример}

\theoremstyle{remark}
\newtheorem*{remark}{Замечание}

\newcommand{\doublearrow}{\twoheadrightarrow}
\renewcommand{\le}{\leqslant}
\renewcommand{\ge}{\geqslant}
\newcommand{\eps}{\varepsilon}
\renewcommand{\phi}{\varphi}
\newcommand{\ndiv}{\centernot\mid}

\MakePerPage{footnote}
\renewcommand*{\thefootnote}{\fnsymbol{footnote}}

\newcommand{\resetcntrs}{\setcounter{theorem}{0}\setcounter{definition}{0}
\setcounter{claim}{0}\setcounter{exercise}{0}}

\DeclareMathOperator{\aut}{aut}
\DeclareMathOperator{\cov}{cov}
\DeclareMathOperator{\argmin}{argmin}
\DeclareMathOperator{\argmax}{argmax}
\DeclareMathOperator*\lowlim{\underline{lim}}
\DeclareMathOperator*\uplim{\overline{lim}}
\DeclareMathOperator{\re}{Re}
\DeclareMathOperator{\im}{Im}

\frenchspacing


\begin{document}

Литература:
\begin{itemize}
	\item Гринлиф, <<инвариантные средние в~топологических группах>>
\end{itemize}

\section{Введение}

Одни из объектов изучения: групповые графы.

\begin{definition}
	Граф Кэли $Cayley(G, S) = (G, \{x \mapsto sx\})$, где $S \subset G$ (ориентированный граф).
\end{definition}

\begin{definition}
	Граф Шрейра $(G / H, \{xH \mapsto sxH\})$, где $S \subset G$ (ориентированный мультиграф).
\end{definition}

В~качестве простой конструкции нетривиальной группы рассмотрим так называемые автоматные группы.
Пусть $\mathbb{A}$~--- алфавит ($\{0, 1\}$). Рассматриваются конечные преобразователи на двух
состояниях $a, b$. На каждый входной символ выдается один выходной.  Мы хотим рассматривать только
обратимые преобразования, поэтому можно показать, что вершины можно разметить на два класса:
1~--- в~вершине выдается тот же символ, что и~на входе, $\eps$~--- выдается противоположный.
Естественным образом у~такого автомата есть два преобразования: преобразовать слово, начав
в~вершине $a$ или $b$. Автоматная группа образована этими самыми преобразованиями
$G = \langle A_a, A_b \rangle$.

Можно рассматривать эти преобразования как автоморфизмы двоичного дерева. Тут удобен формализм
преобразования вершины вида $\eps^k (\xi, \eta)$, где $k \in \{0, 1\}$, а~$(\xi, \eta)$~--- это
преобразования двух дочерних поддеревьев. Заметим также, что $\eps(\xi, \eta) = (\eta, \xi) \eps$.
Тогда в~примере автомата, прибавляющего единицу (adding machine): $a = \eps (a, b), b = (b, b)$,
откуда $b = Id$, а~$\langle a \rangle = \mathbb{Z}$.

Возможные автоматные группы: $\mathbb{Z}, \mathbb{Z}_2, \mathbb{Z}_2 \times \mathbb{Z}_2,
D_\infty$~--- простые примеры. Нетривиальный пример: lamplighter group.

\end{document}
