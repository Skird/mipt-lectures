\documentclass{article}
\usepackage[utf8x]{inputenc}
\usepackage[english,russian]{babel}
\usepackage{amsmath,amscd}
\usepackage{amsthm}
\usepackage{amsfonts}
\usepackage{amssymb}
\usepackage{cmap}
\usepackage{centernot}
\usepackage{enumitem}
\usepackage{perpage}
\usepackage{chngcntr}
%\usepackage{minted}
\usepackage[bookmarks=true,pdfborder={0 0 0 }]{hyperref}
\usepackage{indentfirst}
\hypersetup{
  colorlinks,
  citecolor=black,
  filecolor=black,
  linkcolor=black,
  urlcolor=black
}

\newtheorem*{conclusion}{Вывод}
\newtheorem{theorem}{Теорема}
\newtheorem{lemma}{Лемма}
\newtheorem*{corollary}{Следствие}

\theoremstyle{definition}
\newtheorem*{problem}{Задача}
\newtheorem{claim}{Утверждение}
\newtheorem{exercise}{Упражнение}
\newtheorem{definition}{Определение}
\newtheorem{example}{Пример}

\theoremstyle{remark}
\newtheorem*{remark}{Замечание}

\renewcommand{\le}{\leqslant}
\renewcommand{\ge}{\geqslant}
\newcommand{\eps}{\varepsilon}
\renewcommand{\phi}{\varphi}
\newcommand{\ndiv}{\centernot\mid}

\MakePerPage{footnote}
\renewcommand*{\thefootnote}{\fnsymbol{footnote}}

\newcommand{\resetcntrs}{\setcounter{theorem}{0}\setcounter{definition}{0}
\setcounter{claim}{0}\setcounter{exercise}{0}}

\DeclareMathOperator{\aut}{aut}
\DeclareMathOperator{\cov}{cov}
\DeclareMathOperator{\chos}{ch}
\DeclareMathOperator{\argmin}{argmin}
\DeclareMathOperator{\argmax}{argmax}
\DeclareMathOperator*\lowlim{\underline{lim}}
\DeclareMathOperator*\uplim{\overline{lim}}
\DeclareMathOperator{\re}{Re}
\DeclareMathOperator{\im}{Im}

\frenchspacing


\begin{document}

\section{Спектральный анализ операторов и~динамических систем на графах}

Пусть $(X, \mathcal{A}, \mu)$~--- измеримое пространство с~мерой $\mu$. Пусть $T: X \rightarrow X,
\mu(TA) = \mu(T^{-1}A) = \mu(A) \forall A \in \mathcal{A}$.

$U = \hat{T}: f(x) \mapsto f(Tx),\ \hat{T}$~--- унитарный, $\hat{T}^{-1} = \hat{T}^\ast$.

\begin{exercise}
	Пусть $k_j \rightarrow +\infty$~--- последовательность натуральных чисел. Найти все матрицы $A$,
	такие что $A^{k_j} \rightarrow \frac{A + E}{2}$.
\end{exercise}

\begin{theorem}[Спектральная теорема]
	Пусть $U: H \rightarrow H$~--- унитарный оператор, $U^\ast = U^{-1}$.

	Пусть $\exists h_0: Z(h_0) = Span(U^k h_0: k \in \mathbb{Z}) = H$.

	Тогда объекты, указанные на диаграмме существуют и~она коммутирует:
	$$\begin{CD}
		H           @>U>>  H\\
		@VV\psi V              @VV\psi V\\
		L^2(S_1, \sigma) @> M_z: \phi(\lambda) \mapsto \lambda \phi(\lambda) >> L^2(S^1,\sigma)
	\end{CD}$$

	Притом $R_f(k) = \left< U^k h_0, h_0 \right> = \int\limits_{S^1} z^k d\sigma$, то есть мера
	$\sigma$ есть преоразование Фурье корреляционной последовательности $R_f(k)$.
\end{theorem}

\begin{theorem}
	Пусть $L_2 = \left< t, s \mid s^2 = 1, [s^{t^i}, s^{t^j}] = 1 \right>$. Тогда $Sp(\Delta) = \{\pm
	\cos \pi \frac{p}{q}: \frac{p}{q} \in \mathbb{Q}\}$.
\end{theorem}

Рассматрим графы де Брёйна: $B_n = (\{x = x_{n-1} \ldots x_0, x_i \in \{0, 1\}\},
x_{n-1}\ldots x_0 \rightarrow \{x_{n-2} \ldots x_0 0, x_{n-2} \ldots x_0 1\})$. У~них есть
несколько естественных раскрасок (рёберных):
\begin{itemize}
	\item $x = x_{n-1} \ldots x_0 \mapsto f(t) = x_{n-1} t^{n-1} + \ldots + tx_1 + x_0$. Тогда два
		действия (дописывания 0 или 1) выражаются как $\tilde{a}: f \mapsto tf$ и~$\tilde{b}: f \mapsto
		tf + 1$ (действие в~факторкольце $\mathbb{Z}_2[t] / \left< t^n \right>$, необратимое)
	\item Рассмотрим отдельно первый бит и~обозначим $a(0x) = x_{n-2} \ldots x_0 0, a(1x) = x_{n-2}
		\ldots x_0 1$, а~$b$~все то же самое, но с~флипом последнего бита. То есть действие такое же
		($\tilde{a}: f \mapsto tf$ и~$\tilde{b}: f \mapsto tf + 1$), но
		$f \in \mathbb{Z}_2 / \left< t^n - 1\right>$.
\end{itemize}

Можно заметить, что это на самом деле граф Шрейра группы $L_2$.

\end{document}
