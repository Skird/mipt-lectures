\documentclass{article}
\usepackage[utf8x]{inputenc}
\usepackage[english,russian]{babel}
\usepackage{amsmath,amscd}
\usepackage{amsthm}
\usepackage{mathtools}
\usepackage{amsfonts}
\usepackage{amssymb}
\usepackage{cmap}
\usepackage{centernot}
\usepackage{enumitem}
\usepackage{perpage}
\usepackage{chngcntr}
%\usepackage{minted}
\usepackage[bookmarks=true,pdfborder={0 0 0 }]{hyperref}
\usepackage{indentfirst}
\hypersetup{
  colorlinks,
  citecolor=black,
  filecolor=black,
  linkcolor=black,
  urlcolor=black
}

\newtheorem*{conclusion}{Вывод}
\newtheorem{theorem}{Теорема}
\newtheorem{lemma}{Лемма}
\newtheorem*{corollary}{Следствие}

\theoremstyle{definition}
\newtheorem*{problem}{Задача}
\newtheorem{claim}{Утверждение}
\newtheorem{exercise}{Упражнение}
\newtheorem{definition}{Определение}
\newtheorem{example}{Пример}

\theoremstyle{remark}
\newtheorem*{remark}{Замечание}

\newcommand{\doublearrow}{\twoheadrightarrow}
\renewcommand{\le}{\leqslant}
\renewcommand{\ge}{\geqslant}
\newcommand{\eps}{\varepsilon}
\renewcommand{\phi}{\varphi}
\newcommand{\ndiv}{\centernot\mid}

\MakePerPage{footnote}
\renewcommand*{\thefootnote}{\fnsymbol{footnote}}

\newcommand{\resetcntrs}{\setcounter{theorem}{0}\setcounter{definition}{0}
\setcounter{claim}{0}\setcounter{exercise}{0}}

\DeclareMathOperator{\aut}{aut}
\DeclareMathOperator{\cov}{cov}
\DeclareMathOperator{\argmin}{argmin}
\DeclareMathOperator{\argmax}{argmax}
\DeclareMathOperator*\lowlim{\underline{lim}}
\DeclareMathOperator*\uplim{\overline{lim}}
\DeclareMathOperator{\re}{Re}
\DeclareMathOperator{\im}{Im}

\frenchspacing


\begin{document}

\section{Спектральная теорема для унитарного оператора}

\begin{definition}
	Говорим, что для унитарного оператора $A$ наблюдается явление кратности, если $\exists L_1 \cap
	L_2 = \{0\}: A|_{L_1} \simeq A\mid_{L_2}$.
\end{definition}

Что такое кратность? Модельный пример~--- система из двух одинаковых маятников, независимо
колеблющихся. Более формально, $H > L_1 \oplus \ldots L_m: \hat A\mid_{H_i} \sim \hat A\mid_{H_j}$.

Общие спектральные инварианты: $([\sigma], \mathcal{M}(\lambda))$, $\mathcal{M}$~--- функция
кратности, $\mathcal{M}: S^1 \rightarrow \mathbb{N} \sqcup \{\infty\}$.

Пусть теперь оператор самосопряженный: $A^\ast = A$. Нужно с~помощью дробно-линейного преобразования
перевести спектр на окружность, применить теорему там и~применить обратное преобразование.

Пусть $\Gamma$~--- граф Кэли $G$ по отношению ко множеству образующих~$S = S^{-1}$,
$\mu = \frac{1}{|S|} \sum\limits_{s \in S} \delta_s$ (будем считать, что $\nexists a: a^2 = e$).

$$M_\mu f = \mu \ast f = \sum\limits_{t \in G} f(t^{-1} x)\mu(t) = \sum\limits_{s \in S}
f(s^{-1}x)\mu(s).$$

\begin{remark}
	$M_\mu \cdot M_\nu = M_{\mu \ast \nu}$, где свёртка $\mu \ast \nu$ определяется как распределение
	произведения независимых $s_1 \cdot s_2$:
	$$(\mu \ast \nu)(x) = \sum\limits_{t \in G} \mu(xt^{-1} \nu(t).$$
\end{remark}

\begin{lemma}
	В~пространтсве $L^2(G): \|M\| \le 1$.
\end{lemma}
\begin{proof}
	$M$ есть выпуклая комбинация элементарных действий: $\| Mf \| = \left\| \sum\limits_{s \in S}
	\delta_s f\right\| \le \sum\limits_{s \in S} \left\| \delta_s f \right\| = \| f \|$.
\end{proof}

\end{document}
