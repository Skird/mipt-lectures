\documentclass{article}
\usepackage[utf8x]{inputenc}
\usepackage[english,russian]{babel}
\usepackage{amsmath,amscd}
\usepackage{amsthm}
\usepackage{amsfonts}
\usepackage{amssymb}
\usepackage{cmap}
\usepackage{centernot}
\usepackage{enumitem}
\usepackage{perpage}
\usepackage{chngcntr}
%\usepackage{minted}
\usepackage[bookmarks=true,pdfborder={0 0 0 }]{hyperref}
\usepackage{indentfirst}
\hypersetup{
  colorlinks,
  citecolor=black,
  filecolor=black,
  linkcolor=black,
  urlcolor=black
}

\newtheorem*{conclusion}{Вывод}
\newtheorem{theorem}{Теорема}
\newtheorem{lemma}{Лемма}
\newtheorem*{corollary}{Следствие}

\theoremstyle{definition}
\newtheorem*{problem}{Задача}
\newtheorem{claim}{Утверждение}
\newtheorem{exercise}{Упражнение}
\newtheorem{definition}{Определение}
\newtheorem{example}{Пример}

\theoremstyle{remark}
\newtheorem*{remark}{Замечание}

\renewcommand{\le}{\leqslant}
\renewcommand{\ge}{\geqslant}
\newcommand{\eps}{\varepsilon}
\renewcommand{\phi}{\varphi}
\newcommand{\ndiv}{\centernot\mid}

\MakePerPage{footnote}
\renewcommand*{\thefootnote}{\fnsymbol{footnote}}

\newcommand{\resetcntrs}{\setcounter{theorem}{0}\setcounter{definition}{0}
\setcounter{claim}{0}\setcounter{exercise}{0}}

\DeclareMathOperator{\aut}{aut}
\DeclareMathOperator{\cov}{cov}
\DeclareMathOperator{\chos}{ch}
\DeclareMathOperator{\argmin}{argmin}
\DeclareMathOperator{\argmax}{argmax}
\DeclareMathOperator*\lowlim{\underline{lim}}
\DeclareMathOperator*\uplim{\overline{lim}}
\DeclareMathOperator{\re}{Re}
\DeclareMathOperator{\im}{Im}

\frenchspacing


\begin{document}

\section{Аменабельность}

Пусть $G$~--- топологическая группа.

\begin{definition}
	Левая мера Хаара~--- это такая мера $\mu$, что $\forall B$~--- борелевского $\forall g \in G
	\mu(gB) = \mu(B)$.

	Аналогчино определим правую меру Хаара. Будем называть меру просто мерой Хаара, если она
	одновременно левая и~правая.
\end{definition}

Очевидно, что мера Хаара существует для некоторых видов групп:
\begin{itemize}
	\item Абелевы
	\item Конечные
	\item Счётные дискретные группы
\end{itemize}

Мы хотим дать определение аменабельной группе. Неформально можно сказать, что аменабельность~--- это
про существование эффективного усреднения по группе. Рассмотрим несколько подходов к~этому
определению:

\begin{definition}
	Пусть $\xi: B(G) \rightarrow \mathbb{C}$~--- усредняющий функционал, линейный (конечноаддитивный),
	притом $\xi(1) = 1$. Если он существует, то группа называется аменабельной.
\end{definition}

\begin{definition}
	Пусть $m: 2^X \rightarrow \mathbb{R}_+$~--- конечно-аддитивная мера. Если она существует, то
	группа называется аменабельной.
\end{definition}

\begin{definition}
	Пусть есть последовательность $F_n$ компактных множеств, тогда если $\forall g \in G
	\max\limits_{g \in G} \frac{\mu(gF_n \oplus F_n)}{\mu(F_n)} \rightarrow 0$ то эти множества
	называются Фёльнеровскими.

	Аменабельная группа~$G$~--- такая группа, в~которой есть последовательность Фёльнеровских
	множеств.
\end{definition}

\begin{definition}
	Пусть $T: G \rightarrow G$, тогда оператор Купмана $\hat{T}: f(x) \mapsto f(T(x))$, где $f$
	работает на гильбертовом пространстве $\mathcal{H} = L^2(G, \mu)$.
\end{definition}

\begin{lemma}
	$\hat{T}$~--- унитарный, если $T(x) = ax$
\end{lemma}
\begin{proof}
	$\left< \hat{T}f, \hat{T}g \right> = \int\limits_G f(ax) \overline{g(ax)} d\mu = \int\limits_G
	f(y) \overline{g(y)} d\mu = \left<f, g\right>$. Также $\exists \hat{T}^{-1}$.
\end{proof}

\section{Lamplighter group $L_2$}

В~классическом варианте преобразования $A_a: x_0 x_1 x_2 \ldots \mapsto (x_0 + 1)(x_1 + x_0) \ldots$
и~$A_b: x_0 x_1 x_2 \ldots \mapsto (x_0 + 0)(x_1 + x_0) \ldots$.

Рассмотрим действие на производящих функциях на $\mathbb{Z}_2$. $\hat{a}: f(t) \mapsto (t+1)f(t)$,
$\hat{b}: f(t) \mapsto (t+1)f(t) + 1$. Хотим сделать такую замену $t + 1 = z$, но в~записи $x_0 +
x_1 (z - 1) + x_2 (z - 1)^2 + \ldots$ бесконечное количество слагаемых при 1. Поэтому будем
рассматривать действие только на финитных последовательностях.

Получается другое представление нашей группы: рассматриваем $\hat{a}$ и~$\hat{b}$ на кольце
Лорановых многочленов (ограниченная положительная или отрицательная степень, притом коэффициенты,
конечно, по модулю 2):
$$\hat{b}: f \mapsto zf, \hat{a}: f \mapsto zf + 1.$$

Классическая интерпретация такого действия: фонарщик на бесконечном ряду фонарей. Его два возможных
действия: перейти вправо или перейти вправо и~зажечь лампу. Можно записать с~точки зрения этого
фонарщика следующие преобразования:
$$\hat{b}: (c_j) \mapsto (c_{j+1}), \hat{a}: (c_j) \mapsto (c_{j+1}) + \delta_0, \hat{c}: (c_j) =
(c_j) + \delta_0.$$

В~базисе, $b$ и~$c = b^{-1}a$ группа записывается проще всего, но в~терминах исходных автоматов
выходит нетривиально.

Группа довольно большая, у~её графа Кэли рост экспоненциальный, но тем не менее, она явялется
аменабельной.

\end{document}
